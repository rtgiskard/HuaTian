% <<<1 自定义章节命令
\newcommand{\newsect}[1]{
	\begingroup \clearpage
	\par \vspace{2.8em}
	\LARGE\bfseries\centering #1
	\phantomsection \addcontentsline{toc}{section}{#1}
	\par \endgroup %\vspace{2ex}
}

% <<<1 定义诗文
\newcommand{\poem}[1]{
	\begingroup \par \vspace{2em}
	%\large \centering \bfseries #1
	\large \centering #1
	\phantomsection \addcontentsline{toc}{subsection}{#1}
	\par \endgroup
}
\newcommand{\poet}[2]{
	\begingroup \par
	\centering \small #1
	\ifthenelse{\equal{#2}{}}{}{\footnotesize(#2)}
	\par \endgroup
}
\newcommand{\subpart}[1]{
	\begingroup \par
	\vspace{2ex} \centering \textbf{#1}
	\par \endgroup \nopagebreak[4]
}

% <<<1 env for indent, 缩进环境
\newenvironment{indentenv}[3]{
	\begingroup\par
	\list{}{
		\setlength\parindent{0\ccwd}			% 段落缩进
		\setlength\listparindent{0\ccwd}		% 每项的悬挂缩进值
		\setlength\itemindent{\listparindent}	% 每项标签前缩进
		\setlength\leftmargin{#1}				% 左边缩进
		\setlength\rightmargin{#2}				% 右边缩进
		%\vspace{1ex}							% 环境前水平留白
		\setlength\topsep{0ex}					% 环境后水平留白
		\setlength\parsep{1ex}					% 段落间距
		%\linespread{1.1}						% 行间距,1.1倍
		\setlength{\baselineskip}{1.1\baselineskip}	% 行间距,1.1 倍
	}\item[]#3\ignorespaces
}{ \par\endlist\endgroup }

% <<<1 env for poem,诗词文环境
\newenvironment{shortpoem}[3]{ %<<<2 诗
	\vspace{1em}
	\begingroup \setlength{\intextsep}{0ex}
	\begin{figure}[H]
	%\begin{minipage}[t][][c]{\columnwidth}
	\setlength{\parindent}{0ex}					% 设置缩进
	\setlength{\parskip}{0ex}					% 取消段间距,注意标题与作者间距
	\ifthenelse{\equal{#1}{}}{}{				% 无题
		\poem{#1} \poet{#2}{#3} \vspace{0.4ex}
	} \centering \small
	%\setlength{\baselineskip}{1.2\baselineskip}	% 调整行距,作用于环境,不必考虑后续影响
	\setlength{\parskip}{1ex plus 1ex minus 0ex}
}{
	%\end{minipage}
	\end{figure}
	\par\endgroup\vspace{1em}
}

\newenvironment{longpoem}[3]{ %<<<2 词
	\begingroup \setlength{\intextsep}{0ex}
	%\begin{figure}[H]
	\setlength{\parindent}{0\ccwd}				% 设置缩进
	\setlength{\parskip}{0ex}					% 取消段间距,注意标题与作者间距
	\ifthenelse{\equal{#1}{}}{}{				% 无题
		\poem{#1} \poet{#2}{#3} \vspace{0.4ex}
	}
	%\setlength{\baselineskip}{1.2\baselineskip}
	\setlength{\parskip}{1em plus 1ex minus 1ex}

	\setlength{\leftskip}{4\ccwd}				% 左缩进
	\setlength{\rightskip}{4\ccwd}				% 右缩进
	\small
}{
	%\end{figure}
	\par\endgroup\vspace{1ex}
}

\newenvironment{writing}[2]{ %<<<2 文
	\vspace{2em} \begingroup
	\setlength{\parskip}{0ex}					% 取消段间距,注意标题与作者间距
	\setlength{\intextsep}{0ex}					% 防止浮动体间距太大

	\poem{#1} \nopagebreak[4]
	\ifthenelse{\equal{#2}{}}{}{
		\begingroup \par \centering \footnotesize #2 \par \endgroup
	} \nopagebreak[4] \vspace{1ex} \nopagebreak[4]

	%\setlength{\baselineskip}{1.2\baselineskip}
	\setlength{\parskip}{1ex plus 1ex minus 0ex}
}{ \par\endgroup\vspace{2ex} }

\newenvironment{poempreamble}{ %<<<2 序
	\begingroup
	% also refer to: `texdoc source2e'
	\list{}{
		\setlength\parindent{2\ccwd}			% 段落缩进
		\setlength\listparindent{\parindent}	% 每项的悬挂缩进值
		\setlength\itemindent{\listparindent}	% 每项标签前缩进
		\setlength\leftmargin{\parindent}		% 左边缩进
		\setlength\rightmargin{\leftmargin}		% 右边缩进
		\vspace{0.4ex}							% 环境前水平留白
		\setlength\topsep{1ex}					% 环境后水平留白
		\setlength\parsep{0.4ex}				% 段落间距
		\linespread{1}							% 行间距,1倍
		\small
	}\item[]\noindent\ignorespaces
}{ \par\endlist\endgroup }
% 2>>>

% <<<1 封面
\newcommand{\mytitlepage}[1]{
	\thispagestyle{empty}

	\ifthenelse{ \equal{none}{#1} }{ %<<<2 简短,同目录页
		\title{花田半亩}
		\author{田维}
		\date{}
		\maketitle
	}{
		\begin{center}

		\ifthenelse{ \equal{simple}{#1} }{ %<<<2 简化版
			% 位于页首的 vspace 需要用 * 强制
			\vspace*{64pt}
			\begin{minipage}[c][][t]{72pt}
				\fontsize{72pt}{72pt}\selectfont
				花田半亩
			\end{minipage}
		}{ %<<<2 详细版
			%\fbox{
			\setlength{\unitlength}{1pt}
			\begin{picture}(200,520)(10,28)
				\put(72,428){ %\fbox{
					\begin{minipage}[t][][s]{72pt}
						\fontsize{72pt}{72pt}\selectfont
						花田半亩
					\end{minipage}
				} %}
				\put(148,200){ %\fbox{
					\begin{minipage}[t][][s]{34pt}
						\fontsize{34pt}{34pt}\selectfont
						田维
					\end{minipage}
				} %}
				\put(148,72){ %\fbox{
					\fontsize{14pt}{14pt}\selectfont
					「文集整理」
				} %}
			\end{picture} %}
		}
		\end{center}

		% 换页,页码重置
		\clearpage \setcounter{page}{1}
	} % 2>>>
}
% >>>


% vi: set ts=4 textwidth=94 formatoptions+=2mM foldmethod=marker foldmarker=<<<,>>> :
