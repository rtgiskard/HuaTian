
% <<<1 preamble
\documentclass[12pt,a4paper]{article}
\usepackage{rt}

% <<<2 页面布局及留白
%\geometry{left=2.8cm,right=2.8cm,top=4cm,bottom=4.2cm}		% 页边距
\linespread{1.1}											% 1.1 倍行距
%\setlength{\baselineskip}{1.2\baselineskip}				% 改变行距的正确姿势
%\setlength{\parskip}{1ex plus 1ex minus 0ex}				% 段间距, 0.8 - 1.4 ex 自动
\def\blankrev{\vspace{1ex}}									% 文中留白

% <<<2 封面
\newcommand{\mytitlepage}[1]{

	\thispagestyle{empty}

	\ifthenelse{ \equal{none}{#1} }{ %<<<3 简短,同目录页
		\title{花田半亩}
		\author{田维}
		\date{}
		\maketitle
	}{
		\begin{center}

		\ifthenelse{ \equal{simple}{#1} }{ %<<<3 简化版
			% 位于页首的 vspace 需要用 * 强制
			\vspace*{64pt}
			\begin{minipage}[c][][t]{72pt}
				\fontsize{72pt}{72pt}\selectfont
				花田半亩
			\end{minipage}
		}{ %<<<3 详细版
			%\fbox{
			\setlength{\unitlength}{1pt}
			\begin{picture}(200,520)(10,28)
				\put(72,428){ %\fbox{
					\begin{minipage}[t][][s]{72pt}
						\fontsize{72pt}{72pt}\selectfont
						花田半亩
					\end{minipage}
				} %}
				\put(148,200){ %\fbox{
					\begin{minipage}[t][][s]{34pt}
						\fontsize{34pt}{34pt}\selectfont
						田维
					\end{minipage}
				} %}
				\put(148,72){ %\fbox{
					\fontsize{14pt}{14pt}\selectfont
					「文集整理」
				} %}
			\end{picture} %}
		}
		\end{center}

		% 换页,页码重置
		\clearpage \setcounter{page}{1}
	} %3>>>
}

% <<<2 脚注
\renewcommand{\thefootnote}{\roman{footnote}}

% <<<2 自定义章节命令
\newcommand{\newsect}[1]{
	\begingroup
	\clearpage
	\par \vspace{2.8em}
	\LARGE\bfseries\centering #1
	\phantomsection \addcontentsline{toc}{section}{#1}
	\par %\vspace{2ex}
	\endgroup
}

% <<<2 定义诗文
\newcommand{\poem}[1]{
	\begingroup \par \vspace{2em}
	%\large \centering \bfseries #1
	\large \centering #1
	\phantomsection \addcontentsline{toc}{subsection}{#1}
	\par \endgroup
}
\newcommand{\poet}[2]{
	\begingroup \par
	\centering \small #1
	\ifthenelse{\equal{#2}{}}{}{\footnotesize(#2)}
	\par \endgroup
}
\newcommand{\subpart}[1]{
	\begingroup \par
	\vspace{1ex} \centering #1
	\par \endgroup \nopagebreak[4]
}

%<<<2 env for indent, 缩进环境
\newenvironment{indentenv}[3]{
	\begingroup\par
	\list{}{
		\setlength\parindent{0\ccwd}			% 段落缩进
		\setlength\listparindent{0\ccwd}		% 每项的悬挂缩进值
		\setlength\itemindent{\listparindent}	% 每项标签前缩进
		\setlength\leftmargin{#1}				% 左边缩进
		\setlength\rightmargin{#2}				% 右边缩进
		%\vspace{1ex}							% 环境前水平留白
		\setlength\topsep{0ex}					% 环境后水平留白
		\setlength\parsep{1ex}					% 段落间距
		%\linespread{1.1}						% 行间距,1.1倍
		\setlength{\baselineskip}{1.1\baselineskip}	% 行间距,1.1 倍
	}\item[]#3\ignorespaces
}{ \par\endlist\endgroup }

%<<<2 env for poem,诗词文环境
\newenvironment{shortpoem}[3]{ %<<<3 诗
	\begingroup \setlength{\intextsep}{0ex}
	\begin{figure}[H]
	%\begin{minipage}[t][][c]{\columnwidth}
	\setlength\parindent{0ex}					% 设置缩进
	\setlength{\parskip}{0ex}					% 取消段间距,注意标题与作者间距
	\ifthenelse{\equal{#1}{}}{}{				% 无题
		\poem{#1} \poet{#2}{#3} \vspace{0.4ex}
	} \centering \small
	%\setlength{\baselineskip}{1.2\baselineskip}	% 调整行距,作用于环境,不必考虑后续影响
	\setlength{\parskip}{1ex plus 1ex minus 0ex}
	\blankrev
}{
	%\end{minipage}
	\blankrev
	\end{figure}
	\endgroup
}

\newenvironment{longpoem}[3]{ %<<<3 词
	\begingroup \setlength{\intextsep}{0ex}
	\begin{figure}[H]
	%\begin{minipage}[t][][c]{\columnwidth}
	\setlength\parindent{2\ccwd}				% 设置缩进
	\setlength{\parskip}{0ex}					% 取消段间距,注意标题与作者间距
	\ifthenelse{\equal{#1}{}}{}{				% 无题
		\poem{#1} \poet{#2}{#3} \vspace{0.4ex}
	}
	%\setlength{\baselineskip}{1.2\baselineskip}
	\setlength{\parskip}{1ex plus 1ex minus 0ex}

	\vspace{1ex}							% 环境前水平留白
	\indentenv{4\ccwd}{4\ccwd}{\small}
}{
	\endindentenv
	%\end{minipage}
	\end{figure}
	\endgroup
}

\newenvironment{writing}[2]{ %<<<3 文
	\begingroup
	\setlength{\parskip}{0ex}					% 取消段间距,注意标题与作者间距
	\setlength{\intextsep}{0ex}					% 防止浮动体间距太大
	\vspace{2em}

	\poem{#1} \nopagebreak[4]
	\ifthenelse{\equal{#2}{}}{}{
		\begingroup \par \centering \footnotesize #2 \par \endgroup
	} \nopagebreak[4]

	%\setlength{\baselineskip}{1.2\baselineskip}
	\setlength{\parskip}{1ex plus 1ex minus 0ex}
}{ \par\endgroup\vspace{2ex} }

\newenvironment{poempreamble}{ %<<<3 序
	\begingroup
	% also refer to: `texdoc source2e'
	\list{}{
		\setlength\parindent{2\ccwd}			% 段落缩进
		\setlength\listparindent{\parindent}	% 每项的悬挂缩进值
		\setlength\itemindent{\listparindent}	% 每项标签前缩进
		\setlength\leftmargin{\parindent}		% 左边缩进
		\setlength\rightmargin{\leftmargin}		% 右边缩进
		\vspace{0.4ex}							% 环境前水平留白
		\setlength\topsep{1ex}					% 环境后水平留白
		\setlength\parsep{0.4ex}				% 段落间距
		\linespread{1}							% 行间距,1倍
		\small
	}\item[]\noindent\ignorespaces
}{ \par\endlist\endgroup }
%3>>>

% <<<2 float spaces
%\setlength{\floatsep}{0ex}
%\setlength{\textfloatsep}{0ex}
\setlength{\intextsep}{2ex}

% <<<2 titlesec 定义格式

% <<<2 ctex 设定章节格式
\ctexset{
	section/format = \LARGE\bfseries\centering,
}

% <<<2 hyperref
\hypersetup{
	pdftitle={花田半亩},
	pdfauthor={田维},
	colorlinks=true,
	bookmarksdepth=3,
	bookmarksnumbered=false,
}
% 2>>>
% 1>>>

\begin{document}
	\pagestyle{empty} \pagenumbering{Roman}
	% <<<1 封页
	\mytitlepage{detail}

	% <<<1 目录
	\pagestyle{headings} \pagenumbering{roman}
	\pdfbookmark[1]{目录}{\thepage}
	\tableofcontents
	\clearpage
	% >>>
	\pagestyle{plain} \pagenumbering{arabic}

	\setlength{\parskip}{1ex plus 1ex minus 0ex}

	\newsect{文} %<<<1

	\writing{有关咖啡}{2002年12月14日} %<<<2

		什么时候开始喝咖啡的呢?我早已记不得了。

		关于咖啡的最早记忆恐怕是在爷爷屋子里那套包装精美的咖啡礼盒上了。漂亮的盒子里有两个罐子
	,一罐是纯咖啡,另一罐则是伴侣。不知是谁送来的,爷爷却是没有喝咖啡的习惯,在他眼中咖啡那苦
	苦的滋味是怎么也比不上他手上那壶茶的清香。于是,精美的礼盒便被冷落在了角落里。我想它绝未料
	到自己会遭受这般的冷遇。终于有一天,爷爷奶奶意识到那咖啡若再不喝便要坏了,一辈子勤俭的老人
	怎能接受这样的浪费呢?无奈之下它们只能做出最后的选择,但他们却只取出伴侣冲开来喝,据奶奶说
	那味道有点像奶。于是后来几个月,爷爷奶奶的早餐便成了伴侣加油条。而那罐纯咖啡就那么一天天地
	在罐子中渐渐坏掉了,它是爷爷奶奶宁可视其浪费也不能接受的。

		都说咖啡是苦涩的,而我对于它最早的记忆却偏偏怀着几分温馨。

		后来,是我的初中时光。那时,一切都是很朦胧,对于未来,谁也没有把握。很长一段时间我都以
	为自己考不上高中,我曾经和我的一位朋友说:“我前边是一片黑暗,我不知道自己会走去哪去…… ” 没
	有原因的,我就那么在阳光里莫名地担心着、迷惘着、挣扎着…… 一种单纯的近乎于绝望的东西整日笼
	罩在我心头。或许正是那对未来的担忧在不知不觉中引发了我对生活的最初的一些思考。未经世事的我
	也会偶尔坐下来静想些事情了,于是偶尔也会去麦当劳买杯咖啡来喝,加入很多奶,再续很多次杯。初
	衷只是觉得在咖啡的味道中来看这世界会纯净清澈了许多。

		混混沌沌地就走到了中考面前。

		同学们都喝起了咖啡,却不是去麦当劳。那是一种易拉罐装的咖啡,味道还不错,只是冬天喝起来
	太凉,而且它价格也不很便宜。至少对于我来说是这样的。我便买了速溶咖啡到学校冲着喝,那时正值
	隆冬,午后一杯香喷喷的热咖啡是再诱人不过的了。我的几个朋友也都加入了我得行列。于是,在被冬
	日的太阳照得泛着冷光的楼道一角,每天都多了几个单薄的捧着盛有咖啡的纸杯的身影,不时在初三楼
	道的凝重空气里嗅到几缕苦甜苦甜的香气,传来偶尔的一两声放肆的笑,初三的那个冬天就这么在苦与
	甜,冷与暖,凝重与活泼的协奏中过去了。

		春天终于要来了吧。我们一个个走进一间腾空来充当照相室的教室,坐下来对着镜头微笑。在“咔”
	一声里印记下我们在那个冷冷的楼道中挣扎的最后一段时光。那天,我的头发很乱,一边的头发毫不留
	情地翘了起来,任我怎么努力也压不下去,焦急之中终于还是那样翘着一边的头发被印入了一寸的小底
	片里,或许是头发的缘故吧,我笑的也很不自然。这一串串场景,也一并伴着那如今仍能依稀闻到的速
	溶咖啡的味道被印在了我对于初中最阴晦的一段记忆里了。

		就在中考来临之前,我却突然被宣布不能喝咖啡了,甚至,也不允许去上学,也许还不能参加那让
	我担心了三年的考试。这简直就是噩梦!但它又不是噩梦,我无法从噩梦中醒来,我面对的是真真切切
	的现实。于是,我的世界很久不再见咖啡。取而代之的是弥漫在苍白里的药品的味道,消毒液的味道……
	它们伴着刺眼的白色深深刺痛着我的心。

		春天明明已经来了,怎么却又被锁在了窗外呢?这里似乎是被春天遗忘了的角落,只有那些柳絮似
	是怜惜我们这些被禁锢于屋子里的孩子而飘飞到我的床前,为我带来一些春天的消息。我的朋友们在最
	后冲刺中抽出时间来看望我,它们带来了鲜美的草莓,晶莹的果冻,飘香的果汁,却唯没有我们在那冬
	天一起喝的咖啡。我的心底却永远抹不去那飘散在凝重里的芳香。躺在洁白得怕人的床单上,我的心情
	始终无法平静,终日地胡思乱想着,有时竟会惋惜起那罐一天天坏掉的咖啡,我以为我也会在这床上如
	那咖啡一般一天天地坏掉。

		还好,我终于没有就那么坏掉。在春天已经逝去的一个早上,我走出了那伤心的没有咖啡的屋子。
	屋外已是一派入夏的景象。我的春天呢?它已经不见了踪影。

		我去参加了我曾那样担心过的考试,一切却没有料想得那么糟,我还是顺利地考入了高中。经过了
	那没有咖啡的日子,我的心变得轻松了,我不再为莫测的明天过分地担忧。我懂得了,过好每一天的生
	活。我学着怀着一颗如我最初在记忆中印记下咖啡时的心去面对生活。才发现,当我怀着一颗简单的心
	去看这世界,世界也就没有了那么多的烦恼,原来生活可以简单快乐地像一杯咖啡。

		现在,我又捧起了那盛有咖啡的杯子,记忆如水从我心中流过,苦甜苦甜的气息又萦绕在我的身边
	,一支笔,一份心情,让我又回忆起这关于咖啡的所有。

		有人也曾问我为何不去咖啡店品尝一番那地道的咖啡呢?咖啡店总是一派古朴的风格,灯光是昏黄
	的,精致的杯碟整齐地放置于格子桌布上等待着人们去在这般优雅之中享受那份情调,那份醇香。的确
	,那确是种享受,咖啡是属于那种格调中的,而我却从未走进过咖啡店点上一杯来临窗品味。或许那样
	的咖啡会别有风味,或许那才是真正的咖啡。而那却不是我的咖啡,我的咖啡是儿时那在精美中坏掉的
	咖啡,是在冷光里泛着热气的咖啡,是简单中苦甜苦甜的芳香。所以我不去品那经过太多装饰的咖啡,
	即使它很美,很美。

		咖啡给了我冷夜里的温热,也给了我无数个不眠的夜。我在无眠中伴着它的芬芳回味着一段段有咖
	啡和没有咖啡的日子。或欣然,或心酸,却把生活沉淀得简单起来。心头掠过一丝说不出来的滋味,原
	来也正是咖啡的味道。

		或好或坏的心情,台灯之下,一支笔,一份心情,一杯飘香的咖啡,一颗平静安宁的心 —— 或许这
	才是生活。

	\endwriting


	\writing{婴宁}{2003年2月19日} %<<<2

		—— 读《聊斋志异》之《婴宁》

		花枝间是她华一样的笑脸。\par
		“丛花乱树中,隐隐有小里落。……舍宇无多,皆茅屋,而意甚修雅。”\par
		婴宁是属于这般诗意的风景的,她是在恬静中烂漫着的女子。\par
		她,“拈梅花一枝,容华绝代,笑容可掬”。\par
		她,“年已十六,呆痴裁如婴儿”。\par
		她,“笑处嫣然,狂而不损其媚”。\par
		这样的婴宁纯真可爱,少了封建女子的脂粉气和泣涕泪眼。\par
		她的笑与当时封建教条的反差,使“满室妇女,为之粲然”。

		不合礼教的笑,止不住,“犹掩其口,笑不可遏”。婴宁近于疯狂的笑就像一条皮鞭,声声鞭打着那
	昏昧可笑的教条。

		她的笑却没遭到人们的拒绝,王子服因笑痴迷,王母因笑解忧,邻女少妇因笑“争承迎之”。

		婴宁笑得率真,笑得痛快。即使是在婚礼之上,仍是“女笑极,不能俯仰,遂罢”。她确是像个婴儿
	—— 纯纯白白的,一尘不染。

		婴宁坐于花枝间,她不知,她自己俨然就是一朵最美不过的花。她是属于那里的,她是大自然的。
	于是,婴宁可在王子服提及“夫妻之爱”时问:“有以异乎?”能在王子服说“夜共枕席耳”时答曰“我不惯
	与生人睡”。

		这样可爱得如婴儿一般的女子,应该永远在那乱花丛中微笑,本应永远让她天真烂漫地挥洒。然而
	她终是离开了谷底桃源似的生活进入了俗世。

		于是,最后,她的笑终被凡尘磨灭,婴宁终于因为一场由不诡之人引起的风波“不复笑,虽故逗,
	亦终不笑”。母曰:“人罔不笑,但须有时。”也许婴宁确不该向那西邻子“不避而笑”。而那般憨痴的婴
	宁又如何料得这许多,且西邻子的下场也是应得的。心灵的伤害无以填补,于是曾经笑容可掬的她,永
	不再笑。一个烂漫的女子终是被无情的教条变成了一个终日不笑的少妇。婴宁并非无心之笑,她的情谊
	饱含于她的笑容里,却最终以泪水释放,她“凄恋鬼母,反笑为哭”。

		婴宁的心清澈透明,流于心底的本是一溪欢畅,却都被酿成了苦涩的泪。

		这笑与泪的反差,怎不令人动容?好一个如花的婴宁,好一个可人的婴宁,好一个重情的婴宁。

		但婴宁的故事读起来却总觉得是悲剧,总有暗暗的心痛涌动。

		因为什么?

		因为再不见那个花枝间花一样笑着的她了吗?是的,她的笑,正是那嗅之则笑不可止的“笑矣乎”,
	使“合欢忘忧”无颜色,“解语花”在她面前亦显做作。

		还好,婴宁的儿子和她的笑一样,“大有母风”。

		朦胧间,竟恍然见婴宁依旧立于乱花丛中,南风徐动,她笑着,笑着,一样的纯真,一样的美……
	一如从前,婴儿一样。

	\endwriting


	\writing{耳畔的声}{2003年3月1日} %<<<2

		喜欢那样的声音,那样温柔地在心田里漫漫散播,小心翼翼,像是怕吵醒了梦着的人。

		声,流淌在天外,伴月而来,载那一朵朵芙蓉似的云起舞翩翩。我多想独坐山间,静享这天籁。奔
	涌的溪水,脉脉流去,遗予的是一身轻盈明慧,带去的是我满心的污浊。溪水在耳畔,轻轻抚摩久已迷
	失的爱,久已无影的甜。

		甘冽的不是自如桀骜的水,是回响着的溪的轻柔话语。静思冥想,就在天际或也有这般安然的一溪
	水声,那便是漫天的雨,洋洋洒洒,幻一样,雾一样。

		雨落本无声,听到的只是云里坠落的灵魂在呜咽。

		不知道曾有多少个窗下听雨的午后。在初春、在夏末、在秋风乍起时,在未成雪的日子。一点点微
	微细声,敲着渐已迷蒙的窗,划下道道生的轨迹。没有尽头一样,绵延在我润湿的眼。雨雾间又有多少
	挣扎绵延如这细声的生命,在潜行,在摸索。寻那一条回去云端的坦途。

		细声回旋,像在欢笑,又好似哀叹,听不清。

		我勉强地平静着,佯装一副漠然。在乎的只有耳畔那些无可捕捉的声。

		原来,我是一汪谷底的碧蓝,深深将所有的美丽藏匿,只给这世间一汪如此凄寒的蓝。宁静,没有
	声响,在孤寂里沉睡亿万春秋。其实需要的不过是一个声音,一个足以唤醒我的声音。也许,是一粒远
	方随风而至的沙石;也许,是迷失路途的旅人;也许,是泛舟水上的如花女子。激起一阵涟漪;唤我无
	人知晓的名;或用你的兰浆拨弄我柔软的青丝。让貌似的平静苏醒,重生。

		泛舟女子轻唱幽歌,水波附和着,轻轻地漾。荡在亿万次的落花声里。

		娇柔的身子,总经不起秋的洗礼。

		在阳春,听过花开的声音。号叫着、挣扎着、用尽全部心力地撑开层层厚重。我俯耳过去,在终于
	冲出的一刻,听到的竟是生命的陨落,一种醉人的厉声。全部的过程只是挣扎在花苞,而不是摇曳在春
	色里。你说,你的美丽不过是生命完结前最迷人的谢幕。我于是细听你的一切,一切的挣扎乃至最终欢
	愉的释放。而在坠地那一刻,什么也没有。

		来年又是春光,你重生,依旧挣扎着号叫,然后,为生命再次美丽地完结,谢幕。

		完结,原来可以这般动人。

		寒夜灯下,再无什么声将我的心魂拨剌。只有你,墙角深处,不知何所的一股风,为我吟你遥远的
	诗篇,悠悠在我昏黄的灯下盘旋。你来自沙漠,那里有灼热的沙,把路人的眼迷乱;你来自冰原,那是
	茫茫透明的蓝;你要去开满花的山冈,你要采一朵送你最心爱的姑娘;你要去涌起风浪的海上,你爱那
	种漫卷的激荡…… 你的诗篇,书写在远古,飘荡在我的耳畔。太多的太多,在你低低的声音里,我梦着
	你的梦,那些醒来便灰飞烟灭的美丽。

		就像我永远不再的童年。

		淡淡的记忆支撑不了一颗未了的心。还好,还有飞扬在儿时黄昏里的那一丝声。橘色的天空下响着
	的是什么已记不得,是暮鼓,是轻哼的歌谣,还是漫天飞舞的鸽哨?越用力去想,却越是一无所获。

		才发现,陷落在光阴里,无处寻觅的声,最令人销魂。

		声在耳畔,低语、轻歌,或吟、或不见。轻柔地在心的深处栽植片片芬芳的花园。在夜近更阑,在
	梦醒的清晨,在醉着的午后,把短促的生薰染得清澈动人。

		忘情在声的世界,细品每一次叶的颤动,每一轮无声坠去的月,每一点花间的声响。

		生,原是这般动情的声。

	\endwriting


	\writing{树}{2003年3月10日} %<<<2

		不经意地,在依旧的风里嗅到一缕春的味道。

		明明只是3月出头,明明还是一身厚重,而春的味道就这么悄然渗到心底来。我知道那是萌动着的
	暖,舒缓地在校园,在街边,在荒野,在每一个凄凉着的角落把明媚的序曲奏响。我明白,就在不太远
	的那么一天,到处都会洒满崭新的阳光和如昨的烂漫。

		一切的遐想与憧憬,都只是源于那颗树。

		一颗我没注意到的树,它安静地立在路边,和所有的树一样,似乎没有什么特别。于是,我差点和
	它擦肩而过,于是差点路过了那一树的美丽。

		“孩子!看那颗树,向上看!”

		“哇!”

		孩子的惊叹声拉回了我的脚步,我回过身,下意识向那树望去,满枝的花苞!在这样的冷风里,竟
	生出一树的娇柔!我仰着头,吃惊地望着它。

		“看见了吗?它比别的树都早了一个月呢!”母亲抚摩着孩子,眼中透出点点温暖。孩子吃力地仰着
	头,一脸的明媚。

		天,灰蓝灰蓝地低沉着,风里继续哀吟冬的悲歌。花在枝头,在被压抑得几近窒息的空气里,轻轻
	摇曳柔弱的身躯。淡淡的一种紫,让我知道它是紫玉兰。它为何如此急迫地开出花苞来,是为了让路人
	早些感觉到那即将遍布的暖,还是,它怕错过了什么?

		于是,想起那样的诗句:

		\longpoem{}{}{}
		如何让你遇见我 ~ 在我最美丽的时刻 \\
		为这我已在佛前求了五百年 \\
		阳光下慎重地开满了花 \\
		朵朵都是我前世的盼望

		当你终于无视地走过 \\
		在你身后落了一地的 \\
		朋友啊那不是花瓣 \\
		是我凋零的心
		\endlongpoem

		一颗开花的树,俨然是它,那个个花苞里饱含的不是花,是一颗颗守望着,祈祷着的心啊!

		我却差点错过它,差点负了它500载的艰辛。

		我立在那颗树下,久久地凝望。渗进心来的不只是暖的气息,更多的是一种淡淡的忧伤。它是否真
	的求了500年?它又多少次在路人的匆忙脚步中,飘落了它凋零的心?

		是这样吗?所以她便这般急迫地将自己装点,怕那又一次无视地经过。

		所以,即使还只是3月,即使还是一身厚重,它已经倔犟地让自己的美丽立于路旁了。

		就在这样平凡的路边,就在这样料峭的风里,不知还有多少颗树在等待着,期盼着。

		它们的美丽也许不过是为了赢得你惊喜的一瞥,如此而已。这样的生命,在等待中美丽着,在脆弱
	里坚强着,默默地独立。

		3月,才只是3月,这天还太阴寒,这风还不够暖,你为何如此急迫,如此急迫地把美献予这世界?

		我怀着莫名的哀伤和感动,悄悄离去,心里不知塞了什么,温温的。

	\endwriting


	\writing{做梦的孩子}{2003年3月12日} %<<<2

		一直喜欢做梦,喜欢在平淡里怀着一点点美丽的憧憬,像隆冬里对春的期待一样,让心头痒痒的,
	滋生出来未知的快乐。常常坐在窗子前看黑暗中摇曳的微微灯光,灯下的人你可在梦着?灯光闪闪烁烁
	,给我无声的对答。夜,静谧着,梦应该释放了,无论是醒着或睡着。纯粹的世界里梦也是纯粹的。

		醒着,我透过重重围墙望那远山的轮廓,山渺茫得如一幅淡彩画。微蓝的薄雾里藏匿着亘古的秘密
	。鸟儿偶尔掠过,无比婉转地卖弄几声。我看见了你,在桃花掩映的柴扉后,你桃花似的笑,你说山后
	的蝶儿归来了,去年一样,依旧舞着闹着,我于是去看,才又见那满坡的杏花,雪白雪白地飘散。一阵
	欣喜,一阵心酸,乱乱地在我心头摩挲。泉水叮咚,为观蝶而跋涉的我唱起莫名的歌。蝶儿又在何方?
	记不记得,想起的只是桃花似的你,和你桃花似的笑。山渺茫着,渺茫着,渐渐离去,最终竟不见……

		眼前,只剩一面苍白苍白的墙。

		原来,夜依旧是夜,墙终归是墙。我挣不脱,梦也挣不脱。

		于是,便睡去吧。

		夕阳在天边伸展,是紫色的天,紫色的云,紫色的海和沙滩。抓了一把沙,凉凉的,在我手心渗透
	阳光的最后一丝温存。海风柔和起来,把沙轻轻抚慰,为它抹出道道柔和的弧线。月亮还没有出来,海
	被诡秘的气氛包围,我脱了鞋子,光着脚在沙滩上走,再没有余热,海水阵阵涌来,凉凉地漫过了沙,
	漫过了我的双脚。远处飘来一缕琴音,悠悠荡荡,似乎是竖琴。是谁,在这阴郁的海上拨弄琴弦?是谁
	,在这无月的夜晚分享着我的孤单?琴音依旧,海也依旧,深暗的蓝在每一处弥漫。

		我伸手触那天空,顺着手尖缓缓流下些透明闪着光辉的液体,那是什么?是飞溅的浪,是天上的琼
	浆玉液,还是我的泪?它清晰地滑落,渗在沙里,没有痕迹。一阵浪涌来,淹没了它,和睡熟了的我。

		窗泛起亮色,刺眼却又柔软着。风慢慢拂起窗帘,帘子飘忽在屋中,摆动着,发出特别的笑声。我
	起身,才知道是早晨了,而窗外,又是风起的日子

		所有的所有,依旧,依旧。

		歪在床上,想起梦中的蝶儿来,又想起飘在海上的琴音。也许,真有那么个世界,也许,真有那么
	个寻蝶的我,那么个光着脚踏沙的我。天花板上没有花,和墙一样,一样的苍白。于是,下决心把墙刷
	成粉色、紫色,或者其他的什么能有些梦的色彩。白色,太单调,不是梦的居所。

		继续的是没有梦的生活,现实得不容我再去梦什么。白日里的梦,是被视为浪费时间的。而时间又
	在哪里?苍白的墙壁里吗?或者,苍白的面孔里?如果,时间这么苍白着,没有了梦,又怎么去度过,
	我怎么去将短促的生命耗尽?那又将是怎样的煎熬,我不敢去想,我怕这时间的苍白,这时光的苍白,
	这一张张面孔的苍白。没有梦的润色,光阴只不过是白纸一张,单薄没有重量。

		所以,我躲进粉刷一新的房里,放肆地梦着,有什么不妥吗?也许它们荒唐,也许它们可笑,但都
	是梦,是多彩的、美丽的。不是所有的梦多来得及实现,那么又何妨一梦呢?

		来,梦我们的梦吧。

		在光天化日下,在有月或无月的夜晚,梦你心的归所,梦你我远逝的昨日,梦所有的奢望与狂想,
	自由自在,那里有另一片天地。

		梦一切可梦,梦一切不可梦。

		在色彩里,找到迷失在苍白里的自己 —— 那个苍白着面孔不知所措的孩子。

	\endwriting


	\writing{水仙}{2003年3月14日} %<<<2

		是我爱上这严冬,还是你高洁的娇柔,让我眷恋着灰沉的天,眷恋着屋外肆意的风响。

		总是在新春来临的时候,你怯怯地露出头来,在一片的火红里放着几分恬适温馨的纯色。我喜欢把
	你放在我空荡的窗台上,让一大块沉闷着的天做你的背景,这样的安排,让人匪夷所思,却是我不变的
	决定。

		从还是一颗笨笨的“葱头”开始,你就被安静地放置在那里,映着身后的灰和平静的黯淡。我精心在
	你的身体上雕琢,刻出一颗颗藏于厚重下的幼芽,我知道它孕育着你惊人的美丽。于是,我非常地小心
	,捧于手心的已不只是一株花儿,而是一件珍宝,一件艺术的化身。我日日坐在你的面前,美丽地期待
	着,久久凝视稚嫩的芽,默默地念着:就在这里,就在这里,就要盛开出如昨的纯洁与美丽……

		天,灰蒙蒙地压抑着所有生命,像是要吞噬了我,吞噬了意念里星星点点的期待。我不惮天的威胁
	,我明白,这就是冬天,冬天就是冬天。

		窗子不很结实,在冬的号角声里吱吱呀呀地呻吟,偶尔还有一两声咚咚的巨响。我缩在厚重的被里
	,听着,心也就吱吱呀呀地疼起来。看着安然的你,没有任何动静,像个吓坏了的小生命,小小的芽死
	死地立着。我于是开始怕,心头轻轻颤动起来。

		阳光在某个午后偶然透进来,舒心地在窗上摊开,却依旧是冷冷的,在雪白的墙面上映出你单薄的
	影,剪纸似的生生硬硬地浮在那里,我看着看着,就骤然升起一缕失望。这样的痛苦煎熬着我,我怕着
	,每天继续着莫名的担心。也许,我不该把它们安放在那样一个冬可以随意来去的角落。我有些后悔,
	虽然,那是我不变的决定。

		终于,在一个不太冷的清晨,我见到了你抽出的叶。嫩生生地竖着,油油的绿荡漾着我的心,花,
	已经不会远了吧?星点的期待被瞬间点燃,在朦胧的晨光里,我看见那叶嫩嫩的笑。

		家里人开始忙起过年的事来,气氛渐渐露出一点粉红,隐隐的,却又分外清晰。那一天,我登上凳
	子,把一幅火似的春联贴上,条条的火红燃着阴冷的空气。慢慢空气泛起些暖意,身边的气息叶被烧得
	火热,闪闪烁烁地冒着喜庆的火光。是新春了,一切如此真切地飘在冷风里,于是,风也不再那么凶,
	也被薰得多了几丝温柔。叶已碧绿成一片了,看起来你已经不那么柔弱,坚强了许多。

		像是一阵浪涛,你的美丽在除夕那夜涌来。雪白的花瓣,鹅黄的蕊,昂扬地挺起胸膛,激荡地美蕴
	着几分含羞,娇柔的姿态中透出一心高洁。脱出尘世的清明亮彻,伴着红艳艳的新春,散发的不止是芬
	芳,更是动静之间悄然生出的喜悦与希望。

		我空荡的窗前满是你的动人,满是你给的暖。我恬淡地笑着,眼里闪着你给的光辉。你的身后是依
	旧的灰沉,依旧的黯淡,却早已不是你美的遮盖。吱呀的窗不是在呻吟,而是在为你的胜利,你的高洁
	,你的傲骨而高歌。

		这样的你,如梦纯粹,如天高远,在最惨淡又最火红的光景中怒放。于是,人们认定你是天仙的化
	身,于是,人们这样诗意地唤你:水仙。

		你悄悄地告诉我:在不完美的世界也可以燃起迷人的火光,压抑的灰暗里也能生出芬芳,只要你愿
	意,你也可以,你也可以让隆冬因你而美丽……

		你的话语,像一首隽永的歌,熟悉又陌生着,在不再嚎叫的冬夜缠绵低回。让我灰着的心欣然徘徊
	在你给的世界,在心里反复低歌着,你给的旋律。

	\endwriting


	\writing{浪漫·在一瞬间}{2003年3月23日} %<<<2

		浪漫,动人的字眼,它浅浅地迂回在心底,却又似是无可触摸。它近乎是一种奢望,一种平淡生活
	中的幻影。浪漫,似乎太远,太远。

		我像是没有力气将它找寻,四目一片凄荒,街道的景象像一部老电影,光纹闪闪烁烁,一切笼在灰
	白的调子里。我用力撑开了伞,迎着冰冷的风缩在伞下,微微叹了口气。雨并不太大,淅淅沥沥地漫天
	飘洒,像是委屈的孩子躲在角落里偷偷抽噎着。春天下些雨本该是惹人欢欣的,而街上又有谁有一脸喜
	悦?人人都只是漠然地在光纹里匆忙潜行,再在雨雾中渐渐消失了自己的身影。我没有任何不同,一样
	的漠然,一样的匆忙,一样在灰白里消失了自己。我麻木地撑着伞,麻木地举步,却不知道方向。或许
	该回家去,或许该去趟书店,或许,就该这么毫无目的地走走,毕竟心在屋子里被拘役了太久。没有主
	意,哪一种决定都像是合理的,却又都好像不合理。于是,就这么徐徐地在灰白的街头踱步,不经意中
	已走了很远。眼前是座车站,黑压压的人拥挤在一小方雨棚下,齐刷刷的目光都向着车来的方向。我于
	是停下了脚,立在雨棚的外面,我发现自己喜欢那一张张急切期待着的脸。那么多的人,拥挤在那么一
	小方无雨的天空下,一起期待着共同的期待。这样的安排是谁的决定?让这些本毫无关联的人有机会被
	这样联系在一起。他们的目光就像背负着共同的命运一样,严肃而专注着。这便是所谓的缘吗?给了他
	们共同的一场雨,于是有了头上这共同的一小方天空,给了我这偶遇的莫名感动。我心里恍然暖和起来
	,觉得有什么东西闪着一点光亮。

		我和雨棚下的人俨然是在两个世界,一个单薄而灰暗着,另一个则散发着缓缓的温热。想收了伞躲
	到那里,却发现举伞的人在那里是另类,是不合常理的。于是,我只是远远地站着,安静地伴随着雨声
	轻轻抽噎起来。我不明白,为什么会流下泪来,也许是对那一点亮光的感动,也许是满心阴晦的肆意,
	也许那只是雨水。生活与梦总是差之千里。梦里的阳光一天天阴暗下来,浪漫的心绪渐渐枯萎,现实涌
	进我那仅存的一丝天真。梦里的太阳不再火红,我知道,有一天它会冷却成一座冰山。

		那时,我该怎么活……

		我低着头,伞掩盖了我狰狞的脸。

		猛然,在雨的微声中传来一个男人的呼喊:“不要挤!不要挤!谢谢,谢谢……”那声音刺痛了我的神
	经,它太尖锐,太猛烈。我抬眼顺声望去,于是看见那个呼喊的男人,和被他拢在怀中的那一大束红玫
	瑰。他挣扎一样地扭动着身体,竭力拢着那花。红玫瑰很漂亮,尤其是在这样的灰白里,他是令人为之
	惊叹的一抹亮彩。鲜红吸引着我,在隐隐约约的闪动里,我看到男人与玫瑰不太相衬的破旧衣着,一件
	洗得泛白的劳动布上衣在人群里格外刺眼,我听到他声嘶力竭的声音。他艰难地上车,艰难地拢那一团
	红色。而身边拥挤的人群却毫无顾忌地推搡着,争抢着。他们好像没有看到男人扭曲的姿态,没有见到
	那一束惊叹,甚至也没有听见男人那带着哭腔的叫喊:“不要挤好吗?我的花!这是给我妈的!她在医
	院,她需要它!她需要它……不要挤,谢谢,谢谢了!我的花……不要挤……”

		男人一遍遍地叫着,喊着,嚷着,却没有任何回应。我放低了伞,再次遮住了自己的双眼,我转身
	准备离去。但还是忍不住回头最后望了一眼,灰白里,男人继续扭曲着身体,继续着呼喊,人群也继续
	着拥挤。

		我于是疾步走开了。那一幕却一次次缓缓在脑海划过,抹不去。

		一束火红,一束绽放的爱。我回味着那个扭曲的身影,和他医院里期待着的母亲。不知男人最后有
	没有停止叫喊,不知那束红玫瑰是否安然,不知他母亲有没有在红色的幸福里甜甜笑了。也许,她会把
	花放在透明的瓶子里,放在床前,也许,阳光会刚好透进来,柔柔地抚着花瓣,会有一些美丽的露珠挂
	在上边偷偷地笑……

		会是这样的吗?但愿。

		浪漫忽而在心中涌起,母亲需要的不只是一束花而已。是浪漫,是温馨,是花里蕴着的浪漫,是芬
	芳传达的温馨。浪漫总是深深埋在心底,在脆弱和无助中才勇敢地迸发吗?

		或许,世界依旧苍白着,光纹依旧闪动,而只要有了那一抹亮色,即使身边满是凄荒,冷淡与麻木
	,风也会吹得柔和,心也能寻得一份温存。

		很狭义的浪漫,不是花前月下,没有风花雪月,不过是麻木生活中的一点鲜活。而我真的需要它,
	不过是一点点亮色而已。那些听不到呼喊的人的心,又是怎样的?是否和我一样,在几近枯竭的生活中
	,寻这一点鲜红,一点平淡的浪漫?我不知道。

		浪漫或许并不太远。

		浪漫也许只是春天萌芽的一颗小草,只是夏夜水边漫天飞舞的萤火,只是深秋姑娘颈上那条随风飘
	忽的丝巾,只是冬日里孩子向天空吐的那一口白气。

		所有的情节原来都可以归结于浪漫。只是我的脚步太过匆忙,我的目光太过麻木,错过着它。

		于是,学着精心地去走路。

		因为,浪漫闪过,只在一瞬间。

	\endwriting


	\writing{唱一只歌}{2003年3月28日} %<<<2

		一直不善于唱歌,也曾经在小学时参加过合唱,但也不过是陪衬的角色,张张不出声的嘴罢了。

		我的小学很小很小,没有漂亮的教学楼,没有一坛又一坛的花,有的只是蓝色墙面的几排瓦房和房
	前房后的槐树,朴实无华,单纯得如整日嬉闹于其中的孩子,很夸张,每个年级只有一个班,人虽然少
	,却没有多少清净的时候。就在第一排瓦房,有我们的音乐教室。漆黑的被擦拭得能映出人影的钢琴;
	明黄色的好像跳跃着的椅子;黑板上画得分外齐整的五条线和高音谱号;这一切,在我幼小的心灵中高
	大上着,神圣着。这里是间教室,更是一座殿堂。光洁闪亮的窗透进5月的柔光,漫漫地洒在我们身上
	,轻轻灼烤着,散发出一丝丝温温的5月独有的味道来。坐在那高高椅子上的我们,挺着胸脯和着老师
	的琴音在练声,高高又低低,一遍又一遍。老师脸上笼上些妩媚,稚嫩的声漂浮在校园。槐花开得正好
	,层花间透过一线线柔软的光芒,细细的,亮亮的,调皮地一闪一闪。小些的孩子喜欢一群群地站在树
	下,仰着头看那高高的花。花,纯白白的,和孩子们高仰的脸一样。风抚着花,花弄着风,稚气的歌掠
	过无云的碧空。

		那槐花是可以吃的。我家门前也载着两颗槐树。高高的枝斜斜地伸向天空,遮住了大半个院子。我
	和哥哥于是常常爬上房去摘槐花来吃。恍恍惚惚的绿枝白朵间,哥哥那张快活的脸我依旧记得。他总是
	坐在最粗壮的那个枝上,一把把地摘着吃着,并喃喃地哼着正流行的曲子。这时的我通常是怯怯地坐在
	离枝较远的房脊上,远远地看着他,听着他轻轻哼着的歌,这么多年过去了,他唱的歌我却分明记得,
	正是那一首《水手》:“风雨中那些痛算什么……”我怎懂得何谓“风雨”?那些坐在屋脊上日子,只知道春
	天很好,槐花很甜,哥哥的歌声很小很小。

		在学校上课的日子里,我总是趴在小桌上向外看,有时竟会出了神,奇怪地担心在合唱比赛时老师
	会不会让我站在第一排,有忽然想起暑假时的一场雷雨刮倒了好几颗小些的槐树 —— 那一次真的吓到我
	了。那时候,我有资本去畅想,我有的是时间。

		学校的墙一遍遍地刷,几乎每个学期都要翻新一次,于是蓝色也不停地变换,有水蓝,有淡蓝,有
	古蓝……但始终是蓝色,一成不变。我曾经问过老师为什么总是蓝色,记得她笑了笑说:“蓝色会让人安
	静,唱起歌来也会好听许多。”我没有明白,痴痴地望着那水蓝色的墙。槐树的绿色舒适地摊在上边,
	绿印出斑驳的影,在蓝上缓缓地摇,舞一样,很安静,很动听。

		毕业那一天,我们表演了合唱,在树影婆娑里唱完了我小学的所有时光。我站在第一排,穿着蓝色
	的长裙,第一次唱出了声音。

		下台时,一大滴眼泪洇湿我的长裙,我第一次明白了,蓝色不只是动听的,也是忧郁而感伤的……

		后来的时光,唱歌的机会越来越少。

		我小心地走自己的路,时间一点点缩短,整整的一天仿佛不再有24个小时。偶尔也只是在回家的路
	上高歌一曲,当然,是在无人的小街。却常常有人在这时突然出现,我于是只得尴尬地闭嘴,逃似的跑
	掉。我的脚步渐渐有些蹒跚,歌声塞在了侯间。

		但不知为什么,唱歌的欲望却一天天强烈起来,像是要喷涌的热情在我心底奔腾。我想唱歌,并且
	是极大声地唱!所以学会了在心底唱。

		我会闭上劳累的双眼,一语不发地唱歌,让一个个奇异而安静的音符在心头掠过。没有声音,却格
	外动听。不经意间,嘴角便露出一缕浅浅的笑。因为那歌声里有从前的无瑕,从前树们跳过的舞,从前
	哥哥轻哼过的调子。

		那声音在心底,清晰又虚无。

		也会坐在阳台上唱,多半是在夏夜里,开着窗,任晚风夹着燥热徐来。我穿着水蓝的浴衣,定定地
	立在那里,浴衣的带子缠绕着,摆动着,在湿热的空气里拍出快乐的声音。我淡淡地唱,偶尔忘情地闭
	上了眼,偶尔听见蚊虫在耳畔嗡嗡地闹着。就任它们闹,或许它们也是在唱歌,我想。在记忆里,我在
	阳台上唱的歌大多是感伤的。那声音飘飘缈缈,在寂静的夜空盘旋,不知传去了哪里,又落在了谁的窗
	前。

		我的歌声却始终没有悠扬过。所以,我只是在低低地唱,像是母亲在唱的摇篮曲,柔柔的,轻轻的。

		于是,开始羡慕有美丽歌声的人。我的生命因为有了歌声已经很美很美,而他们一定会更美吧。

		曾经的时光一点点在闪烁,细细的,亮亮的。在一片又一片的蓝里舞着,舞着,有一点点感激和欢
	愉,一点点迷惘和惆怅。

		常常地想,也许有一天我会爱上一个有美丽歌声的男孩,就像爱这唱歌的时光一样……

		一声声,一响响,所有的歌都被悠悠唱起,似是没有尾声,或纤细或激昂。

		我依旧反复吟唱着自己的歌,在那一个个骤然风起的日子。

	\endwriting


	\writing{曾经}{2003年4月27日} %<<<2

		有一些日子远远地去了,像是陷落在某个时光的转角,再也找寻不回。

		\blankrev
		曾经,一样是4月,我微笑着徜徉在被阳光暖得温热热的小街上。天是久违的水蓝色,鲜亮中透着
	甜甜的气息。花自然是一串又一串,红红或粉粉地闹着。我的心汩汩地涌动着一些什么,洗去了,整个
	冬天沉淀下的灰沉等待。春天看在眼里的是明媚,嗅到心里的是悠长悠长的香。我喜欢这样的小街,一
	切都那么自然而然地散发着迷人的希望和春光。

		这样的日子里,母亲总会为我添置一两件新衣。她说,春天是女孩子的节日。她说,春天应该打扮
	得更漂亮才是。那时的我虽然尚小,却一样是爱美。母亲总会在商店中一件又一件地帮我试穿衣服和鞋
	子,她的目光里充溢的是幸福,那一种幸福的目光,我记忆犹新。

		那一年春天,我们最终买回的是一条幼粉色的棉布连衣裙,和一双光亮亮的红色漆皮皮鞋。我迫不
	及待地把它们套在了身上。母亲用五彩的皮筋为我束起两个纤细的小辫。看着镜中的自己,我自认很美
	。接着便急匆匆跑到了门外无边无际的春光里。

		春天真的是节日,不只是因为那一两件新衣,更是因为那满心满目抑制不住的欣喜与躁动。

		曾经,一样是4月,我坐在湖边的长凳上沉沉地睡了。阳光在斑驳的树影间时隐时现。湖水是多情
	而温柔的,徐徐送来一阵阵微风,慢慢拂过我红润的脸,再小心地摆弄起我纤细的发丝。

		一切都好似浸在了无声的绿意中,和我一样沉沉地睡着,陶醉着。此时的心是安静,是平淡,是一
	种超然的境界。

		不再是小女孩的我依旧爱着这节日,心依旧是被塞满了无尽的欣赏,而对于春的体味却多了一份恬
	美少去了几分躁动。

		春天,永远只是在窗外的世界中。春,不能是锁在园子里的。

		曾经,我带你去山上看那漫山遍野的花们。看它们在天空下自在的舞姿,听它们花瓣碰撞间的微声。
	你拉着我在淡紫的世界里跑着,耳畔想着呼呼的风,温热了冰凉了一个寒冬的耳朵。那一刻我分明听到
	你心底的声音,朋友真好,春真好,世界真好。

		在4月,我觉得自己的肉体是鲜活轻盈的,我觉得自己复活在阳光里。在4月,我触摸到生命的轮廓
	,光滑却崎岖,笔直却蜿蜒。

		这一切,都只是曾经。曾经远远地离开了我模糊的视线。4月,不再是欣喜和甜美,4月,突然变得
	离奇恐怖起来。

		今年,我看不见你微笑着的脸,你的脸被隐藏在了层层叠叠惨白的纱布后边。我嗅见的不再是香,
	而是刺我心疼的消毒水的气味。

		一向,很害怕被蔽起的面孔,很害怕弥漫的可怕气味。世界瞬间变得有一点昏暗,比冬天更阴沉,
	比秋天更惨淡。我不再敢用自己的口鼻来自由地呼吸,我怕,我真的怕,即使依旧是在我那么深切爱着
	的春光里。

		今年的春天,不再是节日。

		4月不再是4月,恐惧一点点侵袭我们本已脆弱的生命。窗外的世界变得好像危机四伏。我不懂,为
	什么会这样。这一个春天我该怎样将它生生消磨。我是胆小的,面对种种我无力举步,不敢向前。

		而总是要有人先勇敢起来,一群穿着白色衣衫的天使无畏地去了最危险的战场。她们是一点希望,
	一点力量,给了我希望,给了我力量。

		我于是一遍遍地对自己说:“你要勇敢,一定要勇敢起来。”

		我却是一刻没有停止过对于曾经的那些怀念,我不愿见你被惨白蔽起的面孔,不愿见你无助而失望
	的眼,我想念曾经,想念曾经的4月。

		曾经或许会回来,因为,世界还是世界,只要我们都还能好地呼吸着。

		当曾经回来的时候,轻带我去添一两件新衣,请陪我在光亮的小街上漫步,请让我靠在你肩上在湖
	边沉沉睡去,请与我挽着手臂再上山去,请同我一起感谢所有那些勇敢的天使们。

		曾经会回来的,我相信……

	\endwriting


	\writing{浅浅浮过}{2003年5月2日} %<<<2

		日子慵散地过着,轻描淡写。

		淡淡的惆怅,浅浅的伤感,一切都那么悠长地在不经意间划过。像一叶小舟,轻帆一卷。孤寂中载
	着沉默,随时光的逝水飘忽而去。伸手探出小舟,想触摸光阴的身形,却是捉回一手的愁云,一心的迷
	茫。颠簸着,遥见前方,浪又高了几尺,白蒙蒙的一片水光。我俯身出舱,仰望穹窿,灰蓝里坠下点点
	沉重,压得欲哭无泪的人没有了长啸的勇气与力量。

		轻声地,我摇动尘封了不知几世的兰桨,唱咏起秦风里失传已久的音律。

		“蒹葭苍苍,白露为霜,所谓伊人,在水一方……”

		我在荡漾的流水间,忘情地唱着,唱着。淡淡的惆怅再次掠过,不留痕迹,只给了我满舱的荒凉。
	薰风徐动,似是那伊人身旁浮来的芬芳。而歌者已逝,今昔遥望伊人的他又让寂寥的孤魂停留在什么地
	方?

		鱼影穿梭,依旧是种浅浅的青色,依旧在清澈中划出道道淡淡的水波。舟的心被这太温情的柔波轻
	抚着,默默沉醉。

		我的歌荡入微蒙的水光,依稀中在渐渐平息的浪里起伏。没有人来回应。小舟上的伊人身披罗绮,
	若有若无地微笑。缕缕青丝在愈浓的雾里轻柔地舞着,飘渺间,小舟被团团烟雾隐去,消失在青与白之
	间……

		恍然,有什么堕入深潭。

		习惯着在流水之上孤身一人,形影相吊。慢慢在空阔的水上等待夜的降临,只为那一轮常是残破的
	月。不过是些许微薄的亮,却把醉人的流光倾泻于这一片辽远空寂之上。

		月出于东山之上,令我想起同是在这般流水之上歌咏的东坡居士,还是这样的夜,仍是那昔日的月
	,同样拂面而去的清风。继续着遗世独立,羽化而登仙的清高与孤独。

		而此时月下,我心盛载的不过是流光逝水间那亘古的忧伤。

		慵散中,小舟顺流而行,愈行愈远,忘情的歌声却留于每一方清澈的水中,每一朵汹涌过的浪里。
	悠长悠长地在雾中飘散,泯灭,没有方向。

		轻帆一卷,淡淡流淌……

	\endwriting


	\writing{荒}{2003年5月31日} %<<<2

		\longpoem{}{}{}
		月起时你用你苍凉的手势 \\
		划在空中 \\
		一条不尽完美的弧 \\
		隐没在缓慢的风声之间 \\
		我静静走近 \\
		看见了你一样苍凉的脸 \\
		苍凉的双眸 \\
		在往昔与现实的边缘 \\
		你挣扎 \\
		终于无力逃脱 \\
		此时冷成了唯一的滋味 \\
		心底的言语 \\
		也只在梦里沉沦再现
		\endlongpoem
	\endwriting


	\writing{清晨的随想}{2003年6月9日} %<<<2

		\subpart{今日}
		这一早,我从迷蒙中醒来,见窗外已是一片湿淋淋的清新。吸进肺来的空气凉凉的,薄薄的,夹着
	水气的滋润。没有什么风,只是一两丝细致的小气流偶尔吹进房间,擦着我的面颊流过。我伫立在窗前
	,感觉着这似曾相识却又独一无二的清晨。有一些凉意,却不去添一件长衣。依旧站在那里,任那些小
	气流侵袭我脆薄的身体。因为,我爱与它交汇,摩擦的滋味。

		很久,没有为下雨而欢喜过了。忘了是什么原因。而今日,在这个双眼还模糊着的清晨,我的唇角
	又因飘洒而下的水滴挂上了一两点笑意。

		雨,小心地散落。从渐已陌生的昨天一直落到此时。

		\subpart{从前}
		再没有了在屋檐下观雨的日子。儿时,下雨对于我来说是无比快乐的事,是上天伴我游戏的日子。
	我躲在屋檐下,看着雨在屋檐上一点点聚集,再一点点坠落。等待那些透明晶莹的水珠滴入我早已整齐
	排好的玻璃瓶中,静听那声响,碎玉一样。我凝视它们的降落,恰好地,降落在我的瓶中。纷纷扰扰的
	,是喧闹的雨声,我单纯的目光渐渐迷失在层层叠叠的雨雾之间。

		就是那样一种简单的心情,如今,却再也找寻不回。我却久久怀念着那些童年里有雨的日子,怀念
	我挂满水珠的屋檐。它们湿漉漉的,清晰明了得宛若滴入瓶中的雨。

		从前,迷失在时光的某处,不见了踪影。遗在心头的,是点滴的破碎不堪的回忆,是荡漾在云风间
	的气息。

		\subpart{伞}
		长大一些的我开始渴望拥有一把长柄的漂亮的小伞。这不是我小小的虚荣心。我只是喜欢手中握着
	一把小伞的感觉,好像有了它,我便会快乐许多许多。于是,在我8岁生日的那一天,我得到了那样一
	把小伞,和我想象中的一样。红色的底子上印着可爱的小猫图案。我便开始一日日地盼望着下雨,终于
	,那一天,一个同样下着小雨的清晨,我第一次撑起它去上学。走在湿滑的路面上,我看见小巷泛着一
	阵又一阵幸福的水光。我不时地抬头看一看那一片红色的小天空,心里欢喜着,兴奋着。那一刻,我感
	觉自己简直拥有了全世界。我听到雨点敲击小伞的微声,那节奏是跳跃欢愉的。

		原来,快乐原是这般伸手可及。只要那么一把伞,只要我撑起它走在路上,幸福就能在心头溢满。
	那样一种快乐迷失在路途中。成长,使我们学会了一些什么,却也同时无情地丧失了一些什么。总是要
	猛然回首,才明白自己丢失了。

		红色底子的小伞如今何在?如果能再一次撑起它,我便会在雨雾里少一份对昔日的留恋,少一声浸
	没在小气流之间的轻轻叹息。

		魂牵梦绕的小伞,被今日细密真切的雨声淹没……

		\subpart{爱}
		爱雨,只因爱它的细腻。千千万万的纤丝绵延漫天,不小心,心绪便被卷入其中。灵魂于是被它层
	层缠绕了,渐渐,也化做千千万万的纤丝细腻柔软起来。

		爱雨,只因爱它的无限。深深浅浅,近近远远,听不清是哪一个角落的微声。着莫测的声,唤醒了
	麻木的神经,它轻手为你合拢上双眼,让你从听觉的出口寻到一个无边无际的世界,辽远而神奇。

		爱雨,只因爱它的多情。有时,它簌簌地独自黯然神伤;有时,它静默地思索,一语不发;有时,
	它奔涌咆哮,纵情宣泄着情感。无论哪一种,都令人不禁为之动容。想细听它的思绪,一定是无比率真
	的飞扬。

	\endwriting


	\writing{火车}{2003年6月13日} %<<<2

		火车在墙外呼啸,一次次划过我脆薄的耳。着声音总令我恍然神伤起来。因为它总是疾驰而去,往
	往还夹着长长的悲鸣。这滋味击打我心底最敏感的地方。那声音,让我想起时光,想起流逝,想起一次
	又一次的经过和徘徊。我现在房子所在的地方正是曾经我小学的旧址。它靠近城市铁道,相隔不多时便
	会又一列火车呼啸而过。所以,可以说我是听着火车的声音长大的。

		小学时,曾一度很迷恋火车。我常常坐在操场的领操台上等着火车从眼前疾驰而过。努力地想看清
	那一张张探在窗口上的人脸,想看看他们的模样和神情。却没有一次看清过,火车总是载着那一张张我
	所未知的脸匆匆忙忙地去了。这也注定了,我与那些面孔只能是这样一种未知的“一面之缘”。而越是看
	不清便越想看。有空我便去看火车,想象它远去的身影会开去哪里,我想到了天边,想到了云霞,想到
	了在遥远的火车经过的某个村庄也有一个和我一样痴痴望着火车的小孩,他也许有黝黑的皮肤,也许有
	绯红质朴的脸颊……

		如今,我住在了曾经看火车的土地上,学校仿佛是在一夜之间被拆掉,随之是一幢幢俗气的楼房被
	一一在原地建起。我住进那楼房,躺在床上,在每一个无眠的夜里听见火车从远远的地方向我号叫着奔
	来,再悲鸣着仓皇远去。我不再关心火车上的面孔。只是一次次想象着它在黑夜中飞驰的形象,长长的
	坚实车身上亮着一串明亮的小窗,微微的光芒把一路上黑夜里沉睡的碎石和杂草一一照亮。这样的夜,
	我隔着一面厚重的墙听它,感觉它与我之间有一种奇妙的共振与关联。

		我喜欢火车上的旅行,让自己闲散地倚在窗边,看小窗外的风景一点点后退消失。那景色大多荒凉
	,却也总会有欣欣然的田园风光,有行色匆忙的农人从田野里穿过,有玩耍的灰头土脸的小孩在草丛深
	处凝望着我们的火车,有长胡子的老人赶着他洁白的羊群行走在绿油油的小山坡上。我从容地看着经过
	我视野中的一切一切。这一切,让我知道在我狭小的生活之外还有这样一种生活,这样一种宁静和纯粹
	的美。这令我的心恬淡,释然。

		我接来热水泡一杯绿茶放在手边,茶叶慢慢自然地舒展开,腾腾的热气弥漫在空气中,我深深呼吸
	这空气,有一种纯纯的淡雅清香。突然觉得,火车的鸣笛也许并不是悲鸣,而是一种舒心的长啸。

		我总是说:“无论去哪里,我都更愿意乘火车去!”朋友会反驳我说:“若是让你坐上几天几夜的火
	车,你就该叫苦了!”是的,我也听说过因坐火车太久而双腿浮肿的事。但是,我还是固执地喜欢坐火
	车,无论去哪里,无论有多远,也许这是从小便植入我内心的一种情结。也许,我会永远地对火车有这
	么一种特别的感情,时而令我神伤,时而又让我欢欣,时而又因它兴奋不已。

		坐火车去旅行!带上我心爱的小狗,在我无比复杂交错的心情中,伴着火车那熟悉的节律,听心海
	一阵狂似一阵的汹涌,让情感肆意拍打我回忆的海岸,任思绪自在飞翔。

		今夜,凌乱着头发,我躺在床上。远方的火车又呼啸而来,我继续想象着它引着风声疾驰的形象。

		火车本只是火车。火车却早已不仅仅是火车。

	\endwriting


	\writing{夏天·一}{2003年6月20日} %<<<2

		午后,坐在窗边的我偶然发现外面的世界已是蝉声四起。那一种沙哑的焦躁的声音又重新回到了我
	的耳朵。如果说四季都有它独特的声音的话,那么春的声音是花丛间蜜蜂忙碌的嗡嗡,秋是凉风习习吹
	落树叶,那树叶悄然落地的轻声,冬是大学之后靴子踏在雪地上的咯吱。而夏的声音,便是这窗外无穷
	无尽热闹着的蝉鸣吧?也许正因如此,夏天是火热而又有一些浮躁的。

		夏天,永远有挥发不尽的能量与热情,于是,我爱夏,爱附属于它的一切。它令我的思绪燃烧起来
	,迸发出一串又一串美丽的火花。夏天在记忆里是一个模糊的形象。春与夏之间似乎没有一条清楚的疆
	界,常常是还以为活在春天里却恍然知道已经是入夏了。于是,记忆里总是剔除了初夏的片段,留下的
	只是仲夏的种种。

		儿时的夏没有炎热与难熬,我只记得它是清爽而温馨的。因为关于儿时的夏,在我记忆中残存的是
	它的傍晚,它的夜。傍晚,一家人会把餐桌搬到院子里来,食物多是凉面、麻酱面一类属于夏天的食品
	。每一个人都吃得津津有味,头顶上的天空一点点地由浅蓝化为一种迷人的幽蓝,有时还会浮着一两朵
	淡淡的泛着玫瑰色的云。吃过饭,是我和哥哥的游戏时间,我们坐在柿子树下的藤椅上,开心地搅动着
	玻璃杯里的啤酒,为的是让啤酒的泡沫越搅越多,甚至溢出杯来。我记得,在傍晚微微的光线里,玻璃
	杯中的啤酒显现出一种美妙的透明色,格外迷人。啤酒的泡沫慢慢聚多,不断地升高,最后溢出杯来,
	流在了我的小手上。哥哥叫着:“快喝!快喝一口!”我于是急忙喝了一口杯里啤酒的泡沫,嘴上又沾上
	了一些。那些泡沫极可爱地挂在我的唇边,伴着我冲哥哥的傻笑。那时我少有的几次喝啤酒的经历。

		接着,是一个畅快轻松的澡。我坐在澡盆中,轻轻揉搓着头发,泡沫一个个生出来,渗在我的发丝
	里,穿梭在我的指间,并散发出一种清新的芬芳。然后,我穿着略大的睡衣从浴室踢里踏拉地走出来。
	月亮已经出来了,悠悠地挂在槐树的枝头,明黄黄的。我坐在藤椅上梳理起湿漉漉的头发,四周没有光
	亮,只有倾泻了一地的皎皎的月光。每当这时,哥哥总会跑过来,说要给我讲故事,可他偏偏总讲一些
	厉鬼幽魂的故事,每每吓得我哭着跑回屋去。后来,我明白他是故意在吓我,洗完澡之后便不去院子里
	坐,径直地回屋去了。

		床对着的是一扇窗,窗外是祖母植的一藤丝瓜。夏夜里,我和母亲躺在床上,她轻轻摇动着蒲扇,
	一阵阵清风从扇的缝隙间徐来,此时的夏天没有了浮躁,化做了一种极诗意的平静与安宁。床头挂着盛
	有萤火虫的玻璃瓶子,它们在暗夜里放着诡秘又神奇的光,黄色,绿色或淡淡的蓝,都一样的美丽。那
	是哥哥捕来送给我的,作为吓我的赔礼,因为由于那件事我几经生气有3天没理他了。躺在床上,我却
	久久难以入眠,于是缠着母亲求她给我讲故事,可是母亲显然是不善于讲故事的。每一次她都会讲同样
	的故事:小白猫和小黑猫钓鱼。没讲几次,这故事我已经烂熟于心了。后来上了幼儿园便常常炫耀一样
	地讲给小朋友们听。但即使是这样重复的无聊故事,我依然爱听,母亲一次次地讲,我一次次地在那故
	事中沉沉地睡去。

		最遥远最模糊的夏天就是这个样子,有泡沫、有星空、有月亮、有挂在床头的萤火虫瓶子。那样的
	夏天被时光消磨成一个轮廓,却也雕琢成了无数动人的细节,闪着诡秘又神奇的光,像那时的萤火虫一
	样。

	\endwriting


	\writing{夏天·二}{2003年9月23日} %<<<2

		当一个个夏天匆匆来了又去了,你记住的是什么?深深刻入你生命的又是什么?夏天,如此平凡,
	周而复始。

		我的夏天带着被阵雨淋湿的味道,安静地立在回忆里一个小小的角落。我仿佛看见自己陷落其中,
	湿了头发,湿了衣衫,无力自拔。我却怀念着,久久地怀念着夏天,此时的,或者从前。夏总是引出我
	无尽的想念。即使在那昔日的夏雨中我一次又一次地泪流满面。

		夏天的雨,总是来得如此突然。那一年,那一个夏天的傍晚,教室的屋檐下躲着一群焦虑而恐慌的
	孩子。屋外,是漫天飘洒的雨,天色沉沉地压下来,一切都被黑暗和那杂乱的雨声吞没。风唱着低沉的
	调子,从窗外颤抖的树枝间穿过,雷声凶猛地撕破了天空,摧毁了教室,又钻进每个孩子的耳朵。我怕
	黑,我是在阳光里长大的孩子。眼前的这一切来得太过突然,太过猛烈。我紧紧闭了眼,两只小手狠狠
	地捂住耳朵,我像只受惊吓的小动物一样蜷缩在门的后边。雨,依旧汹涌着,黑暗在蔓延,仿佛就要吞
	食了这世界。直到那只单薄的手臂将我拢在怀中,我才知道一切都会好起来的。那一年,在那一个夏天
	的傍晚,是我单薄的老祖母冲破了层层风雨在黑暗中将我拢在怀中。她是那一天唯一一位在雷雨中来到
	教室接孩子的家长。而她,我单薄的老祖母,仅仅是个体弱的老人。我知道,她是听到雷声就急急从家
	里赶来的,我却不知道,也不敢想象她是怎样拖着蹒跚的脚步穿越那一层密似一层的风雨,和一道寒过
	一道的雷电。在那个惨淡的傍晚,我拥在她那么单薄柔弱的身上,看着她那双不大明亮却充满温情的眼
	,听着她一遍遍轻轻地说:“别怕,乖,没事的。”祖母用她苍老的手指抚着我的头发,缓缓的,柔柔的
	,像是怕惊动了什么。那一天,我们撑着伞向家的方向走去,她的脸那么欣慰和快乐,在渐渐消失的雨
	声中,我听见了爱在流淌,那么可爱,那么美丽而动听。单薄的老祖母在这声音里变得坚实而高大,变
	成了一面我可以安心依靠的墙。

		那一天,雨过之后,我看见了一道浅浅的彩虹,遥遥地悬在天边,轻轻地向我微笑。

		夏天,我可爱的夏天!童年的夏雨停顿在原地,不再向前。隐约中我看见祖母立在雨雾间,慢慢地
	挥手,她依旧单薄的身子却一点点模糊了,她的眼里还是闪动着温情的光,却夹杂进留恋。祖母留恋着
	什么?

		转眼,又是夏天,花香肆意的季节。我看见祖母定定地躺在床上,雪白的墙,雪白的床单映着她那
	一张苍白的脸。我俯下身去,她紧紧抓住了我的手,盯着我的眼却一语不发,我知道她在积蓄着什么。
	祖母的眼混浊了,我的眼湿了。最后,她只是喃喃唤着我的名:“维……维……”什么都没有再说,什么也没
	有留下。

		我不敢相信,祖母就这般撒手而去,撒开了我那双孤单的手。

		一顷刻,我的墙塌了。我的心满目荒凉。

		花香肆意,我坐在小窗前,一夜间,盛开的茉莉散落一地,苍白的身躯孤零零地躺在那儿。依旧是
	夏天,一切却都是物是人非。那晚,又下起了倾盆大雨,雷钻进耳朵,我坐在黑暗中,不开一盏灯,静
	静听着那无情的雨声。雨在哭泣,号啕大哭。

		我问茉莉,祖母在留恋什么……

		茉莉没有回答,风没有回答,雨也没有回答,夏天沉默着。

		我湿了头发,湿了衣衫,无力自拔。夏天匆匆地来了又去了,我沉溺在此时或从前。夏天那么长那
	么长却不够我用来释怀。我怀念夏天,每一场雨,每一响雷,每一个可爱的片段,和那每一点闪动的留
	恋……

		不知为了什么,每当嗅见夏天的味道,我总是泪流满面。

	\endwriting


	\writing{夏天·三}{2003年9月23日} %<<<2

		当我走进夏天,我会想起你,戴着晶亮的眼镜,在灿灿的阳光里冲我微笑。

		朱,让我以夏天的名义给你写一封信吧。

		朱:

		我总是这样叫你的吗?不,当然不。你知道我有那么多种唤你的方法:小木棍、朱朱、宠宠、傻傻
	…… 真是太丰富了。你乐意接受这千奇百怪的名字,因为我们是朋友,很好的朋友。

		是那样一种偶然,让我们这两个本不相干的人成为了朋友,在艳阳高照的那一年夏天。记得吗?我
	们是怎样在回家的路上坐在路边聊天。记得吗?我们是怎样庆祝找到了一起犯傻的伙伴。那么多回忆印
	在那一年夏天碧绿的叶子上,弥散在那一年如火的阳光里。

		感谢上天,让你我相遇。

		我们一起唱歌,一起在大街上蹦蹦跳跳,一起捂着肚子笑得直不起腰。这一切都是属于我们的,无
	论何时,我都会好好珍藏。

		你的生日在夏天,你说你喜欢夏天。每一个暑假我们都会一起外出,即使太阳那么毒,空气那么闷
	湿。我们去游乐园,一整天,我们兴奋地玩得不亦乐乎。那一天,我们都像被水洗过一样,然而心却是
	无比快乐的,当我坐在旋转木马上,我感觉一切幸福都化做了风,一一从我耳畔刮过。我常常地想谁还
	能陪我这般疯狂地享受这份快乐呢?

		我们常常说,我们是一类人,是啊,物以类聚,人以群分。我们都是长不大的孩子。我们单纯地笑
	,单纯地梦,单纯地妄想拒绝长大。我想告诉你:和你一起犯傻,我很荣幸。

		然而,小孩子间的摩擦却是不可避免的。她们都有一点任性,一点固执,一点情绪化。每当这时我
	总是默默伤心,却又不肯让步,顽固地一语不发。真的,我们该学会原谅了。

		深深地埋植在我心里的是你我近5年的友情,你能明白吗?你称得出它的重量吗?昨天,我在抽屉
	的深处无意间找到一幅画,那是我在初三毕业那一年夏天画的。画上是你和我头顶头地躺在草地上,幸
	福的样子。那个夏天,你记得吗?我沉沉地病着的那个夏天。我永远不会忘记是谁在我最艰难的时候鼓
	励着我,是谁冒着高温在初三的繁重学业中穿越半个城市来医院探望我。一直,我只是在心底悄悄地感
	动着。关于感谢,我没有提起过一个字。但是,朱,你知道吗?一切的一切我深深地记着。

		我多希望我们都能幸福快乐,如同那张画上的一样。而如今,我时时担心着明年夏天的那一场考试
	,我心情有一点沉重,你知道的,我的身体不好,这沉重的几个月我不知道自己是否熬得过。而我,又
	梦想着上一所不错的大学,念自己热爱的专业,这一切,需要艰苦的努力,但我开始怀疑自己的能力,
	因为最近我已经感到疲惫不堪,我真的怕自己的身体顶不住。于是,我变得有一点极端,有时因为一点
	小事就会作出很大反应。如果,不小心伤害了你,我请求你的原谅。

		如果,你仍不肯原谅我,如果你已不再怀念我们在一起快乐的种种,那么我又奈何呢?只能愿你今
	后的每一天都像我们的从前那样快乐。

		我多希望我们都能幸福快乐…… 但愿,我们想的一样。

	\endwriting


	\writing{夏天·四}{2003年9月24日} %<<<2

		夏天,海是永远的主题。

		海,你会不会责怪?当我离你这般远了,我才开始我的想念。

		17岁一样的夏天,夏天一样的17岁,我站到海的面前。清澈的海滩,凉滑的沙,海的夜如此真实地
	贴近我。我曾向往着你那么久那么久,从童年的梦一直绵亘到年少的文字中。我不说爱你,我只在心底
	将对你的情感一次又一次地回味,温习。

		我不去赞美你,因为有太多人叫嚣着空洞的美丽字眼。对于你我想,两个字足矣,那就是:辽远。
	在你面前,我看到人心的狭小,我发现了自己短浅的视野。在你面前,我有想飞的冲动,在海的尽头是
	什么呢?有没有人立在彼岸也遥遥地望着我在的海岸?但是,要怎样才能够飞起来?

		抬头,海连接的是一样辽远的蓝空,而我,没有翅膀,我无法乘着那强劲的气流飞越海洋,飞越这
	人世的浮华万千。别哭,没有天空上的自由,你还有脚下坚实的土地。

		赤裸的双脚踩在正午被灼烤得滚烫的沙上,有一点疼,沙是那么细腻,银白白的一片,被海浪温柔
	地浸湿,就变得更加妩媚。我踏浪而行,浪击打着我的双脚,我却站得那么稳,绝不跌倒。

		我是多么开心!即使不能飞,我还可以这样忘情地陶醉和行走!

		远方,泊着几只有着高高桅杆的小船,它们像是沉在了低低的云里,像是在那一朵又一朵柔情的云
	朵间游弋。我看得出神,梦想着自己能有那样一只小船,就那么悠悠地在海上漂泊,任海风吹,吹乱我
	的发,吹偏我的航向。我愿意在这海上等待每一次灿烂的日出,守候每一次宁谧的月起,偶遇一次次绚
	丽的流星。我愿意到海的尽头寻找坠下的太阳,相逢隐去的月亮,拾起那一颗颗遗落在深海的明星。

		给我一只小船,我会做一次动人的旅行,在这无比辽远的海上。

		海,你还好吗?此时的我听不到你的歌唱,感受不到你汹涌的脉搏。而我会想念。想念你,想念海
	风通过的沙滩,想念我未知的彼岸,想念17岁的夏天。此时,我离你这般遥远。但你不在远方,你在我
	紧闭着的窗外,你在我依稀微笑的甜梦背后。虽然,我一遍又一遍读着北岛的那一句:风孤零零的,海
	很遥远。这里要起风了,秋天即将冲进花园。然而风不会孤零零的,因为有我带着17岁的梦回想着那个
	夏天。我如此想念你,以至于我想写一封信给你,用我那支生了锈的钢笔,蘸上纯蓝的墨水,用我的小
	船载着它漂向你,在下一个迷人的夏夜,让如水的月光轻轻念给你听。

		你会收到吗?海,你是否明白?请等待着,等待我拥有了一只小船……

		我期待着与你的再次相逢 \par
		我的海 \par
		我的海南

		\blankrev
		船

		\longpoem{}{}{}
		你说你想要一只船 \\
		小小的扬着一面小小的帆 \\
		你说我们去海上 \\
		大大的只望见一片大大的天

		我说给我一支长长的桨 \\
		长长的划过我们长长的旅程 \\
		我说我们去岛上 \\
		短短的云上涌着短短的浪

		我们会有一只船 \\
		小小的船小小的帆 \\
		无声的暗夜里月亮会挂上桅杆 \\
		回忆在温柔的海中弥漫

		你给我一支长长的桨 \\
		我在风里游弋 \\
		短短的浪飞翔 \\
		你吹起风笛 \\
		立在云河的彼岸

		你说我们去远方 \\
		于是我们的船成为沧海之上的 \\
		白帆一点 \\
		我看见沉在桨里的水波 \\
		那姿态翩翩 \\
		那神情浅浅……
		\endlongpoem

	\endwriting


	\writing{流光的灰白浅影·一}{2003年10月25日} %<<<2

		微冷的清晨,我坐在窗前写这些字。

		\subpart{孤独}
		孤独在门外徘徊,我不让它进来。

		“当我一个人推着车走进那片黑暗是,我看见邻家的黑猫瞪着一对发光的眼睛正狠狠盯着我,那一
	刻,我怕极了……”她说这些的时候,声音微颤,目光里塞满了孤独,她还是这么年轻着的女孩子,那一
	片深不见底的黑暗,她不愿一个人强装着勇敢经过,黑暗里,会有一双双恶毒的眼盯在孤独的外衣上。

		孤独,它不是单纯的寂寞,寂寞是肤浅的,孤独却深刻。

		你是否会记得自己的孤独?会记得在来时的路上有那样一片黑暗?

		你曾强装着勇敢,把胆怯咬在舌头下边,像个盲目的人一样,从孤独中走过去了。没有人知道这一
	切,因为,孤独是你自己的,没有谁能真正了解。记得那一片黑暗吧,当她说起孤独,你便能够让她稍
	稍平静。

		“我怀念原来的日子,如果我那时知道现在会变成这个样子,我希望每一天都过得慢一些再慢一些
	,我希望每一天都有180个小时……”她说着,你默默心酸。那些日子竟然真的远远去了,一去不回。

		也许,人就是这么可笑地活着。

		也经历一段日子,然后匆匆告别,又奔向另一段日子。在这中间,当你立在生活的缝隙里,安静地
	看一看,你会明白,已有太多太多的时光被抛在遥远的身后了,永远找不回。我们,就这么可笑地活,
	一年又一年,期许着明天,又怀恋从前。我们在每一个新年快乐地唱啊跳啊,却又在冷冷的夜知晓了青
	春正在悄悄告别。

		你终究是匆忙长大了。和她的故事,一天天淡去。窗上挂着水蓝色的窗帘,记得14岁的炎夏你和她
	坐在清冷的小街上,那个下午你仰着一张稚气的脸对她说:当我搬家了,我会挂水蓝色的窗帘,风吹进
	来,一定很美……她微笑着听,也说起她的梦,她想要白色的窗纱。

		而今,你正坐在这挂着水蓝色窗帘的房间,风吹进来,很美。就像你们一同经历的从前一样。

		也许,只因为你匆忙地长大了。她竟被你拖进了孤独的深渊。

		可是,这不是你的意愿,你甚至没有知觉,你以为一切的一切从未改变。只是她,轻轻地告诉你:
	一切,都不一样了。

		“我特别希望妹妹来,这样,我就不会觉得孤独了。我发现,我竟变得如此胆小。每一天我一个人
	上学,天黑,又一个人回家,原来,不是这样的。我发现,我如此孤独。”你语塞,你如何应答?你本
	应同她形影不离的,曾经,你们一同上学,一同蹦跳着回家,两只幸福的小麻雀一样。然而,由于另一
	个人必然出现在你生命中的人,你竟如此自私地离开了。你没有任何变化,依旧蹦跳着回家,同另外的
	人。

		而她,只有一个人,经历那一片黑暗。

		而你,说:人总要经过黑暗。即使你不愿她去经过。你没有办法,也许,这是时光的选择。

		有时你觉得孤独是一片汪洋,当人沉浮其中将迷失所有方向。所以,孤独的人很怕黑。

		你想起那些寥落如枯叶的日子,那一整个漫长的夏天和一整个清冷的秋。那时的你,却从未向谁诉
	说过孤独。每一天你一个人向家的方向走去,拖着你被病魔折磨得面目全非的身体。你的脚向家走去,
	却明白家里什么也没有,家,不是忧伤的终点,亦不是快乐的希望。你一时间惧怕着人群,你怕别人对
	面目全非的自己指指点点,你却又依赖着人群,你怕那无人的孤独。于是,你喜欢走在路上的感觉,身
	边充溢着人群,却没有人过问你的事,没有人知道你来自何方,又身向何处,没有人会注意到你小小的
	存在。这样便好了,立在陌生的人群中,你的孤独被一点点淹没。

		然而,回到家,你仍是单单的一个人,立在空荡的房间里。没有电话,你进了这房间便同外界断绝
	了一切关联。你于是每一日近乎疯狂地擦地板,虔诚地跪在地上,一点点仔细地擦拭着,就像擦拭自己
	蒙尘的灵魂。之后,你躺在光洁无比的地板上,等着阳光悄悄到来,把温暖洒在你的身上。那是你唯一
	一点抓得到的温暖。夜里,你总一个人伏在桌前和自己说话,写那些孤独的字,字里却夹着感恩的心情
	。即使,那一刻,你是怎样地孤独着,你曾写:

		\longpoem{}{}{}
		孤独的月亮 ~ 照着我孤独的窗 \\
		我孤独行走 ~ 还好月亮没有抛弃我
		\endlongpoem

		虽然我们都是孤独的,但是,还好,月亮没有抛弃我。你记不起自己究竟是怀着怎样复杂的心情写
	下它们的。

		你讲这些给她听,把你久久埋藏在心底的那一片黑暗给她看。她没有说话,但你知道此刻,她是孤
	独的。而此刻,你最真实的灵魂,也同样孤独着。

		如果你的灵魂不孤独,你便不会写作。只有在钢笔同纸张的摩擦声里,灵魂才寻得快慰和伙伴。

		她会明白吗?当你告诉她:人始终是孤独的。

		肉身同灵魂是分离的,因此,无论何时,在最深层的感触中,我们始终孤独。在黑暗里,总有双闪
	着寒光的眼盯着你孤独的外衣,而不是你孤独的心。所以,孤独的心反而不那么畏惧黑夜到来。我们都
	渐渐长大了,我们都会渐渐不怕黑。因为,孤独植入了我们的心。

		“当我一个人躺在漆黑的房间里,孤独正徘徊门外。我不让它进来。灵魂的孤独此时令我勇敢,让
	我一点点深切地明白:生命中,我们总要经历黑暗。孤独是片汪洋,我始终沉浮其中,失去了方向,而
	孤独却引我向真实的人生,引我深切地思索……”

		这个微冷的清晨,你握着笔,在纸上对自己这么说着。

	\endwriting


	\writing{流光的灰白浅影·二}{2003年10月26日} %<<<2

		夜,我拿起笔。

		\subpart{路口}
		多希望,在每一个仓促的路口,都有个身影,为你守候着。

		\blankrev
		他送你那块表,你心仪已久的 Swatch 手表。你喜欢它因为精准的时间和精致的外表,这也正如你
	人生中的方向 —— 你希望自己也成为那样一个内外都卓然不凡的女子。

		而那,仅仅是你美丽的希望。此时的你不过是个稚气未脱的女学生罢了。然而,你努力鼓动着自己
	的信心,一次次告诉自己:切勿妄自菲薄。眼前的他,伴你走过人生最美丽季节的人,刚毅而温柔,给
	着你想要的呵护和溺爱。你爱他,从16岁的那个冬天开始,你轻轻哼着伤感的短歌走过风雪的冬天。
	你站在苍白满目的雪地里,狠狠地哭了一回,忘记了另一个人,为了专心致志爱他。你总是对女伴说:
	在16岁前一定要伤心彻骨一次,然后在17岁谈一场轰轰烈烈的恋爱!因为,只有伤心和挫折能让人体味
	到生命的冷暖,而17岁,在你最美丽的日子里,是应该好好地爱一次的。这样,在未来的路上,当我们
	回首才不会盲目而不知所向。生活或许是需要用一些特别来做标记的,以便让我们找到回忆的路。

		你总觉得,你的青春是应该沿途留下一些标记的。

		这时,结果已经不重要,你们不说永远,你们只谈明天。你望着他一日日被你的天真洗涤着的双眼,
	微微笑了。忘了是哪一天,你发现你们的言行变得越发相似起来。一样是单纯,一样是快乐,一样只因
	对方而生的小心眼。

		17岁,你真的恋爱了吗?你仰着头看10月的天,风清云淡,心中默默算数着你们在一起的时间。17
	岁,应该是这样的罢。而你原来的日子呢?

		原来那些深深浅浅爱恋过的面孔呢?仿佛只是一刹那,都不见了,他们远走他乡或隐没在了这个城
	市不知名的角落。你将从前一点点抹去,你告诉自己:你没有爱过,从没有。

		没错,那都不是爱。

		14岁的你还什么都不明白呢,17岁呢?你此时真正明了爱的深意吗?此时,你真的在爱恋吗?

		“我不知道,我只知道和你在一起我会得到幸福和快乐!”似乎是经典的电影对白。而你却也的确只
	知道这些。于是,你不会去计较了,你自顾自快乐地享受和他在一起的日子。

		你把那手表安放进一只透明的盒子,一瞬间,你感觉它那么神圣和高洁,那俨然已不是一只手表,
	甚或一只精致的,令你心仪已久的手表。安放在透明盒子里的,分明是他那颗充满温情的火热的心,分
	明是你所说的,却又反复怀疑的爱。

		你想起每一个疲倦的黄昏,当你们走在归家的路上,他那双大大的紧紧地将你拉着的手。你想起在
	那个人来车往的路口,他保护着你安全走过车流的坚实身影。你渴望着的,正是那样一双手,一个身影
	,让他温柔地撑起你对生活的全部希望。

		这世界,人山人海,你庆幸自己在最美丽的日子里遇见他,一个全心对待着你的人。

		坐在这深黯的秋夜里,你开始明白了自己的幸福。

		闭上眼,在每一个仓促的路口,他的身影,清晰浮现。

	\endwriting


	\writing{流光的灰白浅影·三}{2003年10月28日} %<<<2

		\subpart{疼痛}
		我是个始终疼痛着的孩子。

		我记得,我是在母亲的疼痛中来到这个世界的。

		那一天,我于疼痛中出生。于是,我注定成了个疼痛着的孩子。

		即使春天在那一刻正爬进产房的窗,即使母亲欣然的微笑像4月的阳光一样笼在我幼小的脸上。母
	亲说我为世界带来了春天,她软软地歪在产床上,像一只驯良的母鹿散发着迷人的母性的温柔。窗外冷
	冷的枯枝正一天天努力地发芽,抽出嫩生生的碧枝。

		我仿佛依稀见得,母亲怀抱着弱小的我,走到4月初春的绿树红花间。在那样一个温情的春天,我
	如此幸福地在母亲的疼痛中出生了。她的疼痛深深印在我心上,成为一份永久的疼痛,我成为一个疼痛
	着的孩子。

		于是,我常常想念4月,想念轻轻柔柔的春天。当我闭上眼静静数自己的心跳,便知道那一份疼痛
	还在那里,就能够安下心来,继续勇敢地生活。

		在冷寂的夜,我靠在单薄的小木床上,风总会如约而至地呼喊着奔过屋顶。每当这时,我的思绪总
	是将我领回一日日模糊的童年。我又看见那个流淌着芬芳的花园。

		淡紫的云悬在祖母的屋角,化不开。她总是被她的那些花包围着,她总是浅浅地笑,拖着单薄的身
	体照顾她那些美丽的精灵。起风了,祖母立在花丛中央,风抚起她花白的头发,又偷去了一朵朵娇弱的
	花。风将它们吹成花瓣,吹成彩蝶,飞舞漫天。年幼的我总是喜欢追逐风的足迹,任她的悲歌灌满我小
	小的耳朵,我只是一直地向前跑,让花瓣一片片撞上我的前额,又溺在我稀疏的发里。

		我却总是跌倒,深深地跌倒在风吹过的、坚实的土地上。每一次都是伤痕累累。年幼的我,双膝多
	数时候是一片青紫或殷红的。我知道那是我疼痛的颜色。在跌倒中,我从未哭过,我恍然发现我原本如
	此坚强,从我还那么幼小的时候开始。也许,这只是因为我是个疼痛着的孩子,对于疼痛我已不那么畏
	惧。我总是咬着牙站起来继续向前跑去,因为我知道风在前方。

		我是这样一个单纯而倔犟的孩子,在祖母小小的花园里追逐着无形无影的希望。疼痛把它美妙的颜
	色以各式图案绘在我幼小的双膝上。

		在冷寂的夜,当风呼喊着奔过我的屋顶,我便记起了那时的自己,记起了祖母被风抚起的花白头发
	,记起了怀抱着芬芳的悲歌灌注入我小小的耳朵。我的疼痛浮现,水中的倒影一样,摇摇晃晃地映着我
	遥远的记忆,唤起我苍白已久的感情。

		冷寂的夜,我在疼痛中失眠,又在疼痛中醒来。明白了自己,是个疼痛着长大的孩子。

		15岁我开始喜欢走在阳光里,我开始明白阳光存在的意义。我们有太多阴湿的心情,需要让它们彻
	底暴露在阳光里,把过多的水分蒸发。或许正是那一年,我变得脆弱而爱哭,于是我总是走进有阳光的
	地方。是谁告诉我:成长是湿的。我笑笑,成长是水做的。

		没有原因,我被迫地跳进那一片水里。

		母亲在岸上温柔地望着我,一如我出生的那个春天。孩子,你要勇敢。她总是这么对我说着。

		但是,当我倒在了惨白的病床上,母亲也变得脆弱而爱哭了。我将母亲拖进了这无底的深潭。至于
	我突然患上的病,我则解释为我深埋的疼痛的涌发。我的手疼着,脚疼着,头脑疼着,心,也沉沉地疼
	着,我的疼痛折磨起我,它变得无情而残忍。医生说,你不会死,只是疼痛。

		母亲簌簌地流泪,孩子,我多愿替你疼。

		我强作笑颜,轻轻抓住她一日日老去的手。

		疼痛,是我自己的。而母亲用疼痛换来我的生命,此刻,却又想用自己的疼痛换取我的健康和快乐
	。她无法替我疼痛,却用爱融释我的疼痛。我知道,母亲的心此时比我的肉身承受着更沉重的疼痛。

		病房里挤满了疼痛着的灵魂。有个人在我的床头遗下一张白纸,上边用铅笔清浅地写着:“快些仰
	起你那苍白的脸吧,快些松开你那紧皱的眉吧,你的生命它不长,不能用它来悲伤,那些坏天气,终于
	都会过去……”是朴树的个歌,我知道这些字是那个今天离开这里开始新生活的女孩为我留下的。昨夜,
	我和她躺在黑暗中反反复复唱着的,正是这支歌。

		疼痛延续,而我停止了悲伤。

		我是个疼痛的孩子,从15岁开始,这意义变得深刻。孩子,你要勇敢。母亲依旧温柔地说着。

		16岁的夏天,祖母却带着我童年的全部美好匆忙地去了。直到最后一刻她才平静了下来,她躺在那
	张单人床上,离开了。我没有哭,甚至还有一丝轻松,因为我知道,她病着的日子是多么疼痛。癌细胞
	侵蚀了她的肋骨,我的祖母,一个忍受了无数疼痛的老人,却抑止不住地被此时的疼痛折磨得惨叫。让
	祖母去吧,离开这无法再容忍的疼痛吧,离开了,就不会疼了。于是,那一天,在夏日郁闷的风里,我
	久久站着,不发一言,不掉一滴眼泪。

		对于祖母,疼痛是个终结,死,成了最有效的药剂。

		与她的疼痛相比,我的疼痛显得微不足道。我不过是偶尔地疼痛着,在疼痛过去之后,还会得到一
	点点幸福和轻松,让我明白没有疼痛的日子是多么晴朗。

		疼痛简直成了我幸福的调味。

		然而,当我因疼痛而辗转反侧难以成眠,这疼痛的酸心和沉重只有我自己能够体味。疼痛,注定了
	只能一个人默默承受。

		我于是害怕母亲知道我的脚又在痛了,总是强忍着,装作若无其事地经过她的眼前。然而,每一次
	,她总能发现。

		我不希望母亲的心承载我的疼痛,我不愿她再次因我而疼痛。

		我走在自己的路上,磕磕绊绊的,一路莫名地摔倒,又坚忍着站起来。17年,短短的17年,我就像
	当年追风的自己一样追逐无形无影的希望。双膝绘着疼痛的图案,

		嘿,你得勇敢。

		我是个始终疼痛着的孩子。


		\blankrev
		闭上眼,静静数着自己的心跳,我安心了。\par
		继续勇敢地生活。


		\blankrev
		流光的灰白浅影。\par
		我用灵魂的笔,摄下我的影子。

		\blankrev
		喜欢这首词《傲慢的上校》(朴树):

		\longpoem{}{}{}
		总算是流干了眼泪 \\
		总算习惯了残忍 \\
		太阳每天都能照常升起在烂醉的清晨 \\
		像早前的天真梦想被时光损毁 \\
		再没什么能让我下跪 \\
		我们笑着灰飞烟灭

		人如鸿毛命若野草无可救药 \\
		卑贱又骄傲 \\
		无所期待无可乞讨 \\
		命运如刀 \\
		就让我来领教
		\endlongpoem

	\endwriting


	\writing{流光的灰白浅影·四}{2003年11月30日} %<<<2

		\subpart{远去}
		幸福和青春都不过是一串断续着的省略号。

		\blankrev
		“有一天,我会记起你,会记起和你一同坐过的楼梯,会记起我们度过的那一个个悠长的下午。”当
	你说起这些,我的鼻子有些酸,正是这样的,终于有一天,你会将我和我们的所有装进记忆的柜子。我
	们被时光遗弃了,在我们一个个不经意的日子里。

		当我面对学校这长长的走廊,当我独立在无人的操场,我如何学会不去伤感?终于有一天,眼前的
	人们各奔东西,终于,一张张如此鲜活真实的面孔被锁入回忆,一声声亲切熟悉的呼唤化做遥远陌生的
	东西。我们将去哪里?我们是否会在岁月的彼岸回望此时?此时,这如花绽放的一切?

		我们为何总是反复地相聚,又匆忙地散去?

		你坐在楼梯上,脸颊上笼着11月荒凉的光。我仿佛是把自己的青春禁锢在了这条走廊,而你,也一
	样。还记得初中的走廊吗?它不那么长,只是终日低沉着灵魂。在那里,我和你站在走廊尽头的饮水机
	旁边喝速溶咖啡,当时,我们都还小,混在咖啡的香气里,我们幸福地笑。而现在,面前这长长的走廊
	,它令我们恐惧。它太长了,困苦地承载着一日日穿梭其中的那些渐渐破灭的天真梦想。

		我们怎么就长大了?我记得,我还那么小。

		远去了,我们彼此安慰着学会承认这事实。

		是谁说过:生命是一片纯白的空地,孤独的人们反复徘徊。在这一片纯白之上,我哭了又笑了,一
	点点明白人世所谓的道理。当生命终于也随浮华远去,我终于得到安宁。

	\endwriting


	\writing{花}{2004年9月3日} %<<<2

		下雨天,给自己买了花。

		纯白色的龙胆花,花瓣的边缘是淡淡的粉红,这颜色让我想起你。我曾经让你用一种颜色来形容我
	,当时,你便是说:纯白底子上有几片粉红。

		看上去有些凄凉的花,我抱在怀里。龙胆花,一种有着纤弱姿态的花,看了便令人生怜的花。也许
	正是因此吧,她的花语是:爱上忧伤的你。

		爱上忧伤的你,让我动容。

		龙胆是忧伤的。

		亲爱的,就让我抱着你,别再伤心。

		雨还在下,似乎不会停。

	\endwriting


	\writing{渡河}{2004年9月4日} %<<<2

		16岁的冬天,写过一首叫《渡河》的小诗,也或许都不能称之为诗。当时是极满意的,现在读来觉
	得行文幼稚,却正因其幼稚而显单纯可爱。重新拾起,细细体味把玩,不禁会心一笑。

		呵呵,那些美丽又清新的哀愁啊。

		\longpoem{}{}{}
		渡河 \\
		昨日已成虚妄 \\
		日子浅浅流去 \\
		没什么方向 ~ 把叹息 \\
		沉淀入河的深处 \\
		让水草 ~ 乘着他绵绵生长 \\
		也许我 \\
		会就这么渡河而去 \\
		一叶孤舟 ~ 没什么方向 \\
		摇晃中 \\
		枯萎了我头上的玫瑰 \\
		把流去的光阴 ~ 系上桅杆 \\
		看一些孤寂在冷风里飘扬
		\endlongpoem

		看这般轻轻地哀伤过了呢,为了青春里那些小小的失望,为了雪花上略显苍白了的诗意。我,是这
	样走来的,从一片淡蓝色的云雾里走来。我,是这样的女孩子,皱着眉头又偷偷幸福着。这样的自己,
	从远处细细瞧来,有些麻烦,有些可爱。

		渡河,此时是渡河,彼时依旧是渡河。渡过了成长的河,渡过了离别的河,又去渡过痛苦的河,幸
	福的河。我们始终是在舟上的人,是汪洋中的轻帆一卷。

		岸,是永远不消失的希望,是永远的一种期待与守望;岸,是我们的最温暖的归宿,又是我们残酷
	而冰冷的终点。

		今天的我,这么想着,关于渡河这件事情。似乎依旧是幼稚,依旧是单纯。那便还好,还好,我还
	没有老道起来,还没有失去天性的本真。

		当我在河上哀伤,当我迎着风里的飞花想起往昔的种种,你是否也正静默地倚着船舷,和着谁的笛
	声把远去的美梦想念?我们都是这样漂泊着,思念着,又失望着呀。带着轻轻的哀伤、悄悄的幸福,一
	路歌唱,一路洒泪。

		就让我载这一舱的甜蜜,一舱的回味,渡河而去。牵挂着明日又思念着往昔,在星夜里,微笑着谛
	听光阴流过的细声。

		渡河,原是哀愁着的幸福。

	\endwriting


	\writing{九月}{2004年9月9日} %<<<2

		\longpoem{}{}{}
			九月 ~ 天高 ~ 人浮躁 ……
		\endlongpoem

		发现《浮躁》是特别适合在秋天聆听的一张唱片。是那么一种恹恹的情绪在作怪吧,突然喜欢上慵
	懒里略带荒凉的歌声来。

		那天从高中门口经过,见学生们三三两两进进出出,穿的正是我们曾经穿了3年的那身深蓝色。不
	太好看的那么一身校服,此时看上去却觉得亲切非常,默默地竟感慨了一次。同时生出个错觉:我今天
	是不是逃课了?我是不是本应去上课的,穿着和他们一样的校服?

		9月,当他们都开学了,我们还在假期里晃。有点无聊,有点厌烦地,晃。好像是谁开的玩笑:突
	然间几个月不做卷子,不测验了,竟不适应起来,生活都没刺激了!是不是该怀念一下有题可做的日子
	呢?是怀念么?还是应该悼念?

		于是,这个9月,显得格外浮躁起来,好像灰蓝天空上被吹起的塑料袋。

		Faye 唱得寂寞,9月里,平淡无聊。

		一切都好,只缺烦恼。

		是受不了,还来不及纪念,便和你们匆忙告别。

		符合情绪的歌名,无常,浮躁,哪儿,分裂,末日,堕落,扫兴,想象,不安,野三坡。

		符合口味的哼唱,从8年前传来,空灵的,香冷迷人。

	\endwriting


	\writing{高处}{2004年9月11日} %<<<2

		今天去安置好了宿舍。知道了以后的日子自己是要睡在高处的了。感觉怪怪的。

		还是不习惯与地面又太大的距离感。那会令人不安,觉得心是悬在半空的,随时会掉下来,摔个粉
	碎。

		我是对引力有恐惧感的孩子,呵呵。

		但是,高处有高处的魅力。

		高处少了甜梦被打搅的危险,高处多了清净的时间。

		睡在高处,或许还会有更多的机会做和飞翔有关的梦。这是我安慰自己的方法。

		高处是好过低处的,嗯,我相信。

		挂上了蚊帐,纯白色的薄纱,非常中意,躺在帐子里闭了眼幸福了好久。

		蚊帐,是涂上了我童年的色彩的,是带着回忆的淡淡香味的。

		多少次,暗暗回想童年的盛夏,想床上用竹竿撑起的那一帐白纱,想卧在帐里幼小的自己,想乌黑
	的发丝间轻轻渗出的汗水,想放在小床头上装着萤火虫的玻璃瓶。

		都好像是那一夜萤火虫的微微光亮一般,如此真实,又如此令人迷幻。

		是一首诗一样,被吟咏过了,又在光阴里渐渐化开,淡去。只留得最简洁单纯的意象,在呼唤,在
	流淌,彻夜不眠地。

		睡在高处,睡在挂了纱帐的高处,听来幸福莫名。是要睡在清净中了,是要梦在回忆的底层了。

		是时候一个人享受无声无息的静夜,是时候把种种意象的美妙重新整理回味。

		大学的生活,给了我这个机会,让我有机会寂寞,有机会独自思想。

		我就要睡在朦胧的缥缈里,把你们想念了。会流下泪来吗?

		我在高处,生活在高处。

	\endwriting


	\writing{幻觉}{2004年9月21日} %<<<2

		一直,都喜欢陷在幻觉里的感觉。

		似梦又非梦地停留在现实与妄想的缝隙间,赖着不走,就任心绪乱了又乱了,就任一种种绚烂和夺
	目将我融化了,化成一滩澄净净的清水。

		那,是幻觉的境界,无所谓了现实与虚妄的境界。于是可成澄净,可成清水。

		由于对现实的全然漠视与忘却。

		好像蝴蝶,翩翩飞舞了,美妙自由至极了,却迷惑了梦里的庄周,蝶不是蝶,蝶成了梦外的自己。
	还是幻觉,蝶的,或者庄子的,游离了尘世纷纷的幻觉,绝美的幻觉。

		我于是惊讶,惊讶于幻觉的神奇。在千年之前迷惑了洞悉天地的大哲,在千万次月光里迷惑了芸芸
	众生,在每一个不失约的阳春迷惑了嫩草与花树。分不清的,不只是蝶与庄周,不只是月影与传说,不
	只是南风与暖香,更是心灵最底层的渴望与偏执。

		我就是这样的孩子。在缝隙里,在影影绰绰的幻觉里,追寻着最初的那一次感动,最朴素的那一种
	快慰。

		你永远不会忘却的,祖母把花瓣捣成颜料轻涂在你指尖的午后,永远不会忘却那一种淡红色的绚丽
	。满指的醉人与香艳呀,滞留在午后被炽烤出清香的日光里,停泊在童年那一处小小的河湾。似乎那才
	是生活本来的模样:单纯而无杂质的喜悦,祖母纤弱而熟悉的身影,好像永远不会离去的时光。

		长长的这一种幻觉,埋在我的枕下很久很久了。关于童年,花,和祖母的一切,从遥远的现实里脱
	离出来,终于皆成幻觉,皆成这一个我此时的幻觉。

		好几个梦里,我分明看见她,依旧清癯的面容,慈爱的眼。分明觉得她拉着我冰冷的手,分明听得
	她对我说:他们都说我死了,其实没有。分明,我深深地信了,深深地喜悦了。我以为,真的如祖母说
	的,“其实没有。”

		而随之而来的是惊起,是沉沉的失望。然而,不是分明的么?分明的。

		只是幻觉吗?在潜意识里,我总觉得祖母还是在这人间,还是住在对街那扇窗子的背后。我分不清
	,什么是真,什么又是幻。

		喜欢陷在幻觉里,化做澄净,化做清水一滩。却不是忘却了现实,超脱了凡尘的境界。是想念,是
	怀想吧,是无止境的那一个亦幻亦真的梦吧。

		这一个秋天,我上了大学,走出教室的一刻看见蓝空下满树金黄的灿烂。在午后被炽烤出清香的日
	光里,我坐在那一颗树下,突然觉得虚幻。我似乎是嗅见了那一种淡红色的气味,似乎是见到了满指醉
	人的色彩。好像要睡着了一样,我明白,是幻觉了,原来,只是幻觉。那一种色彩永远永远地被封存了
	,同那朵朵睡了的花一同,同祖母微笑的侧脸一同,再不相见。

		怎么突然想起,15岁的夏天,对祖母说的那一句:您要看我考了大学,长了出息,给您买好吃的!
	其实,不过是玩笑的吧。

		怎么,怎么却在瞬息里就离去得这般远了。

		怎么,怎么就这样生生地成了空,只剩幻觉。

	\endwriting


	\writing{那一端}{2004年9月28日} %<<<2

		是映着一样温凉的光啊,让什么思念什么哀愁随着昨夜的西风倾泻了一地。喜欢月圆的日子,喜欢
	深暗中被晕湿化开的那一种明黄的光亮。

		我是沉醉于月沉静而不失温柔的气质中了,我是在月的清辉里执拗地误信着种种关于月的传说的孩
	子。

		于是迷信着桂树,迷信着那一只玉兔,,迷信着广寒宫里静默千古的女子。

		于是读李商隐那一句“嫦娥应悔偷灵药,碧海青天夜夜心”,竟忍不住叹息感怀。

		想着此时的天空,这一座雕栏玉砌的宫殿中,她是否仍在,是否依旧着夜夜的悔恨与伤感?想着她
	飞天而去的那一个遥远的夜晚,沾着花香的衣袖曾怎样飞翔舞动过?

		是不是也有哪一种美丽在她飞腾而起的那一刻便永远永远地坠落了,坠入一个亘古的深渊,万劫不
	复?

		你说,何必,何必这般为了一个早已知晓了真相的故事而劳神苦思?

		我笑笑,有时候怎么我宁愿不知道事情的真实模样,怎么我宁愿去长久地傻傻坚守住一个谎言,如
	此凄婉美妙的谎言。

		我愿意,有一个女子确是在千年之前奔向了那一团冷冷的光明了。我愿意,如儿时祖母讲给我的故
	事一般,所有的传说原本都是发生过的。

		不知是谁在什么时候,把月笼在了一层又一层虚妄迷离着的烟雾中。那烟是诗是词是婉转的曲调,
	那雾是画是梦是千古化不开的深情。就让我爱上月,爱上月孤独而空灵的神态。

		要读着令人唇齿留香的诗句,要枕着洒满清辉的高枕,要聆听夜空极远处隐隐的松涛,要梦飞翔的
	梦,要忆思念着的笑容。让句句珠玉在唇齿间滑过,让绵长延续的只属于月的所有思绪与怀想,都在这
	一个朗朗的夜晚注满心灵的小小池塘。

		月没有放弃我,也没有放弃任何人,在小窗的外边,在层层叠叠的喧嚣背后,用最温存的目光凝望
	着你我。

		凝望这一刻的悲欢,这一时的得失。她从不言语,只用那一轮光明,照亮着、温暖着或孤独或迷惘
	的心。月却一样是孤独,孤独了亘古,月一样是迷惘,迷惘了亿万次的潮涨潮落。

		% <todo: 别字: 编制 -> 编织 >
		月是在编织另外的世界,安详而深邃的世界,给所有醉倒在梦里的孩子。

		诗人,你歌唱着月,你把灵魂都托付给了这一种世间最清澈的光芒,是不是也愿奔赴那一座宫殿,
	哪怕,哪怕等待着的是漫长漫长的冷寂?

		只是相信在天空的那一端,有什么在召唤,召唤着暗夜里徘徊的灵魂。

		这一夜,是中秋,月正圆。

		什么悲欢,什么离合,什么阴晴,什么圆缺?月回答着全部,从那么那么远的年代。

		这一刻,月被我望见,被我用心灵磨成细柔的粉末,被我迷信了,又被我崇拜。

		我正走向月,走向那一端笼着烟雾的温凉。我要飞升,要让沾满花香的衣袖飞翔舞动,好像千年之
	前那一夜的传说一样。

	\endwriting


	\writing{纤纤}{2005年07月06日 ~ 18:41:26} %<<<2

		会梦见,身体瘦弱,面色苍白的小女孩,问我云的由来。

		\subpart{一}

		始终看不清面目的她,好象很远,又仿若很近,我分明听着她细小的呼吸,偶尔,还有她清亮却悲
	伤的歌声,有一句没一句地荡。我叫她纤纤。因我记得,她落在双肩那一样瘦弱的小辫子。

		纤纤问我,云的由来。她那么执着地要知道答案。

		我想了很久,竟没有个结果。于是,好多个夜晚,我总是无言地默对着她,笑笑而已。纤纤没有失
	望,我想,她只是乐意问。问这样一件没头没绪的事。她在我梦里,恍惚地活着。很多个夏天。是的,
	纤纤只是在夏天。

		白日里,我也想问,你究竟是谁。纤纤。

		云,是从哪来的?

		你总是问。这一次,你竟是躺在印花的小被子里,只留下小脸露在外边,你望我,等一个答案。我
	似乎是点了支烟,我看见烟丝抽离化为乌有,渐渐弥漫眼前。我似乎是借着指缝间隐约的那一星红光,
	在黑暗里,抚摩了你的额,还那么稚嫩的肌肤。我回答了你,用几乎不可能是我的声音。

		是我在你睡熟时,偷走被里的棉花,挂到天上去啦。

		你就相信了。纤纤。你就闭了眼睛,安静地睡。

		当我醒了,你还睡着么。真实的日光照到我房里来。我随手倒了杯水,从床上爬起来,站在窗口,
	独自懒懒地喝。城市把天空分割成无数破碎的蓝,而那依旧是天空,会有云。我点了烟,烟丝抽离,化
	为乌有,灰色停滞在早晨透明的光泽里。我想起和纤纤讲的话来。我偷了她被里的棉花,挂在天上。自
	己也笑了。

		毕竟,我从不是懂得哄孩子的人。当然,更不会哄女人。所以,我独居在城市中心的一隅,不去碰
	触感情。女人是麻烦的,总容易纠缠不清,而我,是喜欢简单的人。有个叫欧的女孩,一样是喜欢简单
	的,于是我们共同排遣着寂寞,其他的只字不提。我不必哄她,她只是在适时的时候出现。

		\subpart{二}

		欧是一个你永远不会摸清的人。看她眼睛的时候,我总有站在悬崖边一样危险的感觉。她说,我是
	死板的人,说什么都不明白。我就笑,为什么女人总爱问些奇怪的话,又喜欢听不切实际的事情呢。欧
	说,女人都是孩子,但她已经不是了,她老了。而她,分明那么年轻。长发被烫成瑰丽的花朵,总穿着
	花布的吊带裙子。没有人会知道,她在我这里时表现的颓丧和麻木。好多人,以为我们是相爱的。只有
	两个人清楚,什么也没有。

		欧并不太问,我的原来,像其他的女孩那样。或许,是她根本就没有兴趣。\par
		虽然,好多时候,在一同平卧着的夜里,我想和她说起。另一个人。\par
		欧来过夜时,纤纤从不出现。\par
		云,是从哪里来的?

		我竟是被你摇醒,这一天,你笑盈盈地站在我面前,怀里是一团团洁白白的棉朵。我坐起身来,我
	们被棉朵包围了,光芒笼罩了整个房间,空气很轻很轻,我好象漂浮着。纤纤,幸福甜美的样子,小辫
	子落在小小的肩。

		你带我去,把云挂起来。你满怀期望地说。

		我分明是听见,分明,是见到你站在我的床前。在那么光明的一个夜里。然而就消释了,好象粉末
	投入清水,你化得无影无形。如一场噩梦,我惊醒在半夜的黑暗里。我推开门,站在阳台上,看世界的
	灯火虚无地闪亮,又被谁熄灭。纤纤,你是拆了自己的小花被吗。

		你笑了,因我不着边际的一句谎言吗。\par
		夜里,云都躲去哪里了呢。我看见几丝轻轻的痕迹画在天空,像烟丝的样子。

		\subpart{三}

		欧在周四到我的公寓来,是北方夏天难得的一个雨天。

		欧似乎讲过她喜欢这样的天气,不会太浮躁。她总说,傻一点活着要比聪明好。我觉得她是聪明人
	,所以向往着傻而单纯的生活。也许,是她累了。或许许多曾经改变了这个年轻的女人。我也从不去问
	。因为根本没有兴趣知道。她还是穿着花布的吊带裙子,却是光着脚。雨水和泥水沾了一脚。她见我开
	门时惊异的表情,只是耸耸肩膀,诡异而顽皮地一笑。欧是这样莫名其妙的女人,不知所以地做事。她
	就踩脏我的地板,瘫坐在沙发里。天知道发生了什么。

		我点了烟,烟丝覆盖了我眼前这个女人,却听见她的声音:我给你生个孩子吧。

		我说过,我永远不会摸清她。虽然,我们在一起是那么简单的。但麻烦,似乎终于是开始了。所有
	的女人,总会纠缠不清。

		她明白我的心思,她说,不是,你别误会。我不是想和你有什么结果。只是想要一个孩子。真的。
	只是这么想的。一个孩子。

		我苦苦地笑,你疯了吗。\par
		我真的,只是这样。

		欧没有醉,这一刻眼神坚定而带企求似望着我。我一时无语。我不敢在望她的眼睛,那下边,是万
	丈深渊。好象,我多凝视一刻,便会万劫不复。我们就对坐着,什么也不说。房间里昏暗,窗帘闭着。
	雨淅淅沥沥地落,不会停的样子。烟一支支地燃,灰落一地。

		\subpart{四}

		我终于开口,给她讲另一个人的事。

		那个叫婴的女孩。我大概唯一爱过的女子。她躺在回忆里整洁的床单上,表情安详,她睡着,恒久
	地睡。发上插着白色的百合,像出嫁的新娘一样。婴,固执又倔强,喜欢让我背着她。她就伏在我的肩
	膀,咯咯地笑,有时会顽皮地拽我的头发玩。婴是天真,是未染尘世的净。她说,喜欢看不到边际的田
	野,喜欢水牛没在水里睡觉的样子,虽然那都是她未及见到的。她说,要我背着她坐在田野上,要我背
	着她像牧童那样和水牛玩。我略显倦意地笑。婴看得出,我的不确定。失望地生气,用没什么力气的拳
	头打我的背。她总是问,她可不可以做新娘。总是问,她做新娘的时候,我是不是可以背着她。

		我总是没有答案,只是笑。婴便又会生气。\par
		欧睁大了眼睛,怎么会这样,她笑起来。\par
		我爱着,这样的一个女孩,婴。很多年,虽然,我知道自己根本不会娶她。

		因为她的腿从5岁起便不可以走路了。而且,她也活不过20岁。果然,就在她17岁那年夏天,婴就
	躺在那张整洁的白床单上走了。她明明知道一切真相,却还是问。问那些不切实际的事情。婴可以做新
	娘吗?婴做新娘的时候,你会不会背着我呢?

		我想,她多么期待一句谎言。而我,始终没有说出,始终没有。\par
		你个该死的,你笨死啊!欧重重的巴掌打在我的右脸。

		婴死那天,我把白百合插在她发上,她还那么年轻,躺在那儿,像睡着的孩子一样纯粹干净。她可
	以是新娘,我会背着她,让她做我最美最美的新娘。我说了一百次,一万次,但已经了无痕迹。我点了
	烟,没有吸,只是看它们燃。所以到现在,我也是这样,并不吸。

		我喜欢看烟丝的抽离消散。我就想着她听见了。\par
		我说的谎言。\par
		欧没有言语。摸我的脸颊,第一次那么悉心地摸着。\par
		第一次,她看见我哭。\par
		后来,似乎我们都睡着了。雨不知道什么时候停了,醒了的时候,碎掉的蓝空,挂上了云。\par
		欧要走的时候,还是说,想要一个孩子。没有理由的,我竟然答应。

		她问,你是不是不会再爱别的女人。声音微弱。我不懂得她讲些什么,这些话,本来不是她该问的
	。所以我什么也没说。

		之后,欧又断断续续地来过几次,便消声灭迹。

		\subpart{五}

		我的生活,自己安静如常地过。夏天就要过去,我仰卧在阳台上,点了烟,举得很高,看见烟丝在
	天空下消散。这一天没有云,只有空荡荡的蓝,那么那么刺眼。我想起纤纤。秋天要来了,她拆了小被
	子,会冷的。

		晚上,她告诉我,她要走了,请我放心。我没有看见她的表情,只是一个熟识的轮廓,穿了花布的
	裙子,隐约里,手里似乎握着一枝百合。我听见她唱歌,有一声没一声的,轻轻地荡,就这么一直天亮
	。

		我想问她会不会冷。但终于没有开口,也好象是无法开口。\par
		纤纤,真的走了么。再也不回来,那明年的夏天呢。

		我好象掉进一口井。只见到头顶的一点光亮,没日没夜地工作,却没有原由。我似乎并不想要些什
	么。躲在角落,度着奇怪离奇的日月。有人问我欧的去向,而我,又何从知道。或许,她会有一个孩子
	,或许没有,我都不会知道。如果她并不想让我知道。听说,她离开了这座城。

		我看着夏天就这么在指缝的烟丝里殆尽,像一个生命那样。\par
		永远消散在我的世界,我的梦。

		秋,冬,春,我这么过,许多认识不认识的女子经过我的房间,因为各不相同,却又各自相似的原
	因。我们一样排遣寂寞,用越来越孤寂的身体。她们问我,为什么只是点着烟而不吸呢。我说那是信仰
	。

		什么信仰,自己也想笑。\par
		我独自看天空,等着什么,云?或者别的。

		\subpart{六}

		来年,7月的时候,收到纯白的一只信封。齐整的写我的名字。很久,没有人用笔写一封信给我。
	是欧,我早该想到,是她。

		她说,她有了一个女儿。又记叙些生活琐碎,我看出她做母亲的喜悦。她用自己原来花布裙子改了
	条小被子给她。等天气凉时盖。

		她说,孩子那么小,却总很喜欢那小被子,竟从缝隙里抓出些棉花。\par
		我呆坐在一处。\par
		欧在末尾问,你不会爱别的女人了,是么。即使不回信,也请告诉我。\par
		那一晚,我收拾了行李。买了去另一座小城的火车票。\par
		我有许多的谎言,要去说。

		\subpart{七}

		我偷了你被里的棉花,挂在天空。\par
		于是有了云。\par
		我的爱。\par
		后记:又莫名地讲了个故事。\par
		叫纤纤的小女孩。\par
		恍惚迷离的……

	\endwriting


	\writing{光影}{2005年07月08日 ~ 17:14:15} %<<<2

		莫奈,光影的极至。\par
		睡莲,日出,撑阳伞的女人逆光下的脸,他的花园,他的妻。总令我神迷。

		最喜欢的一幅,没有找到电子稿,叫做《花园中的女人》,是印象派作品中的最大幅。为了绘制它
	,莫奈在花园里掘出一道深沟,把画布放在里边,利用滑车上下。画面上,是晴好天气的花园,草木葱
	绿,四个女人静处其间。一例,穿着素色的衣裙,神情安闲自然,手持花朵。而她们本是一个人,莫奈
	未来的妻子,卡美伊。

		我喜欢,那花园里,阳光轻盈的香味,和女人们素色的衣裙。一个通透纯粹的世界,光影重叠,在
	绿木丛荫下。

		终于,暑假来临。我在刹那里,拥有了剩余的夏天。这个苦难着的夏。

		园子里的睡莲开好了。和去年初来时一样,淡漠地浮在小小的池水。稍稍在岸上站定一会,阳光就
	好象要出来。会投下一大片的树影,沾一两丝温热的夏风轻轻地摇。总是如此的夏日。或者日光强烈,
	或者给你阴湿沉闷的表情。并不该抱怨。毕竟睡莲,是那么美的。

		小鹿说,前几天有小孩子跑到园子的池塘里游泳。我就惊异着羡慕。虽然,后来,他们被保安强行
	带走了。

		同学们会去军训。我独自留守,在这城中的闷闷空气里,继续出汗,发呆,写我的字。\par
		也养我的病歪歪的身体。\par
		不管怎样,一定要,好好生活。

	\endwriting


	\writing{七月荷}{2005年07月13日 ~ 17:52:49} %<<<2

		每一年,如赴邀约,七月,吹起荷风,款步而至。

		是温婉如旧的清雅,略带幽思的身姿,盈盈满塘池水。她不言语,不喧哗,默声在那一处,深插在
	那一处,绿水浅泥。七月,荷,芬芳中复苏。在我亦清醒着的时刻。

		与母亲观荷。从斑斑驳驳的雨水,到艳夏里琉璃般光洁的初晨。

		我想着,17岁的夏天,在废园的小舟,听雨与荷的交响。再不回来的夏天,再不复现的风情。废园
	荒芜,而最后的一丝苍凉下的醉美也被人无情夺去。为了资金,或别的什么,在栽植荷花的池底做了手
	脚,于是,就这么死去消亡。连呻吟和悲鸣也没有留下。

		谁去哀思,美丽生灵的不见。在七月,我不禁会想念,而所有的不过徒劳。\par
		我们乘舟,观另外的荷。当晨早来临。\par
		于是小小的舟上,许多个夏天的遗韵开始漫无目的地流散四处。

		水影子晃荡荡映上船篷,我们似乎在流光间梭行。栏外,是碧透浓醉的枝叶和花朵,我倚着船舷,
	看明艳光洁里,飞舞回旋的蜻蜓,看那双双透光的薄翅,斜侧入晨早的光线。它们飞行,轻盈如丝,不
	费力气。那舞一样的的身姿,是七月之中,天地育生的赞美和灵动。于此时,荷仿若已不重要,只是这
	游船之上的一光一影,一吸一呼,才成难忘,才成兴味。

		许多记忆和往事,在无可捕捉的瞬息逃遁。看荷,七月,谁临水独照。谁为谁记一笔流水帐。惦念
	一段混进尘埃的惦念。

		执着在一己的天地里。尘世种种安排,仔细思量。\par
		我转头,望老去身影的母亲。她微笑着,说,荷花开得真好。\par
		我始终不可以明白的,在母亲的眼中已是洞明。\par
		七月,如赴邀约,淡定如此,盛放无言。

	\endwriting


	\writing{六月事}{2005年07月18日 ~ 19:25:29} %<<<2

		在七月,我想起许多的六月。

		闪现童稚无邪的脸孔,一次次,于日光下微笑清晰。是那个孩子吧,不懂得长大的小女孩,穿着白
	色的长筒袜,跑去挂满气球的操场。红领巾,雪白白的衬衫,她像所有的孩子一样,快乐地扬起嘴角,
	为了节日。

		似乎,还可以听到,你们高高低低的合唱。在飘荡,在飞扬,混合那些六月的甜意,流溢四处。你
	的伙伴,叫卉的女孩子,那一天穿着引人注目的红裙子。她喜欢转裙子,她留长头发,她是那么可爱的
	。你觉得她很美,也会有些羡慕。

		你们坐在一起,唧唧喳喳,说许多奇怪又快乐的话。

		六月某天,你邀卉来家里玩。是一个日光充足的天,柿树的影子铺遍了小小的庭院。你们拍照片,
	做孩子们觉得很是得意神圣的事情。于是,卉抱着树干,被你拍下,衬衫上的花朵开得鲜美。于是,你
	站在屋檐的阴影里,表情单一地笑,被卉留在胶片。因为自由地拍照,整个下午充满了满足和欢笑。六
	月的天,空空荡荡地晴,小女孩们躲在小小的庭院,偷走时光的果实。

		然后,你就忘记了么。只记得照片,而照片呢,又散失在何处。或许,躲藏在储物箱的底端,安然
	霉变,或许,夹在哪一本遗失的相薄,永久告别。我们总是难免丢失,是宿命一样的安排,不经意的那
	一瞬息,你就消失殆尽。

		卉,可爱的女孩子,你还留长发吗。你都忘了吧。一个叫田的伙伴。曾经天真在一种笑容里的伙伴
	。

		卉,是早已不会记得,更不必怀念,如你这般。

		而六月的许多。清晰可辨,像前夜的雨水打湿在窗上,留了班驳的水痕。水本是洁净,是混了过往
	里的灰尘,才画了温存如墨的笔触。

		\blankrev

		似乎,就可以看到,哥哥坐在祖父的藤椅上,慢吞吞地啃西红柿。是六月,一样的六月。

		西红柿的红色鲜美,他一口口地吃,浆液顺着指头流下来。

		哥哥,好久不见的哥哥,陪我玩耍着长大的人。他会折纸枪,带我一起玩打仗的游戏。我们在小小
	的庭院中穿梭奔跑,在每一团花丛后遮蔽和躲藏,装作紧张的样子。祖父,会叼着他的老烟斗,缓缓地
	吐烟圈,一个个升起,浑圆,又破散不见。祖父充当司令的角色。哥哥是我的上级,我是小兵。

		哥哥最常说的一句是:你先撤退,我来掩护你。我并不懂得,什么是掩护,只是向前越了月季花坛
	跑到另外一端。

		其实,那并不是我钟情的游戏。我却也玩得忘乎所以,像个地道的男孩子一样。于是,也会像哥哥
	一样穿了海军服,直到上了小学。

		我的哥哥,居住在不远处楼房的一格窗里。而我渐渐不知道了他的生活。只是,在我病着,或过节
	时,才见到他,听他说说有趣的事情,像小时侯那样。我总想,他是在意他的妹妹的。他只有,我这样
	的一个妹妹。是么。虽然,那么那么远了。慢吞吞啃西红柿的哥哥,和穿着海军服的六月。

		隐隐觉察哥哥的烦恼。不再年幼的你我,面对的,是太多无声间的硝烟。你会说,“你先撤退,我
	掩护你。”吗。而我们,又退去何处。

		\blankrev

		这是个无可逃遁的世界。\par
		想起,表示记忆在渴望一种相会。

		现在的我,十九岁的年纪,仰卧在9楼之上,听火车呼啸而去的欢笑或呜咽。而七月,已消磨去大
	半。夏天,把这城困于湿热的境地。我想起,六月,六月的种种细枝末节,种种无关痛痒。那些大概已
	被忽略不计的真实,被我深沉地想念了。这是否值得。

		我喜欢,守在窗口看夜晚的火车。常感激,在我的窗口可以清晰地望见火车。虽然,许多人嫌它打
	搅了生活的安静。我却喜欢,喜欢火车亮着灯光跑过我的视野。轨道边,安放着一处绿色的探照灯,它
	又将火车的车身染成了绿色。我痴迷着,那短短十几秒的穿梭而去。

		火车,携着光明和色彩,奔跑在黑暗,它呼喊着,像我们所有人心底的嘶叫一样。火车,那么真实
	,又那么虚无迷离。好象我们自己的轨迹,远远地沿着铁轨延续,你看到脚下的方向,却永远不知道通
	向哪里。

		六月,在这样的七月,在不置可否的起落不定中,我想起六月。童稚无邪的脸孔,一次次,于日光
	下微笑清晰。小女孩长大了吧。她不再穿白色的长筒袜了。

		六月事,丢失的是大部分。\par
		我们,总是难免丢失。是时光吗,你偷去我的果实。

	\endwriting


	\writing{谁 ~ 等待}{2005年07月26日 ~ 12:10:17} %<<<2

		读到布兰迪亚娜的一首诗,

		\longpoem{}{}{}
		疾病比我 \\
		离我自己更近 \\
		恰似腐烂 \\
		比核 \\
		离果实更近 \\
		正如核只需等待 \\
		夏季过去 \\
		才能从果实中脱落 \\
		我只需等待 \\
		生命流逝……
		\endlongpoem

		病中的时光,简单而慵散地过。许多的等待,已堆砌成坚硬无形的一面墙,洁白的墙。

		我想着,我安静的回忆,想着,一个个模糊了又空白掉的人形。在遥远的,终于陌生的院落里,我
	看见祖父,坐在明净的玻璃窗背后,望树缝间蓝到虚伪的天。

		% <todo: 别字: 她依着老屋弯曲的门框 -> 她倚着老屋弯曲的门框 >
		我看见祖母,穿着月白的棉衬衫,忙忙碌碌地,洗衣做饭。她的手,她的身子,那么瘦弱。祖母很
	憔悴,偶尔,独自掉眼泪,不让我们知道。她倚着老屋弯曲的门框,日子不紧不慢地度去,她低声说着
	:他就这么整天,看他那两棵树。

		两棵柿子树。健硕地长在院子中央。父亲说,那是在他还小时就栽下了的。

		祖父,总是望着。他的树,和树间班驳的蓝。他只可以这么坐着了,康健的日子一去不回。他难得
	地这么,拥有安宁,或许就在前夜,他又咳嗽得整夜无法合眼。祖父的病,家里人都清楚,只是瞒着他
	一个人。就以为,他是不知道的。

		我不知道,他是不是可以好起来。而祖母,是劳累如此了。她照顾着祖父的一切,靠她那瘦弱的手
	,瘦弱的身子。我常常觉得不忍心。祖母,却依旧忙碌着。安静地为祖父梳头,擦洗,做他爱吃的菜,
	坐在大木盆前搓洗着衣服。

		祖父在等待么,祖母在等待么。一种转机,或者,一个终结?

		日子不紧不慢地度去。

		夏天,某个午后,祖父在门前的槐树下独自坐着。玩耍的我,听见祖父和路过的问路人说话。怎么
	,老爷子身体不好?什么病啊。我得的是癌,好不了了。

		原来,他全都知道的。我没有告诉谁。

		祖母,穿着月白的棉衬衫,她好象风的缝隙中吹来的一屡青烟。午后的影子,又大又轻。好象许多
	年以前。

		我看见,另外的祖母,祖父。

		祖母的腿被一辆摩托车撞坏,她不可以走路。康健的祖父悉心地打来热水,每一天为她按摩双脚。
	祖父蹲在那,高大的身躯,弯成精美的弧。年幼的我,早已忘记其他,只是记得那一段弧,和弧形中的
	祖父。

		在祖母过世后,我才知道,他们是私奔出家结婚的。

		好象小说中的情节。祖母十几年离家,没有一点消息,几个姐姐都以为她死了。那一个兵荒马乱的
	年代。而祖母,本是定了亲的姑娘。

		祖母,十几岁的祖母,会也穿着月白的棉衬衫吗。或还会梳着乌黑的辫子。她会是茉莉花一样的姑
	娘,会是羞涩而勇敢的爱人。是么,遥远的那一年,那终于陌生的往事。他们相爱,用尽有些唐突潦草
	的一生。

		而平实的幸福,却是真实。\par
		让我望你的老去,再望你的消逝。

		祖父走的那天,是秋季。柿子树结满鲜亮的桔色果实。天,蓝成虚伪。祖母瘫坐在树下,许多人搀
	扶着她,她却无法站起。她瘦弱的身子,那么重那么重了,无处可藏。

		她反复说着:只要他活着,我伺候他也好……\par
		吃饭时,祖母拿起筷子,就掉下泪来。\par
		他还没有吃呢。她喃喃着。

		尽头的等待,是终于的安宁,也是终于的空白和虚无。只落下回忆,碎成粉末的片刻和片刻,连绵
	成生命。爱,爱人,甜美又乏味,平常却隽永。几十年,日子不紧不慢地过。

		病着的祖父,望他的树,他的蓝。\par
		他不会知道,这一天的我,一样在病中,却想起他来,还有他的爱情。\par
		是否在动荡的年代,人们更容易,坚定而质朴地相爱。\par
		等待着生命流逝,而我依然在这里。\par
		日子,总是不紧不慢。

	\endwriting


	\writing{山想}{2005年07月28日 ~ 20:38:29} %<<<2

		如果可以,我愿意住在山脚。

		就在山不远的地方,被丛生的植物和花朵环抱。门前,会精心用红砖砌了花坛,会种了蔷薇,小菊
	和硕大的葵花。我亲手油漆的木门,是清澈的天青色。

		一处不需要很大的家园,却要充溢主人的爱意和快乐。

		养一只通体雪白的小狗,一只眼神温柔的老牛。或许还会有一群小鸭。有一小块荒芜了许久的土地
	,待我播种开垦,期许着又一场繁茂葱茏。

		安静的时间撒满园地,在荒草堆间,我栽植希望,一架葫芦,几株番茄,一小块青菜。屋檐下,种
	一片茉莉,等着它萌芽,抽叶,长成一夜夜幽雅的芬芳。

		夏夜里,我会坐在墙角的秋千,哼飘向夜空无穷的一段轻歌,想念远方的朋友,会赤了双脚,在我
	的园地里踱步,踩那落了一地的星光的碎片。

		四周是漆黑,是漫无际涯的空洞,远山是虚无,天地是虚无,宇宙亦成虚无。只有,我小小的屋,
	亮着灯火,明黄的一盏影影绰绰。只有,我小小的屋,是真实,是梦境的码头,等我在大千世界的归来
	。

		我将独居,或者,有一位爱人。

		我会在挂了纱帐的床上斜倚着读一两本久远的书籍,念一两首久远的诗歌。会怀了如诗经般纯稚天
	真的心,歆享平常无奇的岁岁年年。

		看三五之夜月出东山,听秋窗风雨的夕暮,想窗外遍野的桃树红了美人如瓷面颊。

		风吹四季,吹在山林,吹向四野,吹向大荒。我默守自己的家园,安于无声无息的生活,煮一锅碧
	色的青菜,独坐花下,每一寸枝叶都是佳肴甘美。

		每天,走去山涧的泉眼打水,途中为自己采一把笑在日光中的草花。偶尔,就坐在山溪边,听它的
	歌唱,想起索德格朗的诗:

		\longpoem{}{}{}
		山中的夏天纯朴 \\
		牧场上的花 \\
		古老的庭院微笑 \\
		山溪幽暗的喃喃声 \\
		讲起找到的幸福
		\endlongpoem

		或许,就这么坐着,一直就到了太阳坠下去,也没有知觉,忘却了知觉。等天色都暗淡才踏着一路
	黄昏的橘红,回去我小小的屋。

		独居的女子如此,并无空谷佳人式的清绝和怅惘。我是简单如清水的快乐和安然。

		假若,我有一位爱人。他会陪伴着我,住在小小的屋。\par
		他会是安静的爱人,当这世界需要安静的时刻。

		我会喜欢默默看他在园中浇灌着花朵,呵护一棵青菜,看他纯真如孩童般同老牛说话。老牛,温柔
	地望着他,望我们恬淡的生活。他温柔地望着我,望我眼底清澄的幸福。

		会温一壶酒,在突然落雪的冬,守在窗子里,陪我的爱人共饮一炉,会为他缝补了寒衣,为他系好
	手织的围巾,一同去山坳深处寻一树红梅绚烂。

		他不会像姜夔,为梅谱一曲流芳的歌词,他却会微笑,会讲他的童年,他的快乐,会把欢笑撒满沿
	途的路。我们的脚印会在身后延伸,延伸,我们会在路旁一起堆起雪人,一个风雪中依旧笑靥如画的雪
	人,那么纯洁,那么天真。

		在山脚下,我的门前会开满迷醉的花朵,你只有穿过那团团的花丛,才能到达我天青色的门前。请
	轻声敲门,或唤我的名字,我将居住在那,在生命赐予的或长或短的年华。

		我将独居,或有一位爱人。会是同样的快乐和幸福吗。

		如果可以,我愿意住在山脚。有我小小的园地,小小的屋。

		\begin{flushright}
			05年7月28日
		\end{flushright}

	\endwriting


	\writing{结尾}{2005年07月31日 ~ 19:38:11} %<<<2

		站在七月的尾巴上,天空昏黄着响起雷。又是落雨的傍晚吗,我喜欢这样的天气,让雷声在茫茫然
	里炸开一线光明。

		在这一天,莫飞向南方沿海的某处。在前一天,小鹿穿越北方的向日葵田回到家乡。在暑假终于来
	临的时刻,许多的人瞬间消失,回去,或者离去,往各自的方向。这座城却繁华依旧地充实和空洞着,
	留我驻守,在小小的,没人知晓的窗口。有谁会知道,在这密如鸽笼的窗子中,有那么一扇,藏着我的
	幸福或不幸呢。我因为不被发现,而感觉莫名的快乐。

		仿佛是一处结尾。我对这样的结尾格外敏感。

		翻看原来的文字,竟发现许多是在每月的最后一天写的。好象过去了这一夜,当新的一个月来临,
	一切便是全然的新了。

		在结尾处,我满怀希望中恐惧和回望。而近距离的回望却依旧不真实,都是谎言,我欺骗我自己,
	不是吗。回忆,其实是不可信的。新的东西,终究也会旧,而旧掉的,就不再是事实。

		我的房间是处在风口的。那一扇向阳的小窗,总是敞着,像是对风的欢迎。淡蓝格子的窗帘被拂起
	,很高很高,我这么望着。外边是就要落下的夜晚,又一个夜晚。我喜欢有雷雨的傍晚,喜欢不开灯,
	坐着听雷声,等待一束闪电,那一线光明。永远抓不住的光明。永远处在一处结尾的光明。

		我把自己浸泡在黑暗中,世界就很安静,似乎与我并无关联。\par
		明天开始,我将不太提及七月。\par
		好象一种新。一种未知前途的幸福。\par
		在许多个结尾处,自己矛盾重重地妄想着。我的甜美,和快乐。\par
		洞张着双眼等待。

	\endwriting


	\writing{公主梦}{2005年08月05日 ~ 09:30:34} %<<<2

		小女孩,都在做公主梦。水晶鞋子,白纱裙,英俊的王子骑白马。

		小女孩都相信,有一天,她可以住在童话里。美妙的五彩梦,好象教堂里的彩色玻璃,那么明亮,
	似乎就在那一扇窗的背后,藏了你一生的幸福。正如童话结尾中所说,公主和王子永远快乐地生活在一
	起。

		小女孩相信,会有一个亘长甜美的结局,万世不灭的幸福。

		她们在自己的房间,那小小的想象中的城堡,缓慢而迅速地成长。她们为洋娃娃梳理头发,换一件
	件高贵的纱裙,把它们打扮得如公主那样。仿佛那便是她自己。她喜欢傻傻望她美丽的娃娃,纯澈澈地
	笑着。

		于是,在这小小的房间,滋生着幸福。最简单,也最困难的幸福。她会是公主,穿水晶鞋子,白纱
	裙,英俊的王子骑白马,解救被巫婆施了魔法困在城堡的她。

		小女孩坚定地相信,她会是公主。

		一个公主的梦,每个小女孩的梦。甜滑得有如一块洁白的牛奶糖。

		我并不能够苟同,远在19世纪的叔本华对于女人的一些看法,但是我却确信他所说的:女人的思想
	介于男人与孩童之间。大概他的意思,是与智力有关的,不无贬义。然而,我愿意将思想绝对为思想。
	好象人们说的,女人是永远的孩子。女人,从没有长大的吗,只是身体上的成长吗。女人,却的确是如
	孩童的。因许多梦的不泯灭吗,因心中远比男子长久的天真吗。

		是她心中永远藏着个小女孩。

		即使再穷困的新娘,多数也会有一件嫁衣。在很长的历史里,女子会为出嫁的衣服耗费大量的时间
	来准备。只为了那一个时刻。只为了在那一个时刻上,她将是最美丽的新娘。

		所有的女人,都渴望做一次女主角,在她有限的生命和青春里。而出嫁的时刻,无疑是最好的一个
	舞台,一个错失了就可能不再重复的舞台。因那个时刻,也恰是她最美丽的年华。她要尽可能地美丽着
	,美丽着,那一天,她是女主角,将是全部爱和幸福的拥有者。

		仿佛,她那一个梦,在许多年以后终于实现。

		那一个公主梦,在现实并不如童话的剧情中,别别扭扭,勉勉强强地实现了。

		她穿着光亮的鞋子,洁白的纱裙,这一切是多么地相似呢。只是,解救她的人,不会是王子,也不
	骑白马,而这一种解救,又如何不是另一种囚禁。

		然而,终究是实现了,公主梦。

		她穿着婚纱倒卧在床上,四周是盛开的玫瑰。时光就叉开一个巨大的缺口,露出不确定的光芒,这
	时候,公主在甜美的笑意中睡去了。

		故事的结尾,我们写道:后来,公主和王子永远幸福地生活在一起。

		故事可以搁笔,而生活继续。小女孩终于还是长大了吧,至少,她不再相信,有什么亘长甜美的结
	局,和万世不灭的幸福。她不再读童话,她觉得那是全然的鬼话和谎言。

		而童话,是最最完美的谎言吧,让你总不忍心不去相信。明知是虚假,却坚定地笃信。好象说出口
	,就一天天死亡着的誓言。

		我们的一生是被欺骗的一生。在叠错的谎言中,悲伤莫名,幸福莫名。甘愿被欺骗着,甘愿一次次
	在不醒悟的梦里走失。想起彩色玻璃一样的明亮,想窗子背后,那未知的陌生和美好。我们的臆想,制
	造许多梦幻,许多期望和奢求。

		“一定要等待着你,骑白马来将我解救,否则,我将恒久地困在这城堡,缓慢而迅速地老去。”是谁
	,她这么说着。

		所有的老太太,都是小女孩变的。她也有如花似玉的年华,也曾在梦中妄想,她也曾是公主。只是
	,当年,她洁白稚嫩的手把钻石一样的星星镶嵌在岁月的天空,而今,在霜意里收起一地回味,一地委
	顿,或也是一地经年的温存和智慧。

		她心中的世界安静,而年轮过于灿烂。

	\endwriting


	\writing{开落}{2005年08月06日 ~ 21:21:52} %<<<2

		\longpoem{}{}{}
		我看到许多无声的笑意 \\
		无声中向我走来 \\
		又兀自隐去 \\
		在生命的暗影
		\endlongpoem

		莫名地,最近想起的,多是儿时的事情。点点滴滴,迷离中敲我的心门,一声声一句句地,把我呼
	唤。你还在哪里吗,你还住在那间栽了月季花的庭院吗。我问自己。又好象是问一个从不曾存在的声音
	。她是安静的,一语不发,玄奥地笑着,看我长大。那么残酷地长大。

		和哥哥聊天。不过是了了的几句。却让两个人都感动起来。是长久的无言与沉默了。是什么时候起
	,哥哥成为我生活世界边缘的一个名字或符号了呢。是那么长久的陌生了呀。为了各自的生涯与生活,
	我们好象童年的玩伴那样,两处不见。

		我不知道,他经历些什么,又经历着些什么。甚至,我不知道,长大了的哥哥,已成为怎样一个男
	子。全部,只是模糊的形象。我的哥哥,曾经被嘲笑长着秀气双眉的男孩子。现在,要被称为男子了。

		直到这两天,我似乎才了解了一些,我的哥哥,原来是如此善良而多情。他说,他是多愁善感的傻
	子。我却那么庆幸,哥哥是这样的傻子。一个纯粹的人,有爱的人,多么难得。

		他也在回忆,那些儿时。他想到我的祖父,祖母,他的姥姥,老爷。我们同样深爱着,却已消失的
	人。哥哥掉下泪来。

		许多久远细节在浮动,在下落,世界安静,时空安静。

		夏天,一样的夏天,老家门前的槐树花轻轻飘洒,一地的碎白,一地的清香。那洁白的花,在分秒
	的日光里绽放无言。哥哥去上学了,带着鲜红的领巾,祖母拉着我的小手,站在月季花丛背后。中午,
	他会蹦跳在祖父的大自行车后边,放学回来。我们一起吃午饭,那些现今是只堪回忆了的食物,听哥哥
	说,他学校里的乐事。我总是羡慕,期盼着快些长大,去上学。

		经过老家的位置,我总不禁发呆。我们的院子和童年,被崭新的街心花园掩埋。流落在回忆细微的
	世界,和无处可去的我,一同迷失在这面目全非了的街道。它是漂亮了,整洁了,却成为我的疼痛,不
	可以消失和痊愈了的痛。似曾相识的,只有路边的槐树,要纤细许多的槐树,飘散依旧的白色花朵,落
	了一地。我的回忆也在下落,下落,和哥哥一起,在只有我们两个人生动记得的世界中。大概,也还会
	有人记得,只是,那又是另一个世界了,它不属于孩子,不属于童年。

		\blankrev
		我听不到花开的声音,只见到你的下落。

		我忘了全部过往的真相,却在印象里把回忆填补得清晰。便是全然的美丽,一个光明的世界,满怀
	笑意和纯洁的世界。那些点滴,是一地的碎白,是我儿时嗅见的清香。你会相信吗。你是否也会想念到
	胸口轻轻地疼了。

		\blankrev
		那些我们深爱的人呀。\par
		你们好吗。我们都在完好而坚强地生活。放心吧。

		只是,哥哥他说,时间老人真坏。

	\endwriting


	\writing{立秋。台风}{2005年08月08日 ~ 14:43:52} %<<<2

		昨天,是立秋。是该贴秋膘的日子。据说,五花肉卖得很好。家家户户都在忙碌着补充这个苦夏的
	消耗。在街角不起眼的一角树阴里,我看见夏天穿着花裙子,收拾行李,正准备逃走。逃去南方,和海
	洋,那些我无法到达的地方。

		她是的确已经决定了离开吗。

		母亲说,今年是早立秋。她还要流连一阵的。看来,夏天是迟疑的,我喜欢她的迟疑。


		叫麦莎的台风登陆。一路上房倒屋塌,满目狼藉。我,一个地道纯粹的北方人,从不知道风的危险
	和伤害。

		只是从电视上见到,狂风和暴雨,以及台风中摇曳的家园。安然坐在沙发的我,一阵心惊。

		在并不遥远的地方,有另一种天空的模样,那里的人们正遭受着灾难。我看到失去房屋的老妇,苍
	老的身影,在倒塌的房屋间找寻着什么,老泪纵横。

		我只是看见,而我从不知道,风的危险和伤害。

		报纸上说,台风会波及到北京,是11年以来的首次。上一次,已经是94年的事情。我已经不记得,
	那一年的夏天有什么不同。

		这几天,会下暴雨,所有人似乎都在等待。

		台风会来到我们的城吗。你真的来吗。

		那已精疲力竭的麦莎,带来的只是雨水和清凉,没有危险和伤害。对于北方,对于内陆,她是慈爱
	和温柔如此的。

		我想着莫。她见到麦莎吗。是不是在夜里紧闭了门窗,颤巍巍中听风的经过。


		这个世界就没有安宁过。

	\endwriting


	\writing{夜船}{2005年08月10日 ~ 11:02:00} %<<<2

		在北京,你怎么可以不痴迷什刹海的夜。

		看一潭碧水在夜中被染成暗黑,泼洒肆意绚烂的流彩。看疏落落几丛小荷,幽寂无声中悄然绽放。

		只有清风,缺席明月的夏夜,却是兴味依然。和苏乘舟,在八月浓而不艳的风情万种间。

		两岸,是歌声,却看不到汹涌的人群。我们在船上,在寂寂得有几分凉意了的水面,好象一处遗世
	独立的世界。听闻着远处酒吧中苍凉旷远的歌声,一个可以伸长到很远直刺天空的男声,想着有一搭没
	一搭的过往种种,那些不去想,也不会被遗忘或特别记得的事。

		苏在微笑,另一位同伴为她拍照,时间就停顿着,蹉跎不前。

		我只是痴痴看灯火,红红又蓝蓝的一片片,一条条,把我们的夜,照得迷离虚妄着,在暗黑的彼岸
	之上。那一盏盏光亮里,坐着人群,欢乐的,或悲伤的人群,会有盈盈着笑的情侣,也会有独自买醉的
	女子和男子。

		这城市中,有水的角落,溶释我们许多人的温柔和眼泪。似乎是只有面对这样的一处水面,你才可
	以那么那么地纯粹和真实,放肆地欢乐,放肆地痛苦着。是水的原始魔力吗。水是值得敬畏和心怀感激
	的神灵。

		听说放河灯来许愿的传言,于是我们带了12只河灯上船。只是简单的,用报纸折成的小船,放置莲
	花形的蜡烛。却是充满了虔诚的仪式。

		一支支划亮火柴。我看着同伴手中的一星火光,那么微小,此刻却如此明亮的光,想到的竟是卖火
	柴的那个小女孩。一样是许着愿望,她站在孤独的风雪中,化作了星星。

		光亮可以带来幸福和奇迹,我一直相信。我们总需要一支火柴吧,在风雪中划亮,取暖或妄想,一
	个个难以到达的岸。那便是幸福的了,可以妄想,可以不作计算地妄想。

		蜡烛被点燃,放着醉人的光芒。风中的烛火,偏向同一个方向。小小的船,小小的光亮,承载着我
	们各人的妄想和希望,等待着起航。

		终于,在撒手的一刻,它们荡悠悠地漂走。暗黑的水上,几点不明亮的光,渐渐远逝。目送着船的
	离去,道一声珍重,道一声别离。安静着,谁也没有言语。

		夜在流溢绚烂的灯火和歌声中,睡得沉沉。我们的舟,像凝固在湖心,没有动静。心中的光亮,在
	水上漂远了,为河水送去我们的愿望,等一个真实,一个实现。

		没有星星的夜晚,空气湿而温凉。和苏在通明的地安门蹦跳着行走,像两个不知好歹的孩子。一个
	不可思议的夜晚。我们好象瞬间里,就拥有许多的幸福,它卡在喉咙,让你想唱歌,想大叫,想这么蹦
	跳着向前跑去。是水的魔力吧,它听到我的祈祷。

		在北京,你怎么可能不爱,夜的什刹海。那是容你放肆,容你妄想的天真。

		我贪恋着,这一切。

	\endwriting


	\writing{七夕}{2005年08月11日 ~ 19:58:58} %<<<2

		一声喟叹之后,谁知,又是多少的此去经年。


		美丽的节日,七夕,在神话的光辉里,照耀星光迷人。而今天,我在一个净白如瓷的早晨醒来,望
	见的,是窗口的浓雾一片。一片浓白的世界,模糊了轮廓的楼宇,虚无了姿态的树木,像烟的扩散和弥
	漫,像晕湿的一幅水彩。

		有雾的日子,让人感觉生命的不真。似乎就没有什么是不可以幻化的,一切的一切,你无法抓牢,
	它们和你的眼睛开玩笑,全部可以一个瞬息就莫名走失。在那些你还未及了解的时刻。


		即使如此,我仍怀着极甜美的心境来度过这样一个美丽的节日。


		我静卧着,轻闭了双眼,想象自己的身体被包裹在华丽的糖纸中,于是,有了蜜,从心房和心室的
	小缝隙间流出来,一丝丝卷着小浪花,随着血液,向我肢体的最末端奔腾。

		我便获得了幸福,在小小的一个时刻里,我成了拥有甜蜜的孩子,或者,我本身就是甜蜜……


		没有睡去,我只是用这样的方式,使自己合乎于这节日的欢乐气氛,不至于将它浪费。毕竟,如此
	美丽的节日是缺少的。


		难免要提及爱情,一个说起来难免糊涂的词。


		我不懂得爱情。我没有找到爱,也不曾遇见谁。我是匆忙地像花草一样兀自长大了。无休止地想念
	和回忆。我不拥有爱情,但我想象它的模样。


		苏童写到,“有时候爱情是一种致命的疾病。”那一篇短文中,讲述了一对恩爱的老夫妻的故事。或
	许许多人都听闻过这样的传奇,相伴一生的两人,一方死去,不久一方也离奇死去。多数后者是无疾而
	终,表情平静而幸福。“我从此迷信爱情的年轮,假如有永恒的爱情,它一定是非常苍老的。”

		我喜欢苍老这两个字,尤其用在爱情上。爱情,是苍老的,是相爱那天起就甘愿承受的疾病。这让
	我动容。

		人们喜欢永恒,一切美丽的永恒。而美丽总是力不从心地老去了,变丑,锈蚀。若爱情可以苍老,
	那便是世间少有的美丽的永恒。因苍老而愈加美丽,愈加动人心魄的美丽。

		而我,终究是不懂得爱情。


		我参不透爱情的来世和今生,看不破劫数和命定。于是,我听不懂你的誓言,想不通拥有的所在。
	我没有找到爱,也不曾遇见谁。今天,这节日,似乎本与我不相干。


		我却悄悄期盼和想望着,凝望爱人的老去,用我并不富足的光阴。我不知道自己,是否相信爱情,
	我却明白,我是崇拜爱情的。

		爱情是圣洁的。

		虽然,这世界太多时候,已经将它世俗化,太多时候,爱情几近成为物质的奴隶。我没有放弃信仰
	,爱情,应该是洁白的。好象这一个雾起的早晨。你可以不太清晰,可以不辨方向和远近,但爱情,一
	定是光洁而明亮的。


		我在臆断爱情的模样。

		而我,只有疾病,没有爱情。


		天黑下来,天上的星星就亮起来。织女依旧,牛郎依旧,星光依旧,许多年。


		我们仰望,想是哪一阵悲欢的歌声凋谢成银河。我们或甜蜜或悲戚地畅想,我们的日月,和爱人,
	想自己的一条河,想彼岸的虚无,时空的无限。

		很多,在白日里无法想见的,都一并地盛开了,成就一座花园。繁茂地生长,蔓延,混合着芬芳和
	光芒,让新生的藤,触了你最温柔敏感的那一寸肌肤,触了你平日里麻木不仁的心魂。因为美丽的节日
	,因为美丽的神话,许多个不相识的自己一一苏醒,醒在陌生的花草间,迷惘又惊奇。

		因此,这一个我,也开始莫名地说起爱情。

		所以,在一个不相干的节日,我刻意迎合着气氛。

		于是,甜美的心境中,升起灰蒙的烟,像焚烧着什么一样,发出刺鼻的气味。哦,是回忆吗,还是
	,爱情?

		而我,不曾找到爱,不曾遇见谁。


		我固执如此地坚定着。

		我不懂得爱情。


		我便无须追问爱情的去向和源流。便不需要想念和回忆。我焚烧,焚烧不知如何命名的东西。它光
	洁,它美丽,它没来及苍老,它匆匆死去。


		我依旧躺着,时钟滴答里,光阴就这么荒废。窗口,是早已熟识了漠然了的景色,雾没有退去。我
	的蜜,在身体各处散播着快乐,而我清醒地明白了,我没有华丽的外衣,没有甜蜜的心。这是件残酷的
	事情,残酷在于,我竟然在真实的白日里醒着。如此赤裸地醒了。


		另外的许多个自己,在那个瞬间里,倒下去。没了踪影。

		爱一个人,小鹿会甘愿溺死在他眼窝的湖水里。

		爱一个人,是无须思考和丈量的执意妄为。

		可以很勇敢地去懂得爱情吗。我却终究是不懂得爱情,我只是迷恋它苍老的模样。

		那会是一张简单到乏味的面孔,却是美丽,却是无染的圣洁。

		你可以明白吗,爱情不需治疗和药物,爱情不施粉黛。

		要用多少次的告别,才教你学会。


		知谁,误了多少春风月华,多少红烛良宵。全付一声喟叹。是任我乘浮槎游弋天河,也无法相逢吗
	。多少的离情别恨,只化了沉吟两处的各自心绪。

		星光,依旧是星光,照在你的河上,也照在我的河上。而夏雨的几次滂沱,又如何注满干涸。似乎
	,是全然的徒劳。万事是幻化的,如雾这般。


		没有睡去,我迎合着节日的甜美。而我,无非是空空地生活着。

		因为空,所以有用,可以簪一朵小花?

		美丽的节日,七夕。又被我浪费掉了。

		谁叫,我不懂得爱情呢……

	\endwriting


	\writing{纸}{2005年08月13日 ~ 11:59:36} %<<<2

		我不是多梦的人。我的夜晚多数时候是全然的黑暗和寂静。我总是沉入夜晚深蓝的湖心,没有声响
	,没有思绪地熟睡。


		而在少数的梦中,反复出现一种梦境:瓦蓝蓝的天,醉红色的墙,头顶上方洁白的碎纸片纷纷而落
	,好象一场无端的雨,却又象羽毛的坠落,那样飘逸空灵。

		我不知道,为什么自己会站在那,一个陌生的,色彩浓艳,日光充沛的世界,为什么会在纸片的下
	落中,有想呼喊的冲动。

		多少个夜晚,我好象就一直站在了那,看着漫天的纸片落满凡尘,落在我的身体,并不如雪地落着
	。

		因这离奇古怪的梦,我曾经把昵称改为“碎纸片儿”。

		朋友们说,这是个忧伤的名字,很多人不喜欢它。也许,因为纸是苍白,脆弱,而单薄的。它没有
	什么重量,没有什么内容,纯白干净的薄薄一片,象一张洞张着双眼的脸孔,经不住一阵邪风恶雨的样
	子。

		纸,这样的纸,却让我心生疼惜。常常一个人面对一张白纸发呆。

		多少次,我在上边涂画,从我是穿着粉色小纱裙的年纪。我画花朵,画大树,画小兔子的家,用五
	彩缤纷的蜡笔。那个小女孩,在纸上畅想,最可爱简单的想象。

		纸是快乐的,因为孩子的美梦,因为缤纷明亮的颜色,因为单纯稚趣的图画。纸,总是纯白白中等
	待,一场绚烂,一场欢乐的盛放。我于是对纸心怀感激。


		纸是善良的。纸不去索要,不去争辩,它欣然接受一切,一切可能的创造。

		偶尔,会折一只纸飞机。而我,并没有灵巧的双手。我的飞机,总是外型拙劣,性能也不高超。却
	是凝聚着我小小希望的飞机,它是洁白的,它可以乘着气流飞去。

		我喜欢站在窗口,把我的飞机放向天空。它们就会消失,会离开我的生命,不见踪影。我相信,它
	们会钻进某个匿藏的缝隙,飞去另外的时空。或许,就掉进我的夜晚,我的无声息的梦境。

		会吗。你会飞回来吗。

		也会折纸船。在夏天深情的兰色傍晚,放进小池塘中。有几次,几只小蚂蚁还充当了乘客,并在小
	船上放了槐树的圆叶子,作为装饰。我的船,漂在池塘中,不知道它和蚂蚁们的命运。那一涡小小的水
	,承载着纸的幸福,和我的幸福。

		纸会微笑着吧,它成为了船,入水的一刻,在我眼中,它同万吨巨轮是毫无差别的。我蹲在一旁,
	看着我的船,傍晚就深暗下来。那个夏天,谁也不知道,曾经有一艘名为幸福号的船曾经在这里光荣地
	起航。


		纸,苍白的纸,像神奇的魔法师,你给它希望,它就成就你的一切创造。

		我用白纸给你写信。

		我说,让我在这洁白的世界撒点野。没有格子,没有横线的世界,我任性任意地书写,这多么好。
	我写我的日月,告诉你这里的晴雨,让你知道,昨晚的闪电怎样劈开了我的黑夜,丢下一瞬的光明。我
	想让你好象看见,这一刻静坐在这里书写着的我,我想用纯蓝色的墨迹,把我的快乐和悲哀透视在你的
	心里。

		我的纸,就承载着这一切一切,像是另一个我,被我自己小心折好,装进信封,投入信筒,穿越遥
	遥无际的公路和山林,来到你面前,又被你小心地捧在手中,带着笑意地读着。


		我于是,并不觉纸是忧伤的。而碎掉的纸呢。


		那许多个夜晚,无端闯入的纸片,你们从何而来呢。

		我总是站在那,那陌生的光明世界里。我似乎是等着纸的破碎,等了很久很久。好象纸,无声里等
	着我的那些岁岁年年。

		我不讨厌这样的梦境,也并不恐惧,虽然也许它是什么不吉祥的预兆。

		我竟喜欢,那样站立中的自己,喜欢瓦蓝蓝的天,和那醉红色的墙。任纸片下落纷纷,落在凡尘,
	落在我的身体,直到掩埋我。

		我愿意,在夜晚的湖心,偶尔地醒来,这么站立,等着一场没有解释的雨。是如受洗礼的滋味。在
	一种纯白色的弥漫里,我仿佛望见另一个自己。


		就相信了,那是另一个我,化做了飞机,钻进某个匿藏的缝隙,飞去了另外的时空。

	\endwriting


	\writing{纤痛}{2005年08月15日 ~ 17:00:13} %<<<2

		当痛苦随着发丝飘落,悄无声息,是怎样的滋味。

		在医院取血时,又遇见面色苍白,没了头发,裹着花色头巾的女孩。不知道她的病,只看见她的虚
	弱。我坐在她身后。她的背影安静,像塑像一样凝定着,头巾上的花朵开得绚烂。

		我揣测着她的心,她的境遇,和未来。想象她曾经的模样,想那一头的乌发,一定会是美丽的。

		她坐着等待,好象我这样。似乎就等了一生的时间,在这种空白掉的时间里,我们是忘却了肉体的
	灵魂,干干净净地,看鲜红的血液流出,没有了哀怨,没有了责怪。病,便不过是病了,而不是灾难。


		我却总不禁想问,是如何,又为了些什么,让花一样年纪的女孩,就这么恍然病了,就在一夜醒来
	,荒芜了秀发和美丽? —— 是毫无道理的事情。

		没有人回答,没有人解释。

		她只是一根根拾起枕上的发丝,拾起那丝丝的痛苦。会是一个浓白的早晨吗,她对着陌生的镜子,
	悄悄地哭了。

		是毫无道理的事情。而事中的人,无可回避,只有接受着,接受着。是怎样的残酷,没有武器,没
	有血腥的杀戮一般。

		每一根头发,都会哭泣,都有知觉和生命。它们从我们的血肉中长出,是同身体和肌肤一样,受之
	于父母。

		我的头发是纤细的。人说,头发软的人心软。

		大概,我的确是心软的人。见不得灾难和忧伤。


		好象今天,我竟想为一个陌生人哭泣。

		坐在她身后,我和她飘然逝去的发丝,一同掉泪。

		我不去追问命运,不怨恨上天,不乞求神的怜悯和慈悲。我只想,让无端的苦难一点点稀释,飘散
	,让美丽的生命,在美丽的年纪,自由自然地美丽,没有恐惧,没有疼痛,没有悲伤,没有凋萎和暗淡
	。

		可以吗。

		她凝定在我面前,塑像一样。只有头巾上的花朵开得绚烂。

		她不会知道,我在身后,又想着什么。

		那么许多,不可遏止,纤细的疼痛。好象发丝的生长。在我心上。

	\endwriting


	\writing{雷}{2005年08月16日 ~ 15:53:57} %<<<2

		又是疏落落下起雨的天气。我躲在房间里,却还是感觉到凉意。开着窗,就听到断断续续的雨,和
	互远互近的车轮声。一些,是自然的呜咽,一些,是人世的喧嚣。本来安静的房间,一时并不寂寞。


		今天,没有雷,却想起许多个响着雷的黄昏和夜晚。


		如那枕着雷声入睡的春末。一个浸泡在碧绿色空气中的晚上。在宿舍高高的床上,同莫对头而眠,
	关了灯的房间没有一丝光亮。没有人言语,似乎是都已熟睡,我却知道,我们分明都清晰地醒着,听那
	风雨敲窗,听那雷声滚滚。各自怀各自的心事,有怅惘,有悲伤,也有乡愁。


		都是淡如烟雾的情绪,与雨的飘洒缠绵一处。这样的夜晚,当你无声听着雷,是总难以平复心绪的
	,特别是在分别,和离家的时候。

		那些在日光里被烘干的幽思,会如阴雨天的菌类,一点点滋生,长成形态各异的蘑菇。就在碧绿色
	的空气舒展开,如朵朵晴天里的伞。

		渐渐,困意浓起,我终于在雷的哄响睡去,我不知道,谁还会醒着,听着风雨的绽放。或许,就会
	有一个抱影无眠的的长夜,随着被拉得纤细悠长的雨水滴落到天光亮起。在我无梦的睡眠,只是隐约觉
	察到雷的存在,却不足以唤我醒来。那个春末的夜晚,雨确是下了整夜。


		在雷声落下的时刻,我想着母亲。想着童年里,偎在她怀中,恐慌着的夏夜。


		我是怕雷声的孩子。当半夜打起雷,我总是慌张地跑到父母的房间。母亲就会拢我在身旁,喃喃说
	着,不怕不怕。闪电划过夜空,擦亮了黑夜,我攒缩在母亲的保护中,才慢慢平静。母亲给我最安全的
	包容,在所有危险的时刻。我于是在响雷时想念母亲,想她在家中,听了雷声是否也会惦念起我,她终
	于长大,离家的孩子。


		雷声是凶猛的,孩子们都会害怕,所以,我从没因为害怕雷声而感觉自己胆小。而我,大概终究不
	是勇敢的人。虽然,我已经不再恐惧雷声,甚至爱上了有雷声的日子。


		喜欢有闪电,有光明劈开黑暗。喜欢那轰然的巨响,像是上天的嚎叫。雷电,是真实无比的,是不
	加掩饰的宣泄。雷会劈死妖精,雷似乎还是公正的执法者,虽然这世上大概确实是有些好妖精的。好象
	,一样的雷声里,王太常救了狐狸,于是有了小翠的故事。妖精懂得报恩,似乎更胜于某些所谓的人。
	雷声里,我便总有了这样那样的异想,停停走走地飘着。会想象,是不是有好妖精正逃遁在风雨中,就
	希望它躲到我的房里来。如果这世上果真有妖精,有鬼怪,也是极好的事情吧。


		因为有情,动物才成为妖精。因为有情,鬼才会还阳成人。因为有情,人才会在雨天,在雷声起落
	里长出蘑菇。这个世界,在雨水里清清冷冷,实则隐藏着温度。我们各自的体温,温暖着湿冷的空气,
	血液奔流,饱含着热量。因为是凡人,所以逃不出情空欲海。我们反复地相爱和离别,却依旧是痴痴的
	有情人。小鹿安静地躲在乡下,安静在她的爱情。她不知道,北京下着雨,我想念起你们,和许多。我
	们说,要幸福地生活,却仍然发现,风景是孤独的。妖精鬼怪亦沉溺其中无力自拔,何况于人呢?是不
	必执意的。


		前几天,总有在雾中响着雷的傍晚。我站在窗前,却已看不见对面的灯光。那些黑夜中本来明亮的
	灯光,一盏盏推去,隐在浓白的雾后,不见了。我只面对,一个混沌又迷茫的眼前。雷声在耳畔盛开,
	追逐闪电的光明。就这么,我又沉入很深,直到雨终于停了。

		雷雨的世界中,似乎谁也逃不掉。

		只有灯光,是狡猾的。

	\endwriting


	\writing{尾巴}{2005年08月18日 ~ 13:49:58} %<<<2

		\longpoem{}{}{}
		雨天的手牵著你的衣袖 \\
		雨天的温柔总是选错拥挤时候 \\
		雨天的小指头骚动我虚有的乡愁 \\
		雨天的尾巴让夕阳牵著走 \\
		春天的手拍一拍晒乾的枕头 \\
		夏天的温柔躲在你的画框不走 \\
		秋天的小指头点亮了小镇的烟火 \\
		冬天的尾巴摩擦爱人的双手 \\
		雨天的手啊牵著你的衣袖 \\
		雨天的温柔总是选错拥挤时候 \\
		雨天的小指头骚动我虚有的乡愁 \\
		雨天的尾巴拍打浪花一朵朵 \\
		雨天的尾巴让夕阳牵著走 \\
		我的青春是否你也牵著走
		\endlongpoem


		喜欢陈绮贞,从第一次在深夜电台中听到她的《小尘埃》开始。\par
		喜欢她,喜欢甜美中的清新天真。\par
		简单的歌,简单的心情,就这么哼唱着,不刻意,不雕琢。\par
		她是夏风里,纯白色的花朵。

		雨天的尾巴。四季的温柔与纠缠。\par
		在多云的日子,坐在时隐时现的日光中听着。世界是光明的。\par
		在八月,夏末。

		田说,要向大熊猫学习,努力改造,幸福生活。\par
		她这么决定着。很快乐。

	\endwriting


	\writing{如树,温柔}{2005年08月19日 ~ 14:21:38} %<<<2

		多云的天,浮游不定的光线,在红砖地上铺撒一地碎掉的日影。一个慵懒的下午,就在发丝与手臂
	的温软里生长出来。透着夏末尚存的余热,带着透明,无邪的几许天真。我斜靠着窗,望夏天的离别,
	于不轻不重的体触之间。我甚至是没有察觉,凉气已经从北方启程,一波千里地赶来,它是轻捷而小心
	地,怕惊扰了我们在夏末最后的温柔。

		便让温度尚存的所有,尽情尽意地,将最后的热挥发干净。好造就一个全然肃杀,全然静定的秋吧
	。而你的温柔,我的温柔,在这慵懒着的下午,却低垂着睡眼一双双,像笼了轻纱的几场混梦那样,无
	法认得。我要你懂得,叶子最后的苍翠,懂得,造物的用心良苦。于是,请仰头望一望,那被夏风轻摇
	的树尖,想一两件年少轻狂的往事,再做一番引人洒泪的追忆吧。是该深情的时刻,这夏末。是该我们
	去回首,去温柔,再狠心转身的季节。

		我没有刻意,营造一个诗味的境地。只是,我的生活,确已犹如独自的凌空跳伞,从天空不能够再
	忧伤的蓝。脚下,是深渊万丈,是无可支持的空洞,头上,是漂游无向,朝聚夕散的云朵和烟尘。这一
	种世界,我似乎无处可去,只是安静在凌空,安静在小小的伞朵之下。我没有刻意,为生活的进退悲叹
	,许多的人,是如我这般存在于天空的,自己的,或他人的,一片不稳定的天空。没有安全可言,而人
	的生活,本身便是无比凶险的。于是,我并不介意,命运中的遭受,那不是惩罚,或许,是礼物。

		于是,让最后的温柔,在手心绽放,带着体温,燃起一柱烛火。温暖许多个雨夜和寒冬。把该遗忘
	的遗忘,该记得的记得。我会惦念,冬天的雪落,无声的庭院,远方的朋友。我将握着旧掉的钢笔,在
	灯火里写一封信,在暗淡的光景中睡去,睁开眼已是满眼玉花翻飞。我在等待,那样如昨的一个冬日,
	许多的告别还没有完成的冬日。用我最后的温柔,想望着,想望着,在吹起栀子芬芳的花园。洁白的花
	朵,无醒无梦地美丽着。人是否也可以,没有夜晚,没有白昼地老去呢。花,最终留一具艳丽的尸体。
	谁去收敛。我用这悲情的温柔,把你想念,不因为回忆,只源于空白掉的道别。

		哥哥,用一棵树的姿态,苍凉地等待着爱情的归来。他是树,节节拔高,为了望爱人归来的身影。
	他是树,折自己的臂膀做灯盏,照亮她黑夜的路途。他是树,他化了黝黑丑陋的煤炭,让他的爱人取暖
	。我读他的字,轻轻地感动着,细细的温柔,是哥哥的爱情。如丝线纠缠,绵延,不可明言,只可以用
	另一棵树的心去体悟和感想。哥哥,一个善良而深情的男子,在温柔的目光中,守望一段往昔,守侯一
	句诺言。如树的坚定,如树的挺立,如树的,从未改变的春秋与冬夏。你在哪一处驿站,是否如秋意一
	般,一波千里地兼程赶来?你可想见,哥哥如此的温柔与苍凉?

		我并不盼望着秋天。我不等谁的结局与明天。于空中的我,飘浮着,自己的寂寞和自在。我喜欢这
	样的安静,没有声息,只有飞驰的幻梦和妄想。我是靠沉睡活着我人,我不期许醒来。就在蓝到忧伤的
	天空,睡着,睡着。没有人惊扰,没有人知道。夏末,吹在耳朵,吹在心脏,我向下望,看见大地上,
	满是镜子。那是湖泊,是最温柔的水泽。它们一如我,不作表情地安睡,一年,百年,千万年。而湖,
	是易碎的,哪怕只是一片云的流过。你听不到声响,但湖已是碎片,许多许多。映着天空的影子,这一
	块,那一块。


		\blankrev
		温柔,是不能够触摸的。因为它的易碎,因为它的危险。

		然而,在日影慵懒的夏末,我仍然执意着温柔。最后的温柔。因为温柔,所以我们孤注一掷。用如
	树的姿态,苍凉地等待。望着风,摇过树尖,吹起谁的衣角和发梢。

		\blankrev
		之后,会有肃杀,静定的秋天。

		这一个时刻,只把残存的温柔,小心擦拭,晾干。

	\endwriting


	\writing{旧的。那些}{2005年08月21日 ~ 16:51:51} %<<<2

		有空闲的下午,可以收拾房间角落里,那些书架和抽屉中的杂物。默无声息地,它们躲在日常生活
	触及不到的边缘,只有在这样的下午,才抖落一身尘埃,重新鲜活在了我的生命。

		坐在地板上,依旧温热的风徐徐吹进房来。我被包围在众多熟识却陌生的物件中间。好象一位唐突
	的访者,而不再是主人,被热情迎接的人群吓坏。

		我坐在陌生的一片世界,那么远的,又那么近的世界。

		一件久别的玩具,格子布的长耳兔,(也曾被笑称为面条狗的),一本痴迷过的书,一本中学时的
	日记…… 许多,这些,是完整的,可以记忆得圆满清晰。

		另一些,是残片。

		在堆积的纸张中间,发现一张遥远的试卷,化学试卷,才记起,原来自己也曾认真学习过它的。如
	今,有关化学,却已忘却得干净,透彻的干净。人,对于不喜爱的,确乎是健忘,薄情如此。

		一张字迹凌乱的纸条,写着无关紧要的话,却是开心的话,是初中的吧,只在那个年纪,我们才会
	疯狂地传着纸条。让一小张纸片,在沉闷的课上,飞翔着传达我们的思想。想起来,这又是多么美妙的
	呢。我的初中,也飞远了,和那些不计其数的,不知去向的纸条一同。一同嬉笑着,传纸条的人呢。你
	们还好吗。

		一方印章,两面各印着一张照片。是夏天里,一起去做的。盖在纸上就是一张照片,有黑白报纸上
	边的效果。一面是我,一面是他。忘了是什么时候,被我遗弃在书架的低层,布了灰尘。印油尚存,并
	没有干掉,于是随手捡起一张纸来,印了一下。雪白的纸面上,于是有了微笑甜美的女孩,两年前的那
	个自己。一样的夏天,一样的甜美。

		它们,全整的,和残碎的,却一并是真实无比地旧了。

		有一些挣扎着在记忆里树立,有一些在今日无可挽回地坍塌,一溃千里。像是被拆掉的旧楼房,一
	夜间,就不见了踪迹。只有,零零落落的瓦片,木屑,和飞来飞去的尘埃,貌似坚强地坚守曾经的阵地
	。

		旧掉的,就走向毁灭。

		而我,对旧东西心存眷恋地,情有独钟。坐在这些旧东西之间,虽然陌生,却是喜悦的。像是时空
	的一个小空洞,被我发现,又独自悄悄钻了进去。我遇见,许多个已经陌生的自己,和你们。我遇见,
	不听话的回忆,和希望,遇见没有理由的快乐,和哀愁。我喜欢,面对一种种的旧,一种种被时间钝化
	了的回忆。虽然,那么多,已无迹可寻。只残存,旧掉的气味,在老去的字迹上,和一两点,你的,或
	我的,模糊的笑。

		温帝为彼得潘缝好影子,一同飞去理想国。

		我可不可以也剪坏你的影子,让你无法飞起,就留在温帝的家里。

		这是旧了的童话,和我旧了的想象。

		日子放着白光,我在包围中睡着了。

	\endwriting


	\writing{穿梭}{2005年08月24日 ~ 19:11:48} %<<<2

		谁忍住悲伤,心疼地原谅,全部的错失和浪费;谁用最后的温柔,道一句告别,成就不再流连的转
	眼。小女子,亦可坚决明白如此,把回忆的错觉,通彻地一笔勾销。因为更深切的解悟,我们没有了昨
	天,我们只是品尝,而不沉溺。

		坐地铁,车厢在黑暗中穿梭。

		我把自己浸泡在乱星清澈又低迷的歌声,随它飞奔。站台是光明,人们的面孔迅速后撤着。我喜欢
	,地铁的速度,和因速度,幻变出的迷离。像生活的重重意象叠错在一列行驶中的时光。是黑暗,和光
	明的交接不断,如我们的心灵,一处阴湿,又一处光艳,纠缠连绵。

		在比地面更接近这星球心脏的地方,我听到更真实些的心律起伏,不缓不急,涌动向前,似乎无所
	畏惧。我不是勇敢的人,却分明缺席了恐惧。

		在安静的乘坐中,我遇见陌生而众多的面孔,幸福的,或悲伤,或麻木的。

		我想起谁?\par
		想起一样曾经陌生,曾经也爱恋的谁?\par
		终究仍是陌生了的谁,陪我在地铁里寻找出口的谁?\par
		不是想念,想念该是芳香的。

		众多的谁,你们迎面走来,和地铁中的许多人一样,同我擦肩。我却终于将忘却,将离开,那么多
	辨认不清的面孔。继续走自己的路,甜蜜地幸福,或甜蜜地忧伤着。

		一路洒泪,一路歌吟。\par
		我飞向光明,飞向无数个粉红的梦境。\par
		我感觉幸福。我感谢上天的眷顾。\par
		我终于,看清那些无痛痒的经过。

		如果,青春是白纸,我愿意,印上血红的足印。像刚刚出生时的那样。用一种最鲜艳的方式,把美
	丽纪念。我要在最美丽的时刻,被你看见,被世界看见。

		所以,我这么珍惜,生命里的偶遇和意外。

		我愿意幸福。

		我只愿意幸福。

	\endwriting


	\writing{爱时}{2005年08月26日 ~ 21:42:32} %<<<2

		爱,大概不过是某个无意瞬间的闪念,却成日夜的牵挂纠缠。

		有爱,便是幸福,却又如何奢望一个未来。小鹿,让我来讲,关于爱的这些,给你听。

		我没有爱情。很久,没有爱人,甚至没有心动。而我,却懂得,爱时的滋味。

		爱时,是无休止的混乱,是思想的全然钝化。可以一整天,只想一件事,一句话。爱,让人变傻。
	只可以想着他,全部的言行和举止,细微的几处笑意。是朦胧不见的甜美,因着爱,和思念,生长得葱
	茏茂盛。蔓延成森林,成苍茫无际的汪洋。于是,你无力脱逃,只有迷路,或溺死在爱的幻想。

		爱时,是不分日夜的担忧,是对自己的苛责。有一百种怀疑,他大概不喜欢我,有一千种推测,他
	大概是忘记了我。这样的忧虑,在缝隙里,肆意流窜,扰得你心神不宁。就坠入无底的深渊,又好象一
	口井,如此地深,而寒冷绝望。你要日日地叩问,日日地望镜中的自己,她为什么不够美丽呢,为什么
	自己不可以完美,好有更坚定的资本,来赢得他的爱情,或只是,他更多一次的回眸。

		爱时,是简单可笑的满足,是天真纯澈的快乐。只是站在他身边,便是安心,便全世界的幸福。要
	看着他,一样满足地望你,一样的懂得。要把一同的时刻与分秒,碾成细沫,用清水冲服,要让他记得
	,要让自己记得。想将他的记忆,他的今生与来世全部霸占。因为爱,所以自私而任性。想安静地望他
	老去,甚至,想死于他的刀下。有一切的不理智与疯狂。

		爱时,我不愿说出爱的真相。

		爱时,我缄默住唇齿,望他的目光。小鹿,若我是如你般发热着爱上某人,我会是胆小的。我不敢
	言说,我小心翼翼。怕只小小的差错,就错失了,我的爱恋。


		于是,若我爱着,我选择沉默的守望。\par
		因为,或许,如你夜夜怅惘的,他不爱我。\par
		爱时,我将自己想象成,戴望舒笔下,他的恋人:\par
		“她是一个静娴的少女,\par
		她知道如何爱一个爱她的人,\par
		但是我永远不能对你说她的名字,\par
		因为她是一个羞涩的恋人。”\par
		我将是,羞涩的恋人。却期待你,说出我的名字。\par
		爱时,没有一件事,是未被涂上醉红的胭脂。\par
		小鹿,这些,是美丽的心情。\par
		唯一的遗憾只是,他不爱我……

	\endwriting


	\writing{融化}{2005年08月28日 ~ 09:18:59} %<<<2

		不喜欢自己的名字,嫌它太中性。而然说,田这个姓就很女性化,她说,甜。是啊,甜,我的名字
	是有滋味的。一种近于幸福的滋味。

		如糖的,又像蜜。

		于是,竟以为自己是像糖果的,竟以为自己是甜味的。期望起幸福的光芒来。而匆忙的假期,毫不
	迟疑地飞离,不等我琢磨回味,一个甜美的梦,就被自己惊醒。恍惚惚在凉起的夜里惊起呆坐,糖果的
	滋味,就被深黑的窗口无情地吞噬了,是欲哭无泪的惶恐。我的甜味,从舌尖滑去,成了惘然的空白。

		我是独自酿蜜的孩子,没有长大,没有学会,如何耽于幻想,而不去忧伤。

		我却是执意地,自以为是,在无人的花园横冲直闯。弄得一身荆棘,没有人发现,没有人疼惜。

		我真的像一块糖那样,一丝丝,在夏末的日光里融化掉了。

		是不见的不见,未失去的失去一般。没有缘起和理由地,我只可以,等待着秋天,和满校园的落叶
	。我会像去年,在透明的午后,从樱花园经过,捡几片银杏树页,夹在日记薄。会一个人站在来园的小
	池塘边,看一场秋雨的淋漓阴郁。我将穿上长衣,打着伞,默不作声地,把我的甜,抽离成绵延缠绕的
	雨丝,又好象,不能断绝的丝线。就不会被谁知道,被谁发觉,我绵密的心思。它们随着雨落,流去了
	,哭一场痛彻心扉。


		我要看装满秋的灿烂,却无法把天空洗刷得干净。我怏怏地,行走在这座并不陌生的城,心却是黄
	草蔓生一样的荒凉。我在这荒原中央,在没有月亮的几个晚上,用酿蜜的心,等待花朵的开放。我用许
	多个花开的瞬间,换取一种不离开。而花没有成全,夜晚没有成全。离开,是为了另一场奔赴。

		我已经融化掉了。不可以保存,不可以食用。

		因为是甜的,我变成痛苦。

		笑意是模糊的,我透过玻璃望你,却连轮廓也没有望见。在不可以预料的莫测里,假期死去了。我
	将回到学校,数着我的年月,迎接纷落如雪的日子。没有期待。

		我触摸希望,却发现它空无一物。

		因为是多情的,我拥有痛苦。

		阴天,今天疏落地写了这种,不开心的话。红色,一样可以忧伤。

	\endwriting


	\writing{零碎}{2005年08月29日 ~ 22:29:59} %<<<2

		思想凌乱,不可以不整理。

		\subpart{1,病}

		未长大的孩子,大概是难免恐惧和悲戚的。伴随的症状还有缺乏安全感,怕黑,爱哭,易幻想,易
	悲伤…… 等等。是不可治愈的顽疾,目前唯一临床证实有效的药剂,是光阴。每日服用清水,细数日历
	上的数字,可以催化药剂的作用,使患者增快苍老速率,快速有效地摆脱幼稚,单纯的境况,成为心志
	坚硬而明智的成人。

		但之后,苍老的心,往往又成为新的问题。

		才知道,不能够疯狂的心,也是病态的。

		虽然,未长大,这一刻上,令小女孩痛苦万分。


		\subpart{2,树}

		她要迅速地成长。像春天田野里的小树那样,鲜嫩蓬勃地长。不说话,不唱歌,只等着影子变短了
	,又被拖得很长。日子是良药,于是苦口,她这么想。她多想是树呢,真的树,长在某一处窗口,让窗
	子里住着她的爱人。于是,小女孩,可以偷偷爬到他窗上咯咯地笑,可以在有月亮的夜里,把自己的影
	子也画在他的窗帘上。那将是快乐而幸福的,因为,他永远都不会知道,那一棵小树的快乐。因为简单
	的喜欢,而成全了的幸福。感谢他,小女孩有了秘密。现在,她只想长大。


		\subpart{3,艾米莉}

		把《天使爱美丽》温习一遍,又傻兮兮地感动了一把。

		喜欢属于小女孩的可爱和狡黠,混合着那么一些忧伤和孤独。她就那么长大了,独自居住着,管理
	着自己的生活种种和寂寞。爱和善良却是光芒万丈的。在遥远的法兰西,那些浪漫,那些热烈,像玫瑰
	花似的无羁绽放,灿烂到迷醉。光影游移,日子平常,起风的下午,玻璃杯在白色台布上舞蹈,没有人
	发现。属于生活的角落和细碎,枯燥却又鲜美。是否每个小女孩,都像艾米莉这样长大呢,是否每个小
	女孩都可以奇遇一个深爱的陌生人呢。因为陌生,所以我爱。像Faye的哼唱,只爱陌生人,我爱得比宠
	物还天真,比脸色还单纯。

		似乎叫艾米莉这个音的女孩或女子,我都特殊地喜爱。

		比如,电影里的这个大眼睛姑娘,比如,抱着黑猫头帘齐整的卡通,还有拥有曼妙歌声的 Emilie
	simon,以及,美国那个隐居起来的传奇女诗人。

		许多的艾米莉,一样的纯粹。


		\subpart{4,废话}

		我是中了毒,没有日夜地想写字。生产废话。并不能连缀成篇的,只是东一句,西一句的。一个个
	冒出来,时而欢乐,时而又没理由地失望。据说,写字的人,多是灵魂孤独的。大概没有错。我的灵魂
	始终孤独着。我却很好奇,谁是不孤独的呢。你可以在人山人海中确切地找寻到自己的岸吗,你确信你
	的确找到了吗。

		找到了,就不会孤独吗。

		为什么,在我更深切的地方,永远是一种无可填补的不可知呢。从没有人可以终极到达,所以我只
	能够书写。写给自己。让自己懂得,就可以安心。这是写字的一种解释,也是生产废话的借口。

		所幸的是,现在还没有对废话生产者收取环境污染费。高兴。


		\subpart{5,高空}

		看到一句话:云霄飞车,降落时,我体会到无尽的失落。

		我知道,那一种失重的滋味。许多的游乐器特别设计的滋味。用你的体重,配合引力,来创造出刺
	激。而刺激,对于各人的感觉和意义又是不尽相同的。从高空下落,却是迷人的吧,所以才会有了云霄
	飞车,有了蹦极,也有了众多用跳楼来结束生命的人。

		小朱家楼上23层的一个女孩在清晨跳下了。20岁,只有20岁,花一样的年纪。一个完整而美妙的生
	命就这样被摔碎了,被自己的重量,或者,这个世界的重量。

		因为天空是完满的,所以容不下人的肮脏。活着的时候,我们只可以下落,而无法飞升。人是属于
	地的,只有魂是属于天的。

		高空,始终对于人有着不可替代的诱惑,诱惑你,向一种下落,一钟穷尽奔赴。

		能活着的时候,最好是乖乖在地面呆着……


		\subpart{6,日常}

		居家的假期,天气湿热,于是用洗澡作为消遣。一天三遍的洗,也不觉得麻烦。只是想和水在一起
	,只是想被水包围。因为温度,因为纯净,因为温柔。日常的生活,由于重复的行为,而显得可笑而乏
	味。却甘心做了乏味的人。本来,又有什么滋味呢,不过如是的吧。像清水,无滋味,又是全然的滋味
	。

		在等待水烧热的时候,用手试探着水温,时间久了,明明是凉水也竟有了温暖的幻觉。是因为习惯
	,产生了错误的判断。人是容易迷惑,容易上当的。我们也许可以避免被他人欺骗,却难以预防被自己
	欺骗。

		而许多时候,却又甘愿被自己欺骗,编一个谎出来,反而开心。

		人是奇怪的。


		\subpart{7,藏}

		我想把自己藏起来,在你们都看不见的地方。于是一切的一切都不相关,我成为真空,也逃离所有
	人愉快或不愉快的记忆。好吗。我就不曾存在过。只是空气的幻影一样,轻而透明。我却要看见你们。
	看你们笑,看你们哭,看你学习,看你吃饭。我就会是幸福的,因为不存在而幸福莫名。

		我的存在,不关谁的喜怒,谁的哀乐。所以是轻松的,所以是失望的。

		把自己藏起来,有时是为了被寻找。

		真是用心良苦。


		\subpart{8,罗嗦}

		可以少说几句吗。

		我已经觉得麻烦了。我开始害怕麻烦。偶尔,也就连带着害怕起自己来。

		思想凌乱,不可以不整理。虽然,整理了依旧是乱。就好象,日子不书写就会丢失,虽然书写了也
	照样会丢失一样。都是没办法的事情,明知道徒劳,却不可不做的事情。

		与幻想和想念类似。

	\endwriting


	\writing{零碎。二}{2005年09月04日 ~ 16:36:12} %<<<2

	\subpart{1,问}

		莫在问:云彩上住着人吗?

		我在想:为什么只有四个季节呢?

		\blankrev
		在无端的世界里,让奇怪的问不被嘲笑,让奇怪的时刻,成为透明的水果糖,甜而芬芳。像耳朵边
	吹过去的初秋那样。由于一切的没有解释,人可以自由了,找一处楼顶仰卧,伸了双手,就以为可以触
	摸到天空。不需要怀疑,只耽于幻想。肆意地酣享着,属于我们的岁月,我们的狂妄,在指缝,在青春
	的田野上疯长。

		要问不可能被回答的问题。那是快乐的。

		因没有答案,而恒久的快乐。


	\subpart{2,想念}

		想念是有两端的,一端系在你这里,一端放在空气里。你这里的,你自己慢慢品尝,空气里的,在
	等待被爱人吸到肺里。

		如果没有想念,生命将空洞无依。

		想念,是甜美而痛苦的。我赖以生存。


	\subpart{3,虫子}

		她说不相信前世。

		我却宁可信了这一种可爱的谎言。于是,在一只小飞虫爬过肌肤,会相信,它是我前世的恋人。还
	好,我依旧认得,不然,不知又辜负几生几世的深情。笑自己痴,痴到把蚊子的叮咬想成前世负疚的吻
	来今生报偿。

		我把这些讲给她听,她却依旧不信,那些前世和今生。

		却是美妙的,不是么。小虫子爬过肌肤时,我能明白,那温柔。


	\subpart{4,后感}

		读书有读后感,看电影有观后感。

		许多的后感,在我们生命堆积,好象不能离弃的回忆。要在过后,才来感想。

		却是不真实了的滋味。我们被时光的长度欺骗,而那,原本是人的手指无法丈量的距离。一些在雾
	色里了的人,和过往,是近似于想象的。

		看过<哈特的战争>,大熊让我写观后感交给他…… 我却想问:可以写见到你之后的感想吗?

		那一定是非常有趣而可爱的。\verb|^_^|

		关于<哈特的战争>,大熊有过介绍的,可以点击查看,确实是好电影

		%http://spaces.msn.com/members/bestfarseer/Blog/cns!1pqwiks8pZ_17THxRPDkSkfw!116.entry

	\endwriting


	\writing{6日早上}{2005年09月09日 ~ 16:38:04} %<<<2

		你是透明的孩子,被九月的光芒照着。

		走过一个个树影散碎的下午,化在碧绿的期望。许多个路口,你回眸凝望,秋天以苍凉的姿态,抚
	起叶片,向你微笑。

		生活,似乎在另一个起始处展开。

		你说,人大概是靠希望活着。而此刻,你正感觉着,希望,一个个饱满浑圆的希望,在你身体中徐
	徐升着升着,那么轻盈。

		你于是成为透明的,好象,就可以飞在蓝空。

		我要听你不间断的歌声清越飞扬,我要知道你是幸福,是一日日甜美的时光。

		想在自己的湖放燃着灯火的小舟,亮在一种种黑暗和风雨,要让你望着,那丁点的微光,没有言语。

		让时间静成冬的天空,让日子纷飞着,吹起笑意,吹去叶落的悲伤无数。

		我是永远的这一种想望,望你的时光,在不经意的挥手间,镀了金光,成就只堪思量的年轮日月。

		我的生涯,和爱,在九月流窜。在你的,和我的世界。

		透明的你,不说话,只是微笑得甜美清新,有日光的香味。我想告诉你,一千个,一百个,我前世
	的故事。

		我们在睡着的夜里,相逢那些早已辨认不清的人事。好象一场滂沱过后,淋湿的旧照片,再回不去
	原先的模样。

		我已记不得你,我已记不得自己。\par
		却分明地知道,你的来去,是无声无息。\par
		我是透明的孩子,被九月的光芒照着,有些迷糊,有些甜蜜。\par
		6日早上,躲在宿舍,一个人对着自己胡言乱语……

	\endwriting


	\writing{一种女子}{2005年09月10日 ~ 20:43:09} %<<<2

		一种女子,立在风雨斑驳的桥头,粲然微笑,任时光荏苒,任芳菲开落,不曾减却分毫她的风姿。
	她是坚立的花树,是勇敢,是永远的天真洁白,只是伫立,便成风景。


		她是默默的,是不言语的深情。

		像是道旁不经意间的花朵,夜来香,只开在夜晚的芬芳,拒绝了日光的光鲜,选择了星月的寂然。

		不需要许多欣赏,不需要许多赞美,她只是自顾自地美丽着,她明白,真正懂得的人,即使在这深
	暗的夜,也会辨出她的模样。

		她的芬芳,是等待的无声呼喊,一种岑寂间的期盼,夜夜开成粉红,如女儿家的心思,绯红的羞涩
	里藏了热情。

		一种女子,守独自的窗子,看一轮轮的月起月落,想春秋与冬夏的雪月风花。

		她是水做的肌骨,是三分的童真,七分的善良。穿棉布衣裳,她喜欢这柔软,她说她的爱人,会有
	棉质的目光。


		她深信着爱,她读童话,却不相信美丽的结局。

		她等待着的,从不是结局,而是亘长绵延的故事。于是她对爱人说,爱我少一点,却爱我长久些。

		她喜欢听爱人儿时的那些,捣蛋和淘气,她喜欢回味,曾经凝定在爱人身上的那些光线,它们的轮
	廓,它们的色泽。

		她是温柔的人,有温柔的双手,和心灵。

		她把一个个相遇的日子涂抹淡彩,她总是感激地想,幸好这世上还有一个他,值得去悲伤,值得去
	惦念。


		一种女子,坚定着生活的快乐。

		品尝幸福,也酣享烦忧。万事万物,在天空流过,是全然的云淡风清。清明透彻的她,湖水一样,
	映一切美妙的影子。

		她品自己的茶,读自己的书,守自己的日月,自己的年华。

		她是美丽,却不是招摇,不是张扬,她静在时间的刻度上,懂得哪一些是长久,哪一些是暂时。于
	是,她去追求,她去放弃,她有玲珑的心,她有明慧的视线。她不迷惘,不叹息,她享受红尘间的聚散
	沉浮,却不沉溺,不被囚禁。一种女子,拿捏好种种轻重,种种分寸,好象看破所有,有好象从未去了
	解。

		是迷一样,雾似的女子,有苏丽珍的优雅,却多了从容,有张爱玲的才情,却多了清醒。


		她是不凡,却甘于一种平凡。她愿意隐在这华丽的世界背后,过乐意的日子,过偷乐的生活。

		她逃不脱小女子的小情怀,她撒娇耍赖,她要被宠爱,像每一个幸运的小女孩那样。

		她要用桃木的小梳子,轻轻为爱人整理头发,她要在他不知道的情况下,做几件古怪的坏事,给他
	捣几次小乱。她愿意看爱人欣欣的,或坏坏的笑。

		终究,她是个女子,她逃不出爱情的劫数,她大概也会伤心,也会爱错什么人,她却知道,那是青
	春的礼物,是惩罚。她不曾留恋,只是记得,一些退色的人,一些淡漠的过往。

		她爱上爱情,爱上爱情的一切可能与不可能。她深信爱,并从未怀疑。


		一种女子,有坚强勇敢的芯。如莲子的,她洁白,在深出生一丝淡苦的坚定,那苦味,却有隽永的
	清新,只等那懂得的人,来轻手剥离,一生呵护。

		一种女子,立在风雨,无所畏惧,却又是脆弱,易碎,如玻璃,如水晶。

		好象,圣经上说,女人是脆弱的器皿。

		一种女子,要你的懂得,要你的精心。一种女子,如玉的温润可人,可友更可妻。

		却不是所有人有这样的幸运。却也不是所有的女子,有这样的幸运。

		一种女子,在我浅浅的期盼中,迷离恍惚。

		我从不是,也不会是那一种女子。却愿意遥望,或只是幻想,她的存在,她的风情。

		而我,亦是乐于在夜晚开放的女子。于岑寂间,为一种芬芳的懂得。

	\endwriting


	\writing{凝}{2005年09月16日 ~ 18:54:39} %<<<2

		我说,这夏末与初秋值得被记忆得温柔。


		\blankrev
		在九月,一种种变幻无常的天空里,我凝定许多的时刻,把光线铺散,用甜美的笑意封存紧密。

		我要埋藏,所有不可知的幸福万千,像一只松鼠在冬天来临前保藏一枚松果的精心。

		我将把上天的恩赐,说给睡了的夜晚,让雨天里湿掉的月亮在我的梦境中开成层叠的繁花。

		我是不知满足的孩子,我有无法禁住的贪婪。我愿初见的美好,长久如日光的纯洁明亮。


		你会看见,目光的芬芳流溢,你会懂得,一季季的相逢和告别,不是种偶然。我是夙命的,我是愿
	意编织谎言,又深信的。

		于是,我说那些前世,于是,我眷恋于今生。

		我体会手指的温度,想肌肤之下,血脉的奔涌搏动,我记住气味,记住肩头的骨头,想我的,和你
	的肉身虚无。

		我们是借此身躯,才得以相遇的灵魂。我于是感激,于是要用牙齿,在你的肌肤留下痕迹。

		我信着,不可漠视,不可逃遁的一场奔赴。


		\blankrev
		只是并肩坐着吧,看我们的下午。

		看风,不着丝毫力气,穿越空寂的操场,看闲散了的深情,在我发丝间茂盛。

		当世界被吹得冰凉了,你会依旧坐在这里吗。陪在谁的身边,在纯白静好的雪天,吃一支甜甜的冰
	棒。我会坐着,会于班驳迷离的落叶之后,收藏此刻的细枝末节,又打磨一个个纯白的期待。

		我是如此,无可救药地迷惑了,不分真假,不辨方向,义无返顾地沉溺了,于真实,或者,于假想
	。

		这样的下午却是松松地疏放了,像童年的某天,穿上花裙子站在镜前一样,我在大大的镜面前,天
	真率性地笑起来。

		很多时候,那个小女孩,她不肯离开。是那些幸福的时刻。


		哥哥说,你不要做卖火柴的小姑娘。

		他是不要我,去许一个个空洞的愿望,不要我,在风雪交加里,因虚妄的幻象痴迷和快乐。

		我不去卖火柴,却宁可做一支火柴,只是燃烧。

		我愿意,用微小的热量,成全一个个愿望的实现,我愿意,焚烧了自己,来换取一场绚烂的沉迷。

		我只是愿意认真爱恋的孩子,只是缺乏了智慧和勇气的孩子。

		我相信了爱,和美。于是,不能够轻忽,不能够随意。我那么坚定,在无力把握,和辨认的世界。
	我任性如此了,不奢望遥不可及的彼岸,只珍爱,此刻此岸。


		而我,却终究是无法感觉安全。

		如果,要责怪,只是我那一地碎掉就收拾不起的恐惧。

		我是胆小,是易碎的玻璃。若我,仍不肯轻信,若我,终于是猜疑,那是过分的期望凝在缺失了目
	光的回身,是太重的珍爱,把心压得沉沉。

		我在凉丝丝的九月,穿过校园。我拥有夜晚,我拥有雨天,拥有爱人的温度和气味,却无法安全。
	我像固执的小孩,等一种坚定,等一种确定的不消失,不离开。

		是全部无解的奢望。


		\blankrev
		我把九月的每天放在洁净的瓶子。我把甜美的心包裹在缤纷的糖纸。

		我凝在时刻上的表情,你从未看清。我不明白快乐,不明白悲伤。轻或重的空气,被我吸在肺里,
	被我温柔地记忆了。

		那些,关于你,关于我,关于潦草着却纯白的年轻。

		你凝在时刻上的笑,被我读成,亘长的幸福。

		我们的,夏末和初秋。

		你是真的吗,你快乐吗。

	\endwriting


	\writing{花样}{2005年09月30日 ~ 22:49:11} %<<<2

		开门走进的母亲,怀抱一捧花朵。我欣喜万分地迎上去,是玫瑰,未及拆封,骨朵上齐整整套着白
	色小网的玫瑰。

		母亲说,是新婚旅行的同事,今早一路飞行,从云南带回的。


		于是,一个凉意初起的下午,我洗净墙角的玻璃花瓶,蹑手蹑脚地把一只只白色的小网剥离,将由
	束缚下解放的玫瑰,精心插起。是怀着动容,又混合了悲怆。为了几枝未经受住飞行的花朵,为了她们
	毫不留情地萎去,轻轻动了心思。

		想那些,关于鲜花,和女子,想那些,飘零的故事,和葬花的魂灵。

		花,总是同女子关联的,尤其,是那些不经风霜的弱女子。

		大概,是缘于,相似的娇柔和稚嫩。才有了,桃树下灿烂明媚,即将出嫁的女孩被写入歌声,吟咏
	千年,才有了,妖娆可人的牡丹,修炼了精魂,化作冰清玉洁的女儿。

		我便难免,在与触摸它们纤纤花枝的时刻,暗暗动容,和悲怆。因着同为女子的情怀,和心思。

		在我指尖的簿簿肌肤之下,触碰的是花儿被花匠截断的血流,再如何鲜丽艳娇的它,也已是无根无
	着的身躯了,这是难过的事。

		我不能够长久拥有你,我只是贪图了,你一时的美貌。


		\blankrev
		曾有人作文,谈及插花与盆栽,又引申出呵护和爱恋一位女子的道理来。他说,只有精心于一株花
	草开落与春秋的人,方才是懂得珍爱女子的人。

		我爱极了珍爱这两个字。

		女子,是需珍爱的,因她们的纯澈,因她们的天真。

		你不必相信她们是天使这一类的鬼话,却要明白,她们的生涯,的确是如同花朵的,一样的娇柔和
	稚嫩。

		每一个女孩子,都是花朵,都有自己的季节。在春秋,在冬夏,在晨光熹微的晨早,或是,在月明
	星稀的幽夜。

		你需发现,你需耐心,于一株花草,呵护与灌溉,爱她的开放,更要爱她的寂寞,和岑寂。

		花期,花事,是不得久远与常新的。

		没有女子,会甘心做那被截断了血流,只待默声萎去,被随手遗弃的插花。她愿意,被凝视着凋萎
	和老去,在爱人如酒意微熏的目光。

		那是每个女子的期许,每一株花朵的盼望,那才是生命,活的生命。

		而这一刻上,我手中,触摸的,却是一枝枝艳美的尸体,我把它们插入清水净瓶,倒像似了,一场
	肃穆安静的葬礼。

		所有的哭声,都隐在女子们绵亘了千年的深情,和不安里。

		\blankrev
		昨天,是一个雨天。

		良说,你看,雨是斜的。是的,那斜风细雨,淡淡薄薄地把这天饱含了水分的空气,涂抹成浓白的
	清切。

		我们踏着水花向前走去了,地面上渐渐多出许多的小小池塘,积下雨天的透明,成一汪汪明镜,映
	着灰蓝色,天空表情莫测的容颜,还有,那道旁一树明黄的秋树。

		良说,那些树会在更冷些的时候,变得火红。

		我却知道,这是多么美而失真的谎言。那并不是,会生出红叶的树种。却情愿地相信了,并忆起,
	那首叫“西风的话”的歌来,我似乎,总是在这样的月份想起,又一次次重新哼唱起,重新写下,那句,
	花少不愁没颜色,我把树叶都染红。

		西风是多情,而温柔的家伙,有薄荷糖一样清凉的幻想。

		是淡味的甜蜜,是我的心爱。

		我们在各自花样的华年里,一起经过雨天,和那些美丽绚烂的树,是值得俯首感激的幸福。


		我大概,也曾走过野芳满目的原野,也曾笑意甜美地接受一束花的馈赠。那些,是可爱的自己,和
	时刻。

		我却依旧喜欢,没有芬芳摇曳的雨天,喜欢在润湿的空气里,为自己买一捧龙胆花,怀抱着走回家
	去。

		昨天,良消失在地铁站的人山人海,我独自向回走,我终于是,没有勇气在别离的当口,给我的爱
	人一个拥抱,只能够,把他唇上的余温顶在自己的额头,然后转身。

		我也终于,没能在雨天,怀抱一捧龙胆花,走回家去。路上经过的花店,都不见它的踪影。

		爱上忧伤的你,我默读花语,在这个斜风细雨的日子。

		我想着,去寻一株龙胆花,悉心栽培和呵护,让忧伤的龙胆花,在纤弱的花枝下,生长出勇敢而光
	明的快乐来。

		是我的妄想与奢求吗,我却要浇灌,要等待,用一场场的仔细与认真。

		\blankrev

		在斜卧着发呆的空隙里,我梦见许多温柔,生在一双大手掌之上,来把一株淡紫的花朵捧起。花朵
	,慈心爱娇,纯澈天真,胆小的样子。

		我想着,来临的这并不悠长,却足以用来发呆和荒度的假期,一个人,躲在厚棉被里,笑得甜蜜,
	淡味的甜蜜,我的心爱,不会腻。


		母亲说,记得给花换水。

		我怎么可能忘记呢。那苍凉而美丽的花朵呀。

		很旧的上海,有女子在留声机里唱起,花样的年华,月样的精神。


		你们要尽情地美,莫要顾惜。

	\endwriting


	\writing{散漫}{2005年10月05日 ~ 17:26:10} %<<<2

		十月的日光,神情散漫,如目光暧昧的女郎。我们被温暖的光明包围着,被宠爱得懒洋洋。

		这样的十月,令人迷醉,就甘心沉溺,在我们还没认清风向的时刻里。

		小鹿在回程的公车上浅浅睡了过去,耳朵里,还塞着Faye的靡靡之音,“不爱我的我不爱,不要我
	的我不要”,是一贯不着气力的瑰丽声线。那么,我想,我们都是深爱了秋的孩子,因为,它爱我们。

		我拍下小鹿被夕阳照着的侧脸,她幸福而满足的模样,我拍下她斜挎在身上的龙猫,觉得日子是富
	足盈实的。

		虽然,我们总是习惯于飞与漂浮的,却终于,只能够被平凡的快乐而填充。我们是被爱的,这有多
	么好。


		若干小时前,我们在通明的街上踏过去,又钻进一条条错乱的胡同里。

		老屋排列两侧,屋顶生了丛丛黄草,被瓦蓝的天映衬着,一派明媚地随风而倒。

		我们遇见青灰的,被孩子们图画了的砖墙,遇见被主人送到门外,沐浴着阳光的大棉被,遇见因孤
	单而胆小的小黑狗,它的影子很瘦很长。

		有半掩房门的院落,有坐在墙根晒太阳的老太太,有另一条胡同里母亲教训孩子的大嗓门。

		我们信步向前,心情松散。也会蹑着手脚,溜进人家的院子,拍下伏在屋檐底下无忧生长的紫色豆
	角。我们好像走在别人生活的边缘,在一个晴好的天里,抚摩了他们世俗却晴好的生活。


		我们拜访鲁迅。看他在北京最后的居所被粉刷修葺得光鲜。

		是一座齐整的小院落,书房后边有袖珍的花园。

		北房门前,日影穿过主人亲栽的白海棠,铺散一地。我仰头望了许久,树是瘦小并不壮美的,大概
	却也如鲁迅,是倔强而坚实。叶片,在响晴里,都幻化成翡翠一般的璀璨。

		我们绕到后院,看他小小的花园,中央寂落地躺着一眼井,解说员讲,那是苦水井,不能够饮用,
	只用来浇花的。我突然觉得它的可怜,被白色的木栅栏包围着,这不知是谁的主意了。

		扒在书房的后窗向里看,书桌和藤椅如旧,却是动荡荡了的一个世界,猛士离席,再没有人会在秋
	夜里,点一支烟,写下热血。

		鲁迅会不会料想,多年后,会有如我的小女孩,站在他家的石阶上,向他的书房望一望呢?若他知
	道,也应快乐的罢。


		有人在参观者留言薄上写下,鲁迅同志,我没有看错你。这一句虽不够认真,却总比解说员喊的,
	要学习鲁迅的什么什么的那一大通空洞的口号和废话,要真实而可爱得许多。


		令人愉快的是,我们在宋庆龄家里,找到了两架秋千。

		这后海北沿的家,原是王府,再向上溯去,大学士明珠竟也住过。明珠,并不值得兴奋,令我兴奋
	的,是他的儿子,纳兰性德。

		静好的庭院,有王府的气派,几处山石上建着玲珑的小亭。有人在亭里仰面睡了,而那,的确是午
	睡的好地方,心里艳羡了一阵。我们难得的悠闲下午,就在曲折的回廊里消磨吧,为了时光的无情,或
	者温柔。

		我说,我要飞起来,把秋千悠得很高。我是打秋千的高手,我对小鹿笑谈。她说,她会觉得晕。我
	以为自己真的可以飞到天上了,当双脚离了地面的支持,我们就有了飞翔的勇气。

		我喜欢,在飞着,或者漂浮,虽然,是那么危险无依的事情。


		后海边,有在拍婚纱照的男女。新娘一例是雪白的纱裙,模样不见得漂亮,却是幸福得妩媚。

		看他们相依着,微笑甜美,被摄影师拍下。

		柳丝情意缱绻,水波温情,一个明亮亮的午后,有那么多快乐的人,快乐的面孔。

		后海,是充溢着爱情的地方。

		有拉着手骑车而过的情侣,我掏出相机,追在他们身后,却终于是没能捕捉到,那令我动容的片刻
	,在快门按下的一瞬,那两只手竟松开了。

		有多少时刻,是如此的呢,让我热烈中奋不顾身地去追寻,却冷冷地接受一场空欢喜,和遗憾。

		奔跑中停下的我,呼呼喘气,回身看早已落在后边的小鹿,她笑出声来:我发现,你跑得挺快。

		是的,我会是奋不顾身的孩子。变成不可思议的自己。


		这个散漫的十月,我们拥有了爱,和阳光。就不该再生贪念。

		回程的公车上,Faye懒懒地唱,“爱你的微笑,爱到担当不起。”一个简单的日子,又从我的头顶调
	皮地逃跑了。我靠在椅背上,想着我们的生活,它也是晴好的。


		因为许多许多的幸福。

		我喜欢,被宠爱着。

	\endwriting


	\writing{静静的生活}{2005年10月07日 ~ 15:02:37} %<<<2

		\indentenv{4\ccwd}{4\ccwd}{\centering}
		旋转以后静静生活 ~ 垫起脚尖静静生活 \\
		秋天以后静静生活 ~ 电影散场静静生活 \\
		路灯点亮静静生活 ~ 雷雨过后静静生活

		不过问谁 \\
		谁打破周围的沉默 \\
		谁决定话题的轮廓 \\
		谁穿越空旷的沙漠 \\
		带走 ~ 火红的日落

		搬到海边静静生活 ~ 浮木漂流静静生活 \\
		啤酒泡沫静静生活 ~ 火箭升空静静生活 \\
		地球旋转静静生活 \\
		不让全世界的习惯 \\
		陪我放松

		谁打破周围的沉默 \\
		谁决定话题的节奏 \\
		谁穿越空旷的沙漠 \\
		带走 ~ 灌溉的花朵 ~ 迷路的骆驼 \\
		oh ~ oh

		表演过后短暂问候 \\
		搬到海边浮木漂流 \\
		潮起潮落啤酒泡沫

		带走 ~ 灌溉的花朵 ~ 迷路的骆驼 \\
		时间的沙漏

		火箭升空静静生活 \\
		地球旋转静静生活 \\
		不让全世界的寂寞 \\
		陪我坠落
		\endindentenv

		陈绮贞的新作,第一耳朵,就爱上了。

		静静的生活,在秋天之后。

		是落下空白的夏天,是忘却遗失的火热,我们去海边生活,喝一杯素味的柠檬水,数一整天无力的
	云朵。

		看浮木漂流,好象你我,在这迷茫无际的海雾之中。


		我只想要,沐浴后的清香味道,只想要泡沫的华美和脆弱,我在一场场登台与谢幕之间,选择了静
	静的生活。

		像那飞离的早晨,在浓白色中消散,没有痕迹,没有对错。我是不需要答案的,正如,人是不需要
	岸,只需要遥望,和渡去一般。


		于是,等待着,秋天过后。

		于是,仰头在拥挤的街道,于俗世人群看火箭升空,想地球旋转。就明白,没有神仙的救赎。

		让我编织雪吧,为了,在十二月,抚摩你门前的石阶,为了,在急迫的消融里,让你陪我,看一场
	火红的日落。静静的生活,我喝一杯啤酒,我唱一首你不了解的歌,我是皈依于无声的孩子,虽然调皮
	,却虔诚万分。


		\blankrev
		我喜欢这样,引我遐思的音乐。

		而十月,我们的生活,像杯开水,被西风吹凉了……

		是该甘于静静的时刻。

	\endwriting


	\writing{孩子。故事}{2005年10月15日 ~ 18:28:16} %<<<2

		无知的孩子,在大树上刻下名字,于是大树承担了他成长的疼痛。

		那一处刻刀留下的残忍,随了光阴的婆娑,飞去很多年无声息的变迁。

		大树,总在风的经过里,无表情地唱起幸福或者荒凉。

		它就坚定着站立在那,看孩子换下衬衫,穿了棉衣,看孩子丢了布娃娃,戴了红丝巾,它心里明白
	一切,却不出声音。

		大树等着什么,又凭借什么而活着,活得一派葱茏,一派繁茂?

		孩子不会去想它的心思,她只会把皮筋饶在大树的身上,用另一只手牵引着,独自在落叶纷纷的秋
	里跳皮筋。好象一场绚美的舞蹈,孩子跳得精心,虽然,幼小的她只有那么孤独地玩耍。

		大树不去言语,大树却体会着快乐。它喜欢陪伴着她,看她淡粉红的日子,从冬到夏。

		大树欣然接受了它给的伤痕,好几次,它偷偷俯了身子,看那稚拙可爱的笔迹,于是想起夏天,孩
	子搬了板凳躲在它的树阴里一笔一划地写字。

		她每个清早,背了书包蹦跳着离开小巷,大树用叶片挥手,送去一天的期待和祝福。孩子却从不知
	道它的心思,也曾不去想。

		那疼痛的记号,是默许的信赖,大树这么相信了。

		但是,你要长大了吗,孩子。

		它看见人们把大大小小的行李帮运,它看见小巷在一夜间消失,孩子站在被推土机推倒的房子前,
	穿着最好看的红裙子。天下了雨,灰的云彩坠着坠着,想就要殷湿的一团团水墨。


		大树不明白,大树很悲伤。

		孩子在瓦砾堆里哭,孩子用小小的手摸索着寻找,她说,她要她那封皮上画了小兔子的日记本。

		孩子的妈妈带走了她,答应再给她买一个更漂亮的。他们帮去了新房子,新的家,只有大树没有离
	开,心里装满了疑惑。

		你真的要离开吗?放弃了我,也放弃了那本记满天真的日记吗。

		许多的许多,就这么被遗落,和丢失,在一些无知的年岁和时刻。


		大树不明白,大树很悲伤。

		它自己站着,依旧一言不发,而凉丝丝的雨水落了整个晚上。

		直到天明的时候,还是阴着天。


		无知的孩子,在漂亮的新房子里,拥有了漂亮的新日记本。

		她的日子是淡粉红的,似乎什么都没有改变。

		孩子依旧读她的童话,写她的日记,唱学校里新教的歌。只是,没有了大树,没有了她从未发觉的
	伙伴。


		孩子更加地孤独了,她被锁在高楼的某个编号的房门后边。

		妈妈说,不要给陌生人开门,也不要出声。爸爸说,这世上有许多坏人。她知道的,这就好象森林
	里会有大灰狼,想进小猪小兔的家。

		她于是乖乖地孤独着,偶尔托着下巴,在小小的窗口呆望。

		孩子觉得自己好象被锁在高塔上的公主。只是,在童话的世界外,没有王子,也没有王子的白马。


		孩子,你就这样长大吗。

		封皮上画了小兔子的日记本在废墟里安静躺着,它身边长了黄草,生了一朵淡粉红的小野花。它的
	身体里装满了孩子絮絮的小秘密,它有那么多的故事,那么多的高兴,和不高兴。

		日记本懂得孩子的一切,却似乎永远都不会再对谁提起。

		直到冬天的一场雪,覆盖了它,又覆盖了大树,直到雪花们告诉了日记本,大树的悲伤。几只小麻
	雀帮了忙,带日记本起飞,又安放在大树的肩膀。

		大树很惊讶,啊,怎么是你,你一直躲去了哪里。

		日记本微微地笑了,我们是一起被遗忘在了原地……

		大树和日记本,相互依偎着,在洁白的这个冬天。

		听路过的人说,春来的时候,这里就会开工,盖起漂亮的高楼。


		它们很悲伤,一同悲伤,看雪花翩翩的舞蹈也觉凄婉荒凉。

		它们不期待什么高楼,它们各自体会着冰凉凉的想念。


		来,让我把孩子的故事将给你听,日记本轻轻说话。

		它讲孩子第一次在它身上写字,那时她才只会几十个字罢了。它讲她在生日那天得到一只八音盒的
	快乐,那是她一直梦想着的礼物。它讲她摔坏了手表的难过,虽然她还不太懂得时间的概念。

		日记本神采飞扬地说着,它回忆起孩子快乐的脚步,和她铅笔尖轻轻的有节奏的抚摩。

		孩子不会知道,每一次闭合本子时,日记的不舍。

		日记却忽然要流眼泪,事实上,它还没有被写满。

		大树告诉日记本,孩子曾经回来找它,它不是被遗弃的,只是遗忘。而大树自己,才是真切地被遗
	弃在了这里,和所有曾经排列整齐的砖头一起。一些砖头被附近其他的人家捡走盖起新的房子,获得了
	新的生命。

		可大树,只有这么突兀地站立,貌似坚强。

		谁也不会真正明白,大树的悲伤。


		北风经过,它是西风的弟弟。嘿,一向还好吗,唱一支歌来给我听吧。大树低垂着脑袋,摇摇头。
	它不想,在冬天的残酷里,哭诉什么。只是用枝条摸摸肩头的日记本,它们一并地在时间里沉没。

		大树的身上,那孩子刻上的名字,陪着它生长。它心甘情愿地用痛苦陪孩子一同承担。

		那些无可奈何的变迁,无可奈何的长大。孩子,也会慢慢明白吗,孩子,也会想念起吗,大树,和
	她封皮上画了小兔子的日记本。


		她渐渐知道,她不是公主,她只是如公主一样地被锁起来,只是如公主一般地孤独。

		她还记得吧,原来的夏天,搬了板凳躲在大树下,原来的夜晚,亮起小灯,听见门外的大树哼着夜
	曲。

		有时,孩子就会掉眼泪,而她竟然也已经长大了,眼泪好象神奇的魔药。孩子留起了长头发,她已
	经是真正的女孩子了。是5年,还是更久的时间呢,风和雨水把记忆冲刷得腿色又变形。

		收拾旧衣服时翻出口袋里的一只塑料手镯,她的妈妈笑了,是你小时候的呢,怎么一眨眼,就这么
	大了。

		女孩子把那件粉红的小上衣摸了又摸把小小的手镯套在手上,却如何也带不进去了。

		她呆呆地停住,就在这窗台上落了灰尘,世界落了秋叶的下午,她忽然记得,忽然重重地想念了。

		她穿越大半个城市,寻找大树,和她的日记本。

		这一次她没有哭,却依旧是穿了条漂亮的红裙子。天色是灰,像没有晴朗过。她想抱着她的大树,
	想靠在它粗糙不平的身上。

		而原地,已经面目全非,没有树,连花草也没有。

		唯有绚目的玻璃大楼霸道地伫立。


		大树,你去了哪,日记本,你去了哪。

		那些留在你们身体的文字,一并如蒸汽一般挥散殆尽吗。那刻下的名字呢,那曾默许的信赖呢。你
	们,统统去了哪。


		她不明白,她很悲伤。

		此刻,她更加孤独,一个人蜷缩在棉被,看窗子里透露了冬天冷冰冰的消息。

		她把塑料的小手镯放在枕边,而再旁边,躺着她的丈夫。

		她没有嫁给王子,没有看见白马,却是无可挽回地长大,又日日老去了。

		偶尔,她想哭,她用手指一寸寸抚摩丈夫的脊背,她靠在他的肩上充满无辜。她的丈夫,却只是淡
	淡地说,已经不是小女孩了,别这么没出息的样子。

		已经不是小女孩了,她成为妻子和母亲。

		她给她的孩子买日记本,封皮上画有小兔子的,她带她到公园里玩,告诉孩子妈妈和大树的故事和
	秘密。

		孩子不会懂得妈妈的心思,也不会去想树们的快乐和悲伤。

		孩子不能够明白她,她自说自话着,在一棵并没有刻上过名字的大树下。


		好多个不辨真假的夜晚,她分明听见大树对她说,我很想念,你从不是孤独的。

		大树终于对她说起话来,终于让她明白。


		清早,她把塑料的小手镯套在孩子的腕上,美丽非常。像许多年前一样。

	\endwriting


	\writing{谁或谁的胡言}{2005年10月16日 ~ 17:55:49} %<<<2

		头脑必须清醒 \par
		心地必须纯洁 \par
		肉体必须干净

		契诃夫说的,是许多人诚心向往着,却又难以达到的。\par
		只是头脑的清醒,便已难得。

		我大概便从未是清醒的。多数,是混沌中迷糊向前,一路洒泪,一路歌吟。我走过陌生人的门前,
	踏过不相识的台阶,却又在短暂的交会中成就了熟识与亲切。我没有太多的朋友,却拥有,最善良贴心
	的伙伴。

		那些一同经过的人,让我在岁月小小的天井里,默默看你们的光亮在暗夜的天空,让我从容地溯去
	,同你们逆行,来把一一的开放和消亡都变得光润迷人。

		我是不断地睡吗,在现实与想象那狭小的缝隙,我大口喘气,呼吸迷药一样的气息。我于是,是这
	样地无法清醒,在无穷止的昼夜,思想与思念,许多许多的无济于事。

		只让我心爱的人们,懂得我的心思纯净,只让你们看见,我多么地在意,我们的感情。

		我体会得深刻,虽然,我明白眼前一切的真实可触,原是不曾到来的虚幻。

		好象,远得更远的云,便只是你的想象,好象,我从未降临的生活,却分明在19年的长度中深深
	爱恋又恍惚度去了。

		我站在被吹凉的空气,感觉又一个秋天。\par
		你穿起那件黑色的毛衣了吗,你明白了候鸟眼中的离情别意了吗。

		让我坐在自己的位置,手握一杯滚热的开水,来发觉窗外的树木在日光里的深情。\par
		而我们,惦念着些什么呢,竟在无人的许多时间,嗫嚅出一句句短诗歌。谁也不会读起的诗歌,最终只会一字字死去的文字。\par
		就明白了,或许,我们什么也不拥有,只把握了这一点刻上,自身的悲哀或者欣喜。\par
		谁或者谁,在多远的地方,如你般坐着,也喝着一杯开水,也想着无关紧要的人和季节呢。\par
		是你吗,或者是他。\par
		在秋天里,不可得的是清醒,长久的是沉溺于西风的萧瑟。

		而我,是深怀了感激与爱的。
		我对岁月俯首,对星辰致意。\par
		那个寒冷的夜晚,和良站在空旷的操场,星是散碎,而光亮是清澈无染的。

		我想到宇宙的无限,想到不可挽回,不可琢磨的长度与空间。\par
		那黑漆的世界,仿佛要吞噬所有,却又好象给予了一切。\par
		我不能够懂得,关于无限的事情,我只是痴情万分地享用我卑微渺小的存在。人由于渺小,才需要伙伴,那些可以在寒冬里挤在一团取暖的人。\par
		而相隔的山岳与回忆,把太多的美好,轻易地隔绝了。\par
		原来,我只可以轻轻地把你们惦念。


		\blankrev
		朱,我在这里喊你的名字,我想着你的幸福和甜蜜。我在默默祝福,又默默思念,那些我们快乐的
	过往。田,从来不是个体贴的朋友,却把你,放在心中最体贴的一处。

		你还会记得吗,15岁那年的你我,记得夏天的瓦蓝天空,记得穿着白衬衫的女孩子,笑意天真。

		朱,你生日那天说,我怕你忘记了给我画生日卡片。。。你说,你把每一年我送你的卡片都精心珍
	藏。我又怎么会忘记呢,我每一年,最简单的心意。

		我是幸运的孩子,在年少的无知里,认识又相知了一样无知的你。

		当你就要飞离这座城市,我只有微笑,因为,你是向更多的幸福飞去。朱,要守你的天真,你的纯
	净,那是上天的礼物。


		\blankrev
		八月里,小鹿在火车上发短信给我:我看到,北方的平原上,开满明媚的太阳花。于是,我一个人
	躺在夏的枕席上走失。

		在驰去南方的列车上,小鹿经历着一场繁华迷醉的盛宴。我想象花朵,想象平原上色彩烂漫到无际。

		静,在你求学的路途上,是否也曾经历着这般的梦幻呢?静,那么善良的女孩子,你还会做淡紫色
	的梦吗。

		高中的你,对我讲起你穿了白纱裙,漫步在薰衣草田的午睡。我是羡慕非常的,也想,只有你会有
	飞鸟一样轻捷的思绪吧。

		你还好吗,在长江的那一岸上,在拥有美丽传说的城市。我只有等待冬来,让北方的雪,载我的叮
	咛,飞去南方,落成一阵雾样的水花。


		\blankrev
		要让你们懂得,我的珍爱,我的小心。

		我手捧洁白的花朵,来把所有的际遇迎接。我爱,你要明白温柔,明白许多不言语的甜美。虽然,
	我永远是混沌中模糊着向前,唱着哼着不清晰的诗歌,又总会簌簌掉下泪来。

		我却是坚定的,把我所有的爱恋,投射在不曾来过的生活。我在想象里,我在你的,我的,我们的
	想象里,沉迷了。好象赤脚走在青草丛里的孩子,让踝骨沾上晶莹而冰凉的露水,于是可以勇敢地走进
	现实的无奈。我是小孩子,我要无条件的宠爱,是的,我愿意幸福,我只愿意幸福。

		我似乎在和自己说话,却又仿若有另外的自己躲在另外的空间,狡黠地窥探我的心灵。是你吗,是
	真实的你吗。依旧是不得清醒,恍惚里,我看见秋天,看见希望,和温暖。我对自己说:

		\blankrev
		头脑必须清醒 \par
		心地必须纯洁 \par
		肉体必须干净

		这是我最美丽的期许和祝愿。

	\endwriting


	\writing{回答}{2005年10月28日 ~ 23:07:00} %<<<2

		我被小羽兄点名了……

		游戏规则被点到名的在自己的blog上写下答案,并出一个题目,然后把题目丢给另外五个人。\par
		并且到这些人的留言版上留下:"你被点名了"\par
		这五个人在自己的blog注明是从哪一个blogger那里传来的题目。\par
		然后写下答案,并另加一个问题,再去贴另外五个人。\par
		你自己回答50个题目,你回答完了再加一个。\par
		被你点名的博友就要回答51个题目,如此继续下去。


		提问1:2005年,你的野心是什么!『出题人:狐狸』\par
		答:心底融化着甜蜜

		\blankrev
		提问2:为以下物品撰写一句话。此物品为二锅头。『出题人:葵』\par
		答:绿了瓶瓶,红了星星(红星牌二锅头)

		\blankrev
		提问3:叙想象中的最囧的一次恋爱经历,限原创;字数250字以上。『出题人:栗子』\par
		答:喜欢幼儿园时代的男孩,原因是他每天最早从午睡中醒来,并会躺在床上唱歌,解除了我整个
			中午的漫长寂寞(那时,我是从不睡午觉的孩子……)

		\blankrev
		提问4:一天早上起来,发现自己身边的人都变成蛤蟆似的只会跳,只会呱呱叫,你怎么办?『出
				题人:鬼丸』\par
		答:照镜子,看自己是不是也……

		\blankrev
		提问5:如果发现自己最近衰到极点,你会怎么办?『出题人:星星』\par
		答:写文字发泄

		\blankrev
		提问6:请形容一下你理想(妄想)中的结婚场景吧。。包括结婚对象。『出题人:泡泡璐』\par
		答:与我爱的人,应该是温柔儒雅的,在海上

		\blankrev
		提问7:如果你可以变成动漫/卡通里的角色,你想变成谁,说出原因。『出题人:猫猫HISA』\par
		答:小丸子。是真实天真的小孩子,还有可爱的爷爷。

		\blankrev
		提问8:初吻的地点,时间,对象。哈哈哈哈。。如果还没有,那希望跟谁?『出题人:叉』\par
		答:某地,某天,某人

		\blankrev
		提问9:最想到什麽地方定居,和谁一起去,以及原因。很简单的问题吧。\\『出题人:熊子』\par
		答:海洋中的小岛,要温暖,和我爱的人们。

		\blankrev
		提问10:觉得人生对自己最重要的是什么?『出题人:lulu』\par
		答:感觉

		\blankrev
		提问11:你一觉醒来,发现全世界的人都看不见你,也听不见你说话,你会怎么办?『出题人:樱
				桃猫猫』\par
		答:想必自己是死掉成为鬼魂了,高兴,因为证实了真的有灵魂不死

		\blankrev
		提问12:如果可以从机器猫(也奏是哆啦A梦)那里得到一样宝贝,你想得到什么?『出题人:小
				文』\par
		答:随意门

		\blankrev
		提问13:如果重新让你选择一次已经过完的这段人生,你会想从什么时候开始?换句话说,你对自
				己什么阶段最后悔,想重新来过?『出题人:野孩子』\par
		答:出生,却不是因为后悔,是因为喜欢孩童的纯净

		\blankrev
		提问14:你最后一次ml是什么时候?跟谁?『出题人:阿米』\par
		答:未来,不确定的某天,不确定的某人

		\blankrev
		提问15:你认为孙悟空和黑猫警长哪个更性感点?『出题人:假民工』\par
		答:黑猫警长

		\blankrev
		提问16:死的时候你给送我什么?『出题人:benbenz』\par
		答:一声珍重

		\blankrev
		提问17:人为什么要识字?是为了活得漂亮一点,还是为了忧患?『出题人:sep』\par
		答:为了享受文字之美

		\blankrev
		提问18:你非常丑,只有你最爱的人爱你;你非常美,除了你最爱的人人人都爱你。如何选择?『
				出题人:半缘君』\par
		答:前者(虽然美貌和爱情都是奢侈品)

		\blankrev
		提问19:你非常坚持个性,男/女友坚持你改变个性,如何权衡?『出题人:秋暮晨』\par
		答:我只是我

		\blankrev
		提问20:最计较的一件事情?『出题人:一抹深蓝』\par
		答:身体不好

		\blankrev
		提问21:你认为自己真正爱过吗?『出题人:Lisa文文』\par
		答:是

		\blankrev
		提问22:用一种花、一种饮料来形容自己。『出题人:KK』\par
		答:龙胆花,清水

		\blankrev
		提问23:坚持自我,真的那么重要吗?『出题人:Mirukii』\par
		答:是

		\blankrev
		提问24:如果明天世界就会毁灭,你会做什么?『出题人:可爱的33鬼娃娃』\par
		答:躺在爱人身边,等待末日

		\blankrev
		提问25:你会把心事藏多久『出题人:KYO』\par
		答:无限期

		\blankrev
		提问26:你觉得距离会让好朋友疏远吗?『出题者:maomao』\par
		答:不会

		\blankrev
		提问27:觉得自己什么地方最美?什么地方最丑啊? 『出题者:donny』\par
		答:最美的是眼睛,最丑的是什么?……

		\blankrev
		提问28:说说做过最怪的一次梦『出题者:sakura』\par
		答:打着雨伞,被风吹在天空

		\blankrev
		提问29:如果可以角色互换,最想成为现实中的谁『出题者:Cherry』\par
		答:我自己

		\blankrev
		提问30:你第一次彻夜不眠是为了什么?『出题者:Kei』\par
		答:喝了过多咖啡(因为想试用新的咖啡壶……)

		\blankrev
		提问31:明天早饭吃什么?『出题者:jessica』\par
		答:全麦面包,牛奶

		\blankrev
		提问32:都32个问题了,你觉得别人还有耐心继续下去??『出题者:养猫咪的考拉』\par
		答:应该是

		\blankrev
		提问33:我碰到那种号称是朋友,又开不起玩笑。你说她一句就火冒三丈,她不停说你也觉得理所
				当然。还碰到那种当面朋友,背地里阴险的小人~咋办叻?!总不能撕破脸吧?!『出题
				者:Iciclechilly』\par
		答:无视存在

		\blankrev
		提问34:万物归一,一归何处?『出题者:逝-Windsflied』\par
		答:心

		\blankrev
		提问35:这个变态游戏是谁发明的啊?如果你可以随意处置他…… 会发生什么? 『出题者:yoyo』\par
		答:没有报复心

		\blankrev
		提问36:如果可以让你删除一个季节,你会选择哪一个?『出题者:猿渡 翼』\par
		答:不想删,我是贪心的

		\blankrev
		提问37:如果下辈子变成一棵树,你想被种在哪里?『出题者:balance』\par
		答:爱人的窗前

		\blankrev
		提问38:最长时间不冲凉是多久\par
		答:不会用凉水洗澡

		\blankrev
		提问39:最受不了自己哪一点??『出题者:eiko』\par
		答:情绪化

		\blankrev
		提问40:什么事会让你最开心,什么事会让你最伤心?『出题者:pinkdolly』\par
		答:满足,不满足

		\blankrev
		提问41:你现在在哪里上学/上班?具体的学校班级/公司职位。『出题者:Meimei』\par
		答:北京,北语,汉语言文学专业,群众,学生

		\blankrev
		提问42:当你看到你被这个游戏的点到名的第一反应是什么?『出题者:katieyoyo』\par
		答:没反应

		\blankrev
		提问43:如果你可以当一天的上帝,你会做些什么?『出题者:冷冷』\par
		答:到处飞,给善良的孩子玩具和糖果

		\blankrev
		提问44:你最想回到哪个朝代?『出题者:荼蘼S』\par
		答:汉

		\blankrev
		提问45:回答完毕,真的好辛苦 我可不可以不问问题?『出题者:jojo』\par
		答:废话

		\blankrev
		提问46:怎样让自己忘掉过去?『出题者:徐徐』\par
		答:不可能办到的事情就不要苛求

		\blankrev
		提问47:你有没有遇到过让你可以什么都谦让的"姐妹/兄弟"?是谁『出题者:Athena』\par
		答:没有姐妹兄弟

		\blankrev
		提问48:如果有一个人对你很好,可是你并不是真正的喜欢他/她,你会因为感动而接受他/她吗?
				『出题者:瓶子』\par
		答:不会

		提问49:阅读以上48个问题,从中抽出一下几个字,组成你的答案写出来,谢谢!\par
		第一部分是第29题的答案;\par
		第二部分是第15题答案中的第一个字;\par
		第三部分是第16题问题中的第一个字;\par
		好了,看看你的答案是什么?\par
		答:我自己黑猫警长一声珍重

		\blankrev \blankrev
		下面是我的问题:\par
		提问50:请为幸福下定义?

		\blankrev
		被点名的是以下链接:

		\indentenv{4\ccwd}{0\ccwd}{}
		折翼天使的B儿B儿 \\
		莫的过去 \\
		晴天的伞 \\
		鸟倦飞而知还 \\
		小鹿
		\endindentenv

		请仔细回答啦,嘿嘿


		让我们把游戏继续

	\endwriting


	\writing{芜杂里的生长}{2005年10月29日 ~ 22:59:02} %<<<2

		十月二十三日,霜降。十一月七日,立冬。

		我喜欢,这样数着一个个节气,喜欢把它们的名称含在唇齿间,细细地咀嚼,滋味芬芳或冰凉。

		那一夜,听见风是白色,从屋角拨剌而去。身子缩成一团,躲在没有温度的棉被。

		我拥抱自己的身体,无声里睡了,只有风的低吟由远到近。

		这样的夜晚,我总会有梦,在不可控制的情绪中,不可控制地飞离了现实的生活,见了不曾出现,
	或早已消失的人。

		我于是在猝然惊醒的晨早,问着,谁闯入谁的梦境,谁告别谁的不舍。

		我梦见祖母,梦见一树灿烂的秋天,梦见你,也梦见他,梦见雨花里飞起的小伞,梦见陌生的自己
	对镜子而笑。

		我坐在醒了的现实,发怔许久,而窗外,是又一个光芒普照的黎明,只是,布了浓白的雾色。霜降
	,我默读。该是白色的日子,该是无须挽回地挥霍美丽和快乐。


		而谁,来解读我的梦境,那好象我身体中藏匿的另一扇房门。

		别让我,在那里见你的哭泣,见你们的曾经。


		我在日光里生活得完好,数着日月,安静度去。


		雾不散,我带了雨伞,穿过寂寥的操场,等一场凄清的秋雨。

		把耳朵贴在课桌,用手指轻叩桌面,听那声响,好象发自自己的心脏那样。


		我想着,这年里,一场艳夏的荒芜,想着,水光在桥洞下流变,那空无一物般的静。也想这肉身的
	虚弱,一次次的病痛,无能无力。


		就有愧疚的心,对于母亲,她不曾奢求我什么,只是要我的平安与康健,我却都不能够完满。


		我不能够完满,她那卑微的幸福,我对母亲说,不要责怪我吧,她就要哭出来。我竟是这样不会言
	语的孩子。

		躺在她身旁,我才知道,我和母亲原是一体,我正如她的肢体发丝,是没有区分的。

		她总是温柔而宽容地对待我,这曾经小小的,生长在她体内的生命。她是爱我,用了全部的心血和
	劳力。

		为了这病怏怏的我,为了这不能够完满她幸福的我。


		是因为些什么,我脆弱得好象白纸的苍白了,却有恍然坚强勇敢起来。

		我平卧在这夜的深暗里,把所有的遭遇和不幸都一一原谅。


		我说,我把火花点在指尖,跳跃成宿命的燃烧。我想象着,一个安然无恙的自己,站在十月的田野
	,没有悲戚和怨恨。

		为了那些我爱的人们,为了完满他们的幸福,我选择了微笑,选择了不逃避的决绝姿态。


		因为病,我阅读生命,用别人不会懂得的方式。

		莫曾说,上帝是公平。


		不可以改变的事情,我们就伸开双臂,深情拥抱。

		仿佛这一夜,我拥抱自己的身体,不曾责怪她可能致命的小毛病。

		我那么贪恋,爱人们的幸福和笑容,所以,我努力把自己开放。你会明白吗,你能够听见吗。

		祖母,是否在世界的另一个秘密入口,等着我,等我生了双翼,在如她的年纪飞去。生命,没有重
	量,轻若梦的虚无。


		曹问,你为些什么活着。我说,眷恋与不舍。

		这一刻的我,在秋天的大树下,思绪芜杂。不知所言,感慨万千。

		我不会哭泣,因我是被疼爱的。

		我只有感谢。我只有幸福。


		那一天中午,阴沉在一个瞬间被吹灭。

		原来,它是不可以久持的。蓝空透明,好象重生。

		雾淡了,又散去。我看见你,站在我的身边,给我手掌,和支持。我于是坚定,我于是微笑得真实。


		放心吧,万物美好,我在中央。

	\endwriting


	\writing{关于天空}{2005年10月30日 ~ 12:35:58} %<<<2

		因为仰起头颅便可望见辽远,我们每个人都该心怀善良和感激。


		昨天傍晚,走下公车的我,相遇了蓝至澄净的天空。于是懂得,我始终是被眷顾的孩子。

		几片风筝悠悠在云朵间,红的燕子,绿的蜻蜓,细线被地上的老人牵着,他们目光渺远在天上的蓝
	。

		日光稀疏了落下去,透着朱红的光彩,比白日里更为灿烂,照在街对面的工地,那些土墙和钢铁也
	便着了色泽,有了生机,是一片和美安然。

		我站在电线杆下,看这细高的巨人表情温柔地伫立,它们用丝丝电线相连,不知,秘密里交谈着什
	么。是四时的无声流变,还是将至的风沙和雨雪?它们默默里,仿佛洞察了一切。


		我仰起头颅,表情幸福。用相机拍下,这落日里的一处处光影。又在惊喜里讶异着,浓重着彩了的
	云霞。像节日里盛装的姑娘一样,云朵聚集着,仿佛一片欢声笑语。

		我想起,很多年前的自己,那个穿着白色小裙子的小女孩,想起母亲,年轻的,没有年华痕迹的母
	亲。

		我们曾相依着坐在夏末的田野,看一如今日的云霞,看它们燃烧,它们盛开,有转瞬间涣散不见。

		我是拥有天真目光的我,母亲是光润美丽的母亲,那一个年份,那一个傍晚,多么远了,又多么美
	妙。

		现在,我是不是还可以靠在她的身边,安静地想着心事?她会担心,会问我的世界发生些什么。她
	知道,长大的世界里,丛生着烦忧和危险。我不再可以,无所想,无所顾及地生活。


		尘封的画面和故事,在这渐暗淡的光华里不断升起。我加紧了脚步向家走去。我知道,母亲正等待
	着我,一起吃晚饭。

		常常觉得,这样的惦念和等待,是浓郁芬芳的甜美。人,总须有一些重量,沉沉坠在心底,才不至
	于轻至无形。

		我们的亲人,我们的爱,我们的思念和怅惘,在小小的角落,正牵着你我身上的细线。这是值得我
	们去努力和坚定的原因。

		我在自家的房门前站定,读着门板上的纹理和数字,我就要走进去,回到用暖流包围我生命的地方
	。

		门开了,母亲笑意盈盈,却几分嗔怪地说,怎么这么晚才回来。

		我没有回答,却想,是因为那些云霞吧,是因为天空吧,是因为路上的回忆纠缠。


		谁会是坐下来,陪你数云朵的人。谁能够让你安心地依靠着,没有言语中,将你读懂。我们总是期
	望着尘世以外的生活,我想着小岛,想着海水和花朵。我想穿着洁白的纱裙,成为最美丽的新娘,站在
	汪洋中的一叶小舟。


		我喜欢,肆意自己的妄想,躺在洒了阳光的地板上看窗上的蓝空。好象,这个中午,好象,许多个
	自己,在过去的年份里,贪婪着这小小房间的几尺阳光。

		曾经,想把它唤坐流云阁,但觉自己俗物一个,般配不上这般的风雅,也便作罢。而云却是时常光
	顾,安静地映在我的小窗,忽而,又匆忙里离开。我藏在自己的小地方,做梦,餐云嚼雪,有简单而亮
	堂堂的快乐。


		关于天空,我只有心怀感激和虔诚。仿佛成全包容了我的所有,不加条件。我只需仰起头颅,便可
	望见辽远。这让我们,没有理由不去用心感动。


		就躲在明蓝的帐子里,生活得平常,生活得平安。

	\endwriting


	\writing{Faye 非想念}{2005年11月04日 ~ 17:53:45} %<<<2

		漫散了清雾的傍晚,校园广播放着王菲的歌,于是几分凉意中,结了我清癯的怅惘和欢喜。

		喜欢听她的歌。听那首没有具体歌词的《哪儿》得了结论:王菲唱歌,是不需要唱歌词的,只是随
	意哼着,就已成天籁。她说自己有副好嗓子,并且是幸运的人。她确是幸运的女子,有从容选择生活的
	权利,这是令人艳羡不已的。Sky问我对于王菲隐退一事的看法,我说,那是她的决定吧,那么自有她
	的道理和原因,我只有祝福。他赞同。而更多的人,是在指手画脚她的生活,责问她的选择。是的,我
	们再不能够听到王菲的新歌,但这只是我们的遗憾和不舍,她有她的权利去追寻她要的幸福。只要,那
	的确是她诚心向往的。对于李亚鹏,我想,他大概也有他的特别与可爱。只是,我并不喜欢,那个电视
	剧片里几分憨傻的男人。而王菲是喜欢的,那就是好的。和自己喜欢的人一起,有什么不对吗。没有人
	有理由去对别人的生活说三道四,即使,她是所谓的名人。

		我说,王菲的声音干净。我喜欢干净的声音。还没有哪一个声音,让我体会到更彻底的纯粹。常常
	把自己浸泡在这样的歌声。然后,就有一些人,一些回忆敲开我的房门。它们善意地微笑,低头啜泣,
	又转身离开。故事令人的生命丰美,我的过往,好象化在她声音的液体,注满我的身躯。会一个人听《
	红豆》。一首喜欢很久的歌,和朱朱一起唱,在放学的路上,和姜一起唱,在忘了伤痕的下午。还没好
	好地感受,雪花绽放的气候…… 还没和你牵着手,走过荒芜的沙丘…… 许多的没有,铸就我们的无尽遗忘
	,那么美的,又那么轻。会想起,他把从南国树下拣到的红豆放在我手心,通红而明亮的一粒。我不知
	道,那是不是摩诘诗中的红豆,却依旧相信,那是南风启始地的一团思念。此物最相思,让我读着,读
	着,却发觉,我们的思念,早已消亡空白。它们,都是注定不得善终的故事吗。

		她不再唱歌了,我庆幸去看了她最后一场演唱会,在北京,04年的夏天。有很好的月亮的夜晚。我
	沉醉在歌声的清亮透明,像是月色的。整个体育场,亮起荧光棒,我一首首跟唱下来,快乐到几近癫狂
	。那天 Sky 也去了,他笑说,那才是我和他的第一次见面。在一年前的夜晚,一同沉溺于王菲的夜晚
	。一样是喜欢王菲的孩子,所以我感觉熟悉而亲切,虽然,我还未曾了解他的分毫。

		会在骑车时唱她的《夜会》。让人浮想联翩的歌与词句。

		\longpoem{}{}{}
		原谅你 ~ 和你的无名指 \\
		你让我相信 \\
		还真有感情这回事 \\
		啊 ~ 怀念都太奢侈 \\
		只好羡慕谁年少无知 \\
		二月十三号到此为止
		\endlongpoem


		是一些告别吗。当我的爱人,手上戴了订婚戒指。而我,不是新娘。是凄怆的歌声,飘摇在耳朵,
	伤感却散漫。让一切结束吧,回去你原来的位置,住同一座城市,却望见不同的风景。或许,也有擦身
	,却永远来不及,重新认识。有人对我说,文学青年的悲伤,是矫情的。在这样的歌声里,我不免矫情
	。轻轻呼吸了,轻轻发觉了,轻轻走过了。快乐,却永远是我生命的向往。对莫说,谁不快乐,那就是
	和自己角劲,是自己跟自己过不去。她于是笑着点头,说,我们都做快乐的人。看到有人写,快乐,就
	像一只蜂在荒凉之地为自己栽植一枝花。那么,我是这样的孩子吧,一只微小里嗡嗡不息的蜜蜂,一个
	忠诚热情的花匠。


		告诉自己,你不要遗憾。世界上的太多,要我们微笑接受。因为,那是你无能为力的,所以,就无
	须介意。王菲不再唱歌了,人们遗憾,于是翻出很多年前她的处女唱片来出版。听了一首,叫《多梦的
	童年》。没想到,她有这样朴素直白的歌曲。我更喜欢,后来的王菲,那些模糊的歌词,引发我的想象
	,而不是明白地见到。我要你,给我思想的空白空间。她曾经这样稚嫩地唱着甜歌蜜曲。原来王菲,也
	是一步步长大的。


		雾色迷人,广播里唱着《天空》。兴奋地给Sky短信,他说,他的耳朵里正塞着这首歌……

		许多巧合的美丽。我们措手不及。

	\endwriting


	\writing{临界。留念}{2005年11月12日 ~ 20:26:43} %<<<2

		\longpoem{}{}{}
		从遥远的西天,\\
		从余霞中间,\\
		飞来一片枫叶,\\
		飞来一朵火焰。

		我把它拾起,\\
		作为永久的留念。

		\flushright 1969年
		\endlongpoem

		顾城的《留念(一)》,关于秋天,关于淡彩的流云,和火样的叶片。

		\longpoem{}{}{}
		在粗糙的石壁上 \\
		画上一丛丛火焰 \\
		让未来能够想起 \\
		曾有那样一个冬天
		\endlongpoem

		《留念(二)》,关于冬季,和丛生如野草,如花朵的火焰。

		我读他的诗,天真如孩童的字字。想他说的,人可生如蚁,而美如神。

		总相信,我们的心里,藏着自己的神灵。我们是自己的佛,有慈悲,有光辉。只在心灵安宁洁净的
	黑夜,显现降临。

		我听心跳的节律,符合着四时的流变,顺应着天地的安排。我们站在晴空下,站在雨雾里,站在生
	死的临界之上,无限怀想,无限眷恋。爱万物,爱众生,成全着,时光转眼间的大美与大爱。

		这是不可奢望的境界,一沙一世界,一花一天堂。

		我们的灵光总随着白日的轮转暗淡下去,惟有孩童,保持着天然的恩赐,纯净无染,惟有在黑夜,
	我们蜕去日光里的伪饰,通体洁白。

		在生命的最初,在黑夜的最深处,我们感受了真存,我们发现了不可言,不可解的奥意。

		顾城,是那个黑夜里写诗的孩子。黑夜给了他黑色的眼睛,他却用他寻找光明,那藏在生命里,度
	化自身的光辉。而他,却又是那么任性的孩子。

		我们永远看不清身处何方,我们永远只是被时间无情地流放。我笑笑,看看秋天,叶子就黄了,簌
	簌飞去。我们站在正午的林子里,目送一场闪烁光芒点点的告别,再见再见,我们默许了承诺,要把鲜
	活着的每个日月都倍加珍爱。细数流年,去懂得草木,懂得雨雪,懂得,一颗心的盛开和凋谢。我们,
	也是孩子,我们站在一种临界,无所畏惧。我知道,风起的下午,整个世界都在做着飞行的梦。

		我感觉快乐。

		我看见秋天里的你,笑容恬淡。我听心脏里血液的流动,像一条春天的河,那么奔放,那么轻快。
	我的神灵,在这样不想不梦的时刻一闪,又消失。我是在老去,无可挽回与拒绝。爱,对于天地,对于
	人间,却愈加浓郁。许多滋味,是需要陈年酿造的,有些人,是用来反复阅读和体味。我看见秋天里的
	你,笑容恬淡。于是放心,可以安宁下来,一句句读着汉朝人的古诗,一丝丝陷落在风飘万点的悲怆与
	凄美,那接近了生命本质的境地。是的,是悲伤而美好的一切。你说我是悲观的,我却相信,悲观的人
	更懂得快乐。

		我感觉快乐。

		良短信说,看看火烧云,好象美洲大陆板块。

		正静坐一处的我忙推窗去看。凉风吹进,我探出头去,只是西边的尽头红灿灿的几块碎片。良眼中
	的大陆,被遮挡的高楼切割了。我这里看不到,但是我很开心,你看到了。我说。

		我想着穿衬衫的他,站在我们初见的地方,一个人看天际的云霞。我们都会生了思念,种在彼此的
	心田。你,是走失的另一个我。

		我喝一杯清水。黑夜浸润了房间。我感觉恍惚,因为诗歌,因为爱,和时间。镜子里就有了神情零
	落而甜美的孩子,正在老去的孩子。她想不通了,她不明白了,这个世界。却也是在那一个点刻之上,
	她的身躯还原到通体雪白。是在不解的境况里,我们得到真实,发现光辉。

		任性的孩子在精神里说话:整个下午都是风季…… 你是水池中唯一跃出的水滴…… 一 …… 滴。

		我们是水做的。我们在生死人神的临界。

		苦恼,却幸福万分地活。

		这个冬天,我用些什么,作为留念。我也有,黑色的眼睛。

	\endwriting


	\writing{森林里}{2005年11月18日 ~ 23:24:41} %<<<2

		我在森林裡,藏著好多好多的秘密。有些秘密我也記不得了……

		我看到我遺失的夢,在黑暗中微微發著光……


		属于森林的秘密。孩子们都做过这样柔光依稀的梦吧。有毛毛兔,有树木,有浮动在背景的洁白纱
	帘。我喜欢,幾米的这个绘本。不用缤纷的色彩,只是黑白,一切,淡淡的,却安静而可爱。真的好象
	一种梦境,虚无的,无声的,却也最令我们着迷。

		喜欢森林,虽然我没有真正到达过,却也正是因为它只存在于我的想象,而愈加美丽。那里,该有
	野芳连天,该有小矮人们的小木屋,该有迷路的旅人,该有维尼和伙伴们的快乐故事。

		有红色的气球飞过蓝空的空阔,那是属于朴树的歌声中的,《旅途》,他唱:昨天飞走了心爱的气
	球。你可曾找到请告诉我。那只气球。飞到遥远的遥远的那座山后老爷爷把它系在屋顶上。等着爸爸他
	带你去寻找…… 总会有一些轻而又轻的快乐和忧伤,被我们藏在了自己心中的小小森林。我们可以奔跑
	,可以飞翔,可以爬到最高的大树上,筑一间小巢,和五彩羽毛的鸟儿做邻居。会在起雾的清晨,采一
	把紫色的草花,插在木窗子上的玻璃瓶。或许,在暖洋洋的下午,可以穿过溪水的欢歌,到白雪公主家
	喝杯下午茶。只是手扶着树干,照着温柔的太阳,就可以满足,就足够幸福。

		我想着这样的森林。想着从不存在的一处世界,却执迷地相信了。森林,该是那样的,在几重山的
	背后,还会有住着天鹅的湖泊,会有长着透明翅膀的仙女,在夜空中放着光芒,飞来飞去。毛毛兔在傍
	晚敲我的门,送来新鲜的胡萝卜,做为晚餐。它的眼睛很大,充满了善良和天真。

		我就如,那飞离的红气球一样,在山岭和山岭间飞越,在云朵与云朵里穿梭。我是自由的,比空气
	更纯粹,比风更深情。这么想着,嘴角就绽放出两朵微笑的小花。因为可以在想象中肆意,可以在没人
	发觉的时刻钻进另外的天地,我感觉快乐。

		幾米的图画和简短的文字,是有这样的魔力。让我更容易在现实的纷繁里找到那扇通向心灵异域的
	门。

		看他最早的一本书,是《照相本子》。记住的,是最后的那张图。两个人并肩平卧,在鲜绿的草间
	。后来,草长高了,而我们依旧平卧,像睡着了一样,表情幸福而恬淡。他写:后来,我们甚至不再说
	话了。不再睁开眼睛。不再在乎对方在做什么,想什么?后来我们甚至睡着了,幸福地不知不觉地睡着
	了。后来,我们在彼此的梦中,幸福地慢慢醒来…… 读这本书,是在17岁的夏天,深深地喜欢了这些话
	。那个夏天,是水蓝的,很远很远了,我世界的草,也已经在几场滂沱之后长得很高,淹没了我曾经的
	容颜与笑貌。后来,读到一首诗歌,“现在,我只想安静地躺在一个人身边,让流云的影子,千年如一
	日地浮过我们的脸……”是一样的境地吧。我愿意,平卧在爱人身旁,没有声响,只等待青草生长,覆盖
	你我的爱,和年华。那将是无悔而坚决的,那将是值得长久被回忆折磨着纪念的。只想,许多年以后,
	我将在你深夜的梦境里,无端端地醒来。让你想念,又落下泪来。

		那便是我的全部阴谋,和奢望。

		我把秘密藏在风衣的口袋,和糖果们放在一起。等着,落下雪来的早上,与你一同品尝。

		那天,我会在学校的小树林里,堆一个洁白无暇的雪人。我想象着,他拥有生命,他会听懂雪花的
	舞蹈,会明白孩子们的快乐,和忧伤。我将是幸福的,因为,是在接近了梦境的树林,虽然,它那么小
	,那么小,一点也不像森林,却足以埋藏,许多许多花蜜样的日子。

		毛毛兔没跟我说再见就离开了。

		没有梦的城市,好寂寞。

		星期四的下午,风在吹,

		白色的窗帘,轻轻的飘了起来。

		是谁在窗外吹口哨呼唤我?

		我想再作一个梦。

		我喜欢,是迷失在森林的孩子。飞行在柔光依稀的下午。

		世界都飘了起来。

	\endwriting


	\writing{生}{2005年11月20日 ~ 20:08:41} %<<<2

		我相信这样一句话:每一次睡眠都是一次死亡,当我们醒来,便是全新的生命。在沉沉的呼吸里,
	我们曾遁入黑暗,我们的生,在万物安静的时刻,随了远山的松涛,一并澎湃,一并纯净。

		那一次次睡眠,是我们穿梭于物与灵的轮回,在每日的往复之间,我们获得重生,在晨早醒来。原
	来,我们在如此频繁地体验着死亡,死亡是深刻的,却也轻盈。

		关于生死,我们总是疑问,像隔了山岳几重又几重,带着旅人的疲惫,也带着期待,我们一路奔赴
	。越了山溪,经过如笑春山,几分欣愉,几分恐惧。因着对生的无限眷恋。

		不要说,你无所谓于生死,古人亦叹,知一生死为虚诞,齐彭殇为妄作。此身尚在,便难脱深情,
	纵使是弘一法师,如此高通明澈之人,也不免在临终前写下“悲欣交集”四字。人评,“悲见有情,欣见
	禅悦”。却喜欢那一句,“存,吾顺事;没,吾宁也。”

		生死,不过如此,生时顺化四季天地,死去恒久安宁。让人们微笑在世间,寂静在身后。

		在这个深秋,我在照进窗子的日光里想着这些。轻轻抚摩自己双手被风吹干的皮肤。它们粗糙了,
	不再光滑细润。我却也感受到,在薄薄的肌肤之下,血脉正暗涌鲜红的波涛,带着生命的节律和体温。

		我真实地感觉到活着,感觉五脏肺腑的活力。

		这是我赖以有所知觉的肉身。我抚摩,我对母亲无限感激。我是怎么长大了呢,在她温暖的胸怀和
	液体。在一个缀了花枝的早春,我游出她的生命,成为现在的我。

		于是,想好好保养自己的身体,因那是母亲对我的赠予,无限的赠予。小的时候,她总是说我是她
	身上掉下来的肉。现在,我懂得了一切。成年了的我,依旧喜欢枕在母亲的怀里,我沉溺熟悉的温度,
	我们的生命,本是在一起呼吸的。

		并不是所有的孩子都能够了解死亡的意义。我却是了解了的孩子。我经历亲人的死亡,和无常的变
	故,在我还那么幼小的时候。祖父去世了,我第一次亲见了一个生命的消失。

		他像睡了一样,躺在他的床,面色如纸,祖母和姑妈声嘶力竭地哭喊。我真的吓坏了,躲在房间的
	一角,怯怯地看着发生的一切。祖母瘫坐在门前的柿子树下,那祖父年轻时亲手栽种的树正果实累累。
	她反复问着,你走了,我怎么办呢…… 祖父走了,我隐约明白,他永远不会回来了。

		几年后,是凌晨的一个电话。母亲接了电话,便夺门而出,那是冬天,夜晚的寒意填满了无光的屋
	子。我在被里蜷缩,不知发生了什么,却有不祥的预感,一夜恐惧。第二天的中午,我才从父亲那里得
	知,二舅突发心脏病,已经过世了。而我的二舅,是那么健壮高大的男子。竟就这么,化了烟雾一般,
	不见了,再也不见。我才知道,生命是何等脆弱无力的。我们的呼吸,竟然是不堪一击的。

		我于是开始对死亡充满恐惧。开灯与关灯的一瞬,我总是觉得,人也是如这光亮的。一触便生,一
	触又消散。在肉体的内部大概存在着这样的开关,或者,真的有那么一本生死薄,把一切都已安排。童
	年的我,洞张着一双眼睛,惊讶万分于这猝然的了解和发现。

		也是很远的一个冬天的傍晚,天阴郁着,似乎就要下雪,空气是凉而湿润的。

		母亲在厨房的一角,取了煤火在烧一叠照片。为什么要烧掉呢。那些照片上统统是一个女孩,20
	岁的模样,笑意盈盈。为什么要烧掉呢。她是谁呢。我问母亲。母亲却不回答,只是默默地烧着。火光
	映红了她已经开始生长皱纹的脸。为什么要烧掉呢。她是谁呢。我不断地追问。终于,她轻声说,那是
	她曾经的朋友,很多年前死去了。怎么死的呢。怎么死的呢。这一次母亲没有再出声。

		是在后来,我才知道,她是自杀的。为了年轻,和爱,她抛弃了这世界,这生命。我记不清母亲当
	时的表情。也许是太远了,母亲也已经不再记得那往事的全部。

		而那笑意盈盈的女孩若还活着,也该有母亲一般年纪,也该有一个20岁了的女儿。一定会是美丽
	的女儿——她曾是那么美的。

		为了一些什么,生命也许可以失却重量,变作微不足道。比如所谓大义,比如尊严,比如阮小姐所
	说的,人言可畏。

		我却仍然感觉生之可贵。我们终将离去,我们终将闭了双眼万事不知,这有限的岁月,纵使是屈辱
	和痛苦,也该好好保存的罢。因那是母亲的赠予,无限的赠予……

		汉朝人开始知觉了生命,六朝人更将重生思想发掘到极至。人说今朝有酒今朝醉吧,说及时行乐吧
	,问人非金石质,岂能长寿考?问浮生若梦,为欢几何?因这一身的不可再生,我们珍爱了落花,悲伤
	了秋树,听着残荷临雨,细数西风的归期,感叹着流年暗中偷换,凄恻一场。

		孔老夫子站在千年的水畔,看流水的不舍昼夜,他说,不知生,焉知死。我们总是要懂得去生,才
	有可能望见死的真实。

		而有一些时候,死亡,也许是告别,是成全,是解脱。

		看卢照邻的《病梨树赋》。想身患风疾,痛苦非常的他,侧卧于床塌,望着院里唯一的树木——那株
	“叶病多紫,花凋少白”的梨树,发了生命的慨叹。植物与人,似有通灵,病的瘦诗人,病的瘦树木,在
	那刻上,定是相惜相怜了。说着“生非我生,物谓之生;死非我死,谷神不死”的他,最终还是选择了投
	水而死。那大概是好的归宿,人,本是从水中获得。不堪疾病痛苦的诗人,死去了,我们却与他一并感
	觉轻松。

		而那院中的树呢,它还会开出细小的花朵临风憔悴吗。它是不是也早已远行,随着它的知音患难,
	随着足踏水痕,凌波而去的病诗人。他们,都会是度化了痛苦与生死的。我好象听见他在吟唱:常恐秋
	风早,飘零君不知。诗人已去,化了风里的花瓣。

		对于生命,你有什么精辟的解释都只是徒劳。它不可名状,不可言语。只可以在自己呼吸起伏间寻
	觅真相,只可以隐约地懂得。这一逆旅之上,笑与泪交加。也正是那一句结语,悲欣交集。让你默默思
	想,默默生存,深情而眷恋。

		而今的我,不再恐惧死亡,因为,那永远是人们最恒久,最安宁的归宿。没怎么读过周作人的书,
	却看到又喜欢了他的一句话:大约我们还只好在这容许的时光中,就这平凡的境地中,寻得必须的安闲
	悦乐,即是无上的幸福。

		当我从睡梦醒来,我知道自己是全新的生命,又一次死亡在我的肉身上盛开过了。

		每个清早,我们明白更多。我们不断重生。

	\endwriting


	\writing{十一月的胡言}{2005年11月27日 ~ 19:59:39} %<<<2

		在一处安谧的晨早,你透过结了冰花的玻璃窗,说一句,多好的冬天呵。雪花就落下来,像日子的
	纷飞,冰凉无声。一朵,停在你幼小的指尖,祖母说,那是上天给孩子们的礼物。许多的花朵,青空上
	的,窗子上的,在怒放,在微笑。


		十一月,还没有寒冷。特别是今年的冬天,温暖到好几个午后,竟有了阳春的幻觉。十一月,我看
	着燃烧起来的法桐,高大的枝干,灿烂的秋叶。我喜欢,它们的挺拔,把手臂伸上无限的苍穹,把根茎
	钻进土地的更深。树,用它们一成不变的姿态站立,在高处俯瞰,匆匆的,繁华似锦,又苍凉如寒梦的
	人间。看你,抱了满怀的书本,从教室的窗口望出去,望见树,望见树身上的热情,和火焰。人与树,
	在这样温和的秋天,默默里仿佛一同演奏一曲,清水漫过月色一样的钢琴。


		对莫说,我不再喜欢回忆。

		而在季节的缝隙里,当感觉到冬天的意味从发丝间渗出,又如何不去想起。那些,曾经如秋叶一般
	燃烧了,又落下的世界。想念夏天,母亲在睡前摇着扇子讲起的,小兔子的故事。想念小学的同桌,将
	橡皮切了两块,约定长大后以此相认的伙伴。小兔子的故事,没有结局,母亲总是在没有讲完时就昏昏
	睡去,只留下我,洞张着黑眼睛醒在有小虫子歌唱的黑夜。橡皮早已丢失,在夏天的几场滂沱之后,我
	们也都各自消失掉了,为什么会有那样的约定呢,莫非,在那么幼小的时候,我们已经懂得,离别是不
	可以逃避与拒绝的。离别,让我的伙伴,永远是8,9岁的模样,这也许是幸福的。

		我却不再喜欢回忆。虽然,它们总是美不胜收,让人留恋。它们燃烧,飞舞,好象这个秋天,那么
	光明灿烂,把世界都照得透明晶亮。而我自己,永远在那些快乐或小难过中褪色了,我想说,我爱你们
	,我却不可以太爱。


		傍晚时候,和良走在校园,法桐的落叶布满了街道。是一个阴湿的天,铅灰的云压在树尖,天色渐
	渐暗淡了。穿着红色外套的小女孩,从我们身边经过,手中握满比她的手掌更大几倍的叶子。她一脸无
	邪恶的天真喜悦,继续向前,拾起更多的落叶。叶子是美丽的,有血脉流淌。叶子,安静躺在地上,会
	不会想到一个小女孩的来临呢,它们有没有惊喜,有没有快乐。穿红衣的小天使,仿佛从天而降,叶子
	们会相信,她是上帝派来的使者。我回头望着那女孩的身影,她的背上,似乎真的生长出一对翅膀,如
	雪的白色。


		十一月,不写诗,也不唱歌。只坐在落叶旋转的光芒里,听着风诉说的消息。让寒冷来吧,在没有
	言语,没有声响的时刻,把全年的生活梳理。一些无可奈何的心绪,一些迟到的结局,都在这里,被装
	进时间的气球,一并飘去,离开人间,飞向宇宙的未知。我就可以无所牵挂地躺在床上,享受午夜的思
	念,给值得去爱恋的人们,给不忍忘却的回忆。我等待着日子的告别,等待十二月,一场漫天席卷的风
	雪。让我喝一杯热开水,甜美地笑,满足在年末的疏落和枯燥。因为知道,冬天的花朵正开着,开着,
	肆无忌惮地幸福。


		我不知说着什么,十一月,正告别。你也在想念吗。

		我多想,在一个结了冰花的晨早,醒在你洁白的梦里。我们都是,留恋人间的天使。

	\endwriting


	\writing{零碎。三}{2005年11月27日 ~ 22:34:17} %<<<2

		\subpart{一。面包}

		发现一家可爱的面包店,Bread Talk。大概是好久不进城逛逛的缘故,竟然觉得新奇。店面干净
	,面包师傅在大玻璃窗后面操作,戴着白色的帽子,样子挺帅的。买了做成面包超人模样的红豆面包,
	非常满足地离开。想着回去送给良吃,他一定很开心。

		让良翻译Bread Talk。他迟疑一下,“面包说……?”我就笑啦,是面包谈话吧……。嘿嘿,其实都不是
	,有好听的翻译,面包新语。看他幸福地啃着“面包超人”,觉得他是个傻孩子,像我一样。

		我抢了几大口吃。真的美味哦。

		\subpart{二。感情}

		静打电话说,他回来找她了。那个曾经欺骗了她的男的。

		我无言。心里紧紧的。对静说你自己把握好吧。我对那男的已经失望至极。她说,他不是你想的那
	样的。静在极力维护他。恋爱中的女人啊…… 她陷进去了,又怎么能够轻易摆脱呢。她一直还是喜欢他
	,即使,他做了那么伤害的事情。

		局外的人,是最清醒的,却也是最难过的。希望她好,希望他实现他所说的承诺。虽然,我已经不
	抱太大希望了。

		男人都不是好东西。我听人家这么说。

		感情,到底是怎么一回事情呢。静说,她大概是上辈子欠了他太多。女人,总是这样心甘情愿地沉
	浸其中。是可悲的么,还是让人艳羡的幸福呢。她该是幸福的孩子,她是那么善良的。静该被好好疼爱
	。

		我会祝福,如果他真的是你喜欢的。

		希望我的前生是两不相欠的。我们的今生,是用来好好相爱。而不是还债。

		\subpart{三。花火}

		让花火的小小光芒,映亮我的脸颊。良说他喜欢。良一脸天真的甜蜜。

		我是这么容易满足的孩子。我想拥有的,只是这小小的火光罢了。在我的手里燃烧,在你的目光中
	闪现。我们不贪图白昼的光明,我们接受黑夜,我们将拥有非凡的旅途,始终双手紧握。对吗。

		我的爱人,喜欢望着我。正如,我喜欢在角落,偷偷望着他一样。

		\subpart{四。姑娘}

		与我对头而眠的好姑娘。我叫她莫。

		我们有安静的睡眠,我们有温柔的晚安。四年。在那高高的床铺,我们将一同做梦,偶尔闯进彼此
	的梦里。

		她的,凉月,我的,花田。两个盛开芬芳和絮叨的地方。莫摄下我们的影子,在秋天的凉光里。我
	说,这多么好,我们的灰影子,在地上,也在回忆最温柔的地方。

		小情调里的快乐。我们一起分享。

		我也是好姑娘。

		\subpart{五。自然}

		读德富芦花的《自然与人生》。藏蓝的小册子,质感贴心。留一盏小灯在睡前,然后就在文字里,
	嗅见海那一边,原野与山间的芬芳。他爱雪,他爱霜,他爱一切自然的灵动秀美。写晚秋初冬,“庭院
	中红叶,门前银杏不时飞舞着,白天看起来像掠过书窗的鸟影;晚间扑打着屋檐,仿佛晴夜飘雨。”我
	喜欢,这样清澈的文字。像涓涓的溪流,漫过赤裸的脚踝。生活是在细细的回味里散发香气的。

		我愿意枕着自然的恬梦,依偎我的大棉被,在寒夜里听着北风的声音。

		那是属于天地的吟讴,不加修饰,却近乎完美。

		\subpart{六。文字}

		有时候觉得文字是多余的。尤其是自己的文字。

		每个人心里都有文字,只是有些人罗嗦,有些人沉默。

		我是前者。

		一个挺矫情的人么……?

		我中了文字的毒。

	\endwriting


	\writing{耳朵}{2005年12月03日 ~ 09:46:25} %<<<2

		朱有可爱的小耳朵,人说,那是元宝耳朵,有福气的。认识她的那年夏天,九月,她穿了白衬衫站
	在我面前。一个调皮的女孩子,躲在座位里看金庸,打打闹闹的,没有心肺的模样,留着齐整的短发,
	卡着亮晶晶的发卡。从没有料想,她就是我最亲密贴心的伙伴了,一辈子的朋友。有小耳朵的朱,孩子
	气的朱,爱生气的朱,坐在她身后的我喜欢把她想象成卡通人物,画在她的笔记本上。朱喜欢本子上戴
	眼镜的小女孩。我叫她小木棍儿,因为,朱是太瘦的。那么,她的耳朵,也便成为“木耳”。下雨的日子
	,总是笑说,又会受潮长出木耳了吧?朱会乐呵呵回答,好啊,要是长了就送你吃。


		这样想起来,嘴角就浮现天真如旧的笑意。我的小耳朵朋友。有福气的朋友。我们有许多的秘密,
	从那小小的耳廓间保藏,我们有太多的笑语,被小耳朵听见。小耳朵也喜欢知道,小女孩的悄悄话。


		耳朵,身体上最脆薄柔软的一块骨头吗。我可以看见,阳光透过,笼在你耳朵上,是一层微红的明
	亮。感觉是春风和暖的下午,我一个人站在开了白丁香的庭院。想着整个肉身的虚无梦幻。人是完美的
	,被创造出来,各自独立思想着生存,也独立着寂寞。我摸摸自己的耳朵,它有适宜的温度,我又捏一
	捏,就发觉生命里所有细小精致的秘密,都藏在它柔韧的轮廓之下了。那里的血,大约是最真实而鲜活
	的。所以,小孩子才会被取耳血,作化验的标本。我想,那是并不会很疼痛的,因耳朵,终究是一块骨
	头,而不是敏感的血肉。虽然,它是那样一块,引人怜爱的骨头。


		圣经中讲,女人原是男人的一根肋骨。我觉得这说法美妙。也想象,自己,本是长在爱人身体的一
	根骨头,走失掉,又长成一个美丽的女子。想我的爱人,等着他的肋骨长大,满怀着期许和善良。但是
	,我却又更愿意把这故事变形。我更希望,自己原是爱人的耳朵,骨头中那最温柔深情的一块。在我还
	没有长大,他的世界是全无声息的。当我们终于在失散后相遇,直到那一刻,他才听到整个世界。听到
	浮动的风丝酝着草籽的淡香,听见白鸟的翅膀载了晨曦的光飞向四季,听着我,轻轻的言语,和歌唱,
	在他的耳畔。一切,在刹那里鸣响,在刹那里获了生命。那将是幸福的归宿。即使,这是不可奢求的,
	我也愿意妄想。我们是这样,命定中要一并奏鸣双耳的脆弱的。


		谁在你的耳朵,缓缓吹气,又狡黠地笑。谁冬天冻红了耳朵,站在十二月的开端,叮咛一句,记得
	多穿衣服。我用围巾把自己包裹,走在终于结冰的世界,听见寒冷从耳侧驰过,呼呼作响。我并不灵敏
	的听觉呀,在这个时候变得多愁善感,它听懂了那其中的言语,寒冷走了很久,来到这里,寒冷是翻越
	了万水与千山的阻隔,来赴一场从不变更时间的约定。让人们在冬天冷静,用暗淡消沉的天空,治愈和
	平复所有无从放逐的疼痛。那样,在另一个开了白丁香的春天,我们又能够穿上碎花布的小裙子,在树
	丛里寻觅有5瓣的花朵,来在热望里实现那一个有关幸福的传说,或者奇迹。这些,都是我们必定要经
	过的。要记得,阳光终究会透过你的耳朵,光亮亮的微红。


		我将沉淀,和静定,在零度之下。没有任何事情,是真正值得悲伤。


		冬至那一天,是要吃饺子的。不然会在严冬里冻掉耳朵。22日,那天我会记得。我会好好爱护它,
	不让它受一点伤害。去年,同宿舍的伙伴一道步行着去吃饺子。学校傍边的餐厅挤满了为了饺子而来的
	人。我们只好,走更远的路。却没有吃到水饺,最终是成都小吃的蒸饺。很好吃,但是,少了水的灵气
	和滋润。而无论如何,那个冬天,我们的耳朵都保住了,没有被冻掉,甚至,也没有冻僵过。我们最柔
	软的骨头,安然地长在我们的肉身,继续聆听,也诉说,爱与被爱的消息,还有,离别与想念的温存和
	苦楚。耳朵都听见了,耳朵告诉了我们的心灵,耳朵会悲伤,耳朵也会感觉幸福,小心翼翼酿着花蜜。


		良也长着小耳朵,和朱一样的,元宝耳朵,有福气的。我就相信,他们都会拥有最美好的际遇,来
	享用生命的全部。我没有像曾经那样,用尺子去丈量耳朵的尺寸。我只是看他的耳朵,小耳朵,更显脆
	薄深情的双耳。用手机拍下来,独自的时候看一看,就有天真的笑意。我仿佛还是那个和朱一起大声唱
	歌,放学回家的小女孩。让我任意放肆吧,在你们的爱里。被宠爱,被宠坏。


		我的夜晚落下来。没有雪花的声音,但耳朵懂得,那将不再遥远了。

		十二月,是我们坐在空寂的天地,一起听见的。


		换了紫色格子布的页面,有淡淡的甜味

		正在播放的歌曲 —— 《By The Cathedral》 Keren Ann

		歌声清澈的法国女人。

		% By the Cathedral

	\begin{figure}[H] %<<<3
	\begin{multicols}{2}
		\shortpoem{}{}{}
		Darling \\
		I will remember, darling \\
		I've been \\
		Mellow and tender \\
		I've seen water \\
		By the cathedral \\
		Under the maple \\
		It was in April \\
		You wore a raincoat \\
		By the cathedral \\
		I wasn't able, I was unable \\
		Someday \\
		You will forgive me darling \\
		Someday \\
		You will believe that I've seen thunder \\
		By the cathedral \\
		Under the maple \\
		It's was in April \\
		You wore a raincoat \\
		By the cathedral \\
		I wasn't able, I was unable \\
		\endshortpoem

		\shortpoem{}{}{}
		心爱的人 \\
		我会记得,心爱的人 \\
		我曾经 \\
		芬芳,温柔 \\
		我见到过水 \\
		在大教堂旁边 \\
		在枫树下 \\
		那是四月 \\
		你穿着雨衣,在大教堂旁边 \\
		我不能够,我不能 \\
		会有一天 \\
		你终将原谅 \\
		我心爱的人 \\
		会有一天 \\
		你将相信,我听到过雷声 \\
		在大教堂旁边 \\
		在枫树下 \\
		那是四月 \\
		你穿着雨衣 \\
		在大教堂旁边 \\
		我不能够,我不能 \\
		\endshortpoem
	\end{multicols}
	\end{figure} %3>>>

	\endwriting


	\writing{大雪这天}{2005年12月08日 ~ 14:54:38} %<<<2

		今天的节气,大雪。没有玉花飘零,只有寒冷,拨剌过树梢,叹息在蓝到虚假的天空。


		新闻播报说,烟台遭遇了雪灾。雪的突骤,成为了暴虐和灾难。而看着电视画面上满世界的纯洁,
	我却只能够心生艳羡罢了。就有了想买张车票,坐上南下的火车,到那半岛的边缘去的冲动。看大海的
	落雪,看冬天无声息的冷静清寒。让我一个人去,不带行李,站在空寂的海滩,把尚存余温的双脚踩在
	松散凉滑的砂。想白鸥的不知去向,想温暖终于是瞬间的安慰,长久的,似乎不是夏天里的人声喧哗,
	而是我自己站立中的海岸。我仿佛看见,艳阳里,穿红色游泳衣服的小女孩跑进浅滩的浪花中间,仿佛
	听见欢笑,在并没有远离的记忆,沉浮游走,迷失在退去潮水的沙滩。让我一个人去,看大海的落雪。
	在十二月里,默声的丝丝难过,忽而近了,又忽而远。


		而北京的此刻,是全无纯洁的。我坐在如春的房间,看电视上的雪花。并不以为那是灾难。

		我并不如去年,在干硬的空气里热望一场雪的降临。即使,我愿意,在某天未知的夜晚,雪就这么
	,一朵朵开放在我们的屋顶,我们的树上,我们的十字路口,我们的街灯。那么,在梦里的你我,也会
	有了洁白的触觉,有凉丝丝的融化。然而,我终究是不会再去热切。不会了。因为那是不可以预约的幸
	福吧,只有突骤,才足够绚美感动。虽然,那也许将是暴虐,将成灾难。


		如果,清晨醒来,雪封住我们的木门。如果我们只能够相拥着,在小小的房间,出不得门去。就烧
	一炉火热,让我们静坐,诵一段久远的诗歌,或者,读着你曾寄来的信笺。请你放下心来,陪我,听着
	雪花撞上窗玻璃的声音,回想我们的相遇,人海茫茫里,我们的灵魂是怎样懂得了彼此。如果,真的有
	这样一场雪,存封了我们此时的幸福,我将无怨于生命的匆忙。花会开吧,在不远的春天,你笑着,恬
	淡如旧。我们去哪里呢。不要着急,这些时光,是值得静静相对的。雪在积攒,那是一整个春天,那是
	,我们无尽的希望,和欢喜。困于风雪,又何尝不是幸运。


		这个头脑里充溢异想的下午,陪小鹿去买衣服。她说,田是会选衣服的。我很荣幸。

		最后,买了粉色的甜美毛衫,和红布格子的小裙子,非常女孩,非常甜美的装扮。她穿上,在我们
	面前轻轻转个圆圈。在那几步之间,青春的光彩照亮我的脸庞。女孩子都有粉色情结,曾有谁这么说。
	小鹿看样子也是很喜欢,在镜子间摆弄着,笑如花蜜。她把我们的大头贴贴在紫色的纸上,配上文字,
	又装进雪白的信封,放在我手心。她写:2005。12。07 为了准备小鹿的演讲服,田与小鹿逃文学史一
	节,前往五道口“大棚”淘衣,战果显赫,兴尽之时,立此存照,照毕,两人喜悟,美是太瞬间的事,自
	恋有理!照片上的两人,装可爱,却是实在的澄澈天真。


		自恋,我曾说,这是一种积极的人生态度。因为瞬间,因为不久固的美好,我们懂得融化和春天,
	我们好象雪花。干净的,有点落寞,却绝无苍白。看着鲜活的我们,我相信了小鹿无意中的那句:女人
	是如草莓的,鲜美,却短暂。在洁白的信封上,她用深蓝的墨水写,我们都是水灵的草莓。两颗草莓,
	好的冬天,坚硬的冬天,一起向镜头微笑,灿烂。


		今天的节气,大雪。

		我发觉一些深埋的,如化石的真实。我双手冰凉。

		雪终于会落下来的。那一刻,你不要流眼泪。

		12月7日夜

	\endwriting


	\writing{婚宴。其他}{2005年12月14日 ~ 20:14:32} %<<<2

		一场匆忙的婚宴,混杂人声喧哗,弥漫刺鼻烟灰。我坐在被主人安排好的位子上,看着敬酒的新娘
	新郎在宴席间穿梭,为了配合,我也陪上我廉价的微笑。没有雪白的婚纱,没有美丽的新娘。肥胖白嫩
	的她,穿着艳红的皮裙子,批着白绒毛的披肩,似乎很幸福,似乎很满足。没有花朵,没有祝福。这婚
	宴没头绪地开始,又没头绪地结束。每个人看上去都欢天喜地的样子。却是这样潦草地,新娘出嫁了。
	是这样,丝毫不感圣洁地,成为了眷侣,夫妇。


		这是我所不愿意看到的婚宴。


		天空被冻得惨白。为什么要在冬天结婚呢?母亲说,那新娘已有了三个月的身孕,才这样匆忙地把
	事办了。我哦了一声。他们似乎是被逼迫着趋赶向一处幸福。从新郎转身时,不经意的皱眉我看穿了一
	切。有些残酷,却那么真切的事实。

		他们是相爱的吗?只有相爱的人才可以结婚吗?相爱就要结婚吗?许多的问题,一一涌出,我卒不
	及防。我没有答案,因为,我大概仍是不懂得爱情的。


		静的姐姐也要结婚了。静说,姐姐的命好。新郎是她暗恋了许多年的中学同学。没有更多的感情经
	历,和磨难,只把爱纯粹粹给了自己十几岁就喜欢了的那个人,又终于让他懂得,让他心里也生了爱苗
	,把你好好疼爱,一辈子。的确,姐姐的命真好。谁还能有这样的幸运呢?静说,他们已经拍了婚纱照
	,挂在装饰一新的房子里,只等着来年的春天一起住进去。完满幸福得近乎童话。许多人一生都不敢冀
	望的甜美。


		她会发糖果给我们吃,她会穿了白色的婚纱,洋溢青春的光芒。我想象着,不远的那个春天。一定
	要请我去啊。我对静说。她笑说,只要你不抢新郎,不捣乱。我又怎么会呢。那将是安谧神圣的时刻吧
	。我只能够微笑凝视,默默祝福的吧。漂亮的静,和她漂亮的姐姐,我也时常羡慕,这样一对姐妹花。
	我也会想象自己有个姐姐。可以在有星星的夜里,一起躲在被子里,说女孩子绵密的心思,说那些过去
	,说那些妄想。或许她会笑我,或许,她也看见自己的影子。当我幻想这些,静就会说,哪有你想的那
	么好,她从小就和我抢东西,欺负我呢。但是,我相信,姐妹的情谊,正是从这样的小吵闹里生长,又
	茂盛的。如果我有一个姐姐,她一定会比我美丽,比我懂事。


		我总想成为婚礼中的一员,而不只是旁观。于是,愿意做一位伴娘,在新娘身旁站定,捧满怀的粉
	红花朵。就总是劝身边的朋友将来一定要早结婚,请我去做伴娘。她们却坚持说不行,要看我先嫁掉。
	而我,大概是会迟迟不愿走入的那一个人。我喜欢看你们的幸福绽放,看你们的笑靥如花,映在蜜一样
	的歌声和舞曲上,步入另外的生活,和你们的爱人。我想自己有天会坐在人群中间,得知曾经的爱人,
	成为谁的丈夫,又快乐地生活。我最终将是孤独的吧…… 我常常几分悲戚地想。但是,心里又升着希望
	,水泡一样,晶亮浑圆的。


		我会穿着白色的纱裙,站在潮退的海滩,我会是最后的一抹阳光里微薄的呼吸。我会发糖果给你们
	吃。我被时光趋赶着,迫不及待地美丽,愿被你望见,被你深爱。我好像骄傲,实则卑微。不可得的,
	不可去爱。Faye 唱,不爱我的我不爱,不是我的我不要。而我,竟分辨不清什么是爱的,什么是需要
	的。只是耽于无穷尽的幻想,听着你的话语轻柔温暖。我以为,上天可以成全,在今天赠我一场飞雪的
	漫天席地。而没有,它依旧把蓝瓦瓦的颜色涂在脸颊,像用颜料弄脏了的孩子。不经意里,我仿佛站在
	夜晚的轨道边,等一列灯光的驰过,载我的迷惘与回忆,轰鸣着消失在平原的无际。谁会发现这些不可
	言喻的细末呢。我把自己撵碎了,等着一场朔风的吹散。


		那一场混乱的婚宴上,被嘈杂淹没的我,表情甜蜜,心如荒原。很远很远,红色的绸缎包裹女子的
	玉体,很远很远,母亲啜泣你的远嫁,在雁去时候,不成言语。都不会被我看见了,关于爱情,我们是
	捧着婚姻的黄土将它掩埋,还是,用更深更沉的心日夜相守?我不会知道,不会看清,时光的河水很长
	,谁陪我在秋风的瑟缩里,携手走上河梁,望一望日落的缤纷,说一说少年的惆怅?我不敢去想,一个
	即将成为大人的自己,无可选择地面对,所有的所有。成长,在于我却是不动声色的开放和隐匿。


		看父母的结婚照。头相依着,年轻的面孔。母亲手中轻握一束塑料花。那不会凋败的花朵呀,永远
	以最初的姿态示人。老去的是我的父母,手中的花朵,大概只是蒙了灰尘,如所有的爱情。像张楚唱的
	,你说我们的爱情不朽,它上边灰尘一定很厚。这样想,人就会绝望,也获得莫名的慰藉。只让我记得
	一切可爱的片刻,记得你眼底曾流下的温柔。让时间死去,让相爱的人,在彼此的记忆里长存。


		儿时的记忆,父母总是争吵。但现在,当浮华的人世在各自的眼中被现实软化得淡而清晰,他们显
	出的,是生活琐碎的溪流下,恩爱的石子。她总是担心父亲的身体,他总是默默做了曾经从不伸手的家
	务。我看着,想婚姻,究竟是怎样。如何度去爱人的美貌,如何磨灭初识的激情,又如何教你懂得,两
	个人的日子,怎样相扶相搀着,守着敝室草庐一间。走在路上,看见一对老人,听着他们商量中午是吃
	饺子还是米饭,吃炒芹菜,还是小黄瓜。那些细小零落的生活呀,原来,是要一个人陪你去静数和拣拾
	的。


		总是说,若有人为我写一首如东坡那首江城子一般的悼亡词,纵使死一千次我也甘愿。悼亡,多是
	丈夫为亡妻所做的。其中不乏感人至深的篇章。潘郎悼亡的深情与真挚,竟足以掩盖他人格上的不足,
	而令我深心倾慕。我在唇齿间含着古人的词章,想当我死了的时候,你是否也会如贺铸一样,哀鸣一句
	,空床卧听南窗雨,谁复挑灯夜补衣!这世上大概是不乏有情的男子,少的只是能够在妻子的坟上焚一
	页诗稿的难忘。总是越多情,便越易忘情的吧?


		人说,你不要对别人苛责。情感的事情,总是无法澄净通明的。只明白,当你说一句告别,道一声
	再会,就将是彻骨的悲寒。这些,有多少人,曾经发觉,多少人,仍然迷惑,如我。我看见,新郎疲惫
	的侧脸,也知道自己赔笑举杯的模样是多么可怜。那些悲伤无奈的碎屑,诉说了爱情,或者根本无关爱
	情。我好象盲了,伸手向这个冬天的风里索求,却不知道想获得些什么。你说,我活得幸运,我说,我
	幸福得不知去向。


		为了真正的深爱,或许,我舍得卑微的性命。请写一首诗给我,我会收到,读给天使们听。风吹向
	南方,吹灭我的妄想,像那些深夜里脆弱的烛火一样。


		人人都貌似欢乐,人人都很悲伤。

		我没有活在童话。我活在我的爱情。

	\endwriting


	\writing{发现}{2005年12月17日 ~ 22:24:31} %<<<2

		尝试着换了几种界面,发现,我只能够在洁净的白色上写字。\par
		用纯粹的语言,粉饰一些,或许并不完满的生活。\par
		我自己,不曾欺骗世界,却常常被世界欺骗。\par
		发现,我只能够饱含天真,装作善良,莞尔一笑。\par
		身不由己,在无法被隔绝的时间里,我慌忙地美丽,疯狂地爱恋。\par
		因为,一切的一切,正燃着你我的身躯,向另一处安宁靠近。\par
		发现,手指是最敏感的,我察觉了所有冷暖,所有不可以拒绝的欣喜和意外,却没有气力去承担,你轻盈的步伐,不定飘忽。\par
		我需要的,只是安心,只是温暖的一句思念。\par
		你会说,你从未停止。\par
		我愿意被花糖纸包裹,肆意芬芳,和甜美。\par
		我要享用这全部,关于你的,关于梦想,关于遗憾。\par
		发现,枯坐的午夜,适合点一支香烟,像我蓝色的朋友,在十三层的风声里,去了解离别,独饮悲伤。\par
		虽然,我不会吸烟,却依旧喜欢,看那灰烟,与火光,充满玄奥的猜测,一丝丝消磨殆尽。\par
		云在不远的铁塔旁歌唱,粉红的一片。\par
		我在房间里品尝这个黄昏的沉默,望窗外的灿烂,化为乌有,组成黑夜的凝重。\par
		这盛况,另外窗口里的小鹿,也一同目睹。\par
		发现,我总是难于控制,去想念,一个不曾存在的,穿白衬衫的男人。\par
		他走向我,说着老去的我们,不忍回味的温柔。\par
		他在爱我,我确信了,于是幸福。\par
		发现,自己只能够,生存在你棉质的目光。\par
		我只能够是甘心情愿的小女子,写莫名的句子,义无返顾。

		播放中:我的心里只有你没有他

		\indentenv{4\ccwd}{0\ccwd}{}
		我的心里只有你没有他,\\
		你要相信我的情意并不假。\\
		只有你才是我梦想,\\
		只有你才叫我牵挂,

		我的心里没有他 \\
		我的心里只有你没有他,\\
		你要相信我的情意并不假 \\
		我的眼睛为了你开 \\
		我的眉毛为了你画 \\
		从来不是为了他

		自从那日送走你回了家那一天 \\
		不是我把自己恨自己骂 \\
		只怪我当时没有把你留下 \\
		对着你把心来挖 \\
		让你看上一个明白 \\
		看我心里可有他 \\
		我的心里只有你没有他,\\
		你要相信我的情意并不假 \\
		我的眼泪为了你流 \\
		我的眼泪为了你擦 \\
		从来不是为了他
		\endindentenv


		悲伤的歌,最喜欢的是这个麦兜里的版本。

		非常喜欢麦兜呢。是个有思想的好小孩。

	\endwriting


	\writing{12月某晚}{2005年12月23日 ~ 21:51:15} %<<<2

		现在,北京时间二十一点零四分。我坐在电脑前敲字。\par
		门外响着哗哗水声,超超在洗衣服。\par
		更远一些的地方,有长笛的乐音响起。

		夜晚早就被扯下来,从阳台上望见,树梢晾着淡黄的月亮。我觉察了安静的生活,正在进行得不紧
	不慢。

		大学里的日子,是这样,与他人好像全无关联,却又在深刻而微妙的细小里,彼此依赖。正像这一
	刻上,无端发生的所有。


		周末时和莫一起买的画,被我挂在书桌前。两朵洁白的花,开得繁盛,却哀婉。我喜欢这样洁白的
	花朵。素心一颗,纯净无邪的模样。

		一些最柔软的心情,会在与它们相视无言里悄然盛开。

		想起许多场春天,微雨中,孩子们走过缀满粉红的树下,一路嬉笑一路歌唱。

		看墙上贴的照片,7个女孩子,站在白碧桃的芬芳间,微笑灿烂。那是05年的春天,在西山的植物
	园。

		正看得出神。隔壁的长发粒子敲了门进来。

		于是,我们一起说着那些有关无关的,回忆的碎片,带着花瓣上残存的清香,从唇齿间流离,充满
	了小小的房间。

		在这个北风呼嚎的冬日里,她对我讲她家里窗外的桂花,讲雪后的洁白和毁坏,说花开了终于要落
	的,雪下了终于会被弄脏,不免悲戚和伤感。

		却也在心里升着带着蜜香的想象:那些盛开的桂花,被粒子发现时,深深地惊喜,在一个花香四溢
	的早上。然后,这晚上就没有那么冷。

		在等雪的心情里,我把回忆小心整理,归档。

		很多陌生了的人,又从新跳出来,站在曾经的雪地里,与我重新打一会雪仗,堆一次雪人。那将是
	奇妙和快乐的,时间逆流,我们再次认识,一个个完全陌生的彼此。

		所有的遗忘,也在这样的瞬间里被宽恕,有了退色后的情味隽永。\par
		最柔软的心,被寒冷包裹。\par
		很平常的生活,晶亮亮的,好像玻璃纽扣一样,被我用彩色的丝线,缝在吹灭了灰云的天空。


		后来的一天,我耳朵里塞着Faye的歌声穿越无人的操场。想起,几天前的某晚在这里,一刹那里望
	见流星。那是我第一次目睹它邈远急促的光芒。

		是冷色的,一次坠落急转,一次难以捕捉的美好。\par
		我来不及许愿,甚至来不及记得,那片刻的光亮。\par
		消亡入黑暗,留下原地的我一脸惊讶幸福。\par
		好多时候,我们是这样相遇的。\par
		在没有回过神的瞬间中。离别了最可宝贵的一切。\par
		于是,学会珍爱万分,不放过,一次展眼之间的美好。


		想睡在夜空,想向更不可知的天地飞去,用爱的魔力,用年岁的痴心。


		我以为,这样无雪的冬天不会太长了。\par
		晴朗的困顿终于会结束。


		现在的北京,只听见安静。

		05.12.某日某晚

	\endwriting


	\writing{气味。魔咒}{2005年12月25日 ~ 22:18:50} %<<<2

		谁吹熄了夜游的烛火,让你在茫茫黑暗里发现,小荷初绽的清香细细地从田田的叶间流散整个庭院
	。没有月亮,你也不需要月亮。只皎白无言的香味,便足以记得这个夏夜的全部温柔与风华。


		那些不可遗忘,无以遗忘的气味,仿佛一道道魔咒,禁锢着记忆的碎片,零落在时间无情的幽蓝色
	湖面。

		你时常乘着小舟游弋其上,没什么目的,没什么方向。

		有时,听着船下澌澌的水流,你就有了幸福,或者感伤,于是会顺着水势,捞起一两块来,细细把
	玩回想。

		将手插入水中的瞬间,那滋味,是冰凉,还是温暖,只有你自己明白。

		\blankrev
		你闻见,遥远的花香,在许多没有归途的季节。你看见,一个陌生的孩子,奔跑在石子路上,一朵
	朵纷飞的花瓣,撞上她洁白的额头。

		你仿佛并不认得她,她却向你跑过来,又一刹时停了脚步,睁了疑惑的眼睛看你,冲你笑。\par
		你于是蹲下身来,摸摸她的脑袋,问一句,你怎么会在这呀,怎么还不回家。\par
		孩子从背后掏出一朵明黄的小野菊,插在你右侧的发上,歪头看了看,一脸天真满意地跑开了。\par
		你认得她么,你突然怀疑。\par
		你站在溪水边照照自己,那朵小花明媚灿烂,你也学那孩子的模样,歪头笑了笑,一样地可爱。\par
		你默默问,我怎么会在这呀,我怎么不回家。\par
		你捧起水,洗一洗脸,那个世界竟就顺着溪流从你的指缝间流走,又消失了。

		\blankrev
		只眨一下眼睛,你正趴在母亲的腿上。\par
		午后的阳光洒在你们身上,散发出橘红色的淡香。那是太阳的香味。\par
		母亲的手轻轻地,万般小心地,在为你掏耳朵。\par
		疼吗,疼就说话啊。她重复着这样的话。\par
		你安心地闭上眼睛,听见那个小小的勺子,在耳洞里发出声响来。\par
		你喜欢这样的午后,有好的阳光,把世界照耀着。

		门外,正晾晒着棉被,它们将在那里,安静地铺展一整天,直到太阳落到山的后边去。这样,被子
	会有了阳光的气味,陪伴你紧紧拥着,度过许多个寒冷的睡眠。

		你似乎真的睡过去了,听不见那小勺子的声音。\par
		接着,是洗衣粉独特的香味。

		\blankrev
		那个衬衫白得耀眼的男孩子,在你身旁坐下来。\par
		他的红领巾很鲜艳,总是一幅认真的模样,有小孩子假装出来的严肃。\par
		你总是闻见,他身上的香味,起初你不知道那是什么香味,后来他告诉你,那是洗衣粉的味道。\par
		你于是奇怪,你的衣服也是用洗衣粉洗的呀,怎么没有香味呢,他说,他妈妈用的是一种叫“奥妙”的洗衣粉。\par
		你恍然大悟了,那是广告上播过的牌子,比其他的洗衣粉都要贵上一些。\par
		你喜欢那个香味,却不敢开口对妈妈说。

		于是,只有羡慕地闻着他身上的香味,一天又一天。\par
		然后,把那个男孩子深深记住又嫉妒了,好几年。


		后来,班里的同学都说你喜欢他。\par
		你怕极了,不再和他说话,不再向他借削尖的铅笔,甚至,不敢再看他一眼。\par
		只是,依旧可以闻到那洗衣粉的香味,飘到你桌上来。\par
		很长的时间后,你被罚做值日的一次放学,当同学都走光,他一个人悄悄从后门溜进来,帮你倒掉了垃圾。

		你还是没说话。\par
		听见他站在身后问,你真的那么讨厌我吗。\par
		你停下手里的板擦,咬紧了嘴唇,却不敢回头。\par
		你闻到,洗衣粉的香味。\par
		他说,也许你讨厌我,但我是喜欢你的,别和别人说。\par
		然后,抓起书包,跑掉了。只留下你,傻傻对着黑板,嘴唇几乎咬出血来。


		长大后,你家里也开始用那种洗衣粉了。\par
		你的衣服上也有了那种香味。穿白衬衫的小男孩却永远消失掉。\par
		只在偶然的时刻,想起他来,然后轻轻地一笑。\par
		你想,你应该也是喜欢他的,至少,喜欢他衣服上的香味,至少,是不讨厌的。

		那个面对黑板的下午,你心里大概是开心的,虽然,之后的日子,你依旧没有对他说半句话,直到
	毕业,直到你们各自消失在各自的世界。


		\blankrev
		那个孩子是你吗。\par
		你不敢辨识这一切。因为距离的阻隔,因为时间的无情。\par
		幽蓝的湖水仿佛要将你吞噬,不再给你机会去回忆,你捞起一块块碎玉样的片断。\par
		那些无家可归的气味,在这里,那里,充斥了天穹,化了颗颗明星,组成璀璨而冰凉的河汉。\par
		你仰视屏息。

		是什么力量,把你抛在这个无限未知的天地,是谁,带走你曾经真实触摸着的所有,都锁在这湖水
	的深邃?

		你不能够解悟,独独地仰卧在舟上,想让自己无所留恋,无所挂牵。

		然而,只一阵蓦然而起的南风,又把你,和你的思绪卷向无垠,好像被龙卷风卷走的桃乐丝,你在
	不存在的,却分明真实的另外世界,经历着一场场惊喜,一种种悲伤,一次次冒险。


		我要记得此刻。

		若我无法记得你的容颜,那么便记住你的气味。

		良送我

		GUCCI的香水,ENVY ME。作为甜蜜的纪念。\par
		我喜欢的清而纯净的香味。我喜欢的粉红色花纹的玻璃瓶子。

		看他细心包装好,又郑重放在我手里。良的善良和深情,仿佛被装在了这小瓶子里,等着全世界的
	人,来把我嫉妒。

		我记住了,这香味。\par
		他也会记住,属于我们的每个日月,和香味。\par
		谁细数这四季的变迁,谁对我说,别忘记穿上毛衣。\par
		谁把我们真切爱过了,又默声离开。

		\blankrev
		我闻到一个个花团锦簇的春天,闻到你给我的糖果的香甜,闻到被我空掷荒度的年华,开出单纯圣
	洁的芬芳。

		我很幸福吧,在多数的时候,我这样满足,心怀感激。因为,有记忆,有期许,混合在无处可逃的
	生活里。

		我很想,可以拥抱着整个冬天,把冰凉的鼻尖贴在冻结的湖面。这样,我与那些碎片便没有任何间
	隙,没有任何分别。

		\blankrev
		你吹熄了夜游的烛火,让我在茫茫黑暗里发现小荷初绽的清香。\par
		你仰头指着河汉的灿烂邈远说,那是你今生前世全部的记忆,全部的爱,和光芒。\par
		我深深信了,歪头微笑着看你,像那陌生的孩子一样……

	\endwriting


	\writing{初雪}{2006年01月01日 ~ 21:59:33} %<<<2

		一个灰白的早晨,北京落了雪。

		我被小鹿从梦里摇醒,仿佛,是在依稀的梦里,看到斜飘的雪花从窗口飞离。这一天,2005年12月
	31日。我们一脸欣喜和惊异地,站在年份的末尾。


		我问莫,如果昨晚我们再多说一会话,是不是就可以第一个发现了雪。她说,会的,只是后来我们
	都累了。我于是回想,三个人坐在楼道里时的情景,午夜已过,只有安全出口的灯,还固执地亮着。莫
	和小鹿,抱头而哭,我拉着她们的手,没有言语。许多无可名状的情感,在这无底的深夜,一并喷薄,
	盛放。我们不知道,在这几小时之后,雪会静默里积在我们的屋顶,不知道,是一些什么,把各自的故
	事,悲伤或遗憾的所有,稀释融化,成了咸涩的眼泪。


		我说,有些人,拥抱时很远,有些人,思念时很近。

		靠在小鹿的肩上。这个坚强得近乎倔强了的女孩子,正热切而痛苦地爱着。她说,如果,他会是幸
	福的,那么她付出的情感便没有白费,她便可以如他所愿地放弃。我听了心碎。闭了眼睛,有一个世界
	,向我靠拢,又向我告别。那些爱情,那些年少的决意,我们如何明白,又如何无有保留地投入,哪怕
	粉身碎骨。我想,这一切是困难的,却又是无比美丽的。


		本来,我们可以一直坐到天明的,可以遇见,第一朵雪花。

		如莫说的,只是后来我们都累了。太多事,不是我们不想,而是无能为力。

		于是雪在我们的梦境里落下,落下,等着你醒来,好在你睡意朦胧里,造一个世界的玉洁冰清。雪
	花,你是善良的。穿着睡衣站在阳台的我,惊叹得喊出了声音。


		我突然明白,有时候我们需要晴空,有时候我们需要冰凉。

		我会很勇敢,从薄薄的雪地上走过的时候,我这么决定。

		而那些雪,是昨夜的泪吧。

		这一年就这么过去,来不及整理。

	\endwriting


	\writing{自习室。几日}{2006年01月12日 ~ 22:01:23} %<<<2

		我想,我们终于都可以美满富足地生活,在心灵,在别人不知情的笑颜,心照不宣。

		层楼之上的层楼,一片安谧中,自习室的绿窗帘被日光照耀,书卷在手。\par
		这个下午,我读着自己,流溢四处的心思绵密。\par
		这个冬天,让我们都仔细地看书,仔细地生活,不假思索。\par
		所有人都沉默。北语,在这样的时候,才瞬间清净下来,褪尽惯常的浮躁喧哗。\par
		我于是喜欢5层的自习室,还有,那一个个昏昏欲睡中,头脑充斥了幻想的午后与夜晚。

		在那里,偶尔也写一两笔日记。\par
		芜杂的语言,生长在字句,许多宝贵的光阴,就这么被我空掷向独自的境况,快乐,或者寂寞的。\par
		转头,看身边的小鹿,想关于青春和女子那些古旧的问题,想我们的世界,还剩余多少缝隙,来藏匿我们各自心爱的面孔和声音。\par
		会是义无返顾的吗。\par
		多数时候,我们还没有足够的勇气,来承担,所有的细枝末节,所有的所有。

		有的下午,我们站在自习室外走廊的窗口,看天井里漏下的天空,灰白的,或者是无情的浅蓝。\par
		后来,会有飞机飞过去,留下长长的,云朵一样的细长尾巴。\par
		我说,尾巴会是难过的,因为飞机的离开。\par
		小鹿说,尾巴会是幸福的,因为被我们看到。\par
		两个女孩子,一样是稚气未脱的模样,在窗边的暖气上温暖十指,像个文青似的,说了许多矫情的话。

		她说会和一个不曾有爱情的男孩去海边,想象在阳台上偶然回首的自己,看到皮肤被日光晒得黝黑
	的他走过来。

		我说,若田老了,若田那时身边还有爱人,并且是相爱的,她就带他去海边,把对爱的誓言和对生
	的遗言,一起写在无人的沙滩。等着潮水涨起来,冲刷掉一切,归于完整的平滑。他们将在那时候跳海
	自溺,在珊瑚的身旁,恒久中睡眠安稳。

		再后来,我们又回到自己的座位,看书,看西边的太阳掉下去,通红浑圆的样子。\par
		那天,我们都很开心。只因为许多简单却无可名状的东西。

		在冬天,我们都很冷静,或许可以唱一支没有调子的歌。因为,这生活的真实,和懒散。\par
		日子过去,绝望与希望轮转不停。\par
		一些无意义的文字,像垃圾一样被我生产出来,又弃置不理。

		然而,小鹿,我想,我们终究都可以是可爱的人,并深信不疑。\par
		都是活在幻境里的人,于是,我们有了新名字,我叫梦话,小鹿叫呓语。

		有时,睡着,却比貌似醒着的,更为清醒。


		\blankrev

		*正在播放的---陈绮贞《小尘埃》

		田从16岁开始迷恋的歌

		有好的回忆,可堪细品琢磨。朱在她的叶子上贴了GIGI的《关于爱》,于是,我想到曾经的时
	候,曾经耳朵里的声音。

		田很微小。


		\longpoem{}{}{}
		他在五点十分醒来 \\
		他在清晨时候离开 \\
		透明的玻璃窗沾满小尘埃

		我在这里 \\
		比黑暗更深的夜里 \\
		张开眼睛陌生人寄来一封信

		我在清晨时候醒来 \\
		带着他的暗示离开 \\
		在跌倒的地方 \\
		勇敢站起来

		我在这里 \\
		我在你昨天的梦里 \\
		一片乌云一座神秘的小森林

		谁来拥抱我 ~ 保护我 \\
		或是伤害我 ~ 放弃我 \\
		拥抱我 ~ 保护我 \\
		带着我逃到黑暗的尽头 \\
		拥抱我 ~ 保护我 \\
		或是伤害我 ~ 放弃我 \\
		拥抱我 ~ 保护我 \\
		带着我逃到黑暗的尽头

		等着他

		他在午夜时候回来 \\
		带着忧伤的歌把回忆敲开 \\
		我在这里 \\
		手提着沉沉的行李 \\
		迷失在我和你未完成的旅行

		你在哪里 \\
		一道解不开的迷题 \\
		隐藏在我和你未完成的作品

		隐藏在我和你未完成的作品
		\endlongpoem

	\endwriting


	\writing{圆圈}{2006年01月15日 ~ 16:01:53} %<<<2

		在书屋上读到辛笛的诗,《航》。有这样的话:

		\longpoem{}{}{}
		从日到夜 \\
		从夜到日 \\
		我们航不出这圆圈 \\
		后一个圆 \\
		前一个圆 \\
		一个永恒 \\
		而无涯涘的圆圈

		将生命的茫茫 \\
		脱卸与茫茫的烟水
		\endlongpoem


		心中竟就瞬时安静,我说,此刻令我魂灵得以满足的,唯有平和淡定的心境。而生活,以我们惊异
	的模样,绽露眼前。冬天灰白地浮在半空。看这人间的荒诞悲喜。你我又何异于水上行船的水手,在夕
	阳残霞的粉红里,听着风声悲啸,问起雨和星辰。眼泪或者欢笑,都如可笑可叹的表演,何须在意。似
	乎,我只有在退回自己的世界的时刻,才能够真正懂得,人本不需要那么许多。独自的周遭,一如彼岸
	花树,枝繁叶茂。


		谁对我说,知道吗,命运是无际汪洋,你只可以沉浮其中,却永远游不出去。这些无源无尽的水流
	啊,教人黯然销魂。我从这里经由,我向远处告别。每一个简单的月份,每一季日期的更迭。看旧照片
	的时候,我才记起你,原来,我们曾经如此快乐和年轻过。从前的一切已足够美好,只是后来,注定的
	失散,无可奈何。那些人事,仿佛不曾真实过,仿佛是你们,穿着草编的鞋子,走过我栽满桃杏的堤岸
	。鞋子会湿了,沾了花香,人却渐行渐远,消隐在彼此的生命。我们好似画着圆圈,跳进去,又自己把
	痕迹抹掉,一次又一次。


		她说,后来的幸福,是淡忘。唯有淡忘,可以成全,此时的快乐。

		我笑了,用笔记在日记上。原来,我们不需要谁的赠与,和救赎。一月,田写下,所有的囚笼都是
	我们意念的制造。对Sky说,我相信,人多数时候是被自己围困的。总是说,要幸福呀。总是微笑得甜
	美可爱。而那,是真实的吗。我的需要,处于无限度的假想。或许,也已经得到。对于不可碰,与丧失
	的所有,何必苛责。终究是个圆圈,可以是受困,也可以是享受。


		晚上的公车空荡。冬天的夜晚,马路上灯火辉煌。车子飞驰,我们轻轻唱歌,Faye的,不爱我的我
	不爱。“我只能感谢,你能够给我的一切”。我无力抵御,自己微弱的声音背后的脆弱。我说,我需要一
	只胳膊。只是,想紧紧抓住些什么。用浑身的气力。即使,我已精疲力尽。

		\longpoem{}{}{}
		将生命的茫茫 \\
		脱卸与茫茫的烟水
		\endlongpoem

		人们把这两句刻在墓碑,作为辛笛的墓志铭。逝者已去,而生命的所有,正以相似或相同的轨迹轮
	转,恰如圆圈。人离开的时候应该是干净的,不落遗憾,和污垢。如果一定要执意留下什么,那么便是
	文字的默声无言吧。


		生命玄奥,参悟者难免有“智慧的痛苦”。而人心是经不起诱惑的。对于不可明言的未知,我们苦苦
	探求。而没有人真正发现全部。如果,我尚存小女子的悲伤无助,那么也请原谅。毕竟,我努力去思想
	过了,只是,我没有获得答复。我只能够,手握白粉笔,在地上反复画着圆圈。


		读诗经,喜欢那陈风里的一句:

		\shortpoem{}{}{}
			彼泽之陂,有蒲与荷。\\
			有美一人,伤如之何?\\
			寤寐无为,涕泗滂沱。\\
		\endshortpoem

		我们的血液,流淌多少深情的温度,用来日夜思念。我们曾怎样哭泣过,大雨如注。跳进圆圈的我
	们,是被迫的选择,也是自愿的沉溺。


		我终于不能够足够坚强。然而,这个一月,我有所希望。

		这便是美好的,值得莞尔一笑。

	\endwriting


	\writing{彼岸}{2006年01月20日 ~ 19:15:09} %<<<2

		\longpoem{}{}{}
		看见的熄灭了 \\
		消失的记住了 \\
		我站在海角天涯 \\
		听见土壤萌芽 \\
		等待昙花再开 \\
		把芬芳留给年华 \\
		彼岸没有灯塔 \\
		我依然张望着 \\
		天黑刷白了头发 \\
		紧握着我火把 \\
		他来我对自己说 \\
		我不害怕我很爱他
		\endlongpoem

		Faye的《彼岸花》,荒凉得教人心疼的歌。

		在我们每个人的汪洋之上,是否,都存在着这样一处彼岸。度过,便是花树繁茂,灯火辉煌。

		而彼岸,未尝不是一场醉心的谎言。如果,你一定要我相信,那么我便彻底虔诚,尽全力投入一次
	无目无求的奔赴。

		彼岸花朵,是毒,是蛊惑。你我却甘心沉迷,义无返顾。

		朱讲到名叫迷迭香的植物:

		“迷迭香具有防老化,增强心脏和大脑的功能,所以,它可以帮助人们增强记忆力,留住回忆。

		莎士比亚的戏剧中曾经写到[迷迭香能帮助回忆,亲爱的,请牢记在心]”


		我想,若真有彼岸,那上边一定遍布这种神奇的花朵。

		让我们记忆,所有的幸福,或者难过。

		让你用苍老的容颜,深刻爱人的只言片语,如花笑靥。

		许多的岁月过去,许多的惆怅,在窗子前开成雪花,又兀自融释。我们只有被昨日拯救,获得希望
	和思念,用来歆享余生。

		她说:“我很想拥有这般能留住回忆的花草。几十年后,当我不再年轻,也许那时已是疾病缠身,
	我可以耳聋,也可以眼瞎,但是我不能失去记忆。如果我有这样的植物,真的能够让我留住回忆,那么
	,我不再害怕年老,不再惧怕死亡,因为我可以记起年轻时,即便那些人和事带给我的曾是痛苦,但是
	我想回忆起来更多的会是快乐。”

		朱是这样勇敢而可爱的朋友。

		田说,记得,也是需要勇气的。无论幸福,还是痛苦。

		你情迷彼岸的花朵。就等待光阴把我们从原地带走,不见踪迹。

		看见的熄灭了 ~ 消失的记住了

		我们轻握生命的片刻,度过微小的叹息和愉快。这一程水路,我愿与你同船,并庆幸,有你同船。

		而有时,此岸恰如彼岸。

	\endwriting


	\writing{之后。愿望}{2006年01月24日 ~ 21:34:07} %<<<2

		之后的夜晚,我们的世界火树银花。我点小小的花火,就让时光记住了一切。

		顾城说,让列车静静驰过,带走温和的记忆。

		因为住在火车道附近,我从小便喜欢看火车。好像海边的孩子看海潮的起落,山中的孩子数云朵的
	游移一样自然。它呼啸而去的时刻,站在原地的我,显得渺小不真。火车,载着多少热望期许,又了结
	多少远方的悲伤。它真的带走我们的记忆吗。那个坐在操场双杠上看火车的孩子,是永远消失了,在墙
	外杨树一年年,迎风的瑟缩歌声中。


		今天的天空,灰白地覆盖,仿佛错过的幻觉。

		那些深爱过的故事,被我们的冬天,在玻璃窗上轻轻哈气,又轻易擦去。我仰卧在床,想起许多模
	糊的侧脸。想起四月,淡绿的一抹,在我的头顶盛开桃红的花枝。远去的自己,站在陌生的教学楼,阳
	光蒙在我的身上,像一场甜美的阴谋。记得,你曾陪我坐在这楼梯上,你说,有一天我们都会想起这个
	时刻。


		之后,让我,把谎言般的记忆,讲给你听。让我们都笑了,感激着生命的恩赐。

		有时,我们相对而坐。喝一杯茶,讲新的旧的感情。你还是温柔可人的模样,只是经过的人,一个
	个错失分别。我想起我们的十六岁,想起那个冬天的我们,一起抱着大兔子,走过十字路口。红灯灭了
	,又亮起。我们在许多的路口停留,又各自放弃,奔赴希望。我想起,两个女孩子的默契,心照不宣而
	明净于心。

		我确实在感激,生命的恩赐。让我可以挽着你们的手臂,一路欢笑歌吟。

		火车,带走你,火车,带走我们的生活。向着不可知的深渊,托付你我微小的青春。答应我,要幸
	福,无论海角天涯。只有这样,我才能够安心,微笑着心满意足。当我望见火车,便会想起你。在深夜
	的列车上,你或许睡了,轻靠在陌生人的肩。你是微笑着,大概,正有童话的梦境。

		你却又闯入我的梦境来。坏笑着调皮。

		我不知道能够说些什么。这时候,我的心停在静到破碎的渔港。流云正流过天际的莫测。

		之后的你我,会站在隔岸的灯火里,各自懂得了许多,几分疲惫,却满足而平静。我想,那会是美
	好的结局。当我们都老了,仍要围坐着,看彼此的脸上又多了几条皱纹,互相骂一句老妖精。

		这是我最可爱的愿望。你要帮我实现。

	\endwriting


	\writing{转瞬光焰}{2006年01月29日 ~ 22:14:05} %<<<2

		消散瞬间的绚烂,留四处火药的硝烟在空气弥漫。

		这一夜,和母亲捂着双耳躲在角落,看深暗的世界中,繁花丛生,拥挤人们期望的目光。

		我们的城,在某一特定的时刻,幻化成光焰火蛇营造的绮丽梦境。你伸出手去,是不可触碰的幸福
	,混合在鞭炮的呼号叫嚣,引人下泪。

		我看到,你们盛开,遮蔽了我的窗口。\par
		烟火之美,或许,正在于美到窒息的爆发,猝不及防。\par
		你短信说,你看醉了烟火。我想,这山水茫茫的两地,上演的大概是同一场,虚无绝美的幻觉。

		记忆中的烟火,似乎要小的许多,不像今晚。只是孩子手中的点点光亮和火星。家门口的小卖店里
	就可以买到。价格也是非常便宜。

		童年的春节,仿佛瓷娃娃脸颊上的两朵桃红,热烈甜美,而不失天真。

		爸爸会在除夕的夜里放一挂鞭炮,我们就扒着窗玻璃向院子里看,心随着炮声咚咚地跳,炮声熄了
	,热饺子也出锅了。

		现在的我,也时常想念。

		那时候,还会和哥哥姐姐一道去庙会,买回木头的刀枪,撕打一番,或是几个人学了大人的模样搓
	几圈麻将。

		如今的我们,都也成为了大人。各自奔忙,却不知所向。

		我们,大概很难再一起拉着手,跑到庙会上花几块钱去玩一次套圈了,很难在一无所获之后,再满
	心遗憾地回家,撅着嘴骂套圈老板把好看的玩具都放在最远。

		有些时间,是我们无可挽回和追悔的,只要,在当时的我们,有所记忆并深感温馨,就足够的了吧。

		是的,我并不能够奢求,你的停留。

		仿佛火车的驰去,原地的我,不过轻轻地挥手,祝福你的旅途。\par
		你将经过风景,在深夜或清晨,会有放羊的孩子,站在荒芜的山梁,对你善意地微笑。

		而我,会尽我全部可能来想象。一处纷落的雪白,一处日暮的彩霞,很多很多,属于你的快乐,和
	难过。

		我们的记忆,我们的童年,在彼此的生活里刻下深印,各自封存珍惜,却从不提起。这是我们,最
	美的默契。

		大舅带他的小孙子去看一棵洋槐树。

		他指着楼房间孤零零的一棵树说,那是爷爷种的,原来爷爷的家就在那棵树的地方。房子,早已不
	见,人,早已搬离或者逝去。只有树,那幸存的一棵树,近乎倔强地执意生长。

		它在等待什么吗,是否,就是一个双眼洞张的小男孩,像曾经把它种在这里的那个小男孩一样?

		我听着大舅说起老家的旧址,我看着他老去的目光里饱含了岁月的温情。树会知道,那个小男孩已
	经是有了皱纹的老人了。

		那些绚烂的烟火,点缀了寂寥的日子。而日子,仍旧把一切带离,从未迟疑。\par
		这一刻的自己,想拥抱着什么,用心来深爱。毕竟,我没有那么许多时间,再弃置挥霍,无所顾及。

		认真地度过,如一株植物的萌发和挺拔,那么,即使是烟火熄灭后的黑暗,我们的心也能够和平富
	足,繁花迭起。

		在盛开的光焰迷人中,世界的轮回,变得轮廓清晰。

		能够掌握的,只是这呼吸的片刻。\par
		转瞬而去,你的面孔,我的面孔,不无留恋。

	\endwriting


	\writing{给朱朱}{2006年02月03日 ~ 22:07:00} %<<<2

		眼前的你,还是一样,瓷白的脸孔,七分天真,三分狡黠地笑。

		这让我想起,很远的自己,很远的你。坐在学校月季花坛前吃冰棒的两个女孩子。白衬衫,无风的
	夏季,一个又一个喧嚷中寂静下去的日子。那么真实,仿佛一张瞬间的留影,我们的所有,没有刻意记
	忆,却成为恒久。


		你会想念吗。那时侯,生活的简单。

		爱迟到的田,爱生气的朱朱。她们是最亲密的朋友。一起给老师们起奇怪的外号,一起在语文课上
	乐作一团,一起在楼道里犯傻,又不幸被同学看见。我们,仿佛是永远的小朋友,学校,是我们的幼儿
	园。

		我们善于自娱自乐,把同学想象成各种动物,通过气味判断中午食堂的伙食,把塑料袋吹得高高,
	并且比赛。你我总会有突发奇想,在枯燥的学习中。


		那个年纪的你我,因为无所知,便无所奢求。所以,可以无所顾及地与人群逆行,并以此为乐。

		我以为,一切的一切都不会离去,或者,我从未想到过那些。


		校园里的树,摩擦着空气,沙沙作响。一个个季节,从树枝间穿越,飞离此刻。在田生病在家的时
	候,老师会说,看见了朱朱,总觉得田就该在不远的附近。我们是如影随形的朋友。这令人嫉妒。

		不知道,你是不是忘了。那年的11月,一个下午,我们并肩坐在通往顶楼的台阶。你说,会有一天
	,我们将想念这个时刻。我把这些记录在日记。


		现在的我,是在想念了,或许,正是你说的时刻。

		11月的天空,浅浅的灰色,玻璃窗上布了雾气。来年的春天,我们就将起身,各赴前程。这很残酷
	,生活在我们还未来及珍爱的转眼,就已将所有改变。


		我们的教室,它依然在那一座教学楼,那一扇房门后边。

		然而,当我再站在它跟前的时候,却明白,它早已不是从前的那一间了。它很陌生,我们的教室,
	随着消失的伙伴和同桌,一并不见。

		它在时空的某段漂流,无家可归。

		旧时的老师依然,我却分辨不出当时敬爱或者憎恶他们的模样。


		都与我们无关了,像所有的过去一样。湖光山色,浮光掠影,你我不过小小的扁舟一叶,经过了春
	天的花树芬芳,又错失一场艳夏的风华。


		留下幸福,也留下纪念,不过如此。我的深情,已如大雨倾注后的天空,彩彻云霁。

		田不能够是诗人,不能够在冷却的事事中,读出飞翔的诗意,为我们的青春歌咏。田只能在季节的
	离别后,在你的心里,我的心里,轻轻惦念着,并不感伤,并不遗憾。

		因为,有你在那里,瘦小的模样,用最真挚的情谊等候时光的无情。因为,有我在那里,在每个记
	号上,为你画一张小小的卡片,用粗糙的笔触,绘成你我完满的快乐。

		田已经足够幸运。遇见太多的明亮甜美。

		记得,我们一起在冬天洗草莓。偷逃的体育课,无人的男厕所,冰凉的水。你推开我,说我的身体
	不好,别动凉水。洗好的草莓,被装在木头的小碗里,我们相对坐着,看这一捧鲜美,都不忍下嘴。而
	那一天的草莓,是我吃过的,最甜的草莓。

		那鲜美的模样,仿佛生活的隐寓。我一直记得,并深感温暖。

		朱朱保护着田。我知道。


		田生病的时候,在最黑暗无光的日子,朱总会用孩童一样的天真,驱散乌云。田不愿说感谢,因为
	,彼此的心灵,能够明白一切。那些,是无须言语的表白的。只在嘴角的轻扬,在眼底的光亮,我们就
	足够知晓,对方的心思。我瞒不过你,正如,你瞒不过我一样。

		后来的我们,懵懂里,都有了各自的爱情。

		而那,你说,大概并称不起爱情的。

		小鹿也说,爱情是有重量的,只有你为对方义无返顾地付出了什么,才有资格说起。

		那么,我想,姑且让你我以为,我们已经义无返顾地付出过了吧。毕竟,有难过,有纠缠的痛楚,
	流在了血液。

		如果,那些都是真的,那么两个小女孩,真的长大了。


		静回来的时候,我们坐在一起喝茶。说起的是各自的情感。

		明明是三个女孩子,我却仿佛看见她们身后隐藏的另外三个,或者更多的男人。

		爱情,让女子头晕眼花。我终于明白了,这一切的可悲之处。


		为什么,我们要把自己塞进某一个人的心里,才心满意足呢。为什么,我们不能甘心让他保藏在我
	们的心里,却永远隔离在我们的生活之外呢。也许,这是无谓的问题。

		没有人,不去期望爱的报偿和回应。然而,正是这,成为我们的苦痛之根。\par
		毕竟,我们无法控制他人的心思,特别,是这世上多变又虚荣的男人。\par
		女子的悲剧,总是由痴情开始,由痴情告终。


		银河凋谢成寒夜的1月,田写,相逢前,离别后,万念俱灰。\par
		爱情是如季节的。你不可以怨恨风季的匆忙,也就不能够遗憾爱人的离去,深情的消散。

		我说感情虚妄,人心虚妄,你不信。你说那些幸福是那么真实。而爱情确如烟火,绚烂美好,却经
	不起时间。缤纷去后,又是绵绵黑夜。

		幸福,不是因为谁的给予,谁的守侯,幸福,是我们自己捏造的幻觉。\par
		只有我们自己,可以把握。\par
		曾经的你,想放开抓不住沙子的双手,而望见双手的空空,你又不禁悲伤。\par
		那些注定相逢的人。是我们命定的快乐,还是遭受呢。\par
		那些不可挽回的昨日,是你我亘长的失去,还是一无所失的圆满呢。\par
		如果我们可以以为,那些爱过的人,本不是必然的经过。或许,许多的不舍便能够释然。\par
		虽然,你会问,既然如此,又何必遇见。


		原来,全部的故事,只是不可琢磨的圆圈,从没有他的生活中起始,又回去没有他的世界。这看上
	去,是多么公平的事。

		仿佛一次无目的旅途,风景过后,我们又回去最初。没有什么值得去感伤。

		要多爱自己。最爱自己。选择了,我们就要勇敢地承担,无论是幸福,还是痛楚。正如你说,这是
	成长的必然经受。


		最初的田,最初的朱朱,还坐在月季花坛前吃着冰棒。\par
		那一个你,那一个我,还未遇见那么多荒唐的相爱和别离。\par
		无风的下午,万物明净,心如晶亮的水晶,在阳光里透明。\par
		我们真的不曾失去什么,因为所谓的那些拥有,都已是电光石火后的如梦云烟。

		那个时候,我们因为不奢求,而能够拥有所有美好。\par
		现在的我们,在生活的脸上擦着胭脂,它不堪地微笑,那模样与我们的期望,差之千里。\par
		何必强求美丽和完整,无所欲求,或许,我们的心灵才能安宁。\par
		不要让我听到,你难过的话语。不要让我知道,谁伤了你的心。\par
		我愿,你永远是眼前的你,瓷白的脸孔,七分天真,三分狡黠地笑。\par
		要知道,任何的悲伤,都不值得我们失去快乐的天性。享受生活给我们的一切,而不只是接受。

		你会幸福,我会幸福。\par
		我们都会幸福。答应我。


		\blankrev
		田 ~ 零六年冬末


		\blankrev \blankrev
		对不起,又写了这么长。朱曾经抱怨,田写字太罗嗦,都没耐心看完了……


		那天收拾东西,找出你送我的水晶,一块是拼图,一块是小猪。拼图我戴在脖子上了,不知道你的
	还有没有。我的是黄色的,你的是蓝色的,我记得很清楚。

		明天就立春。春天真的要来了。


		喜欢昨天看的麦兜。笑得我们脑袋都碰到一起了。麦兜的生活像水彩画一样明亮。今天,我满脑袋
	都是那个老中医不停问的那句…… 嘿嘿,从略了,影响我主页的整体风格。

		我这花田半亩里,是不怎么施肥的,哈。……

		不知道你能不能读完这么许多,不知道你会不会笑了,觉得田真是个傻孩子。

	\endwriting


	\writing{雪天}{2006年02月06日 ~ 23:16:18} %<<<2

		北京终于漫天飞雪。

		清寒的早上,浓白的空气,用雪花的舞蹈将我围绕。就有微寒的凉意,小心滴落在皮肤,又无声里
	化开。

		许多的落雪,纷纷如此,令人心醉沉迷。只为那相遇的片刻,渗入冬日的中心,用一处雪白的天地
	,纯净了所有的恐惧和不安。仿佛,一声不刻意的安慰,一只温热的手掌,从我们的耳边滑落,在我们
	的肌肤融化。

		整个冬季,我都在等候,这样的时刻,从天而降。

		05年的末尾,雪在睡梦里飞进阳台。06年的1月,雪在我拉开窗帘的上午,撒满世界。

		每一次,我都心怀感激地站立在雪中。一年里,你只拥有这样几个日子,来触摸雪花的盛开。

		Faye唱,还没好好地感受,雪花绽放的气候。

		它们,是如碎玉一样,零落在生活的时间。难过的是,我们总是一再错失。来不及,在雪的包围下
	,放肆地哭泣和狂奔一次,来不及,把一只红色的气球抛向天空,完满童年的美梦。

		雪总会在你我未经意的缝隙里,悄然融化,悄然消亡,不见了冰清玉洁的模样。

		雪,像是上帝的礼物,又恍若迷题。

		你无法预知它到来的准确时间,你无法掌握,它的开放,和凋败。

		雪中的我,幸福莫名。又似乎在隐隐的地方,藏着无尽凄恻苍凉。雪的洁白,令我想到一切美好,
	又预感到一切美好的脆弱。

		对于种种热望,我不知,究竟该用怎样的神态来期待。许多时候,越是急迫,越是换来失望的空无
	。而往往,又恰恰是在那一种空无里,闪现了热望的结果,点燃整个天穹的光明。

		而我们拥有的,却永远多不过等候的过程。用折磨与挣扎,构建自以为是的甜美。

		这一切,正如雪,你不可以计算着日期,来把它期望,只能够装作无所用心。\par
		雪喜欢给你惊喜,而不是信守约定。\par
		或许,这也正是许多美好的意义所在。它们不曾约定,也就无须履行。\par
		这样,等候是人自愿的束缚,雪没有错误,痛苦,是人自己的选择。

		于是人,是不可以奢求太多的。\par
		他们说,最令人中毒的东西总是,才握时有,一撒手无。

		雪花,在我的手心一丝丝死去。鲜花的尸身,是萎黄不堪。而雪花的尸身,却是澄净的清水。

		这与生命同源的液体,同样存在于你我的肉身。于是,我想,在一定的温度下,我的血液是否也会
	结晶成花朵,开满骨骼与经络间呢。

		葬身雪山的攀登者,永久地睡在纤尘未染的圣洁,他们大概会知道答案。

		堆一个雪人,然后教她微笑,教她如何与路人合影留念。

		雪孩子,在冬天的路边,一脸天真无邪。你看到她,会觉得自己用双手创造了一个生命。你知道,
	雪孩子会懂得,她认得你,她正感谢你。

		想起,小时候看的动画片。一个关于雪孩子的故事。结尾,她冲入火海救出别人,自己却融化不见
	。记得那时候,每当看到这,我总会难过得掉眼泪。而之后,对雪人就总怀了异样的感情,把她们当成
	真的孩子,真的人,有善良而勇敢的心。

		人,是无法如雪的,那么干净。人却能毁坏雪,不几日,雪已是面目全非。看到一些雪原的照片,
	不禁悲叹。人的双脚,人的车轮,本不足以破坏她的圣洁,而千万的双脚,千万的车轮,却足够把雪的
	美丽在瞬间化为乌有。

		这城市,创造无数传奇和财富,这城市,又扼杀多少天真和烂漫。在更远更古的时代,当这世上还
	没有那么多的人,当人还在虔诚仰望自然的神秘,是不是,就没有一夜之后,满目的肮脏泥泞?

		他的诗歌在流传:君不见,高堂明镜悲白发,朝如青丝暮成雪。

		在如白驹过隙的日月,我们可堪固守的,又有多少,又值得几分价值。陷落于城市的红尘霓虹,你
	我连对星空的目睹,也渐成想象。

		哪一处,还能注满你灵魂的空杯子,像雪的融化。

		充斥了恐惧与不安的世界。我看到,你们争夺撕杀,不露声色,却比野兽更残忍凶狠。

		只有,在雪花绽放的片刻,只有,在寒冷锁了人间的房门。我才稍微安宁,从另外的窗口,洞见一
	个真的人间。

		北京终于漫天飞雪。

		整个冬季,我仿佛都用来等候,这样的时刻,从天而降。

	\endwriting


	\writing{十指}{2006年02月10日 ~ 20:15:39} %<<<2


		冬季的末尾,指尖温度尚存。肌肤之下,多少深情,暗自酝酿了,又悄然消亡。

		这十指纤纤,曾抚摩花季的风丝游移,触动爱人的臂膀,曾轻握初雪的单纯凉意,捏碎安静的光阴
	,在树影婆娑的白墙。

		那么多零碎的知觉,从指缝间流失,一去不返。

		如一把握不住的沙,不断地下落,下落。让你懂得珍惜分秒的相聚。


		二月的风,从我的屋角飞过去。

		瑟缩在棉被里,想像一只冬眠的动物那样,一睡不醒,直到淡紫的小花朵开遍山野。

		我伸出双手,投过灯光,它粉红透明。想起枫说,她习惯出门戴上手套,因为他的不在。

		于是,这小小的手,那么轻易地,成为你我的痛处。

		他们说,没人疼的孩子,总会双手冰凉。十指之间,藏匿多少女孩子的幸福或悲戚。

		这个冬季,有多少双冰凉的手,无人温暖,无家可归。


		所以。

		如果你有爱人,请你,一定温暖她的双手。

		让她知道,她是被深爱的孩子。


		他的双手,流出钢琴如水如光,他的双手,流出诗句惊世,他的双手,为她抚去发上的雪花。

		这一双手,可以被历史记忆,被艺术铭刻,也可以被岁月凝定,成爱人心上永不退去的思念。


		会记得,每一段指节的构造和宽度。

		会记得,每一丝微妙的气息,混了洗手液,香烟,或咖啡的淡香。


		我要如何,用这无力的双手,完满整个空的世界,这时间,这灵魂。触摸下一季的空气。

		只想,坐在四月的花园,让阳光撒满。只想,轻靠着他的肩,听他讲一个甜美的童话,却忽略结局。


		让那些沙,兀自下落。让我们安心实意,期盼一株花树,接受一场盛会的邀请。

		让我忘记,把天空吹成深蓝的晚风。


		我数着十指等候。我数着时光告别。

		我于是感谢,此刻的存在,让我终于望见幸福的机会。紧握双手,并不需要一枚,叫做诺言的指环。

		所有的所有,不过十指之内的默想,和计算。

	\endwriting


	\writing{细沫的凌乱}{2006年02月11日 ~ 22:02:21} %<<<2

		头疼整夜,忽睡忽醒,做离奇古怪的梦。被恶人追逐,又坠入深谷,然后有青蓝的天空,无涯的汪
	洋,我们的独夜小舟,穿越山河和大洲,迎送刺眼的黎明……

		不知道多久之后,我终于停下来。才复又清醒在无光的暗夜,昏沉却了无睡意。

		于是,有他们无端走进我的房间,点起篝火,与通体冰凉的我,围炉夜话。穿着当时的衣服,从我
	的记忆还足够到达的深处走出来,微笑或者沉默,说一两句早已旧掉的话语。火光映着面孔,我认清每
	一个人:童年的哥哥,表姐,病着的祖母,坐在座位前方的L君,还有和他相爱的女孩。

		一个充满疼痛,也充满爱与悲伤的夜晚,就这样被造就。

		哥哥和表姐丢失了当时的天真,他在奔波,她在劳碌。祖母在夏天永久睡去,我未及一句道别。L
	君带着他心爱的女孩,消匿在我18岁的冬天,不复相见。

		这是多么幸福,又多么残忍的夜晚。让我这样把来处的足痕仔细抚摩。

		让我得以,在人事皆非的此时,同当时的你们围坐倾谈。

		那些,都是曾经的如果。那些,都已是往事,疏淡在失去和获得的体触里,惟余微妙。

		小鹿讲的笑话:从前,一棵洋葱,走着走着,就哭了。

		为什么,所有的故事前边,总是加上从前。这个笑话,是禁不住细想的,你若从这里纵身而下,便
	会发现,谁不是那一棵洋葱呢。只是,哭的内容不同罢了。每个人都是母亲用水做的。真正的人,总是
	被泪水泡大了,一些人流出来,一些人独自饮下。

		空说,在抑制悲伤,而悲伤是因为思念,是命轮里一季又一季的风雨和脚步。我看着这些字,想起
	曾经一起走过的路灯几盏,忽明忽灭的。后来,我们终于也都会忘记。思念飞着,为彼此不同的期盼。

		再醒来时,天已大亮。澄澈透明的蓝。

		窗外,远处的屋顶尚有残雪。又是多好的一天。我深深打了个呵欠。

	\endwriting


	\writing{心}{2006年02月12日 ~ 13:19:27} %<<<2

		你用水彩绘画,那么多缤纷,芳香。仿佛爱情的模样。\par
		我用彩纸折出,纸的心房,来把所有的故事珍爱。

		之后,我们终于能够相视而笑,相拥而泣。\par
		之后,如水的日光下,你我的沉默如鱼。

		完好的世界,不需多一句的叮咛。\par
		安闲的几处梦境,在你的手掌盛开,投射在我的湖水。\par
		我要他门前的一颗石子,一捧沙土。\par
		我守着偌大的城,望一夜夜花灯如昼的繁华。\par
		一边想念,一边遗忘。


		很多文字,与真切的生活无关。\par
		很多思想,不着边际。

	\endwriting


	\writing{此恨}{2006年02月13日 ~ 17:57:58} %<<<2

		放假的时候,偶然拿到一本小说,王安忆的《长恨歌》。便顺手读起来,不知觉里有了情味。

		这一晚,上个世纪四十年代那个繁花锦簇十里洋场的旧上海铺展开来。竟叫在北方出生又长大的我
	,于空气间嗅到丝丝桂花糖的酥香。

		夹竹桃层出的花朵,老虎窗外,氤氲着女人脂粉气的天空,留声机里回转往复的“四季调”,有轨电
	车不休的“当当”,一切恍如隔世,却真实可触似的,从字里行间浮上来,一丁点一丁点地清晰。

		那个上海弄堂的女儿,王琦瑶,静等着她的时代,也用美貌与聪慧酝酿着一场致命的悲剧与传奇。
	她是那种有魔力让男人一见倾心的女子。因这,她有了非凡的经历,有了用来挥霍的资本和勇气,也有
	了数着日影度日的寂寞无赖,有了干涸的眼里惟余的一颗老泪。

		她被男人捧着,爱着,宠着。她穿着婚纱走上选美的舞台,却想着,“也许穿上婚服就是一场空,
	婚服其实是丧服!”。

		后来,王琦瑶的预感应验,她没有真正地做过新娘。她被命运的流变碾过如花似玉的年华,她在男
	人的世界一败涂地,无限风光的“三小姐”,用骄傲和那害人害己的聪慧与美貌,断送着一切幸福的可能
	。

		她知道,像她这样的女子,是不能够结婚的了。

		那镜里的美人,风韵不减,心却冷如灰烬了。最后的日子,她的依靠与慰藉,是那一盒金条。王琦
	瑶明白男人是靠不住的,却还是取了那装金条的木盒子,想把最后的赌注压在男人身上,“王琦瑶挣着
	手,非要开那盒子不可,说他看见了就会喜欢,就会明白她的提议有道理,她是一片诚心,她把什么都
	给他,他怎么就不能给她几年的时间?”。而一切,终成虚空。

		大概谁也不会记得,那间阁楼上住着怎么样的一位女子。或许,有人偶然想起,也会自然把她归为
	“那种女人”的行列。

		旧上海,沉浮着多少如王琦瑶的女人。她们住在如爱丽丝公寓那样令人无限遐想的房子,她们总是
	那些正经女人的不屑,总是街头巷尾四起流言的主角。而有多少女孩,又终是经不起那花花世界的诱惑
	,义无返顾地纵身跳下了呢。百乐门的歌舞不休,这不夜的城市,充满了纸醉金迷的气息,由不得人清
	醒。甚至,那些正经人,话里话外也在羡慕着这许多的王琦瑶,不屑的口气里竟有着嫉妒的成分。

		住进爱丽丝公寓的,总是抱着女人中的佼佼者一般的姿态。她们多有一张光鲜可人的面皮,有足够
	多的光阴和年轻。而这些,或许已经完全能满足一个女人的虚荣。

		她们仿佛是被特别爱着,眷顾着的。又分明是被幽禁与弃置在了华丽的囚笼,一半是这空荡的公寓
	,一半是那浮华过后的虚无幻灭。她们却永要保持着那佼佼者的姿态,哪怕生活已经满目疮痍。

		我隐约读出女子的悲哀。几千年都是一个模样。正是那句红颜薄命吗。

		美貌是人们梦寐以求的,获得的人却又往往因它围困了一生。“坏女人”多是漂亮的,“那种女人”少
	有不是独具风情的。有人说,红颜的不幸是男权社会的罪孽。萧红说,她一生的不幸,都是因为她是个
	女人。

		在那个旧上海,我却看到女人是甘心被男性控制与评判着,并以此为荣似的。好象王琦瑶的妈妈说
	的,她的贱是自己作的。而悲剧的始末又怎能归结于一个弱女子的自轻自贱简单了事?

		王琦瑶只是那茫茫背景下,一个随波逐流,无依靠无目的的浮影罢了。

		她以为男人能给她世界,却把世界给了男人,一无所获。她是太美的女人,连自己的女儿也要心生
	敌意。可见,女人确是生来的天敌。这个充斥了红男绿女的人间,何处才是得以喘息的港湾呢?王琦瑶
	心里没有答案,只是紧守着那一日日削减的风韵,徒然地经过,又消失在摇晃的灯影里。

		人们读王安忆的上海,就想起张爱玲的笔下。那里,也有一个上海,花园洋房,绅士淑女。而张的
	上海,透着细小微妙的精致,有小姐呼吸里吐露的香水气。她讲的爱情,总那么勉强与无奈,又带着狐
	步似的优雅和轻佻。

		这长恨歌里的上海,总觉得小家碧玉一样的真切。那些姑娘家细碎的心事,弥漫在弄堂的姨娘们的
	不满与闲话,一个衣冠楚楚的少年,一句无心的却引出伤心的调侃,那么自然平常,又处处隐着悲剧的
	伏笔。

		王安忆在讲的故事,是人与人无端的相聚与失散,是爱与虚荣的悲怆和无情。她没有一针见血,却
	一点点撕开伤着的皮肤,露了血肉给你看。

		这长恨歌的“恨”字不知究竟该哪一个解法才合适。是遗憾吗,若从古汉语的角度来看。还是仇恨呢
	。最后的王琦瑶,横陈床上,因那一盒金条断送了性命。小说的最后一句“对面盆里的夹竹桃开花,花
	草的又一季枯荣拉开了帷幕。”是否在预示着什么,那夹竹桃是否在等着另一场悲剧的开演,于这城的
	艳美和躁动中。

		读了小说,紧不住找了电影来看。

		关锦鹏的片子,郑秀文主演。然后,我有了无限的悲哀,为文字里那个王琦瑶,为已经萦绕在我嗅
	觉里的桂花糖的酥香,为王安忆,为了茫茫如我的读者们。

		当然一部改编的电影,你自然不能去给它许多期望,因那期望多数是要落空,让你摔个精实。

		但我实在无法容忍,粗制滥造的电影改编了,当然,或许编剧也是下了苦心的,并不是随意。而呈
	现在影片的结果是,我不忍把它与我读到的文字和情节做任何的联系。

		我只有想,根本不是从那小说改编的,心里才能稍微平静。那一个样貌痴憨的王琦瑶,扭捏作态的
	王琦瑶,莫名其妙就和男人躺在床上了的王琦瑶,不是我望见过的,同情过了,也遗憾与气愤过的王琦
	瑶。

		银幕上,没有那个上海的丝毫痕迹,没有传奇,惟有作为导演的失败和悲剧。

		此恨绵绵,我很遗憾,我去看了电影。它把我所有的想象和幻觉都撕得粉碎,体无完肤。


		王琦瑶,该是那永远锁在文字深处的弄堂女儿,该是永远的沪上淑媛,在旧报纸上微笑。让鸽子飞
	过,让它们知晓人间那些亦真亦幻的风闻与传奇,秘密和悲歌。

		女人的不幸,一世世继续,一幕幕上演,你看如花的女子,在镜台前的梳洗,听她们妩媚的裙角略
	过风丝的迷人和婉约。任男人用尽全世界的财富和殷勤来把她们热爱,也任男人不加思索地回转了身躯
	,把一切抛掷和遗忘。

		不过这么回事。男人女人,不是你折磨了我,便是我荒废了你。

		我不再去想那电影,只希望各位导演们,今后手下留情。

	\endwriting


	\writing{午后}{2006年02月14日 ~ 14:40:24} %<<<2

		这个午后的明亮令我吃惊。

		坐在地板上,剥一只桔子。我的狗在脚边睡着了。阳光的香味,弥漫四周,让人升起感激和希望。

		看窗外的云朵如水流去。一会儿遮蔽了日光,又瞬息里散开。房间忽明忽暗,仿佛六月里的阴晴不
	定。

		为桌上的植物浇水。一棵白菜,一株杜鹃,还有两盆不知名的小花。它们欣欣向荣的模样,翠碧的
	叶,娇媚的花枝。像是忘记了季节的更迭,承载着千万的希望开放。

		眼下的所有,完满充盈,一无所失。桔子的汁水浸润在指缝间,是略酸的清香,它来自田野的草木
	。我闭了双眼,就有春日散满。

		生活在惯常的轨道上滑行。节日的快乐也显得苍白麻木。而明亮的日子,从没有被遗弃。在毫无吝
	啬的阳光里,我细心收藏温暖,满心欢喜。

		无论在哪里,无论沉迷些什么,总有简单的获得,总有轻易的失去。

		那是我们心的诡计。

		你想到什么,这世界就存在什么。

	\endwriting


	\writing{静别}{2006年02月16日 ~ 22:40:09} %<<<2

		\shortpoem{}{}{}
		携手上河梁,游子暮何之?\\
		徘徊蹊路侧,恨恨不得辞。\\
		行人难久留,各言长相思,\\
		安知非日月,弦望自有时?\\
		努力崇明德,皓首以为期。\\
		\endshortpoem

		汉乐府的字句依旧,千年的光景却已随逝水东去。桥上的友人,曾有的惜别留恋,被年年的风雪覆
	盖,听不到一声马嘶,唱不起一句阳关。

		水总是世上最销魂,又最冷漠的表情。忽而,是桃花面上的一滚泪珠,忽而,又成多少,不舍昼夜
	的流澌,响如佩环。度去,岁月的面目,模糊不可辨识,直到误识了天际归舟无数,让那桥上的玉人花
	白了头发。

		我于是庆幸,静是乘火车离开的。\par
		或许离情的怅惘才不至被水的波纹扩大。

		然而,我还是在这样的日子,想起那些送别的词句。于是,便也有了水畔的心情,几许寥落,几许
	恍惚摇曳。

		电话的这一头,那一头,有了水声,有了满天如席的风雪。\par
		仿佛见了你我,立足千年前的古桥,我执酒,你的眉目却已模糊,消失在冬的末尾,这北国的最后纯洁。\par
		雪意留君君不住,从此去,少清欢。我手中的酒冷掉,而离人已隔山岳。\par
		又一轮新月初升。


		江南的春天已在门前。

		叮咛着你,千万和春住。烟雨里,静会如那丁香一样的姑娘,隐约在诗中的雨巷。你是那样的女子
	,用脉脉的情深,爱得勇敢而坚强。我记得,你编织的围巾,记得你的巧手慧心。总说,你是贤妻良母
	。也玩笑,若是男人定要娶你。是一辈子的福气。

		静的火车开动了。我的手机响了。

		记得想我。你说。

		要相信,路上的风景,正等你来目不暇接。

		你这北方的姑娘,已被江南的熏风和暖浸泡得一身娇媚温存。你要被好好疼爱,而不是去独自坚持
	一份注定不得善终的情感。

		静应该幸福,没有不幸福的理由。

		我会想念,在每一个下了雨的日子。当世界被淋湿,我知道,你那一头的河水又将要涨满。这看似
	残缺的生活,瞬间里,竟一无所失。我便思量着,荷会在酝酿了,在春末的镇江,也在飞絮的北京。等
	着你的归来,一并盛放,拥挤我们同乘的木舟两侧。

		是6月吗,或者,更早的一些时候。

		你说,是4月的末尾。因为姐姐的婚礼。

		静的姐姐,一样是美丽的姑娘。她要穿上婚纱了。她也邀请我和朱出席。那将是甜美得近乎幻觉的
	日子吧。

		我将赠予祝福,她将抱以最幸福的笑容,把五彩的糖果发给每一个人。\par
		我们为姐姐高兴。我们又多么嫉妒呢。与初恋的爱人,共步红毯。\par
		我们总会有泪花,在青春的一场场轮回之中。惊讶着不期的相遇,还有那,无可奈何的失散。\par
		因为年轻,所以貌似脆弱,又故作悲伤。

		很幼稚,却也天真。好像16岁的春天,被校门口的一棵杏花感动,好像写在本子上的那句:花是多
	么倔强呢,固执地一定要开放,熬过了漫长的冬天。

		后来才发觉,原来,我们也一样倔强。

		我在这里,我在那里,我在谁的思念中浮现,又在谁的明天等候。静,这是我们注定的经过。记住
	了,又遗忘了,或者,珍惜了,又荒废了。总会有笑意明媚的你,在下一处十字街头,等绿灯亮起。

		不必匆忙,水流的声音还不够令我们懂得吗。要仔细地度过,它终究是要从桥下流去了。

		百川归海,千古不变的景象。我们要么沉浮挣扎,要么携手桥上。你选哪一个呢。欣赏水的流去吧
	。谁让我们无以拒绝这一切,惟有去接受,又热爱。

		即使是乘火车,却仍要祝福你,一帆风顺。

		等着,我们的4月。

		我信了那句,黯然销魂者,惟别而已矣。

		田说,她会想你的。

	\endwriting


	\writing{雨水}{2006年02月19日 ~ 15:02:02} %<<<2

		今日雨水。

		对于节气敏感。想着它们在日历上的浮现和沉没,就有了细数年华的滋味。仿佛一串零落的脚印,
	深深浅浅,无言里让你深记。

		立春,雨水,惊蛰,春分,清明,谷雨。是春天的一个个名字,一点点精细的刻度。

		你看那窗纱上的日影,等着庭院里的一株小梅,在被节气划分了的时光。

		可以嗅见气味变化的微妙,起初是冰雪消融时的湿土味,然后有花朵酝酿的淡香,等丁香开了,便
	是太阳炙烤衬衣的暖暖气息,把人心都融化成蜜。

		而现在,我好像停留在季节的当口。看着窗上的鸽子扑棱着翅膀,凌空而去。

		远远地,我能也望见自己,站在那里,和你一起,和朋友们一起,和花朵似的女孩子们一起。穿碎
	花的布裙子。

		春水注满池塘,注满湖泊,注满天地,默默无语,却解救一个冬天的干涸枯淡。我们等着,玉兰,
	迎春,碧桃,目不暇接。让照相机记住那些可爱的表情,以待日后回忆。

		一切,却像一场不曾清醒的梦一样,不可捕捉。生活奔流而去,我们一张张撕下日历,好像剥落年
	华的表皮。

		这一日之后,降雨增加。

		我知道,会是如烟似雾,或许,还伴着春雷的轻哼。让该到来的,如约而至,让将别离的,决绝得
	义无反顾。

		春天,像是从罐子里出逃的精灵。要把我们刚刚习惯的世界颠覆,再装饰一新。

		而激动慌张的我,语无伦次。

	\endwriting


	\writing{她}{2006年02月22日 ~ 13:36:45} %<<<2

		他们都说,我们长得像,只是,她年轻时没有我漂亮。她于是微笑,几分得意与骄傲。

		她也是小女孩。她住在砖墙四合的院落。门外是枣树林子,更远些的地方,有一块块碧翠的菜园,
	和弥漫着草柑味的玉米地。

		她穿素色的布衣裳,挎着竹篮子从田野阡陌间走过,篮子里装满喂猪的野菜。她有点沉默,有点安
	静,不像别的孩子吵闹,清高或者孤独的样子。


		她喜欢和守院的小黄狗玩,她喜欢看爸爸粉刷着碎砖块砌成的围墙,又画上漂亮的图案。她和哥哥
	们一起收获枣子,一起挑到工厂门口去卖。她站在粉刷一新的院子里,阳光撒满,小黄狗追着尾巴转圈
	,她觉得生活崭新明亮,充满了莫名的希望。


		她不爱打扮,却也喜欢穿上连衣裙,挎上白色的小包,和几个好姐妹一起去上班。她们做一样的工
	,吃一样的午餐,怀着一样的简单的幸福。田野渐渐消失,枣树林子也不见了踪迹,成为了马路和各种
	各样的工厂。


		流行起墨镜的时候,就一起买了墨镜,又借了相机拍下照片,学着时髦女郎的模样。她们坐在展览
	馆的水池边聊天,被新时代的光芒照耀着。她们尝试着一切的新鲜,喝了味道怪怪的可乐,穿上格子布
	的衬衫,牛仔裤。


		一群女孩子,嬉笑打闹着,她喜欢这属于年轻的热闹。

		二哥哥给她介绍了几个对象,有作曲的,写诗的。她嫌他们不踏实,都一一回绝。

		她却织了毛衣,送给同厂的一个男孩,粗线的纯毛毛衣,清新的青蓝色。

		她坐在他的车后,任他飞快地蹬着车,穿过闹市和小街。他不曾作曲,不会写诗,他单眼皮,他抽
	烟喝酒,爱打架,他仿佛一无是处。

		他却是正义凌然的人,真正的年轻人,一脸热情,一脸勇气和自信。


		他们就这么有点莽撞和草率地相爱了,没什么波澜,又领了结婚证,她成了他的妻子,在一个和暖
	的五月。

		结婚旅行,他们到上海,他们在黄浦江畔留影,两个人笑得甜美。

		她住进他家的院子,院里种满月季花,婆婆是爱花的人,善良和善,公公做得一手好菜,喜欢写几
	笔书法,是退休的老干部。生活安宁平和地开始。

		她对我说,是从那个时候开始,她不再为很小就失去母亲而悲戚,因为她有了完整的家,完整的明
	天。

		她穿着牛仔裙,拿着大皮管子为花坛浇水,她在阳光充沛的日子晾晒被褥,一通拍拍打打扬起细小
	的尘埃,在空气里旋转飞舞,亮晶晶的。

		初夏的晚上,星星的碎片散落一地,她抱着小小的我,从房间的这一头走去那一头,口里欧欧哼哼
	的,湿热的空气充满花露水的香味。他跑遍北京城买回红色的塑料浴盆,为我洗澡,又涂上洁白的爽身
	粉。等我终于在她怀抱里睡熟,他小心翼翼地剪着我稀疏的头发,她满是慈爱与幸福地看着。


		她悉心在本子上记录,用她娟秀的字迹:“梦中会出现微笑…… 会侧着睡觉,会吃大拇指…… 9个月会
	叫爸爸,10个月会叫妈妈…… 1987年5月22日会自己走了……” 这些用蓝色圆珠笔写上的文字,依旧清晰美
	丽,读起的我,心底总是盈满了甜若花蜜,又净若清流的液体。


		她为我梳头发,用彩色的皮筋扎着小辫子,她也为我剪头发,效果却不好。

		她买许多花裙子给我。

		春天带我去动物园看熊猫,夏天到景山看荷花展览,国庆节时去游乐园玩蜗牛车,被他抱着坐在屋
	顶看远处的烟花,大雪后在院子里堆三个雪人,两大一小。

		圆明园办灯会,他们又欣然前往,我骑在他脖子上,越过人群看惊险的杂技表演。

		许多次,我们在湖上划船,夕阳照着,一切都成迷人的橘红。

		我还那么那么幼小,小的手掌,小的脚丫。在北戴河的海滩一路跌跌撞撞地跑去,留下精致的脚印
	。艳阳高照,她坐在太阳伞的阴影里,穿着红色的泳衣,皮肤烫烫的。

		她骑车送我去幼儿园,骑车带我去学美术,我的出行都是坐在那小小的车座后,刮风时蒙一件红纱
	巾,下雨时躲进她的大雨衣。


		她坐在沙发里打毛衣,看我一天天长大,戴了红领巾,戴了三杠,蹦跳着放学回来。

		她曾经的姐妹们大多失散不见,偶尔,她取了高柜子上的相册翻看着。我会缠在一旁,问这问那,
	她一一解答:这是她最好的朋友,那是她最喜欢的一条裙子。有一年冬天,我看到她悄悄烧掉一些照片
	。

		她把织好的毛衣在我身上比了又比,是清新的青蓝色。她有点严厉,不准我放学后在外边逗留。她
	要我准时回家。而孩子总是贪玩的,好几次她生气了,让我罚站。已经记不得了,但我似乎是哭了。

		过几天,她买了苹果样子的转笔刀给我,让我知道她的生气是因为对我的担心,让我明白她爱我。

		后来,我开始为她梳头发,为她染头发。


		她听别人夸我懂事聪明,欣然笑着。她仿佛很满足,又好像不怎么在意一切。

		日子惯常地过去。

		三个人搬到楼房里,原来的院子被铲平,成为街心花园。有时,我们一起走到那里,他和她总会停
	一下,努力辨识着曾经的位置,想起些旧事,笑说自己突然就老了。

		我上了大学,他们的生活就又回去最初的模样,两个人,一张桌。

		当我周末回去,她总是早早买好许多食品,塞满冰箱。她常怕我在学校不好好吃饭,就学着发短信
	,提醒我要多喝水,多吃水果,几乎每天。显得有些繁复和唠叨。

		她的担心是那么细密,那么多。我不在她的怀抱,不在她的视线,她就总有无法抑制的惦念和牵挂。

		谁让我是她身上掉下的肉呢,她总是这样解释。

		春节里,烟火充满了我们的窗子,比许多年前国庆时在屋顶看到的更加绚丽。

		我和她就一起躺在床上看。

		远的,近的,美妙的烟花起起落落,耀着我们的眼目。仿若尘世的繁华,仿若许多的年华光阴,在
	我们的眼前上演又谢幕。

		是近乎虚假的夜晚。

		我们就那样靠在一起躺着,一言不发。我抚摸她渐丰腴的身体,想着前前后后,那么多个她,那么
	多个自己。

		冥冥之中,我是如何成为她的孩子,成为她爱的寄托。她不再年轻,在突然的一个瞬间里一样,无
	可挽回地告别了曾经的光华。

		夏天的时候,我们还是一起去看荷花,拍下许多照片。冬天的午后,她坐在南窗前的阳光里,脚边
	睡着小黄狗,却早已是另外的一只。


		她喜欢为我削一只苹果,喜欢剥一块巧克力递给我,于是,我嘴里盈满了莫名的希望,生活崭新而明亮。

		她总是心怀善意对待所有人,她人缘极好。她依旧沉默安静,一个人在房间看报读书。她却不再孤独。


		她把洗净的床单铺好抚平,躺下,翻过身闻一闻洗衣粉的香味。她在这些细枝末节里快乐,无所担
	忧,无所惧怕。她好像是世界上最幸福的女人了,她看似并不拥有什么,却分明又拥有了全部。

		前天,是她的生日。我们一起围桌吃吃长寿面。他做的面条,总是很好吃。没有礼物,没有热烈的
	祝福,唯有无数记忆的零碎细微,无数的感恩,在她的,在他的,在我的心田。


		有时我们争吵,有时我们彼此生气,但是,那都是经不过夜晚的插曲。

		我爱她,我不曾亲口说出。

		她却明白一切。正如她也一样无言里爱着我。


		我也喜欢为她削一只苹果,喜欢剥一块巧克力,塞到她嘴里。这一世的恩情,我将怎样报偿。——然
	而,你给予的太多,太多。


		妈妈,生日快乐。我想,我们是世界上最相爱的母女。这所有,我却只能在文字里轻轻对你说。

		我们原本是同一个生命。我深深相信。


		他们都说,我们长得像,只是,你年轻时没有我漂亮。你于是微笑,几分得意与骄傲。

		我说,你很美,我们是一对母女花。

	\endwriting


	\writing{观众的下午}{2006年02月25日 ~ 15:24:29} %<<<2

		那是一个有好阳光的午后。我们的阳台上撒了淡绿的光芒。穿白裙的女孩,赤着双脚,就要飞起来
	。一只粉气球,一丝不被察觉的凉风。仿佛安静,仿佛呈现与消隐间的距离,她的姿态轻盈,游离于现
	世之外。

		我站在这个午后的面前,很久很久。

		那离地的双脚,投下瘦的浅影子。眼前,不是画家精心营造的幻象,而是,一整个不被真实容许的
	世界。

		偌大展厅,不过三三两两的观众,细细地度着步。一扇门,一栋建筑,轻易隔绝了尘世。一扇扇窗
	,被安放在白墙之上,等你的经过,等你停了脚步,伫立着凝视,把空荡荡的心用里边那个油彩的世界
	充满。

		是这样的参观,我和小鹿,成为默默的观众。被色彩与光影,线条与明暗迷惑,又欺骗。进进出出
	一处处虚假的,却更接近真实的空间。

		中国美术馆。一个不经意中就偷走一个下午的地方。

		我想到很小的自己,一个做着画家梦的孩子。那时侯,我手中握着彩笔,我的画纸上有最奇异美妙
	的图案。我背着大画夹,和另外许多孩子来到湖畔,来到春天,画远的近的亭台,画北海的白塔,颐和
	园的万寿山。不过是幼儿的涂抹,没有比例,没有透视,却是纯真可爱。便有倾斜的房子,粉红色的天
	空,飞翔的小白兔,在纸上成为真实。我知道自己在创造,并有了创造的快乐。

		或许真的,每个孩子都曾是艺术家。是在时间的路途上,我们把那些最可宝贵的东西遗失一地,而
	全然不知。那些稚气天真的图画,被母亲一一收起,放在牛皮纸的袋子,保存在柜子里。它们几乎被忘
	记,只在搬家的时候被翻出浏览,然后,我知道了自己的失去。

		当人们问起梦想,我再不会毫无犹豫地说出画家两字。

		当人们问起梦想,我已经无言以对。

		小学的第一堂美术课,老师教我们画蜗牛。那是一个星期二的下午,只有一节美术课。花白了头发
	的中年女老师,她容貌慈爱,身材瘦削。我画得很好,她给我了我全班唯一的5分加两个星星。我记得
	很清晰,爷爷接我放学的时候,她把画交给他,夸我有天赋,色彩感觉特别好,想推荐我参加区里的比
	赛。

		那是从家里走入学校,走入集体的世界后,我第一次受到肯定和表扬。因这,幼小的我,竟将对于
	学校的恐惧一扫而光。我喜欢美术老师,她姓刘。她让我感觉亲切,没有对其他老师的紧张感。她对我
	很好,让我成为美术课代表,并和一个姐姐一起负责美术小组的活动。一次春游,我丢了带来的零用钱
	,她就拉着我的手,带我吃饭,又给我买冰淇淋和果汁。

		后来,刘老师退休了。我时常想起她,却再也不曾见过。

		那些比赛的获奖证书,也一并放在那个牛皮纸的袋子里。沉睡着,一个个春,一个个秋,并不知晓
	我的长大,和人间的失散。

		你问我什么时候开始放下了画笔。

		我说,是当我发现太多东西,我根本无力画出。

		于是,我惟有向文字求助。我在文字里,寻找头脑中时隐时现的影像,色彩,气味。

		图画与文字,于我,都不过是工具,来救一个淹没在内心世界的孩子。要将那片海水汪洋,倾泻在
	色块线条,流散在字里行间。

		或许,我始终是生活的弱者。才在肉身的更深处,为自己保留了那么许多。

		现在的我,被自己迷惑,无力自拔。在文字,在幻觉与错觉,在虚伪与真实,穿梭往返,而乐此不
	疲。看看镜子,看看日历,什么都不曾等待我们去明白。阳光在这一刻照着,蒙在我的脸上,光明到无
	法睁开眼睛。我总相信,那滋味就像幸福。

		回程的公车上,小鹿又靠着车窗睡去了。像每次的回程一样。

		我在沉默里看拥挤的车厢里人们的沉默。夜幕又将落下,把我们的白天藏在地平线之下。我想到在
	美术馆遇见的午后:穿白裙的女孩,粉气球,徒劳却固执的一次起飞,淡绿的光线……

		画的名字叫,阳台上的气球。不知为什么,我感觉莫名亲切。

		也许,有曾入梦的夜晚,有不远的阳台,有自己,有尚存梦想的孩子。

		是轻盈,是缺失重量的静,是不被容许的世界。

	\endwriting


	\writing{二月。留言}{2006年02月28日 ~ 18:55:04} %<<<2

		回到学校,生活继续。

		一样匆忙的早晨,窗外又落了薄薄的雪花。天空洁白,不曾图画的宣纸似的,等着一朵灰云,一缕
	青烟,泼洒如墨。

		湿润的空气,在二月温煦的风丝间流散。我们提着书包穿过篮球场,混入人群中,模糊了形迹,被
	这人间的热闹淹没。

		在北语,这人群多是嬉笑着向前的女学生,总有花样不断翻新的衣着,各具风情的笑靥。像花海的
	流动,迎着晨曦和清风的爽利,把青春的活泼和幸福洒满。

		我喜欢这样的早晨,让你觉得世界是新鲜的,不无激动与希望的。

		作为女学生队伍中小小的一员,并没有被淹没的低落,却有融入这宏伟美丽的快乐。好像一株花园
	角落里的小花,安静着开放纤弱的白花朵,心满意足地享受着短暂的春光。

		来园的腊梅开了。鹅黄的小花,毫无张扬,素默无言,并不因是第一株花树而炫耀骄傲。不过是用
	尽一冬的力量,盈满一树美好,来装饰你的视线,你的小窗。我于是感动于梅的一切,姿态,神情,和
	傲骨娇花间吐露的坚定。

		便想到石评梅,想到城外荒冢里葬着的如光如电的青春。她爱梅,人也如梅。她说,“世界既这般
	空寂,何必追求物象的结果。”

		没有来源,没有去处,谁又不是生命的流亡者呢。

		她的人生短暂,她的才华却光润耀人。世界空寂,在生前的碧波,在身后的湖畔。安宁的人世,不
	曾有真切可握的实体。好如这暗香疏影,引你留恋下泪,也不过是此刻的幻象,不得留驻封存的。

		陨落的生命,睡在心爱的湖岸,同心爱的人。十年,百年,千万年。仿佛是夭逝凋亡的,成全了恒
	久,遗憾叹惋的,铸就了不灭。让我们,得以把她的人生深记凭吊。

		我不能够明白所有。

		春天却已不远。未解的真相,且留待明日追究。今日是云淡风轻的和暖温存,今日便有清朗辽阔的
	心境,浮在半空漫游的冥想。

		如果我说我快乐,你会相信吗。

		如果我说我悲伤,你会心疼吗。


		学校的生活平铺直叙。而我,只是看花的孩子。

		从腊梅,到玉兰,到丁香,从槐花,到睡莲,到美人蕉,看过桂花,就等候着雪。


		二月的末尾,我把最后的雪花接在掌心,感觉融化的温度。

		因为幸福,连时光都碎了。

	\endwriting


	\writing{情怀}{2006年03月03日 ~ 15:15:15} %<<<2

		收拾书架,无意在底层翻出一本《席慕容经典作品集》来。素净的封皮,开着纯洁的白花朵,下边
	齐整书写分辑的名称:七里香,无怨的青春,时光九篇,请柬…… 每一字也是芳香溢满,引人遐思。

		是中学时爱不释手的书。席慕蓉,有美丽的名字,写美丽的文字,却是一个并不美丽的女子。读她
	的诗,或许是每个少女都曾深心沉浸的事。

		随手翻动几页,对身边的莫迟笑说,我该细细重读,来温习少女情怀。

		这些字句,依旧安静地排列,带着陌生又熟悉的表情。其实其中的很多,我都曾反复抄写,记在随
	身携带的笔记本上。一些抄写,还是在课堂上完成的。

		那时,我的座位下,总会藏着诗集,徐志摩,北岛,顾城…… 这本席慕蓉,还有最挚爱的宋词。中
	学的我,大概是个大胆的学生,会把这些书包了写上数学两字的皮子,堂而皇之拿到桌面来读。那个自
	己很可爱,凭着那么一点小聪明,混迹在好学生的范围里,为所欲为。

		现在的自己,显然是愚笨了太多。

		想到中学的自己,就想到蓝的裤子,白衬衫。想到一道搭公车的静静,想到她把新买的本子递给我
	,请我在上边写些字。那是我极骄傲的事,于是总是挖空心思,想出华美的词句来,再一笔一划小心写
	上去。

		她说,她还留着那些东西。

		字迹该还是16岁时候的模样,而我们,记起那个16岁的自己已经隔了重重雾色,如烟似幻了。时光
	是小偷。


		中学的对街,有一处居民花园,周围种着许多玉兰,桃杏一类的树木。午饭后,我们总是喜欢买一
	只冰棒,用舌头舔食着,到那里散步。

		有时,说一说昨晚看的连续剧,有时,谈一谈班上的男生女生,就是在这样的谈话里,不知道多少
	谣言被制造又迅速流传。许多谣言或许也会与自己有所瓜葛,但是女孩子们依然乐此不疲。没有人会因
	为怕谣传有天降临到自己头上而甘心牺牲这谈话的乐趣。况且,有谣言,大概也是一种受人关注的体现
	,多少会满足虚荣的心理,填补调剂无聊的课业生活。


		在那些花树下,留下女孩子们的闲言碎语,留下青春的苦恼和悸动,留下她们花季里的倩影。她们
	在春天的第一棵杏花下拍照,笑得比花更娇美。她们郑重地收起这些照片,夹在最爱的书页里,仿佛留
	作年轻和美丽的证据。

		曾有那样的一张照片,被我夹在席慕蓉的书里。正是写有《一棵开花的树》的那页。


		在心里偷偷问了上千次:如何让我遇见你,在我最美丽的时刻……

		从没有人来回答。多少个女孩子,如我这般,虔诚在诗歌,虔诚在年光的佛前,等待一个人的来临
	。那个你只愿在最美丽的时刻遇见的人。那将是万分幸运的,于芸芸众生的洪流,于亿万年时光的无涯
	。是恰好的时刻,恰好的美丽,恰好的他。这相遇也许不过换得无视的走过,也许,花瓣零落一地,但
	这相遇已足够美丽了。我们,有什么理由去奢求完满。有时,是遗憾和悲伤,保鲜了爱情原始的模样。


		重读席的文字,在经年的改变后,它们有了崭新的意味。

		她说,“在年轻的时候,如果你爱上了一个人,请你,请你一定要温柔地对待他。”

		她说,“我们去看烟火好吗。去。去看那。繁花之中如何再生繁花。梦境之上如何再现梦境。”

		女子,是为爱而生的动物,是为情所困的生灵。

		从古至今,她们痴心,她们神迷,她们物我两忘,惟见爱情。所以,有人嘲笑女人的感性与不理智
	,有人默认女人无法成就大事。

		她们细致,敏感,她们望着南归的大雁,一次次神伤,长跪读那远方托鱼腹传来的家书,泣不成声
	。

		张爱玲有名的那篇《爱》里的女孩,历尽人生劫数,到老了还记得那春天的晚上,记得桃花,和那
	后生的一句:“噢,你也在这里吗。”细微的瞬间,可以用一生一世来记忆,枉然的经过,也能够入梦夜
	夜,只有女子,只有真正的女子,能够把情爱这样深沉地收藏,又温柔擦拭。


		从江南采莲的女子,到桃树下的佳人,她们总比男子有更热烈的爱,却更迷惘的神情。男权的世界
	里,女子因这情深悲哀。我却愿歌颂,女子的美好,不只是形态身躯的优美静好,更是整个生命的天真
	纯澈。所以,那些嘲笑女人的人们,该反思下自己受到了几多世俗的毒害,还洋洋自得。


		少女情怀总是诗。或许,不过随口的言语,也是唇齿生香。年少的日子,该好好珍爱,该用心来度
	过。让三月的阳光照着每寸肌肤的细润洁白,让熏风浮动吹过发丝散发着的花香。女孩子们挽着手臂走
	在春天,每一年的风景似曾相识,只是裙上的花朵变了,只是唇上涂了新的唇彩。


		这生活亮晶晶的,花树在酝酿,春的诗歌也在酝酿。却又有多少情怀是随时间一去不返的呢。我却
	不会有丝毫失损的悲戚,因为是三月。她说:

		\longpoem{}{}{}
		让我相信 ~ 亲爱的 \\
		这是我的故事 \\
		就好像 ~ 让我相信 \\
		花开 ~ 花落 \\
		就是整个春季的历史
		\endlongpoem

	\endwriting


	\writing{瞭望}{2006年03月09日 ~ 22:50:30} %<<<2

		我想着,许多个春天的风晴日暖,用花朵点缀着季节的苍凉流变。想着,许多的你们,微笑或者忧
	伤,我们携手走过湖岸,看落英缤纷的四月。

		母亲在花树下,用温柔的双眼望春水荡漾的碧波。天空清亮,像从黑暗中苏醒的蓝水晶。用力呼吸
	这空气,吸入我的心肺,润滑着骨肉血脉。我们在着湖岸上漫步,迎着眼前一团浓似一团,一片幻如一
	片的粉红的云霞,在树梢,在枝头。

		仿佛所有的四月,都是这样明亮。像低低哼唱的一段歌谣,荡过山尖,荡过波纹,在人的心海里投
	下一颗糖果似的石子,并不再索回。于是会有甜蜜的涟漪,一圈圈,画着圆满的弧线,在记忆里留下完
	美无瑕的轨迹,可堪思念。

		四月,读着,便知道它的美好。可以脱去冬衣,穿上布格裙子,可以抱一本诗集,徜徉,或者静坐
	,默读,或者发呆。在毫无吝啬的阳光里,一切的时光的流逝都可以忽略不计,一切浪费都是最好的珍
	惜。就让四月无所事事,游手好闲。把每个晶亮的日期空掷入无言的沉醉,不只是春风,柳絮,不只是
	桃树,花枝,踏向春野的脚步声,已足够愉悦从冬天融化的双耳。

		我们总是在四月,到那湖岸上去,载着欢笑,拍下照片。我说,我最爱那长长的堤,爱那杏花不胜
	凉风的娇羞。时间给我们一个充分的借口,来爱惜花开的时节。有时,我们走累了,就坐在湖边的石凳
	子上。然后,远山成了水墨中的风景,船上的歌声,把我们的思绪带离,飞去比山更远的天地,不知去
	向。

		我感谢,我是一个四月的孩子。母亲在终于温暖起来的日子把我从她的生命唤醒。于是,我在万物
	复苏的节气,清明这一天降生,成为春天的孩子。我的生命从一种复苏里开始,我感觉我生命的节律是
	与大地同一的。很快,第二十个年份便将来临,一霎时,我感觉自己的微小,是多么美妙的一种存在。

		母亲总会在生日那天为我点燃生日蜡烛。我总会满心期许与希望,将它们全部吹灭。我总记得那烛
	光,和烛光摇晃里的母亲,她的目光,她含笑的嘴角。她的小女孩长大了,在一年年的愿望里。那些愿
	望,忘记了是否终得实现。所有的生日,却成为永恒的记号,烙印在平凡的岁月,母亲的皮肤,我的心
	上。在这样的四月,我是在迎接一岁的重生,更是在不断告别。

		我像是永远走在湖岸之上。夹岸的花朵,正向季节的深处凋零。而母亲是湖,是围绕拥抱着所有的
	柔波。春天的气息,透过皮肤,渗透入我的筋骨,我的血脉。遗忘所有的冬季,我可以安心地闭上眼,
	轻靠在花树挺拔娇美的身上。

		天空清亮,像从黑暗中苏醒的蓝水晶。季节离开了,而我们依然在这里。湖岸,我最温柔的寄托,
	在许多个春天复活。用四月,祝福崭新的生活,要你擦拭净眼睛。面向阳光,就看不到身后忧伤的影子
	。

		为了幸福,人必须勇敢。

	\endwriting


	\writing{风沙}{2006年03月11日 ~ 21:53:24} %<<<2

		独行在季节的边缘,三月的雪花从天而降。

		苍白的日光,预示着风沙,预示着我的回忆,被抛入深渊,像落在深井,捞不起的月亮。要用多少
	的热情,才足够把我们的岁月填满,永不枯竭。风沙将至,从春风未识的塞外,从飞沙走石的山岭,从
	湮没帝国与美人的荒漠,日夜兼程,一路奔赴。

		\longpoem{}{}{}
		我动身回北方 ~ 四处打听 \\
		她身世飘零 \\
		那时间的幽灵穿越爱情 \\
		听哀号的声音
		\endlongpoem

		梁朝伟用他并不美妙的歌声沉沉唱起。我曾在身上落满黄沙的春天街头,反复聆听这首叫《风沙》
	的歌。它让我想到遥远杳无的那些时代,想到旅人脸上的沧桑迷惘,想到那千年前的一句悲叹:饮马长
	城窟,水寒伤马骨。后来,自然又加入Jay的那曲《娘子》,便更有江南的烟雨如愁,折柳的娘子迟暮
	。

		一道道关卡,早已不是简单的一种人文或地理分界,而是情感的天堑。高墙的那一头,是大漠孤烟
	的凄荒悲壮,是戍客思归的夜夜,群山的这一边,是长安的月光如练,是零落无望的年年。远方的人,
	有琵琶哀怨,有埋骨蓬蒿,关内的人,默泣在春风沉醉的故园,一季季枯竭着风华,随那小塘的荷花,
	固执地问着问着:何日平胡虏,良人罢远征。

		断肠人在天涯,在爱人的视线无法触及的风雪。分别时,嘱咐着努力加饭餐,却明知,这之后已是
	永世的不见。你会在世界的某处执意等待,我们终于可以在未知的一种天地间相遇,作来世的比翼之鸟
	。——他们的相信,他们用勇敢和眼泪书写美到失真的童话。

		那些可以执手相看泪眼的年代,随着旧时安谧多情的月色,一并灰飞烟灭。

		风沙却从高墙的那一头,飞行到我们的城市。取代了曾经的锦书鸿雁。它们遮蔽了日光,用黄沙覆
	盖目光所及,落入你的发,你的眼,你的回忆。风沙无情,风沙多情,用近乎严酷灰暗的表情,哭诉大
	地的伤害。它是复仇者,正义的复仇者。

		\longpoem{}{}{}
		风沙 ~ 要带我去哪

		是她 ~ 我一生挣扎 \\
		爱恨都放不下

		风沙 在断枝折花 \\
		天涯踩在我脚下

		我一路在牵挂
		\endlongpoem

		他的歌声继续。旅人的马瘦了,一片孤城,万仞高山。千古同一,人是情感的动物。男人用苍凉与
	雄浑,女人用悲戚与哀怨,见证那些时光带走的回忆,像那些不再的时代,遥远杳无,灰飞烟灭。英雄
	与红颜,连一具形骸也不见于历史,人的微小,被一次次有力而冰冷地证实。那么,我们还有什么是值
	得争夺不休的。

		只让你相信,来生,我们可以做比翼的小鸟。纵使,你喝了孟婆汤,纵使,我们再也辨认不出对方
	曾经的模样,我也会飞到你的枝头,唱一支涕血的挽歌。

		愿那些远方的风沙,落在你年轻的肩膀。愿你记得这座城的春天,记得一个尚未凋谢的我,无论那
	将是痛苦,还是幸福。

		独行在季节的边缘,三月的天空,飞雪迎面。

		风沙还没有来临,我没有恐惧。只是漫天的雪花,让我感觉生活的虚假。午后,竟有了明蓝的一角
	天空。大风把阴霾吹破,阳光蒙在我的脸孔,微寒的天气里,我开始想念你的手掌。感谢你赠与的全部
	温暖,在整个的冬季。

		而回忆,永远是落入深井的月光。

		那些沙尘,和所有的晴朗一样,被我们锁在日期的柜中,封存入永恒的无限。

		你可以遗忘,你可以享用。

	\endwriting


	\writing{白蝴蝶}{2006年03月18日 ~ 23:12:36} %<<<2

		诗人在静默里熟睡,任时光如书页开合寂寞,若小小的白蝴蝶。

		他的墓,不过朴素的石碑,与众多的墓比邻而居,没有张扬,没有墓志铭和赞美诗。诗人的睡房,
	与他精美的诗句相比,是显平凡与简陋了。


		三月的日光洒满,我站在他的墓前,看松树的碎影子在碑上婆娑,耳畔是远的近的,不可辨识方向
	的风。碑文是矛盾先生的笔迹:诗人戴望舒之墓。1905 - 1950。那日期,划分了生死的界限,用轻描淡
	写的笔触,为逝者的生命作以最庄严的总结。百年的时光,就这么,随一成不变的树影,轻摇如梦。我
	相信,他是熟睡着,偶尔也醒来,听墙外白杨树上歌唱的小鸟,听来访者的脚步悄悄,听他们在他的房
	前读一首雨巷。


		撑着油纸伞,那丁香一样地,结着愁怨的姑娘,从旧时的巷口经过,与行人擦肩,又同光阴一并隐
	逝。微雨的江南,氤氲淡紫色的忧伤,从姑娘的款步,从诗人的眼底。他的恋人羞涩,他的恋人有桃色
	的脸,桃色的嘴唇,天青色的心。我读诗人的诗歌,我想他的安睡,想他爱的炽热与伤痕。你会听到这
	熟悉又陌生了的字句吗,你是不是知道我的到来,是不是在百年时光模糊了的这个午后,在心底酿着更
	凄美动人的诗。为尘世,为不灭的魂灵?


		不远处,是大钊先生的烈士陵园纪念馆,正在重新修葺,工人们来来往往。那里的热闹,更衬托此
	处的寂寞。工人不时向这边看看,用我读不出的眼神。同莫一道,将白色的菊花献上,鞠躬。


		花朵在阳光下,那圣洁又肃穆的情态,正如小小的白蝴蝶,在时光里飞逝,在永恒里凝定,因为美
	,因为纯粹的心灵。我说,生死原本如此完好,一个浑圆无缺的圆圈一般。从混沌无知中游来,又向那
	亘长无限的沉睡中遁去,皈依于大地的拥抱。莫微笑着,对我讲歌德的看法:人的生命如雨水,降生如
	雨,而后又蒸发消散如雨。


		诗人墓的旁边,是一座小而失修的墓。建于民国,葬着一位24岁的姑娘。多少年轻而美好的生命,
	是这样,无情又多情地告别了。我想到,张爱玲小说里那一篇《花凋》。那眼见着幸福远离,而无能无
	力的女孩。那曾是一位美丽的姑娘。此刻睡在这封土之下的,该也曾是一位美丽的姑娘。或许,她也曾
	结着丁香一样地愁怨,在徘徊,在彳亍。现在,她小小的墓,与诗人紧邻。诗人会对她讲起诗中的世界
	吗,她会在笑意里倾听吗,他们会偷偷地讲话吗,在我们这个世界睡熟了的时候。


		早春的风,还不免干燥与坚硬。虽然墙外的树梢,已吐露幼芽的鹅黄,远山也涂了浅浅的胭脂。大
	概,我们是在这个春天来看望诗人的第一个人,我不无幸福地想。他会快乐,我知道的。离开前,我们
	对诗人道别,约好有空再见,那情境仿佛熟识的朋友。


		我们献上的白色花束,倚着墓碑,开放得坦然而安静。没有惊奇,没有赞叹,像太多的日期一样,
	被我们生活过了,被人们爱惜过了,又在日后被未来的某天重新记忆,隔绝了时空,失去了真相的触觉
	,却增添了距离的芳香。我们祭奠逝者,我们没有一滴泪水,我们洞见多少流变无凭的爱恨,多少鲜活
	,多少消散。这土地里,将沉睡着你,也将沉睡了我,不发一言,把所有的纠缠与留恋深埋。


		这墓园,亲见了多少的春天。这墓前,默立过多少的足印。一支早已枯了的火鹤,萎在诗人的墓前
	,我们的小小白蝴蝶,也会一样地平安枯萎吗,陪伴诗人,用短促却爱娇的生命。


		时光被风吹起,如书页开合,转眼便已百年。诗人的诗歌,被印在崭新的书页,一如既往地,散发
	油墨与诗情的芬芳。

		读诗人的诗歌,在他的墓前:

		\longpoem{}{}{}
		…… \\
		翻开的书页,寂寞;\\
		合上的书页,寂寞。
		\endlongpoem

	\endwriting


	\writing{三月二十四日}{2006年03月25日 ~ 22:16:53} %<<<2

		春天逼近的下午,空气中,树木的种子正在旅行,带着三月的光芒依稀,用最安静的方式,为了更
	新的生命而流放。没有目的,它们只是飞散,这里,那里,或许在哪一处积土的屋顶停下,等着雨露,
	来长成一株小小的植物。

		那些努力接近晴空的生命,总令人动容。它们纤弱无依,却是倔强固执的姿态,和所有愿望幸福的
	人一样。一样在拥有着苍白而甜美的脸色,疲惫却快乐。


		我是在这下午,骑着单车经过熟识的街道,仿佛一粒飞行的种子。

		满怀希望,却又无可皈依。只用最倔强固执的姿态,坚持着最朴素,却也最高贵的信仰。


		车轮被我蹬得飞快。很久了,我不曾这样骑着车。这一切应该只属于中学。属于那个十五六岁,穿
	着白衬衫的小姑娘。

		因为经常会迟到,我的车速总是很快。虽然,我更喜欢的是悠闲与散漫。但又有谁能够拒绝风紧贴
	着耳际呼啸而去的快感呢。

		每个早上,我是那样匆忙在上学的路上,为了在睡眠中多停留并无意义的片刻,为了恰好在铃声响
	起时赶到教室。班主任在门口等我呢,我知道的。有很长一段时间,我为此耿耿于怀 —— 她总是要抓住
	我迟到,才心满意足一样。

		被抓,自然是家常便饭。而我竟仍旧长期“恶习”不改。留恋于睡眠,狂奔在路上,陶醉淹没于速度
	的风声。而这一切,也许不过是我懒惰的借口。喜欢骑车却是真切的,无论快慢,无分季节。


		当然,最好还是雨后的仲夏。


		高二那年,由于非典,学校实行小班教学,同学被分为AB班。我是在下午上课。

		六月的天气,偶尔会在午后阵雨。

		听着雷声,从教室的窗口望出去,世界在乌云的尽头徘徊。淋湿的树,淋湿的操场,淋湿的花坛,
	氤氲着泥土的清苦,一种淡绿色的气味。

		我等待着雨停,想着一把把雨伞的闭合,又撑起,被晾晒在洗净的日光下。

		雨声细小,像某种神秘的祷告,敲着整个校园的肃穆,沙沙地响。我的车子在雨中寂寞。老师的几
	何题还在解,黑板上的字迹却已渐模糊。枕着手臂的我,用铅笔在桌面写下短小的诗句,又用橡皮擦掉
	,吹去碎屑,一次一次。

		放学时,雨便停了。天空是水彩绘出的明蓝透彻。

		我可以哼一首歌,可以不在乎调调,可以慢悠悠蹬回家去,或许遇到几处红灯,或许停下来深深呼
	吸几次。空气也是淡绿色的。


		简单的时间,从车轮的转动里迷乱,又瞬间挥发。你说,年华不容追悔。\par
		这小小的单车,承载多少路途的远近。被谁用手轻握住车把,向那目的地一路飞行。\par
		又有谁,曾坐在你的车后,小心地扶住你的胯骨,或是甜蜜地挽着你的腰。\par
		多少如幻的快乐,多少不知情的幸福,被你一一体尝,又肆意挥霍。

		没有人来细数,也没有人能够预言与肯定,谁会是坐在你车后的最后一人,哪一天又是我坐在你车
	后的最后一次。

		对于单车,便不禁拥有了微妙的感情。

		它多数时等候,多数时静默。它见证成长,从一所学校到另一所。它记忆着生活,从这里,到那里
	,有雨露落下,有幸福在倔强固执。

		我努力来接近。因我对存在的深信。\par
		良说骑车的时候不许听音乐,说那危险,并被他强制执行。\par
		他一脸严肃。我只有摘下耳机。他才放行,放心地微笑了。\par
		那一刻我知道,某一天,我会想念,这个下午站在杂乱街头的我们。\par
		而在没有归途的世上,我又该用什么来奉献,来皈依。\par
		让所有人都轻易地忘记,让你疼痛又甜美地记住。\par
		如同夏雨的滂沱,来灌注干涸。让我就这么经过,许多年,许多虚无与真实。

		在日期的背后,有我们缓慢生长的感受。是落在屋顶的种子,等待着雨露,来茁壮,来接近晴空的
	遥远。

		春天逼近。作为一名勇敢的流放者,你要开始相信。

	\endwriting


	\writing{丧失}{2006年03月25日 ~ 23:44:20} %<<<2

		在醒来的十二月早晨,在灰的天空下,她说:

		All we have is how you'll remeber me.

		我们拥有的全部,是你将如何将我回忆。

		她的面容苍白,她的声音却坚定。

		And I need that memory to be strong and beautiful.

		为了拥有,她选择了丧失。然后那些经过的回忆可以获得永生,在无尽的遗憾和心碎里,酝酿着甜
	美。她是聪明的女子。在无可抗争的命运面前,做出惊世骇俗的决定。让十一月就此结束,让我们彼此
	的相爱经由丧失,来抵达永恒。

		尽管,她曾以为在这段约定好起始与结束的感情中,她能够操控一切。经管,她曾以为坚强无情。
	当Nelson将手机与手表,那禁锢他,阻止他作为一个正常人来热爱生活的东西,通通丢入水池,向她求
	婚时,她哭了。她知道自己的不能够,而他,是她唯一想答应的人。

		一切,超出了原有的规划。她无力自拔,在她自己制定的游戏规则里,她第一次并最后一次痛苦不
	堪。

		《甜蜜的十一月》。一部悲剧性的电影。身患绝症的女子,在影视作品里早已滥觞。不同的是,她
	去面对命运的方式。美得令人费解,令人讶异,令人叹息。她是在挥霍最后的时光,她微笑,她貌似幸
	福,与一个个男人约定,同居一个月,一天不多一天不少,然后各奔东西。离开时她总是决绝无情。这
	是约定好的结局,她不曾丝毫留恋。直到他的出现,她以为她能够做到的一切,都被彻底摧毁。他们相
	爱了。以她未敢想象的方式。I never thought I'd have the chance and you gave that to me. 他
	给她爱的滋味,她毫无预期的幸福。

		然而约定,终于要履行。不因为彼此的心灵,只因为残酷的,已被书写的悲剧性命运。他在房间挂
	满了十一月的月历:Every month is November, and I love you every day. 他说。他拒绝离开这甜
	蜜的十一月。她依偎着他睡了,这最后的夜晚,笼着温暖的橙红色。

		为了让你记得,我只有选择离开。

		她用围巾蒙上他的双眼,如无数次做过的一样。在十二月的早上,灰的天空,这桥上。让爱情随吹
	去的枯叶消散,让她离开,让他在黑暗里,忽略离别的疼痛。当他取下围巾,只有空荡的视野。十一月
	,销声匿迹。而他,又如何忍住泪水。

		他们知道,所有的疼痛,是记忆的代价。要那幸福的时光,永远明亮着,在更深更深的拥有中。于
	是,只有让十二月真实地降临。

		相爱,是多么困难的事情。永久的拥有,便更是奢望。有时,在现实的不可得,唯有于丧失中,反
	而能以某种独特的方式得以成全。是记忆,是疼痛,是时过境迁后的念念不忘。

		我们拥有的全部,只不过,给予彼此的回忆的全部。我要那是完美的,无可挑剔的。

		想到《断背山》中最后的场景:衣柜,衬衫,照片,窗外绿油的麦田。我一直紧张的心竟然轻松下
	来,我知道,他们终于在一起了,永远不会分开,永远。

		因为丧失,他们在另外的世界和层面中得以相伴,直到生命的尽头。我知道,他会这么看似孤单,
	实则满足地生活下去,不会自杀。因他要让那些记忆活着。那是他们两人全部的,最珍贵的拥有。

		是谁说,我们曾付出的热情的全部意义,只是在日后用来凛冽地遗忘。

		那些细微的时刻,是不能够磨灭的,一切气味,光线,甚至眼神与体温。时光令我们不断丧失,一
	个个自己,一场场爱情。谁绝情的话,都不过自欺欺人的虚假,如果你真正付出过热情。

		爱情,从不会是理智的。这也是多数爱情会有遗憾的原因。而这,大约是爱情必然付出的代价。

		我只想,用接近完美的过程,来霸占你的回忆。我只想,用有限的青春,来焚尽年华的美丽。生命
	不曾完美,我不曾完美,我们只有接近,而永无到达。那宛如彼岸之上的承诺,让人来举目展望,而从
	不要希冀登陆吧。因全部的拥有,正是此刻,正是手中温热的轻握。

		我将绝情地老去,请你,一定要记得我最美的模样。

		我的丧失,可以成全最浓最厚的幸福,在我们的今天,在我们的明天,在明天的明天。

		我愿意,你在年老时想起我,指着我发黄的照片,对什么人讲起,我们曾拥有的,最明亮的快乐。
	那会让生命的痕迹如雪上的脚印,坚实而圣洁。那会让我,在某处你不再知晓的角落,幸福得默默哭泣
	。

		为了这样的拥有,为了我们的爱。我接受,所有的未知和恐惧。只要你是记得的。只要我们的心灵
	,是那么近,那么近。我敲一下墙壁,你的心房就可以听见。你懂得这一切。你明白我爱的全部秘密。

		我愿是你爱的人,你世界里最善最美的女孩,永远。

		If I know that I'm remebered that way, then I can face anything, anyting.

	\endwriting


	\writing{四月。湖}{2006年04月01日 ~ 22:11:56} %<<<2

		四月,在一个浮着灰云朵的阴天,敲开我日期的窗扇。

		西堤的春柳正婀娜缱绻,随温煦的季风,悠悠斜飞在游人陶醉的视线。观柳,是最宜在这般境况与
	时节。

		春风的剪刀,妆成的碧树,灰云缝隙里犹疑不定的洁白日光,水天一色的湖。造就了一处浑然天成
	的幻境,在四月,未名真假的温暖里,把渺小的生命,投入宇宙的无限。

		天空与水面,只相隔那纤细精巧的一段堤岸,浅绿的,点缀着杏桃矜持羞涩的花朵的堤岸。如佳人
	细腰间轻挽的裙带,美而不妖,是日暮倚修竹般的凄清,是自开自落的年月,是寂寞的孤傲。我于是爱
	那堤岸的神情,几分的倦怠,几分的随意,却是维系了天水间的沟通,用不着痕迹的一笔。如姑娘脸上
	涌起的一抹红晕那样,不是刻意的胭脂,却比胭脂的美更摄人魂魄。

		我以为,这天空是另一汪大湖,在人的头顶,在尘世的高处,凌空俯望着一幕幕悲喜。而人,终不
	过这细细裙带上的花纹,我们站在上边,被上下的湖水包围,感觉着被环绕的幸福与安全,或者,恐惧
	着被围困的痛苦与惊慌。而我们,终于是在这里了,无可逃遁,无可拒绝。在一个水的世界,仰头,又
	俯身,看天湖上的落雨,将人间的湖注满。人间,天上,两处湖水,一样的无言,一样的神秘未知。

		四月,我在这裙带似的堤岸上徜徉,是起风的日子,我的发也斜飞,像所有萌发着希望与季节的树
	木一样。湖水被激起风浪,拍打那护岸的巨石,汩汩作响,陪伴着耳畔寂寞的风声,一片天籁交响。

		来游湖的,多是老人,他们坐在湖畔的石椅,并肩相靠,并不言语,只望着渺茫无际的湖,和那若
	隐若现的远山的影。是多少并肩的岁月,成就了此刻风轻云淡的守望。

		看他们的背影,我无法获知那些风雨浮沉,是怎样把彼此的心灵打磨,终成一块浑圆整洁的石子。
	或许,那便是女娲补天时候,最后缺失的那一块,它从不曾纯在于物质的世界,你不可以触摸与把握,
	只能够在人心的深处获得。

		你知道他在那里了,岁月在那里了,于是,那一块石头在那里了。天空是完整的,不因女娲的牺牲
	,而是因人的多情。

		你会和谁,在后来的堤岸上,并肩相靠,补全天空的裂缝?

		你们是不是也会终于安静下来,不言语,不说笑,只听着风浪,听着柳色,洞悉一生的奥义,将对
	方褶皱的手掌,紧紧握着。

		有一块石子,在岁月的尽头等待,在青春躁动的梦里,调着邈远的色彩,如远山之清。

		我仿佛一只素白的瓷器,未着丝毫纹绘。

		我知道,人是如这样的器皿,等待着叫做灵魂的液体来注满,来成为有所感,有所情的生灵。

		生命,会像眼前的湖泊,天上的,人间的,涌起浓白或银灰的云朵,荡漾温柔或凶猛的水波。湖,
	在我们的眼前,湖,在我们的心田,在最近,也最远的空间。

		让一个个风季,吹进所有的春天,让万物复活,或者新生,像上帝最初造出人类那样,让泥土有了
	呼吸。我听到,那些呼吸,急促的,小心的,如爱人睡眠里微微的鼻息,那么精巧,在明亮的清晨,在
	无光的暗夜,萌发着生动的爱意。

		那爱,在泥土中,在我们的身体,无限扩大,盈满生命的湖水,像许多个夏天的雨水那样,注满干
	涸,映着云朵 —— 那些游弋在天湖的点点白帆。


		万象皆宾客,我终于了解。生命是短暂的降临,浮光掠影。

		小小的器皿,等那水的蒸发,便终归于空。而人的相遇与多情,则是最虚无的美景。

		湖光山色,是我们蓄意的留恋,是你眼底的水色,纠缠着我的前生今世。

		你不见,我的眼睛,含着湖水,一汪天青色的泪。因为珍惜,我用三生三世,来期许一块石子的浑
	圆整洁。

		我以为,天地的大美,是需存着无尽的善心与大爱,来赞美与热爱。只从小小的指尖,让我体会温
	暖,懂得一只瓷器,懂得湖水的动静,和那水面下无所不在的游鱼的沉默与欢乐。


		四月,在这湖上开始。湖的主人,那权倾一时的帝王早已灰飞烟灭。而春是依旧,柳是依旧。在永
	远含混不清的时间上,我们是点刻之上的微尘。只有多情的岁月,是永恒的归属,只有这人间天上的湖
	水,暗示着宇宙的真相,将我们的空瓶注满,将人的肉体充盈。


		我愿意,有一种等待。我愿意,在时光的对岸,在远山的缥缈间,有谁舟楫泛于无尽的未知。

		会有一双手掌,紧紧相握。会有你,在明亮的清晨,在无光的暗夜,把一块石子交付,向天地有所
	感激,有所眷恋。


		我在湖上,我在你的湖上,我在人间天上,沉默里微笑,幸福中哀伤。

	\endwriting


	\writing{如果。告别}{2006年04月05日 ~ 09:49:54} %<<<2

		踏上又一段阶梯,一个笑意纯白的孩子向我挥手,阳光正洒满四月的窗口。是告别吗,她站在楼下
	,一言不发,让清晨的雾色渐渐模糊了身影。

		没有一句再见。于是,我们便再不必,也不会相见。

		相隔于日期,相隔于记忆,汪洋如海。

		上楼,在醒着的午夜,看时针与分针的相遇。然后,一个孩子从身体中出逃,另一个孩子,用手指
	叩着我的额头。她说,生日快乐。她说,要幸福地生活。然后,朋友们发来短信,敲开锁了的房门,送
	上精心包装的礼物。

		我于是知道,有一些时光又离我而去了。并不是感伤的事,失去是常态,获得才是异态。因为这样
	不落痕迹的失去,我们的生命才有可能被温暖包裹。我笑着,打开礼物,回复短信,我笑着,感谢每一
	个珍惜彼此的朋友。

		一个如常的日子。我如常醒来。我看到世界开满了花朵,鸟的侧翼,划过蓝空的透明。二十个春天
	,每一次,都是如此,像一曲完美的歌吟,像风的手掌抚过温热的耳际。离开吧,不要在我的楼下停留
	。她看着我,洞张着一双黑眼睛。离开吧,你该躲进记忆的盒子,像所有的从前一样。不要流眼泪,因
	为,这是多么美的季节。


		我无法说一句再见,只可以道一句别离。

		她离开了,离开了。从我的视野消隐。

		在雾色里,隔了日期,和记忆。

		后来的时光中,我们彼此想念。

	\endwriting


	\writing{一个下午}{2006年04月07日 ~ 23:46:58} %<<<2

		四月,起风的下午,世界摇曳在丁香淡紫的影子里,含着爱娇的神情,面目模糊,语气忧伤。坐在
	阳光里的我,被春天包裹,无所思想,无所牵挂。

		几年前,听到关于丁香的传说 —— 若找到5片花瓣的丁香,便能够找到幸福。这一种姿态纤弱的花
	呀,总是簇拥成团团的芬芳,探出形状精巧的绿叶。爱她,十字的细小花朵,小心谨慎的模样,安静里
	开放,淡紫或洁白。那几个春天,我们经过许多棵丁香树,我们无数次仰起头,寻找5片花瓣的丁香,
	却终于一无所获。而幸福,总是充满了诱惑,让你不禁期待猜想:或许,她就在下一片花丛,或许,她
	也正悉心又焦急地等待,如树下所有痴痴仰望的女孩一般。

		后来,你说,幸福是小概率的事件。在丁香的心里也结着愁怨,向岁月的遗憾致敬,向年年的春色
	道别。我开始渐渐相信,关于幸福,或许不过一场彩绘的谎言。

		而又是谁,在春天里编造着这样美好的传说,让你向一棵棵无言的花树索求幸福的可能?风荡漾所
	有我们无法把握的未知,带走孩子的微笑,飘进一处灰白的世界,没有奢望。

		幸福也许于千里之外,幸福的可能,总是咫尺天涯的渺茫。

		便只在心底安放一支花朵,淡紫或洁白,吐露着5片花瓣。幸福,是多么不真的体验,可以触摸一
	样,却又瞬间消亡。学习耕种心田,在荒凉里洞见繁华,又在繁华里逃遁向荒凉。

		细雨缠绵,花朵在生活的背面,开得完满灿烂。


		这个春天,我仿佛可以开始坦然,可以不去寻找,和仰望。那曾经的女孩,走失在不归的半途,与
	我分道扬镳。你快乐吗,你有过快乐吧?在仰望与寻找的年月,你的心被期待盈满。却因为成长,因为
	懂得,打翻了那水杯,摔得粉碎。


		无所思想,无所牵挂。说来简单,貌似洒脱。


		碎的期望,在现实的地面沉睡,幸福却又在无意的回目间遇见。那5片花瓣的丁香,开放眼前,不
	费吹灰之力,她已开放在我的眼前。在许多个春季的找寻与失望之后,我们这样偶然地相遇,如一场命
	定的安排。好似注定了,我的目光,她的吐露。

		我仿佛看到自己在心底安放的花朵,那么细小,那么小心又谨慎,开着美好的花朵。

		为你预示幸福,给你幸福的可能。


		一切的发生,只在那瞬间的一次无意。我与幸福的相遇,想来不可思议。

		原来,幸福从来是不可苦苦找寻的。


		这个起风的下午,我又站在丁香树下。

		她开满一树的希望,身姿婀娜娇弱。

		我的影子,和丁香的影子交错在初生春草的地面,轻摇着,像永不枯竭的梦境,那么透彻,那么轻
	盈。于是闭了双眼,任四月在耳畔轻揉细语。

		你的幸福,在树梢,在想象,在掌心。


		我仿佛等待,又好像留恋,在午后的园中坐了许久。望那开放幸福的花树。

		她甜蜜的忧伤,模糊着时光的面目。

		你会记得吗,那些迷失在来路上的幸福,我笑着的眼,你飞在风中的裙角。我们的花朵,在那里开
	放,在那里凋败,不用一个花季的长度。幸福,从未在花丛,幸福在我们度过的每寸光阴。是你们,陪
	伴我经过的全部。


		那时的我们,却不会知道,这一切真相。直到,我们失散在汪洋的不见,城市的阻隔,直到,我在
	这个春天的下午,见到5片花瓣的丁香,才终于发觉。所有的度过,是酿着的回忆,是我日后赖以生存
	的指望。


		我要你,借我肩膀,借我手掌。

		我要四月的天空,涂上水色的温柔。为了有所珍爱,有所寄托。包裹在春天的我,假装无所思想,
	无所牵挂。

		幸福,不是小概率的事件,不是谎言。

		是我在不断的告别里,偷来的甜美空闲。

		用来感激,用来奢望。

	\endwriting


	\writing{病话}{2006年04月14日 ~ 14:43:29} %<<<2

		病才稍好,就开始照镜子。却迎面遇见一个憔悴不堪的孩子,连自己都不忍看了。

		歪歪斜斜的几天,在病床上辗转,在昏睡里度日。年光短暂,而一日的折磨却是难耐。没了读书的
	心思,没了闲愁的兴致,肉体在病痛里挣扎,哪还有气力满足你精神的奢侈。四月的天气,忽而风,忽
	而晴的,像十所说的那样 —— 正如女人的心情。我躲在房里,忍住咽喉的疼痛喝着热开水,一杯又一杯
	,只希冀这有和煦的阳光吹灭灰云,明天早上,我的头脑也清醒了,身体也爽朗了,又是完满愉快的春
	。

		病,是你不得对它生气的,虽然,它那么无端地剥夺了你的许多原本正当的欢乐。病,保持着无辜
	的模样,向你笑笑,并不说一句抱歉,只等你用苦的良药来同它们斗争。这便让人无可奈何,它与你的
	肉身同在了,在病着的时候,你和你的病早已融为一体,敌我不分。它大概是恶魔,不落痕迹地把你折
	磨,拽倒一整个健康光明的世界,只留下悲戚与痛苦于你消受。病,它多残忍,不给你一个理由,就轻
	易把你锁在日夜的煎熬。

		病,是那说来极雅,实则难堪的境况。

		你看那诗词里“病起萧萧两鬓华,卧看残月上窗纱。”病中的岁月,总透露着几分哀婉清绝的诗情。
	病中的诗人,大约的确是更适合于诗了。而对于并无从体会其中雅趣的我等来讲,病只是一次肉体的洗
	礼,只是惯常的磨难。是谁说过的罢,唯有在肉体的折磨中,我们的灵魂方能保持自由与清醒。由此观
	之,真正能够解悟生命奥义的人,是非病人末署了。因为疾病,人们在痛苦中思索,因为思索,人们的
	肉体更加虚弱。而痛苦是人类接近美善,接近真理的阶梯,我想,这一点并不假的。一个无痛苦的世界
	决不会是天堂,而是思想者的地狱。我们是在一次次折磨挣扎间,不断丰盈着生命的空虚无依。


		躺在床上的时候,我动弹不得,稍微侧身,呼吸便又急促困难起来。我知道,这病不过小毛病,是
	吃几天药便会痊愈的。但仍不敢掉以轻心,你可以说我是个贪生的人,但我更乐于认为这是对于生命的
	珍视和尊重。虽然,我总以为,生,不过是我们漫长的沉睡中一次偶然的醒来。在我来这世上前,我睡
	在无知无觉的黑暗,在我离这人世而去后,我依旧睡在无知无觉的黑暗。醒来,是片刻的停留,恒久的
	,是未知的无垠黑暗,那里才是我们真正的存在。生,或许是一种异常的状态,因为异常,而愈显得美
	妙而神奇。我可以把死亡想象得很美,却总抵不过对于生的热爱。因为在这一团气息积聚的时刻,我可
	以触摸,我可以感知,我可以爱上一些什么,可以悲伤,可以想念。这些,都是在黑暗里无法体会的,
	纵使,生是苦难的汪洋,我也愿意是无目流放的一只小舟。我已经不再恐惧死亡,我所恐惧的是不能够
	思想,不能够握住亲人的手掌,吻他们生了皱纹的额头。


		病,把人从麻木的生活里强行拖出,给你不安。让你明白,你以为你拥有的一切是多么轻易地就会
	与你诀别。让你对生命怀了真实的敬畏。

		到医院看病,被淹没在病人中的我,在挂号大厅里不禁迷茫。满眼是患病的人,每个人不知道都受
	着怎样的痛苦,他们来到医院,拥挤,推搡,为了与医生短暂的会面,为了一张解决痛苦的处方。生病
	的人是可怜的,而谁又不曾是病人呢。我是个多病的孩子,我的许多回忆多与医院有关。我却并不惧怕
	打针和吃药,仿佛我对苦痛有着特有的免疫似的。我也不害怕那些穿白大褂的大夫,我甚至喜欢他们的
	样子,对听诊器非常好奇。有个点击率挺高的博客,好像叫北京女病人。病人,在人们的眼中是什么呢
	,似乎已经成为某种标新立异,特立独行的代表。病人,是与众不同的那么一群人,因为病而无所顾忌
	,而真正的病人当然不会如此,我见到许多病人,他们有的面色灰暗,有的精神脆弱,有的掉净了头发
	。他们讨厌别人说,“你坚强一点。”他们想骂一句,“废话,得病的不是你,要是你得病了,你给我坚
	强一个看看!”当然,我这里说的,多是得了“大病”的人。我不喜欢有人把自己叫做北京女病人,或者
	什么什么病人,倒不是觉得这是对病人的不尊重,只是觉得她大概没有想过真病人们的感受。你那种病
	,真是无病呻吟的病,或许,也的确称得上是一种病。


		得病,绝对不是什么快乐的,值得炫耀的事情。而小孩子也多有盼望得病的经历。教文学史的张教
	授,在一次谈话里就说起他小时候非常想生病,因为生病了就会有罐头吃,但不幸的是,当他终于生病
	了,罐头也变得索然无味了。看着胖乎乎的教授,想他小时候馋嘴的模样,不禁失笑。他见我们吃冰棒
	,便坦言他也想吃,但是怕长胖,原来他在减肥。胃口好,想必也是某种福气。生病的人,多数没了吃
	的念想,只歪歪斜斜地躺着吧,只把希望寄托在一把把,一勺勺吞下去的药吧。于是,生病的人多是消
	瘦了,蜡黄蜡黄的脸,无精打采的模样。诗人常在病中诗兴大发,多少也与诗人经常生病有关?所以,
	诗人也多是消瘦的,少见如《围城》里描述的那种丸子状肥白的。为了诗,诗人病了,因为病,诗人瘦
	了,瘦了,诗人更容易病了…… 岂不是成了恶性的循环。但也正是诗人们的不幸,成全了读者们的阅读
	和飞翔的幸福。所以,诗人是伟大的。


		像我这样,病情稍有好转就开始照镜子大概是极不相宜的。这不堪的模样,只会引我这般小情调的
	小女子又一阵自家伤感。病的好转却是让人在心底生出许多的希望,我不必再每天卧床,不必再呆望着
	窗上的一小角灰天空而不敢动弹。我可以翻几本闲书了,可以剥一只橙子,弄得一手芳香的汁液,可以
	看看我的狗在地板上幸福地打滚。生活的琐碎都变得很新鲜,因为病时的寂寞枯燥。人原来是不可以不
	去经历些辗转,忍受些折磨。


		你是不得对病生气的,它那么无辜。怪只怪我们没有照顾好自己,让自己又在床上上了几日几夜的
	哲学课。那些不得不吞食的苦药,应该就是学费了。

	\endwriting


	\writing{很美}{2006年04月16日 ~ 11:50:35} %<<<2

		那好像是许多个似曾相识的午后,你们都还在小木床上睡着。蓝窗帘被吹起来,罩不住洁白的日光
	。斑斑的影子泻落到绣着小兔的被子上,棉花糖似的云,正挂在天上。

		只有我,独自醒着。我是不睡午觉的孩子。

		百无聊赖,在远远近近的呼吸声里清醒,看临床的小男孩在梦里一点点流出口水,在枕头上留下清
	晰的湿迹。然后,我开始拆被上的线头,缠在手上,实在无趣,便咬一咬小木床的床栏杆。

		我的午后,在那么静的时光里,如一尾沉默的鱼。一个孩子独自的游戏。

		而不睡午觉的孩子,是不允许起床吃午点的。每次起床时,我便总会装作困倦的模样,摇摇晃晃地
	坐起来,绝不着急。现在我知道,那伪装一定是极其可笑而拙劣的,根本逃不过成年人的眼睛。但童年
	的我,竟然很少被抓,多数是可以蒙混过关,顺利吃到午点的,并引以为豪。如今想来,大概是老师的
	网开一面了。

		午睡的那几个小时,曾是在幼儿园中最痛苦的时间,今天却觉得美妙。

		因为清醒,我记住飘起又落下的蓝窗帘和洁白的日光,记住墙外小贩和老太太们的讨价还价,记住
	摩托车启动的声音,还有楼道里高跟鞋经过的脚步。

		我想,这些可爱的零碎,其他的小朋友是在流着口水的梦里错过了。

		当然,我也错过了流口水的机会。

		我想,我并不是一个讨人喜欢的小孩子。我的嘴不甜,也不会做些让人觉得可爱的事情,还经常莫
	名地哭起来。

		我却从不当着人面大哭。

		有人说,小孩子的哭多少有表演的成分,是为了博取大人的疼爱。而我,似乎从来没有这个意思,
	我甚至害怕被他们发现。于是喜欢躲在被子里,默默饮泣,有时还不自觉地把自己想象成了倍受虐待和
	委屈的灰姑娘似的。我好像乐于体会那种独自的哀伤。

		这连我自己也觉得无法解释,毕竟,我是那么小那么小的孩子。

		老师们只在图画课上表扬我,并把我的画当作范画那样贴在小黑板上。那是我最光荣的时刻。而更
	多的时候,在没有图画课的时候,我像只胆小的小猫,缩在自己的位置上。

		在幼儿园,可以为所欲为的,是那些老师们宠爱的孩子。像我这样并没有人多理睬的小朋友,总是
	投了羡慕的眼光,跟在他们身后,想跟人家做好朋友。

		我记得很清楚,有个小女孩,老师经常把她抱在怀里,还夸她像洋娃娃。我们都知道老师喜欢她,
	便都喜欢和她玩。其实为些什么呢,小孩子哪里有什么功利的想法,但这又俨然是成人世界的缩微一样
	。

		她给我的印象深刻,以至于这么多年后我还能够叫出她的名字。当时并没有美丑的判断,现在翻开
	相册看,那果然是个洋娃娃一样可爱的小孩子,不能怪老师喜欢她了。

		去年夏天经过原来幼儿园,隔了铁栏杆给苏指我们班原来的位置。

		门窗早就换了,只是望过去,窗口依旧飘着蓝色的窗帘,虽然已经是崭新的。

		夏天的绿树,和很多年前一样,只是那树荫浓了许多。

		那年,我们一群小朋友就站在那些树荫里,看一位胖叔叔表演溜溜球,然后又哭嚷着让爸爸妈妈给
	我们也买一只溜溜球,自然是从那位叔叔那里。那天,胖叔叔很开心地离开了,并说以后会来教大家很
	多有趣的玩法,但后来他再没有出现过,我们的期待也落了空。

		我记得树上有“吊死鬼”,一种恐怖的绿色小肉虫,不知道治好没有,现在看上去倒是很健康的样子
	,叶片在烈日下闪闪烁烁。看门的老爷爷早已经退休养老,换成了一个黑脸大汉。那老爷爷,总是笑笑
	的样子,对每个接走的小朋友挥挥手。他管我叫“小钟”,因为那时我家住在大钟寺。其实这有什么关联
	呢?他是个和善又卡通的老爷爷,很瘦很瘦。

		黑脸大汉不许我们进入幼儿园。即使我表明了我曾是这里的小朋友的身份,并说出一大串当时老师
	的姓名,他还是摇头,一口回绝。于是,便和苏在门外拍了几张照片。我才发现,门上多了许多块金色
	的牌子,看来,我的幼儿园已经发达了起来。据说,现在能到这里入托的,都是有钱人家的小孩。我于
	是庆幸,我早生了十几年。

		最近看张元的《看上去很美》。一个可爱的小男孩,方枪枪。当时,剧组在全北京时招募演员的事
	情,还隐约记得。故事发生在幼儿园,却是70年代的。孩子们都住在大殿改造的睡眠室里。我不知道,
	北京是否真有过这样的幼儿园,我想,如果有,那住起来可能是挺恐怖的。(也难怪他们把老师想象成
	妖怪)我还是更喜欢我们的小木床,蓝窗帘。

		片中,孩子们的眼睛通透而天真,看得我自惭形秽。我的确是长大了,长得太大了。再也没有了那
	种眼神,湖水似的,那么深,又那么净。成人的眼睛,怎么看,也是迷惘。这也是我嫉妒孩子的原因。
	因为,他们拥有的,恰恰是我无可挽回的丧失。

		但毕竟,我是拥有过的,可悲的只是,当时的我们,都全然不知,像所有的孩子一样。

		电影中的一些场景给我印象深刻,像旋转在蓝空下的小木马转椅,比如方枪枪探出窗子看外边小朋
	友们玩耍。

		我不喜欢玩转椅,我清晰地记得,那滋味非常痛苦。我没有探出过脑袋向外边看,但我想起,窗台
	上栽种的萝卜花,白菜花,和它们散发的略臭的奇怪味道。

		有多少扇窗啊,在我们的记忆里,开开合合,被风刮上,又被我们推开。

		那小木床上,被我咬的齿痕是否还在呢?

		我总像那个窗口的方枪枪,看着玩耍的小朋友,看着许多个自己,在那里,在阳光里,笑着哭着,
	一路奔跑,又跌倒。

		似曾相识,我想起你们来,睡在流着口水的午后。我想这一切都美极了,并不只是看上去。

		你们还好吗。

		你们都去了哪。

	\endwriting


	\writing{话语。碎片}{2006年04月23日 ~ 19:33:46} %<<<2

		这是一场甜美到繁复的春季。带着光亮的柳絮飞行,从城的这一端,到城的那一边,经过女孩子的
	窗口,经过不识春风的夜晚,充满这喧哗又寂寞的世界。

		我立在无力的风中,绿树的灰影子在我的脚下婆娑。想着一些很远,一些很近的幸福,所有在坚强
	里挣扎的喘息。我的视野,被温暖的日光包围。找出压在衣柜底层的布裙子,翻弄冬天厚重的衣服,沉
	沉的,在手心的重量,正如一整个灰暗的季节。我们,是从那里穿越和跋涉,迎来眼前的满目光明,有
	晴空,有引人沉迷的希望。

		就这样,我们告别着,许多含混不清的情绪,在春风的怀抱。

		海棠花涂着淡粉红的胭脂,谦卑地向蓝空致意,一园的桃红柳绿,让人的身体松软下来,想在午后
	做几场昏天黑地的美梦,而不急着醒来。这温暖,让你觉得,春天一定是幻觉,要么,便是完美的阴谋
	。

		一切的完满明丽,总会感觉不真,好象世间所有近在咫尺的幸福,总好象虚妄得空无一物。

		我伸出手,并不能够触摸你。时间漫过你我的皮肤,你的侧脸茫然。因为不可以感知与干预的变迁
	,我缺乏基本的安全感。仿佛平静似水的湖上,之下却万般波涛暗涌。

		生命的动荡,永远存在,永远无可排除和消亡。

		人大概不可以迷信些什么,人大概又不得不迷信些什么。那些,并不能够轻易描述的需要和相信,
	是简单的,赖以生存的食粮。

		我看到春天,于是慌乱得语无伦次。在我的心底,是相信和依赖着许多,是深深依靠着一些什么,
	它们如草木般生长,等候着雨露。我迷信着,又执着万分。


		你无法听见,我的呼吸,我的微笑。你无从明白,我的梦境,我的呓语。

		所有的日子,都细如烟丝,抽离,又兀自消散。我只可以陪伴你,望人间的事实,却不能够和你们
	诉说,另外的一处,永无疆界的世界 —— 属于我的王国。

		我不会像诗人那样宣布,我是国王。我只是,那水彩绘成的城堡的小小女奴,每一天擦拭玻璃和地
	板,每一天修剪花木,整理书籍。

		我想这王国是自由的,快乐的,不需要任何人打搅的,却也是寥落的,寂寞的,毫无着落的空中楼
	阁。

		而日子,是城前不息的河水,保护着城堡的纯净和决绝,也围困着它自己。

		人,总要住在自己建筑的城里,有的宽敞,有的狭小,有的夜夜笙歌,有的寂寞清冷。热闹的,也
	许在黎明后发觉繁华的空无,无声的,也许在沉思里懂得孤独的神圣。

		生活教我们没有理由和权利来评价别人的选择,人们住在各自的城里,自得其乐,也自饮苦酒。

		小小的女奴,我守着自己一半荒芜,一半茂盛的城堡,度过时节的流变,一年年的悲喜。跳自己的
	舞蹈,记录昨夜的梦话,踏着星辉月露,有时遗忘,有时想念。

		这围困自我的城,成全着绝对的自由,也决定了真实的疏离。而我,终究不可以蹬城瞭望,不可以
	期待遥远的马蹄声声逼近。因为,我的城堡,是不可到达的海岛,是不存在的天地。

		我在漂流,随着命运的洪流,我无法让你听见,我的呼吸,我的微笑,无从令你明白,我的梦境,
	我的呓语。

		它们,都是最美的,最真切的谎言。我在希望里沉沦,望穿秋水,我在你看不到的地方,独自享用
	一场悲怆感伤,又一场火树银花。

		我想说,让我们安放好自己。我想问你,明天的晴雨。

		我知道,眼前的快乐,稍纵即逝。

		去年的日记上端正地书写着,“这是不该再起贪念的春天。”

		而人心,却从不会就此罢手。那也是我们自己的魔,苦痛的根。

		这是困难的事,在繁盛中学习放弃。而穿越的过程,正是人的自我完满,用挣扎,用勇气,用智慧。


		我只是立在风中的孩子,谦卑的,固守城堡的小小女奴。

		呼吸,微笑,在透出明亮的一切缝隙。这一场春季,落在我的目光,如柳絮的无端飞行。

	\endwriting


	\writing{祝福。爱}{2006年04月25日 ~ 14:28:27} %<<<2

		在欢乐和幸福里,我看到粉红,看到金黄,看到她穿着白纱,缓缓走下台阶,托着长长的裙裾。我
	在她身后跟随,抛洒玫瑰花瓣,它们从她的眼前落下,落下,不像一阵微雨,却如一场无声的梦幻。

		她怀抱着百合与玫瑰的花束,笑意甜美,如所有的新娘一样。婚礼,在这一个四月,一个有晴空的
	日子。静的姐姐出嫁了,走上红毯,也走出从前的岁月。她将是妻,是母亲,是美的,善的女子。不再
	是任性的小姑娘,不可以撅起嘴巴,索要糖果和洋娃娃。人生,在短短的红毯上转向,我们终于抛弃了
	一个个自己,迎着崭新的生活进发。这令人充满了希望,无尽的可能在眼前展开,像未知的花园,栽满
	待放的美好。

		我们在祝福,每个人都在祝福。新郎,我们叫他穆穆,是彬彬有礼的男子,眼中充满善良。他为她
	戴上戒指,许下爱的承诺,会永远对她好,请父母放心。她的幸福,流露在一次次目光流转,那么精致
	的,光滑的幸福啊,像这指间的钻石。世上最坚硬的石头,最明亮夺目的石头。

		姐姐出嫁了。人们不约而同问起妹妹的婚期,什么时候喝你的喜酒呢。一袭黄裙的静腼腆地笑,并
	没有回答。我和朱也笑,想6年后的静,当她如姐姐一般的年纪。三个女孩子,天真中带着淡淡的憧憬
	与恐惧,展望貌似遥远,却又近在咫尺的日子。而就在我们的不远处,姐姐换上红色的礼服,在为来宾
	们敬酒。人们起着哄,说笑声,碰杯声充满了空气,粉红的,甜的空气。

		这是一个完美无瑕的日子。隆重的喜筵过后,宾客一一离席,只剩下意犹未尽的我们不断拍照,在
	鲜花前,在气球中间。美好的时刻,总是飞逝。姐姐有些疲惫,婚礼是令人劳累的。她说,以后结婚别
	办事,旅游结婚挺好。我也觉得,爱情并不需要一场庄严盛大的婚礼,不需要许多人的到场和见证。但
	也许,婚姻是需要的。喜筵的初衷总是美好的,想与亲朋好友分享幸福。而这幸福,若分与太多不相干
	的,并不熟识的陌生人,大约是件难免尴尬的事情。毕竟,这爱,是两个人的事情,不需要旁观者的驻
	足。只有喜筵上,新娘的光彩和美丽,是需要观众的,便无论陌生与熟悉。服务人员开始整理残席,这
	金碧辉煌的大厅,又一场婚礼落幕,而明天,明天的明天还有许多的婚礼同样甜美,同样隆重地举行。
	空荡下来的会场,让人觉得几小时前的繁华盛丽,恍如一梦。

		我大约也会出嫁,朱也会,静也会。就好像更早更早的时候,我们想,自己大约也会和一个男孩拉
	起手走过街道那样。我们总是在今天的展望里,又记得昨天的展望。一圈圈,永无宁息的未知,淡淡的
	憧憬,淡淡的恐惧。成人的世界,无比真实,把生活的真相撕裂给你看个明白。那么美的,又多么残忍
	。我出嫁的那天,你会在吗,你会为我把花瓣一路抛洒吗。我愿意为你托起长长的裙裾,看你手捧花束
	,许爱的誓言。

		而我,并不会告知你们,我的婚期。我将秘密地在有海的地方结婚,在沙滩写下誓言,再等着潮水
	将他们磨平。我也许会穿白纱,也许会在发上,插芬芳的茉莉。我会给你们写信,告诉你们这一切,然
	后在信筒边,日夜期待你们的祝福,由邮递员送达。我不需要你们的见证,我的爱情,该是自由的,我
	的婚姻,不必庄严的盛丽。没有观众,我可以是自顾自美丽着的新娘,只要,我的爱人懂得,只要,我
	的爱人爱着。

		这所有,妄想似的明天,明天的明天。一个小女孩纯白白的想象。虽然,我也早已不是可以撅着嘴
	,索要一颗糖果的年纪了。爱情,我们从没有看个仔细,婚姻,更是毫无指望的彼岸之花。我们没有小
	舟,幸福的可能,仿佛无可到达。而所有的所有,终于将有答案,有时间,有岁月,在日记的背面为你
	写下。不必疑惑和心急,安静地走过去,我们都将获得,各自的解答。

		我祝福着,一切爱情。

		我祝福着你,祝福着她,也祝福着自己。变更的故事,在6年后的回首,等待着彼此的想念。

		你是梦中的人,我是梦中迷惑的光线。

	\endwriting


	\writing{旧日记}{2006年04月28日 ~ 14:35:17} %<<<2

		无意间,翻到去年的日记,一样的春天,不一样的心绪和语言。

		年岁过去,日期作为永恒的标注。湿润的许多季节,把我从记忆的牢笼唤醒。然后,就有个孩子,
	在心底呼唤,有些恍如一梦的幻觉,在瞬间里清晰而贴切。

		所有的文字,是我注定的跋涉。

		在这里,它们的出现,只是一种无力却深情的纪念。


		\subpart{05年4月4日 ~ 晴雨}

		晴雨不定。清明时节,本应是浸饱了水分的。

		独自等车,回身时见田野已是油绿的一片。这许多未及发现的欣喜,在初春就要落下雨来的午后,
	向我涌来。天空已阴灰了大半。我便期待这一场雨,或许只是轻轻洒过的雨。在登车的一刻,果然降临
	。躲在车窗里,看这世界被淋湿大半。

		就想,独听的窗前雨,就想,同谁共剪的烛火。对于雨,心中总是情味悠长。宛若隐隐于口的一缕
	清苦,类茶的滋味。只有在风雨飘摇的日子,才及体尝的幽香。因这清苦,才得脱俗的甘味。经常想,
	某个落雨的傍晚,同一知己,临窗共品一茗。那确是人生的大幸,因这同听雨的人,是只可随遇,而不
	可欲求的。多数时,我们只是独自坐着,看这世界被润泽,浸湿。人总需耐得住这无声息间的寂寞。品
	自己的茶,读悲喜交加的天地。

		风雨如晦,鸡鸣不已。既见君子,云胡不喜?

		读诗经单纯朴质的语句,便莞尔一笑。落雨的日子,绵延了千年,咏叹的却同是一心的情思。是无
	须悲伤,只须欣喜的。我们立于雨的世界,守望或翘首,源于思念或别离,我们咀嚼一句简单的言语,
	就恍惚里思绪万千。这是难于言表的幸福,便只是斜倚于门吧。便只是用流转的美目望穿秋水吧。没有
	人归来,此时的自己,热望的或不过是更通透的一个自己。翩翩而至。

		下车时,天业已放晴。

		瞬儿风,瞬儿雨,瞬儿晴。仰头望天,竟蓝得悠远。清明时节,我站在岁月轮换的拐点上。原本,
	万象无常,素无定数。想这逆旅之上,又有什么是真正值得在意。

		心田无涯,悲喜随常。


		\subpart{2005年4月16日 ~ 无心}

		慵散的日子,平淡如常。

		灿烂如许的春,摇摇曳曳地挤满窗外。那园子或许是这校园里最得风情的地方。\par
		小池塘中已注入清水,的确是清水,尤可见池底地砖的纹路。就想起,初入\par
		大学的盛夏,想那池塘中零零散散的小莲。\par
		盈盈的身姿,看似娇弱,内骨却坚实而高洁。\par
		有人倚着池边湖石而卧,读一本书,会是诗吗,我满心欢喜地猜想。

		夏天,总是热烈明艳,在这园子,却也得清幽和风致。只不过是一塘的水罢了,竟就溶释了太多的
	浮躁。

		冬天的时候,水便抽干,偶尔也见人站在业已无水的池子中打太极拳。我是没有如此的兴致,很少
	在园子逗留,觉那抽干的池塘,已成凹陷的深坑,虽然,它是那么浅的,浅到只可称之为池的。

		直到这个春天,那晚上同小鹿偶然从园子走过,才惊见一池塘又已粼粼的水光。

		不由得在池边站定许久。看一泓水泽将隔岸的灯火和树影都流溢成彩,成墨。\par
		有人唱歌,拨动琴弦,听出些已淡漠的忧伤。\par
		和小鹿说起各自的事,都是毫无条理的片段,并不清晰的心绪和过往,却觉真切动人。

		临水的夜晚,不免寒意阵阵,而心却清醒。\par
		好象这盈满的池水,在许多个干涸的日子后,又温存柔润起来。

		春天,总会是重重的惊喜。是天地是本然,如甜蜜的糖果,待谁的素手将层层的花纸剥开。

		后来才知,在临水的夜里,身后树丛间正有一株丁香在悄悄酝酿着盛放。

		我没有发现她急切而小心的心,我也没有听见她小小花朵裂放时的轻笑。而次日的日光中,我终见
	得你细细小小间簇拥的喜悦。

		澄澈的蓝空下,你毫不张扬地默默美丽着。

		人说,找到5片花瓣的丁香便可得到真爱和幸福。我便情愿迷信这传说。丁香,却仍是本初的摸样
	可爱。

		有人曾花费漫长的一个下午在丁香树下寻那5瓣的花朵吗?\par
		有谁曾找见,又将它轻手赠予他的爱人吗?\par
		丁香曾为谁,而竭尽了气力长出5片花瓣吗?

		周五的下午,班里组织到植物园赏花。

		的确是繁花锦簇的世界。有一处叫丁香园的,栽满各品种的丁香。树身小小的牌子上标明它们的产
	地和血统。

		本该是更觉美妙的时刻,却分明少了于小园中赏花的欣喜。

		满园的繁华,原不如一枝的风姿令人动情。\par
		我笑笑,由此观,我们本不必拥有许多的。幸福,是于疏落甚至寂寞中长成。

		所想,只是如池塘一般将身躯用清水注满,只是如独自的丁香一般寂寞地美丽着。\par
		不该再起贪念的春天。


		这并不是一次怀念,所有的怀念终归徒劳。

	\endwriting


	\writing{汪洋}{2006年05月01日 ~ 19:41:43} %<<<2

		那一年夏天,她独自离开,到彼岸的岛国去。她开始不断给他写信。

		时间定格在1936年。他们相识的第四个年份。

		四年前,身怀有孕的她,被未婚夫无情抛弃在旅店。因交不出宿费而被店主百般责难,并停止了饮
	食供给。

		四年前,一样是夏天,洪水泛滥,店主与旅客纷纷逃离,孤苦无依的她被救生船救出。

		是他,接受了这样的她,在整个近于荒芜的残忍世界。

		四年前,仿佛一处光明的希望,他们相爱,相依,共同着自由的理想和热情。

		萧红,萧军,两个总会同时出现的名字。在一个春天的末尾,我躲在房里,读那些遥远了光阴的书
	信,42封,萧红写给萧军的信。

		她唤他,均,军,君先生,三郎。她署名,莹,吟,萧,红,荣子,小鹅。这些丰富而可爱的称呼
	,让人遐想他们的亲密。

		而事实是,她决定离开他,用半年的分别,重新唤醒他的爱情。在信中,她却不提一句渴望与期盼
	,责怪或嫉妒。她只是絮絮诉说着,独在异乡的寂寞,又反复叮嘱,他生活的点点细微。


		“现在我庄严的告诉你一件事情,在你看到之后一定要在回信上写明!就是第一件你要买个软枕头
	,看过我的信就去买!硬枕头使神经很坏。……”“船上买一点水果带着,但不要吃鸡子,那东西不消化。
	饼干是可以带的。”她这样写,这些琐碎的小事,为了他的健康,虽然她自己倒是常常生病:头疼,肚
	痛,发热,接连不断。她从不在意这些,只在信的末尾标注,“肚子好了”,仿佛有欢快的神情。萧军寄
	来照片,是黑了,于是她开心起来,“你健壮我是第一高兴的。”


		泛爱的萧军,他的心,他的爱,是真切,不掺杂丝毫虚假。只是,他没有为谁停留。他爱着女人,
	而不是哪一个女人。

		在第四年,两人的生活开始磨擦不断。那段时间,萧红经常是呆在鲁迅的家里。被神经衰弱折磨着
	的她,写下未曾发表的长诗《苦杯》。这美的,光明的爱,瞬间里化作冰凉的痛楚,令她倍受煎熬。确
	如一杯苦的酒酿,让她独自默声饮下。她决定离开,到岛国之上寻求解脱和清静,并用距离来挽回一切
	的丧失。


		她不断写着信,又一日日等待着回信。然而,他的信是那么少。

		她问,“你近来怎么样呢?信很少,海水还那样蓝么?透明么?……”她不等待回答,她只是一封封写
	她的信,无论晴天,还是飘雨。

		透过文字,我好像可以看到她,一个人住在楼上,躺着,或者歪斜在桌上,喝一杯牛奶,吃半个西
	瓜,又吞下止疼的药片。

		门外,是异乡陌生的言语,和木屐的脚步声。门内,是一个女子孤独的写作,挣扎,释放,和想念。

		她读着萧军寄来的唐诗,消磨一个个白日和夜晚,她听着这乡间的寂静,独自睡去,不知梦见些什
	么。偶尔,她吸一支烟,她的精神细如烟丝,流散在房间。

		“……我孤独得和一张草叶似的了。”

		即使是孤独,她依旧有点赌气地说“你说我滚回去,你想我了吗?我可不想你呢,我要在日本住十
	年。”

		我想,她会快乐,如果他确实在想念。冬天过去,她便回到了上海,而一切都没有改变,所有的丧
	失依旧无可挽回。半年的分别,原来并没有预想的重量。

		他轻轻吹口气,那爱就散了,比风季的改变更轻易。

		如他所言“爱便爱,不爱便丢开。”于是,谁也没有继续停留的意义。

		恍如一梦的爱,曾光照着这残忍与昏暗世界的爱,不落一丝纠缠不舍,就此完结。后来,她为他产
	下婴儿,却是刚降生便死去了,仿佛那些,无力却分明的回忆,永远地,死去了,不留任何余地。


		他们的信在海的两岸穿梭往复,他们的爱情却无法逾越茫茫汪洋。

		歌声里,Faye唱着,就像蝴蝶飞不过沧海,没有人忍心责怪。对于所有夭折的美好,我们总是难以
	正视。

		在经过的地方,尚留玫瑰余香,深爱的人却就此分道扬镳。

		所有的亲密,也终于化作陌生的怀念。

		这令人悲戚,在许多近乎神话的故事上,蒙了灰尘,起了疑心。只是,两个人的跋涉,你怎么可以
	强求到达。

		萧红,不会遗憾,不会后悔,他们的相遇和相爱。她漂泊沉浮的短短一生,有萧军的相知相依,是
	幸福的经过,没有借口,来苛责一个完满如童话的结局。虽然,那是我所期望的。


		她在临终时说,“我一生最大的痛苦和不幸都是因为我是个女人。”这话,令人听之下泪。

		又不禁想到许多,富于才华,而命运坎坷的女子。

		这世上,不幸的人很多,男人也会不幸,但女子的不幸,总会更引人悲伤。在被男权控制的世界,
	在男人仍然控制着这社会绝对权力的时代,女人的不幸,有多少是因由男人造成的呢,也许,这是永无
	答案的设问。萧红的笔下,不乏种种例证,她控诉着,她揭露着,她有时像鲁迅,那么决绝得不留情面
	。

		她在觉醒里,看清人的许多恶毒与麻木,也在最内心保留着对于善和美的最高追寻与崇拜。写下令
	人窒息的《生死场》的她,也写下《呼兰河传》。她在夜的世界里,洞悉着,逆也顺受,顺也顺受的国
	民性的悲哀,又在暗无天日的生活真相下,回忆天真的小女孩,用她纤尘未染的双眼,读这埋伏了无限
	残酷的人间。

		萧红,多么坚强,她的心好像充满了悲愤的城,用女人的善良,女人的隐忍,坚持着书写,那些不
	停止,不消亡的安静的喊叫。


		有人把她与张爱玲并称,“南张北萧”。张,一样是充满了不幸的女子,一样是才华横溢的女子。我
	也爱她的文字,爱她的所有,如一个苍凉的手势般的传奇。

		而张,并不如萧红懂得生命,她的世界凄迷冷艳,芳香迷人,却少有富于疼痛的同情,她淹没在自
	己的悲情里。张爱玲,在文字中将爱情把玩在手中,不费吹灰之力的人,却在自己的感情里,输得一败
	涂地。萧红,她懂得了女人命定的局限与不幸,却无法不依赖着一个个一手造成她不幸的男人。

		所有的幸福,原来都不曾轻易被我们紧握。你以为你明白了全部的起因与结局,却终于无法逃脱,
	命运的玩弄。

		谁让你遇见了,那一个人。谁让你的双眼,甘愿在白日里失了明。

		但一切的经受,从不会是无意义的劳碌。爱情,在细节里记忆着彼此的青春和热情。毕竟,遇见,
	已经是多么幸运的事,哪怕那将是致命的危险,一场毁灭性的灾难。


		无怨无悔,或许是所有人准备相爱的人应有的素质。无需责怪,无需怨恨,如果结局让我们落下了
	眼泪。只要,珍惜的时刻,有真实的心疼,有不弃的勇气,有相知相依,即使不过温暖的幻觉。

		让我们背负着,因爱而起的苍凉与恐惧,坦然向前。

		汪洋,在眼前,是无限的未知。而我们,都是小小的,翅膀单薄的蝴蝶。


		谁曾告诉我说,这一切,不为彼岸,只为海。

	\endwriting


	\writing{航}{2006年05月06日 ~ 22:02:49} %<<<2

		这一端,还是初晨的光线洁白,那一岸,已是如雾的暮霭沉沉。


		这生活的河水,轻描淡写之间,度去许多,无目散落的年月。水光荡漾,我们的船舱狭小。而不息
	的水流,总是轻易带走真实的感触,却把人渺小的躯体,一寸寸投入时空的未知,如夜晚的星辰,落入
	宇宙的苍茫无解。


		我时常做着这样的梦,一条河,一叶小舟,不知去向的流失而去。

		那是坐在船头的我,静听着鸣如佩环的流澌,在暗黑的隧道中的独自穿越。

		我爱这样梦着的夜晚,我以为它更接近着生命的真相。


		凉的河水,凉的空气,凉的月光,温热的手指,插入瞬息间不见踪迹的波浪,风声飞去。

		这所有,形象地形容着,我们在人间的行走。


		全部的痕迹,好像在七月里一路踏过的沙滩,留下精致又寂寞的脚印,等候着涨潮的海水,洗去一
	切。


		船与水,自古被人们在意识里同生命紧密相连。岸,更是美好到达的隐喻。

		我们渴求着一只船,无论大小,来借以度过,借以完成,那些无可避免,无可奈何的苦难。人的无
	力,在不绝的江水与海河面前,显得脆弱不堪,正如面临命运的沉重。每一步,是各自艰难的跋涉,与
	注定的经受。

		当洪水漫过这世界,吞噬了生活的平静,生命成为虚无的梦幻,只要一朵浪花,就堕入无知觉的黑
	暗。我们于是望一只船,在洪荒世界的天地苍茫,在横无际涯的水天之间。


		我不曾经受洪水,不曾亲见过任何灾难。却明白,灾难的致命,也隐约懂得一个瞬间里的存在和毁
	灭。

		如一场海啸,淹没无数鲜活的生命。

		这一刻,那孩子在这椰树下奔跑,下一刻,他已不见了踪迹,埋葬于沙土之下,恒久地沉默,再不
	能露出洁白的牙,在艳阳里微笑。也许,当许多的时光过去,会有一双无限遥远的双手,在这里发现了
	他,一具人类化石,正如,今天的人们,挖掘出原古动物的化石一般。生命,在灾难里寂灭,并不发出
	痛苦的哀号,也不曾落一滴眼泪。

		我想,灾难的痛苦并不属于罹难者,而归于所有目睹它发生的人们。


		死亡并不恐惧,恐惧的是亲见它发生的活人。他们会发觉,有些真实的存在,变得十分可疑。呼吸
	仿佛是假的,人生是假的,那么奢求和悲伤是可笑的,爱恨情仇是多余的。

		而船,它流过我们的命运,又承载着体内的波涛,一日日度去,所有终于无可逃脱的困扰和情绪,
	却是此一刻上,逼真的感受。

		我于是才有了行舟河上的梦,于是爱这样的夜晚,甚至以为,白日是颠倒的梦境,而夜晚的流水声
	,才是真实。

		我发觉我的生命微小:不及一抹流云善变,不比一颗石子恒久。

		我看到,晨昏在日期上切割,绿树在墙壁婆娑,爱情摇曳着身姿,消耗着年岁的热情。


		生活在眼前,而生命在毫无预兆的未知无解。


		远古时,在江畔生活的部族,曾把亲人的遗骨装入船形的棺木,埋入土地。他们相信,亲人的灵魂
	可以乘着这船,穿越黑夜,随晨早的太阳一起,重生于东方的地平线。他们相信,太阳的隐没与升起,
	是与灵魂的死亡与重生一样。生死之间,以一只小船维系。这原始而虔诚的相信,令人动容。


		在另一个早上,深爱的人,可以乘我们亲手制作的小船,同阳光一并再次获得光明。

		我愿意去相信这一切,就好像愿意相信,我们每个人都是在自己的水面,航行着一个个圆圈。


		这情景,如里尔克的描叙:

		\longpoem{}{}{}
		我生活在不断扩大的圆形轨道 \\
		它们在万物之上延伸 \\
		最后一圈我或许完成不了 \\
		我却努力把他完成……
		\endlongpoem


		每个人的船,因为对于生命的敬畏之心而充满了神圣的意味。

		在这河水的对岸,希望的诱惑,把我们凡俗的身体填满。而度去的渴求,在沉浮的风雨飘零里,使
	浮躁恐慌的内心获得安详宁静。

		这是困难的到达。


		而有一种穿越,总要靠我们的意念,在孤独中完成。

	\endwriting


	\writing{楼}{2006年05月12日 ~ 11:45:59} %<<<2

		\longpoem{}{}{}
		生命为什么不挂着铃子?\\
		不然丢了你,\\
		怎么感到有所亡失?

		萧红《沙粒。十七》
		\endlongpoem


		当春风穿堂而去,眼前的小楼,旧去的墙面显出时光的疲惫。散碎的落寞,在歪斜的门板,残损的
	壁炉,一丝丝渗出,溢满房间四处。人去楼空,这废弃书房。当我们站在吱吱哑哑的木板上时,萧军早
	已不在,只留婆娑树影依旧透露着往日,这窗上风景的旖旎。


		老照片上,是笑意盈盈,相依的恋人。他们的故事,好像这生命,不曾挂一只铃子,不曾待年年的
	春风吹起,来告知获得和丧失。丢弃的热情,冷如灰烬,任时光的书页翻卷,请文字徒劳地记忆着并行
	的路途。目送着移情的萧军,萧红只有饮独自的苦杯,怅怅地写下:说什么爱情!说什么受难者共同走
	尽患难的路程!都成了昨夜的梦,昨夜的明灯。


		再没有一盏灯,亮起在萧军的书楼,没有一首诗,用浓情的唇舌,写给新的情人。已成为危楼的旧
	居,好像飘摇于时光的一叶。

		墙壁上,贴着90年代的挂历,定格在12月,空白处用铅笔草草写着,预计搬家的日期。顶棚被油烟
	薰染得焦黄。而房间的角落,堆放着原住户丢弃的物品,落满尘埃,单只的皮鞋,孤零零的红沙发。显
	然,解放后的很长时间里,这里有许多个家庭混杂居住。在楼下晾晒被子的女人告诉我们,这里在7年
	前便不再有人居住。

		% <todo: 别字: 依在楼上 -> 倚在楼上 >
		这曾经必定风光一时的西式小楼,而今在前海的岸边,孤独而决绝地被废弃在原地,眼见一街巷的
	灯火迷离,红男绿女。它也许目睹了,曾经的淑女名媛,绅士文人,进进出出,一栋栋精心安置的楼宇
	,在某一年的春风里,倚在楼上,挥一条喷香的绢帕,握一支海外的香烟。它又亲见了,所有人的消失
	,那些宿命的浮沉,在时代的变革里,每个人都逃不过,命运的玩弄。萧军,在这楼上远眺水波的温柔
	,他不知道,这小楼的明日,想不到几十年后的自己,只可以躲在储物间的一角,小心地安身立命。我
	们站在文革期间,萧军的“蜗蜗居”,阳光从墙上的小天窗漏下,从这洞似的窗口,可以望见邻居的屋顶
	,和他们房上的瑟缩歪斜的黄草。

		谁也不会停留,没有人可以完全占有。我们的拥有,只是匆忙,不及眨眼的瞬间。楼上,楼下,枉
	然千古,多少人怅惘过,迷失了,在不可知的人间。那些不忍回味的思念,遗忘,和变迁。

		燕子楼空,佳人何在,空锁楼中燕。东坡的梦里,有当年风姿卓越,一心痴情的关盼盼。茫茫黑夜
	,却无处可寻那时的风情万般。夫死守楼十年不嫁的盼盼,被古代文人传为美谈。这并无关封建道德,
	只在乎一个女子的痴心一片,是士为知己者死一样的勇毅豪迈。

		古今如梦,何曾梦觉?但有旧欢新怨。全部执意过的感情,在空掉的小楼,随流年偷换了窗口的春
	花秋月。再回首,不过空茫茫的一处无可填补,不过一场人物皆非的浩叹。所有的楼,都藏着美而悲伤
	的往事,所有的楼,都等待着风光的消损,如佳人的玉颜,永无挽回地,一寸寸烧成回忆里,捉摸不定
	的光影叠错。

		当月光如水,水如天的夜晚,这楼上的尘埃会不会积成满天星辰,会不会叹息一声,来吊唁已渐成
	烟的岁月。爱人的脸孔,空掉的楼宇,好像我们经过的那些风景,萧瑟着,投入历史的洪流,殒身碎骨
	,万劫不复。谁还会记得,你们的山盟海誓,谁还会念起,檐下零落的许多风雨。所有爱恨,终将被时
	间掩埋,像所有黄土之下的沉睡,归于恒久的混沌与安宁。于是,全部的新怨与旧欢,都不过云霞似的
	虚妄。轻轻吹过,生命从不出声,不像挂在风口的铃子,清脆地预告着季节的离别。我们各自的亡失,
	在时空的无限中,显得过于微小,而不可以挂上铃子,发出声响。

		老照片上,恋人的脸孔早已破碎不堪,人去楼空后,爱恨情仇,显得徒劳无功。一生寂寞零乱的萧
	红写下绝笔:“生平尽遭白眼冷遇…… 身先死,不甘,不甘。”而不甘的心,在冷冷的坟墓里,也终于会
	获得宽慰和原谅。当暮年的萧军,在桌前整理着当年的书信,并一一作注又编排成册时,是不是也深深
	叹惋起旧的时光,那曾痴心依恋的爱人。春风年复一年,穿堂而过,这一天,我们站在这里,从碎的玻
	璃窗望去,你们昨日的纠缠怅恨。一切,都宛若近在眼前,一切,又都迷失在当时当地,郁结着遥远的
	情绪。淡淡的,不可触摸的痛,却发不出声响。

		小楼的前世今生,同我们的经过一般,是一场安静的宿命。

	\endwriting


	\writing{茉莉}{2006年05月16日 ~ 23:28:31} %<<<2

		那是一次无可奈何的回眸,如窗台上茉莉,苍白无血色的脸孔,依旧含着迷惘的微笑,对这尘世流
	年变换。

		茉莉,平凡里透露着骄傲,淡定中隐藏了哀愁。小的花朵,白的花朵,没有张扬,安静到令人心疼
	。只有四溢的幽香,搅乱了静止的时光,吹过夏夜的凉风,深入记忆的底层,为一切的昨日,沏泡一杯
	清绝的花茶。我曾坐在这样,开放着茉莉的夜晚,映一盏昏黄的灯,读书写字。在世界沉默的时刻,书
	页与指尖的摩擦成为人间的全部声响。我是在这样的夜晚,嗅见茉莉的芬芳,凉的,不动声色的芬芳。

		窗上,是远处高楼的灯火忽明忽暗。枕上是与文字耳鬓厮磨般的纠缠。流散的香气,令人生出哀愁
	,想着,这夏夜的深邃,星辰的升起和坠落。我像园中的植物一般,兀自在房间里生长和老去,等候着
	雷与闪电,光顾生命的空无寂寞。

		这是我度过少女时光的方式。伴随着那些花朵,美而苍白的脸孔,书写阅读,痴心妄想。不知道有
	谁,如我一样,善于矫情在自造的情绪纠结。每个女孩,都有她的方式,来度过那最美的几年。我要拍
	照,要写下日记,作为日后的证据,好在某个老眼昏花的日子,对什么人炫耀,你看,我也曾年轻过。
	而现在,我在无休止,不言其烦地面对着镜子。我们的面目,将被雨水洗净,如那些落在脚边,绝情地
	萎黄而去。我的爱人,你要记得,我最美的时光。

		小鹿穿了白色的连衣裙,在我的视野里,踮起脚尖,转轻巧的圆圈。5月的阳光和暖而明亮。多么
	美好的下午,绿树用翡翠色装点了我们空的窗口。我们享用生命的恩赐,用那么贪婪的心。时间变得洁
	白,如小鹿的裙子,如云朵,如茉莉。却又因为美好,而充满了眷恋的忧伤。

		最近在上映的电影,被封锁了三年的《茉莉花开》。

		讲述三代女人的故事,在上海,那逝去的风月无边。自然想到张爱玲,王安忆,想到海上繁花般,
	梦里凋零着的精巧。

		三个女人,茉,莉,花。各自不同的命运,在三个截然不同的时代。却遭遇了同样的命运,被男人
	抛弃的命运,只是用各自不同的方式 —— 逃离,自杀,或移情。男人没有错,只是女人玩弄了自己的青
	春,把它们挥霍一空,又徒劳挣扎。感情,像这命运的外壳,最薄弱,也最冰冷,仿佛给你无限希望,
	再让你由谎言与虚妄臆造的高空坠下,殒身碎骨,万劫不复。

		影片的介绍,这是一首女性的悲歌。而我,却更愿意认为,这是男人与女人共同的悲剧,不可以预
	防,只能够治疗。

		电影的原著,苏童笔下的《妇女生活》。有人说,电影把小说改得体无完肤,苏童看了会哭。读过
	小说,读拥有另外姓名的三个女人,娴,芝,箫。一样的悲剧,只是更接近着世俗生活的真相,和那些
	我们眼见的生活。没有茉莉,没有那些翠色的旗袍,衣衫,和场景,小说里的女人,和苏童作品里其他
	的女人一样,永远在暗地里进行着一场争斗。即使是亲密的母女。我想到《长恨歌》里的王绮瑶和她的
	女儿。女儿对美丽的母亲充满了不可消散的敌意。这缘由,归根结底是男人。好像苏童所说,女人共同
	的敌人是男人,但女人却是为男人而死,这不是一件公平的事。

		电影把已经被小说揭露的真相,善良地怀着慈爱遮起,用唯美的光影和画面,粉饰一新。我想,这
	更合乎大众的趣味,更符合普遍的接受能力。生活的残忍,总要涂了胭脂,擦了粉,才敢晾晒在电影里
	,来博得适度的同情和感伤。电影用来欣赏,而小说用来思考。

		茉莉,仿佛已成为一种象征。当她们唱起那首《茉莉花》,你怎么不感到岁月那向深渊坠落的无尽
	悲凉。我们在书籍与电影里观望她们的命运,落一两滴,真实的或虚伪的眼泪。生活的现实里,我们又
	各自触摸着凹凸,踏过茉莉玉体横陈的时光,一路向注定或未知的明日奔赴。有孤单,有倔强,有感伤
	,有矫情,却从不敢绝望。那两个字,一碰,就是灾难,是无可挽回的倾塌。


		用我们的方式,度过少女的时光。再用我们的方式,追忆一切美好的,遗憾的过去。全部,是值得
	付出的代价,无论过度的欢乐,无论肆意的悲伤。青春的脸孔,从来是经不住端详。我愿意坐在,栽了
	茉莉的窗里,我愿意有夏夜里的幽香,即使,那会勾起许多,徒劳的情绪。我以为,生命的全部过程,
	正是不断的瞭望与追溯。至于今日,是我们踮起脚尖,是我们在镜子前,消磨的,荒废的,却不肯轻易
	放弃的时间。

		我只需要记忆,无论那苍白的脸孔是否已经呈现出病态或衰老。我要握住,这些命运的仓促,即使
	,一切的拥有或许只是安静的沉重,那也将芬芳而明亮。

	\endwriting


	\writing{鞋子}{2006年05月24日 ~ 23:47:21} %<<<2

		整理房间,鞋子也从柜中取出,被一双双排列起来。红的,洗旧了颜色的帆布鞋;白的,细带子的
	凉拖鞋;深灰的,系着紫色鞋带的运动鞋…… 安静俯在地板上,全然是慵懒又寂寞的模样。属于中学时
	代的鞋子,在被锁入柜子的秋天后,便不见天日,直到我再次想起,把它们从黑暗里救出,重新暴露于
	日光之下。


		鞋子的后跟处,一例地在外侧磨损严重,那是我穿着过的痕迹,因我独特的行走方式。我一只只察
	看,有多少的路途,我就与他们相依相伴着走过呢。在那些已辨认不清真假的记忆里,在不再留恋,也
	不再徘徊的岁月中,好像所有的途经,都已化虚妄,唯有鞋底的磨损,和那缝隙间残存的沙土,犹然依
	旧,真实如此。

		是在哪里,我踩过了那如今细碎落下的沙土?和什么人,说着什么话,心里又装着什么秘密。竟就
	这样,再追忆不得当时的细节或大概。多少的沙土,在我们的脚下,隐藏了时间的阴谋,等候着,一个
	未来,给我们怅然的惊喜。


		很小的时候,我好像拥有过一双红色的小皮鞋。我会穿上它,一路跑着,在老宅的院子里。看祖母
	种她的月季,看月季一天天长高,盈满小小的花坛。阳光仿佛都成了淡粉红的 —— 那月季的颜色,又散
	发出甜味的清香来,随了风,浸润到我黄而稀疏的头发。祖母坐在门前的板凳上,为我梳起小辫,扎上
	红色的丝带,她又捣了指甲草的汁液,涂在我圆圆的手指。我把双手高高举着,向着太阳,为了把指甲
	上的汁液晾干。阳光投射了手指的影子,映在我扬起的小脸。我透过,被照得透明的指缝,眯着眼睛,
	看见通红的快乐,属于小女孩的,无需理由的满足。当指甲终于风干,我脱下那红色的小皮鞋,躺在祖
	母身边,依偎着那老去的,瘦弱的身子,听她讲乏味的老故事,又昏昏睡去。祖母的,黑色的绒布鞋,
	同我的小红皮鞋也并排躺在床下,睡着了。

		在后来另外的一个夏天,祖母的鞋脱下来,便再没有穿上。我不知道祖母走的时候,穿着的是哪一
	双鞋。

		小学的自己,坐在院门前,有点吃力地握着偌大的刷子,一丝不苟地刷白球鞋。空气里弥漫了肥皂
	水的气味。我努力回想劳动课上老师讲授的方法,先刷鞋里,再刷鞋面,等等。因为劳动课与语文课是
	同一位老师,在实践过刷鞋后,我还要完成一篇关于刷鞋的作文。于是,又努力记住自己刷鞋的动作,
	并构思着,怎样在作文里加入遭遇困难而终于克服的情节。那是范文里惯用的套路。鞋子终于刷好,裙
	子却被肥皂水弄脏,但并不在意。劳动的快乐,溢于言表。

		踮起脚尖,把鞋晾晒在向阳的小棚子上,又一张张在鞋面贴好卫生纸(这样鞋子不会被晒黄,也是
	劳动课的成果),然后,等待阳光,然后,发起呆,想象自己穿上白球鞋走进班门的模样。我的鞋子一
	定是刷得最干净,最白的。与此同时,裙子上的肥皂渍正在慢慢风干,变成一块块黄色。


		忘记了我是怎样完成了那篇作文。只记得,当天晚上的一场大雨,淋湿了我晾在棚上的鞋子。我的
	许多幻想是破灭了。鞋面皱巴巴的,透出裙子上一样的黄色。

		眼前,那双洗旧了颜色的红色帆布鞋。我在夏天里穿着它,站在8月的绿树下,看头顶的蓝天,载
	满棉花一样的云朵,那么轻,那么大的,飘忽在我的眼睛。

		17岁的我,用整个夏天,观察云层的变换,用文字记录,我爱的颜色,我爱的声音。然后,下雨了
	,雨水从窗口溅进来,淋湿我的日记,蓝墨水,像傍晚的雾气一样,氤氲一片。我欣赏那些消失在雨水
	中的文字,仿佛记忆里,注定归于消亡与模糊的部分,因为无可辨认,而显得神秘可爱。绿树婆娑,任
	风隐藏青春的苦恼,任窗口饱含了莫名的忧伤,在房间关紧女孩子的自怨自艾。

		我的鞋子,红色的帆布鞋,陪伴我,来荒废那么多日光充沛的日子,来天真任性地生活过,又同自
	己失散。这所有,是年龄对我们开的玩笑,寂寞而美好。

		鞋子,你行走的时候,它寸步不离,你停留的时候,它也执意守候。最近看到一则广告,讲述女人
	与鞋的故事,小女孩穿着纱裙,偷偷穿上妈妈的高跟鞋。哪个女孩没有做过同样的事情呢。鞋是那样大
	,却充满了致命的诱惑。高跟鞋,是小女孩对于未来的一场排演,是对于作为女孩,区别于男孩的一次
	可爱的觉醒。

		我记得自己,拖沓着大大的高跟鞋,从镜子前不厌其烦地来回走过,甚至记得自己插着腰站在那里
	,摆出的姿势,和微笑。那个女孩,有一天终于会穿上自己的高跟鞋,咔嗒咔嗒地走过世界。好像广告
	里演绎的一样,女孩,女人,一双鞋,就窥探出全部的秘密。

		我喜欢拍下自己的鞋子。从夏末到深秋,从隆冬再到仲春。一双双鞋子,就这样随着四季风物,落
	入光影的凝定,也落入我的记忆,带着种种可能会变得虚假的知觉,划定了今日明日的界限。而一些沙
	土,在全不知情的状况里,把我拖入时间的河水。我像个赤脚的孩子,涉水而过。鞋子,排列在我的生
	命,清晰地踩出了甜美的印痕。

	\endwriting


	\writing{回复。给晓琳}{2006年05月28日 ~ 11:17:27} %<<<2

		高中时代,一起看云的女孩。毕业后,只是断断续续地通信。

		我们习惯在文字中相遇。

		这一天,我乘了独木舟,回溯而去,在河水的上游,有你等着我。

		04的夏天之后,一切关于你的片段都归于回忆。属于时过境迁后的云淡风轻。

		温暖的笑,写满铅笔字的牛皮纸本子,还有你的大水杯。

		我感谢,在那些寂寞的日子,有你的存在,读我矫情的文字,听我说许多别人只会诧异的语言。你
	能够懂得,你看得见,我在心底安排的那个世界,明亮的,却蒙上雾与尘埃。


		很长的一段时间,我们都是坐在靠窗的位置。

		我记得我座位上可以看见得风景。挂在教学楼旁的一角灰蓝的天空,远处几棵婆娑飘摇的杨树。我
	在课上发呆,看那些天空下的树,在夏天绿得华丽而忧伤。在本子上写下不知所以的文字,读顾城的诗
	。

		记得不久前你问我最近在读什么,我回答,顾城散文集。你就大呼,你读他要读多久呀…… 我也觉
	得可笑。可是,我依旧在反复读他的东西,不厌其烦。

		那一个孩童似天真的诗人,带我逃离这世界的真实,去造一座城堡,去做永远的国王。就在今天,
	我竟然还把网页上的“倾城”看成了“顾城”,并兴奋地点开…… 我的手机,恰好是“倾城”系列,白色。

		突然也喜欢上“城”这个字,那是一种包围的姿态吧,自我的封闭围困,或者满足。在那里,我们获
	得安宁,获得释放,也遭受孤独。


		而孤独,从来是不可避免的。感情不能够改变我们孤独的命运。因为,有太多美好的事,是不可以
	和别人一起做的,比如做梦。这话,似乎也是顾城说的。他的孤独,杀害了他自己和他的爱人。


		你眼见我的高中,正如,我眼见了你的一样。我们好像站在对岸,观望彼此的生活,有时枯寂荒芜
	,有时火树银花。

		穿越我们的时光,遗忘一些脸孔,告别许多的昨日,重重叠叠。

		在夏天的几场滂沱之后,我们被洗净,又晾晒在后来的日光里,像一床松软的棉被那样。

		我不愿去怀念,因为怀念的徒劳无功。我却无法抑制地陷在回忆的泥沼。

		好像,那里有我们生命的养分,必须定时汲取。

		人大约是不可以拒绝,回忆,不可以了无牵挂地离开。

		全部过程,是我们温习的课业。


		去年的六月高考的那几天,我总是梦见自己再次坐在高考的考场。做数学卷子。于是慌乱一团,怎
	么办,已经一年没碰过数学了,考不上了,考不上了……然后惊醒在黑夜。

		数学像一场噩梦似的,将我反复纠缠,即使在考上中文系,终于摆脱了它之后。就想起,那时上马
	片儿的课,想起,他解题间歇里的低头沉思,我发现那样子竟然酷似吴镇宇。当然,没有吴的帅气,只
	保留傻气,而已。

		中午的时候,大家挤在教室的讲台,找他讲题,然后领取新的作业。我总是无法及时领到新的卷子
	,昨天的题目还没有解完。这让我长期处于痛苦的负罪感中。

		现在,我终于不必学数学了。

		却在某天突然记起正弦函数的曲线,很优美。

		去年春天回过一次学校,装修得很漂亮。

		我却还是喜欢初中时候教学楼的模样,淡绿的墙,朴素的桌椅,上边留着高年级同学的刻痕和字迹
	。那让我觉得,学校充满了昨天,充满了故事。

		而今,它看上去洁净而肃穆了,把我拒之千里。站在楼道里,连气味都变了。在原来教室的门口向
	里面偷看,陌生的老师,陌生的学生。

		才知道,这里早已不属于我们。

		看到了久违的王胖胖,还是教他的历史,还有正在课堂上滔滔不绝的马片儿,还是穿着他的黑片儿
	鞋,多少年如一日地。

		他们讲授相同的课程,面对一届届学生。他们淹没在学生的青春里,悄悄老化。


		据说,学校的校风是越来越差了。也看到,穿着我们校服的女孩子,夹了睫毛,涂了唇彩,坐在五
	道口市场里做指甲。我想,她们只是更爱美了,也更懂得挥霍青春的热量。


		那天,和老李坐在办公室谈话。她的样子很温柔了,不像从前对我那样。

		想起来,从前上课,她会严厉地喊你:Linda! 记得,你总是慢慢回过头来,一脸无辜地说:我怎
	么了。这令她苦笑不得:弄得好像是我错了一样。

		不知道她现在的生活怎样。也听说,她经历了一些变故。

		我们离开了那里,她又还会记得谁,又遗忘谁。这都是无关痛痒的事情吧,终于,我们都将消失,
	在各自微小的生命。只为对方祝福,平安,幸福。


		毕业后,晓琳去了南方。我留在北京。

		她为我寄来照片,安然于手指间的长有四片叶的三叶草,幸福的象征。

		后来,很久没有再写信。在文字中的相遇,却依然继续,以不同的方式,用相同的默契。


		我们一起看云,在秋天。你问,怎么样才能够飞起来。今天的你说,再不是贪恋飞翔的孩子,就这
	样悄然无声的生长吧。


		你把田的房子叫作,安静的张扬。田在这里栽满文字的花朵,等候我们的相遇。

	\endwriting


	\writing{两个女孩}{2006年05月30日 ~ 00:33:29} %<<<2

		三个女孩约定,听过《两个女孩》后,各自写下故事。

		这是在莫文蔚的歌声背后,田的臆想故事。

		\blankrev
		莹的爱,依偎着你的身体,无私地将你包围。日光昏黄,照在她显出稚嫩的脸上,长长的睫毛安静
	垂下,像是睡着。你喜欢她这样,像个孩子,幼小的神态中透露出满足,在嘴角挂着单纯的笑意一
	两抹。她会抚着你蓬乱的发,突然问起:

		喻,你会想念一个人么。

		……想念?

		对,想念。一个人,让你轻轻地心疼,无法停止。

		……

		你沉默,你从来不去说什么。你只会捏她的鼻子,笑她的痴。然后在心里想着,


		如果莹不在身边,你便有了想念。而现在……


		你从来不告诉莹,你的另一处想念,系在哪里的河岸,像一叶小舟似的,随着风飘飘摇摇。你好像
	等候着暴风雨的可能,扯断一些什么,归还你自由。而你明白,你所渴望的海洋,只会将你吞没,落得
	万劫不复,尸骨无存。于是,你依偎着可以把握的温暖和平,在洁白的日光里,看河上涓涓的流水,不
	急不慢,就像惯常的日子,平静到枯燥。


		6月,总是翻云覆雨的天气。只这一会,就涌起几朵浑圆巨大的云来,遮住了太阳,把阴影打在你
	的白衬衫上,灰灰的,陷在背后的白墙里。

		莹闭上眼睛,黑头发齐整覆在前额,娃娃似的。她的红鞋子东一只,西一只地丢在地板上。她是生
	气了么。你猜不出她的心思,也不愿去猜。莹,永远是你初识时的模样,天真,聪慧,却在心底掩藏无
	数的秘密,那秘密,有一些甜蜜,有一些,又饱含了忧伤。


		你记得,她家搬来的那个午后,一个飘着柳絮的春。


		莹抱着她的小毛熊,站在一楼的台阶上。明亮的眼睛,紧张地看着搬家工人把家具从车上卸下,再
	一件件搬进陌生的房间。她的恐惧,被那洞张的双眼,泄露无疑。那天,她也穿着红鞋子。她只有7岁
	。而你,也不过10岁,4年级的小学生。

		母亲让你陪着莹玩。你蹑手蹑脚走过去,她却一扭头,哇地哭着跑开了,留你在原地不知所措。你
	家祖上留下两套居室,一套在一层,一套在二层。父母把闲置的一层,租给了莹一家。从此,你们做了
	邻居,你也似乎多了个小妹妹。有点爱哭,却可爱的小妹妹。母亲总教育说,要让着莹,你隐约知道她
	的家庭遭遇了一些什么不幸,却并不知道具体是些什么。


		你们一起上学,别人都以为你是她的亲哥哥。

		莹不再害怕你,还喜欢跟在你身后,和你一起背着大书包,蹦跳着跑去学校。有时,她要你背她去
	,你也同意了。你似乎是真把她当亲妹妹那样,甚至,比亲妹妹还要好。为这,你被同班的男生笑话,
	他们见到莹摘了小黄花别在你的衣袋,就一起起哄骂你是小姑娘。你和他们打了一架,脸也抓破了。莹
	看得着急,却帮不上忙,站在旁边大哭了起来。你被许多男孩压在地上,她情急之下扑上去,拿起手里
	的跳绳,举手便漫无目的地挥动起来。那些男孩的背,被莹的这几下,抽得通红,甚至有的渗出血来。
	在一阵混乱后,他们呼地散了。一群人骂着跑进巷子不见。晚上,好几家的家长找上门来。你说是你打
	的,不让莹出声。她窃窃躲在你身后,默默流眼泪。


		你哪来那么大气力呀。平时连吵架也不会。

		第二天,你笑着问莹,她咬着嘴唇,半天不答。用那溜溜的,隐着泪似的眼睛看你,半天才挤出几
	个字。

		阿姨打你了没有……


		时光,像一朵纯白的花,开放在那些连你也渐渐辨认不清的回忆。

		你知道,莹心疼你,你也心疼她,总想挡在她前边,为她排除一切危险的可能。你想,这大约是作
	为一个男性的原始本能,但是,只有对莹,你才会奋不顾身,乐意背着她,被她在胳膊上画手表,甚至
	陪她加入女孩子的队伍,玩踢毽,玩跳皮筋儿。因为长得又秀气,你的小姑娘的绰号就叫开了。只是,
	没有人敢再当面叫了,只是窃窃私语的,用那种小孩子的顽皮。那时,你已经是学校里的大队长,白衬
	衫上挂着醒目鲜艳的三道杠,是多少孩子羡慕的对象啊。大家都眼巴巴看着你,穿刷得雪白的球鞋,走
	上领操台,在升旗仪式后宣布少先队在六一的活动安排。

		女孩子们喜欢你,也喜欢你加入她们的游戏。所以,高年级的女生也去找莹来一起跳皮筋儿,她们
	知道,莹能把喻也叫来。叽叽喳喳的,女孩子谣传关于你的一些琐碎。她们羡慕莹,却不嫉妒,她们觉
	得,那是你的亲妹妹。


		是从什么时候开始,莹放学后总是呆在你家。你大约也记不清了。你们一起放学,一起在你窗外长
	着大树的房间写作业,喝母亲冲的高乐高。莹喜欢吃那些甜的东西,却总是慢慢喝下去。她说,她喜欢
	巧克力,但是她妈妈从不让她吃。你就把家里的巧克力偷偷带给她吃,每天三颗,你们说好了,因为莹
	说吃太多怕牙痛。巧克力的糖纸,被莹细心收起来,放在铁皮盒子里,又藏在床下。有阳光的时候,她
	就坐在窗前,一张张地看,她说,你看,有了光,它们就变得那么美。莹喜欢晴天。喜欢你的房间,为
	了那棵大树。她总是拉了你,痴痴看那树,茂密繁荣的枝叶,把你的窗装饰成一场翡翠一样的梦。你乐
	意陪着她,听时钟嘀嗒走过,等着傍晚落下,把对街的一小块天空染成粉红和淡紫。

		她总是不愿意回家,她说她不喜欢那个男人,也不喜欢妈妈。

		你问她,是说你爸爸吗。我爸爸早死了。莹斩钉截铁地回答。那神情冷漠得让你觉得陌生。


		你问母亲,莹家的事情。从前母亲是不肯说的。后来,在16岁那年,她终于告诉你。莹的亲生爸爸
	在她2岁的时候便出车祸死了。没多久,莹的妈妈就与现在这个男人结了婚。很多人都传言,是他们合
	伙害死了莹的亲爸爸。于是她的祖父母不许他们住在原来的房子里,连这个孙女也不认。他们一家人才
	租下了这套房子。

		你听得心惊肉跳,又回忆起莹那冷漠的表情,一时呆住了。

		莹的爸爸对她不好,又嫌弃她妈妈一直没给他生个孩子,总是粗鲁地骂,生气时就打莹的妈妈。莹
	的妈妈,似乎把这种怨恨都归结到莹的身上,从没有对她表现出丝毫母亲的温情。你突然觉得,莹的世
	界风雨飘摇,你突然想紧紧抱住她,甚至把她吞下你的身体,然后和她一起痛哭一场。后来,莹真的靠
	在你肩膀上哭了,18岁的夏天,你离开家去上大学。虽然没有离开这座城,但也是只有周末可以回家,
	见到她了。她抽泣着,身子颤动起来,眼泪浸湿了你的肩,大大的一块湿痕,冰凉凉的。

		莹问,你会想念一个人吗。

		你将她抱紧了,这是你第一次,拥抱着她弱小无依的身体。

		……想念?

		对,一个人,不在你视线里,你的心就要碎。

		……

		你沉默着,什么也不说,只是将她抱得更紧。陪她哭。

		你突然明白,两个人的依恋,是多么可怕的一件事。你想,你是爱她的吧,你第一次意识到这一种
	感情。却又不敢正视,反复告诉自己,你只是想保护她,让她依靠着,不受到任何伤害,哪怕是最细小
	,最细小的。

		你走了,然后,莹每天只有把自己锁在她小而无窗的房间,不停地做习题。她本不喜欢解那些枯燥
	的理科题,却发现这是多么好的,度过时间的方法。她不去你的房间看树了,虽然你母亲经常招呼她来
	家里一起吃饭。她的父母总是很晚回来。莹常常草草吃些饼干,就把自己锁起来。她的世界那么那么小
	了。她仿佛再没有其他的朋友,除了她自己。她等着你给她打电话,等着你回来。


		然而,莹不会是,也不可能是你生活的全部。也许,她只是回忆里小小的角落,只是一只脆弱的玻
	璃杯。你的世界,在那个夏天后无限展开了,你离开那已显出旧陋的楼房,进入名牌大学攻读。你参加
	各种社团活动,又才成为学生骨干,学习一如既往地出色。每天,忙得不可开交。

		你总是那样优秀,从小就是这样,被周围的同学仰望着。

		舍友们都觉得你该有个女朋友,你却并不关心那些特意在你眼前路过的女孩。她们漂亮,有才华,
	却缺少了些什么,你的心似乎满满的,是莹吧。她住在你心里很久了。然而,你确实是渐渐疏忽着她,
	因为你的忙,你的世界。毕竟,你们不再居住在同一块楼板下了。但你知道,那里有属于你的窗口,有
	黄昏的大树,有莹天真无保留的笑。你得回去,守护着她,不让她受伤。

		你这么想着,却也慢慢发觉这可笑。他们劝你,先在学校里找一个吧,别告诉你那家里,不就成了
	?你别总伤人家女孩的心成不成。

		他们说的,是那个在你演讲后,跑上台为你献花的女孩,玲。这一刻,在初冬的楼梯口,长发的玲
	正怀抱着书本等你下课。她发短信给你:想问问你关于入党申请的事,一起去自习吧,有时间么。你犹
	犹豫豫,你拒绝了太多类似的邀请。终于,你还是答应了。你记起,她经过时,那长发流散的清香。你
	没有这样对一种气味敏感过,你发现,自己喜欢她经过时,那微风里的气息。


		你觉得罪恶,却在那罪恶里获得一种极大的满足和快乐。


		她没有问入党的事情,在你身旁安静坐下,掏出书本,对你温柔地笑笑。那香味,又充满了空气,
	渗在你的皮肤,钻到你的骨头里去。你突然想抚摸她的长发。却抱着迟疑了片刻后唐突的一笑,作为回
	应。玲是多聪明的女孩,她已经洞悉了一切。她知道,她把你捏在手心里了。心满意足地低头看起书来
	。


		周末的时候,莹还是会到车站等你。

		你看到她小小的身影,在人群里一点点扩大清晰了,看到她的笑,她雪白的脖颈上,你送的琉璃吊
	坠。你的心一下子,平静得如落入湖水的鱼。胸口温温的,你竟有想哭的冲动。

		莹穿着白色的连衣裙,站在那里,像个天使一样,纯洁到让你感觉到神圣。她跑过来,挽住你的胳
	膊。

		怎么这么晚呢。

		她总是会抱怨你的迟,却是笑着说。

		那一个瞬间,你闻到她发上的香味。忽地愣住了。那气味,竟是分毫无差。莹见你不说话,就晃动
	你的胳膊。怎么了,怎么了。

		你又是迟疑片刻后唐突的一笑。没有…… 莹的裙子真好看。

		是吗?!她幸福地转了个圈,夕阳的影子落在她身后,一地破碎不堪的橘红色。莹的笑,在那影子
	的涂抹里,显出从未有过的情态。她此刻的幸福,令你感觉到无尽的罪恶,和恐惧。你仿佛是用幸福穿
	透了她小而薄弱的心房,你看着那血流出来,阴湿了莹洁白的裙子,惨淡而悲伤地染成夕阳的那种红色
	。你不敢再看她,她的笑,把你推向无可挽回的深渊。你像罪人一样,拉着莹的手,那细小温凉的手,
	一步步向那矮而疲倦的楼房挪去。


		而事实上,你并没有同玲有任何更深入的交往。你没有碰过她,没有说过喜欢她,甚至,没有主动
	给她发过短信,打过招呼。你刻意回避,而玲总是出现在你的视线里。你的目光也不由自主跟随着那个
	背影,那个侧脸,远远近近。

		玲是学校乐团的钢琴手。你看过她的演出,穿上黑色晚礼服的她,静坐在钢琴前,任乐章于指间如
	水流出,那一晚,你恍惚地坐在台下,随着人群不住地鼓掌。他们怂恿你把去献花,又把学校准备的花
	塞到你手里。你忘记了自己是怎样走上台去,又把花递到她手里。你只看到那一双修长的手,淡粉红的
	指甲。你有时开始想抓住那双手。好几次,在梦里,你梦见自己坠下悬崖,正是这样一双手把你紧紧救
	起,绵软的,却带着骨的清瘦和力度。你觉得自己像钢琴似的,被她演奏着,这一首曲,是你的徘徊,
	她的自信观望。她只等你落网,仿佛战无不胜的女将军,又像万人臣服的女王。玲相信,你绝对只会是
	她的战利品。


		% <todo: 别字: 依在你肩膀哭泣 -> 倚在你肩膀哭泣 >
		玲不知道你的过去,那些模糊的,有点霉变了的记忆。不知道一个倚在你肩膀哭泣的女孩,不知道
	她眼睛里的忧伤,那么深那么深的,就要化作彻底的绝望一样。如果她知道,也许她会不去碰触那一只
	脆弱的玻璃杯,但也许,她会更痛快地把它摔个粉碎。

		玲的等待,渐渐失去了耐性。她以为,至多一个月,你便会表白了心迹,至少,不会再继续闪躲着
	彼此的热情。然而,你的回避从未改变。

		你好像是喜欢她的,你想,但那是爱吗。那么,莹呢,莹又怎么办呢。你不敢想,她那未染纤尘的
	眼睛,你怕那眼睛望着你,充满了期待与渴望。

		莹大概以为,你可以给她一切,一切,在她的昨天残破不堪的爱。她需要你包容着她,给她绝对的
	保护和安全。她的世界里从来只是无垠险恶的汪洋,是你,给了她唯一的岛。那有树的房间,那被哭湿
	的肩膀,那曾背起她上学的背,是她唯有的,能够相信和依赖的土地。莹要在这土地上站稳,伸出胳膊
	,拥抱美的,明亮的世界。你也觉得,她应该拥有那个美的,明亮的世界。然而,你的罪恶,在她的土
	地上滋生,你想把那恶苗铲除干净,你恨玲,却无法拒绝她的一切邀请和好意。你的不忍心,令你对自
	己厌恶。你想把莹托向那个她期许的世界,用你的臂膀,你的全部,同时,你又感觉,莹紧紧扼住你的
	喉咙,让你无法呼吸。莹的眼睛,是危险,是责任,是沉重。


		你发觉自己的无力。有时,你觉得自己没有爱。有时,你觉得你的身体,被爱充满着,几近爆裂。
	你不愿去思考,去判断,去选择。任何的决定和分辨,都是伤害,从不会有完满的结局。你在恐慌里,
	吸着烟。你看到车窗外,风景的熟悉,你的心开始疼痛,无法停止,却不是因为想念。当售票员夺下你
	手里的烟头,你才听清他大声的喝斥,听见没有啊,聋子吗,公共场所不准吸烟!现在的人都是怎么了
	。真没素质。乘客们窃窃私语。你看着被烟灰烫红的手指,微微笑了。

		莹在责怪你又吸烟了。她可以敏锐觉察出你身上的烟味。从中学开始吸烟起,她就总是阻止你,甚
	至藏起你的烟来。你知道,莹心疼你。你摸摸她的头,傻丫头…… 你想抱起她,把她举过头顶,但你清
	醒地知道,那已经是多么困难的事情,你的世界藏了太多未知的恐惧,没了力气。你却越发地明白,你
	爱莹。深切地爱,所以如此劳累了。


		你承认了,你也爱玲。那爱令你轻松,让你的身体变得很轻很轻,没有回忆,没有责任,只有目光
	交错时的心跳,和毫无忧虑的快乐。你充满着幻想,想她的双手,只为你弹奏,只挽住你的脖子,抚过
	你的头发。你却不可以向前,骄傲的玲,终于开口问你,你对我,到底是怎样一种感觉呢。

		你的目光停留在她僵直住的指尖。

		我想,是一种疼痛的感觉。

		你徐徐地说。

		玲的眼神落在你的眼睛里。她并不懂得,你说出的,是多么重大坦白。

		莹19岁的生日,你买了蛋糕,为她庆祝。你知道,她的父母是不会记得这一切的。你们在你的房间
	坐了整个下午,你们喝了酒,玲倒在你的床上,昏昏沉沉。树依旧把你的窗口装点成翡翠一样的梦。喻
	,我想吃巧克力。她仿佛是在梦里一样,含糊地说。你剥了巧克力,塞进她嘴里,又把糖纸叠好,装在
	她的口袋。你看着她那么安静地躺着,才发觉莹从来没有离开过你的心。她制造了你的罪恶,也惩罚着
	你的罪恶。你乐意承担着这样的惩罚,好像从前你乐意为她所做的一切。喻,想抱抱你,想哭。她像是
	自言自语那样絮絮着。你侧卧在她旁边,让她怀抱着你,你好像婴孩一样,被她的身体无私包围。日光
	昏黄,照在她显出稚嫩的脸上,长长的睫毛安静垂下,像是睡着。你喜欢她这样,像个孩子,幼小的神
	态中透露出满足,在嘴角挂着单纯的笑意一两抹。她抚摸你蓬乱的发,问起:

		喻,你会想念一个人么。

		……想念?

		对,想念。一个人,让你轻轻地心疼,无法停止。

		……

		你沉默,你依旧什么也不说。莹的鞋子,在地板上相互失散。你们两个,却第一次紧紧相拥着睡去
	。莹的忧伤,像岁月里盛放了,又凋萎的白花朵。风声,经过窗口,经过陌生人的门前。有多少,真实
	的可堪珍爱的时间,在等候着一处完满或者残忍的结局。

		什么也没有留下一样。你们在梦中失散,像那双红色的皮鞋,零落在两个角落。你们的脸上却充满
	着幸福。你们做了美梦,你们把心安放在了装满糖纸的铁盒子。


		6月,总是翻云覆雨的天气。这一会,就灰了天空,簌簌地落下雨来。玲在街那一边的窗口里,奏
	起钢琴。

	\endwriting


	\writing{六月的琐碎生活}{2006年06月05日 ~ 14:11:43} %<<<2

		生活的真相,是一种琐碎。

		所有的片断,像海浪的细沫,汹涌过潮汐,静止在沙土的痕迹。我总是赤脚走过的孩子,踩下深的
	,浅的足印,捡拾遗落的贝壳。

		在六月,这真相直逼每个人的身体。\par
		热起来的城市,湿润中显露出压抑沉闷。天空灰白色,绿树在暧昧的光线里婆娑。\par
		课堂上,聊斋的故事还在讲,窗扇上又掠过轻捷的鸟影。

		万物都沉睡了似的,如小园里,落落开着的莲,等候路过的目光,等候微雨的午后。

		我们经过小桥,经过美艳得毫无遮掩的月季花从,看世界的天真烂漫,在季节里展开。想着,年华
	的美好,往昔的凋败,全然是小女子的闲愁与矫情。

		而这细微入肌骨的体悟,是青春的馈赠。不该被嘲笑,反应被赞美和珍爱。

		这一天,我们还有心情和气力,来关心一朵花的枯荣。这一天,我们还能对着满园的繁华,一场悲
	喜无端。

		生活,在我们的眼底,是通体的洁白,是美,是爱,是希望。

		儿童节的时候,我们互发短信祝福。超龄儿童,恬不知耻地欢度着节日。记得去年,我还收到了一
	颗水果糖,凤梨口味,晶亮的明黄色。我把它放在口袋,装满了甜美一样,满足地走进洒满阳光的大路
	上。

		今年,我没有一颗糖果,口袋空空,心却富足美好。因为,淡如清水的日子,因为无滋味,而成全
	然的滋味。

		六月,在睡前读童话,在早晨读古诗。听着keren ann的歌,一路奔向教室,在主南门口,争抢着
	买到包子,幸福地吃下去。

		这些隐藏在生活里不值一提的小事,细细想起,也在嘴边泛起笑意。

		我的生活,在琐碎还原了真实,带着尚存的善良和天真,平凡地继续下去,并用文字记录。然后想
	象自己老去的模样,想象那个老太太,一脸笑纹地读她的少女时光。

		小鹿拍下她的墙壁,她的水果,她的清洁用品,她的侧面,她生活的一切琐碎。\par
		我留言:自恋是一种积极的生活态度。\par
		女孩子,总是喜欢用另外的眼睛,发现自己的美丽。因为,女子是最喜欢被赞美的动物。

		从另外眼中发觉的美丽,有神奇的魔力。让我们的平凡,也显得特别伟大。

		而全部的琐碎,好像操场边攀生的蔷薇。纠缠而迷醉。

	\endwriting


	\writing{偶然间的回忆}{2006年06月10日 ~ 15:11:13} %<<<2

		快些扬起你那苍白的脸吧。快些松开你那紧皱的眉吧。你的生命它不长。不能用它来悲伤。

		铅笔的字迹,抄录朴树的歌词,《在希望的田野上》。小满姐姐在出院前,和我们一起唱这首歌。

		一首已成为某种纪念的歌。总要用平淡的语气,回想属于那年的一切,那些美好的女孩子。


		15岁的春天,领到化验的我,被带入病房。听不懂自己的具体病情,忐忑而恐惧地听任护士的安排
	,在走廊尽头的一间病房住下。临床,是瘦瘦的女孩,疏落的长头发被小心扎起,垂在嶙峋的背上。她
	半卧在床上,被子掩着腿,怯生生地对我笑笑。


		她不善言谈,多数的时候一个人沉默地看书,是陆幼青的《生死日记》。她只偶尔与我说话,依旧
	怯生生的,却一脸单纯的善意和期望。她从远郊来,去年的时候关节突然疼起来,路也走不得了,开春
	来看病就住院了。我看她瘦的模样,目光里积着许多未知的悲戚似的。她停顿一下,继续说,不过她已
	经确诊了,类风湿性关节炎,吃了药便可以控制的,几分轻松的模样。


		那时,我还没有确诊,每天做各种各样的检查,她于是安慰我,没关系,再等等吧,会好的。她说
	她会算命,就拿了扑克牌,在雪白的床单上铺开,一张张摆弄,为我的病情占卜。她很认真,算了几次
	,结果却都不好。她充满了抱歉,收了牌,反复说一点也不准。我看她紧张,明白了她的善良,那么简
	单。


		她对我谈论陆幼青的书,讲生命的脆弱,细而无力的辫子垂在背上,我安静地聆听。她只同我讲话
	,只能够同我讲话,很少有人来探望她。有一次,她去隔壁有电视的病房看电视,后来哭着回来,那些
	女孩子欺负她,因她不是城里的孩子,因她的内向和软弱。她常常蒙着被子,把头转向窗口躺着,我看
	不到她的脸,只听见隐隐的哭泣。


		她似乎总是哭,白天,和夜晚。


		母亲来探望我的时候,她总是躲到其他病房。后来,我才听她难得来一次的姑姑讲,她的妈妈在她
	三岁的时候便去世了。我对她,便怀了更深的同情,见她郁郁的侧脸,泪的痕迹还在,便走过去,多说
	些互相勉励与宽慰的话。她很感激我,说她不害怕我,却害怕其他城里的女孩子。


		她吞下白色的药片,她一瘸一拐地去领牛奶,她无声无息地盯住天花板发呆,日子这么缓慢无趣地
	过去。她羡慕着我,总有人来探望,还有许多其他病房的女孩子来找我玩,说说笑笑。她告诉我她的羡
	慕,田,你多幸福呢,她洞张着深陷的眼睛说。


		后来的一个早上,她突然出院。她没有解释,或对我说些什么。我从护士的议论里才得知,她从没
	有出现在医院的父亲,因为医疗费的高昂,而不愿她继续在医院治疗。她走了,脱下病号服,依旧扎起
	疏落的长头发,提起不多的行李,跟在她的姑姑身后,从我的视野消失。


		几天后,病房里住进了小满姐姐,皮肤略黑,身体健康的样子。两个月后,她便要参加高考。她带
	了便携的DVD机来,那时还是非常稀罕的东西。她喜欢音乐,人又开朗,很快我们就熟起来。


		另外的女孩,枫,同小满姐姐一样的病。枫的病房与我住的地方只隔了一扇大玻璃窗。我总能看到
	她,要么近乎放肆地笑,要么抱着饭盒,一阵饕餮。

		她很快乐,并没有生病的悲戚与痛苦似的。我住院的第一晚,她便过来和我打招呼,她的热情,自
	然而然。枫比我大几个月,我也叫她姐姐。在这家儿童医院,我们是大龄的病人,仿佛异类,夹杂在此
	起彼伏的孩子们的哭闹声中。


		我和枫,喜欢蹲在楼道里聊天,说那些女孩子的心事。她的故事,我的故事,被统统交换,相互猜
	测着对方的快乐或感伤。她的快乐总是感染了我。见她没头没脑地笑,也便少了生病的疑虑和恐惧。


		她向我借手机,打给她喜欢的男孩,却没有接通。她说,他可能还不知道她病了呢。她的心,向下
	沉去。她对我讲,那是怎样一个男孩,他们怎样认识,怎样相处,又怎样误会,和失散。她只记得他的
	电话,只可以打电话,却始终没有接通。


		他不知道你生病是好的,不然他会多么担心呢,我说。枫没回答,坐在床上,默默吃她最喜欢的蛋
	酥卷。


		在那里,我和枫常常一起被小孩子们包围,逗他们玩。


		其中的一个小女孩,对我的随身听很有兴趣,她喜欢趴在我的被子上,让我给她放音乐。那还是卡
	带的随身听,我带了许多磁带,其中有蔡依林的新歌。她说她最喜欢那首《爱上了一条街》。是节奏明
	快的歌。她会在听的时候忘我于其中,不自觉间跟唱起来,爱上了一条街,迷路也甘愿。自然,她的歌
	声是跑调的。


		她的头发乌黑,眼睛明亮。她总是抱着她红色的毛绒兔子,从走廊的另一侧,跑来找我玩。要我带
	她一起去照镜子玩。

		这是我发明的游戏,两个人站在大镜子前,做出鬼脸,看谁保持的时间长。她似乎特别喜欢这个游
	戏,那次,我只是随意想出来哄她,她却着了迷一样天天来找我去玩。她拉着我向那镜子跑,迫不及待
	的样子。我们站定在那镜子前,看着其中出现我们扭曲可笑的脸,哈哈大笑。


		其他的孩子要加入,被她决绝地拒绝了。她不允许别人加入,属于我和她的游戏。于是,有小孩子
	偷偷跑来对我说,姐姐,她的毛绒兔子掉到尿盆里了,她却还是总抱着睡觉。我装出惊奇的模样来,他
	们就满意地离开了。

		我们的游戏从没有停止,每天我都陪她去,我们不厌其烦。有一天,她向我借随身听,并说用她的
	毛绒兔子作为抵押。她要在睡觉前听那首歌,她解释说,有点不好意思,像是怕我拒绝。我借给了她,
	她把毛绒兔子安放在我的床边,喃喃嘱咐着,和姐姐好好玩,明天早上来接你,要乖呀。然后她一蹦一
	跳地离开了,脚步有力地踏在地板上,嗒嗒地跑远。

		第二天,她却哭着来找我,姐姐我的兔子没有掉到尿盆里,他们瞎说的,她着急地说。我笑了,把
	床头的毛绒兔子指给她看,你瞧,她睡醒了,昨晚晚上它都告诉我了。她于是安心,又带着惊奇。她把
	随身听还给我,并说,她下午便要出院了。


		她拉我到她的病房,她的母亲和婶婶正在为她整理和准备出院的物品。她得的病是儿童糖尿病,饭
	前半小时要注射胰岛素,她小小的胳膊上早已满是针眼,只有6岁,她却学会了自己来注射。由于家乡
	没有足够的药物出售,她必须从北京带一些回去,并定期再来购买。离开的时候,她的母亲对我表示感
	谢,并说她女儿特别喜欢我。她嘿嘿地笑了,躲在母亲的裙子后,向我挥手道别。再见,再见。我想象
	着她还很漫长的人生,想象着那一路的崎岖与颠簸,心里生着不安。


		我多想,她能够像歌声里那样,在一个明亮的日子里,穿着最心爱的皮鞋,有力地踏着欢快的节奏
	,一脸幸福地走在街上,走进无穷尽的阳光。她本该是那样的女孩子,平安完整地长大。然而,这世界
	有多美好,便有多残忍。


		我不知道,那个小女孩还好吗。不知道,她的生活里,有没有昨天一样的天真美好。那镜子里的鬼
	脸,此刻仿佛成为对于命运的讽喻,被我无情地记起了。


		小满姐姐教我们唱歌,朴树的,《在希望的田野上》。你的生命它不长。不能用它来悲伤。

		我总是重复着两句。

		枫同我并肩坐着,好多的时候,我们都是这样一起唱着,唱着,等候着,或者遗忘着什么一样。有
	时,是《天黑黑》,有时,是《年华》和《叶子》。我们重复那些歌词,一遍遍,在心里暗记。喜悦的
	伤感,和忧郁的欢乐。从我们的静脉里流过,像是生命注定经历的河流与天空,有晴空,有乌云,有风
	季,也有暴雨。

		小满姐姐把歌词用铅笔抄下来,送给我们,作为离别的纪念。几年过去,我再次见到,那字迹早已
	模糊。而当时当地,那些沉默的,悲戚的,不安的心,那些勇敢的,纯真的,美好的人,却清晰如此。


		我仿佛可以触摸,曾经由窗口散布入病房的光线,那么细的,令人惆怅,却生出爱与希望。


		只有和枫还保持着联络,我们在假期见面,打扮漂亮,逛街,吃饭,做所有女孩子喜欢的事。\par
		快些扬起你那苍白的脸吧。快些松开你那紧皱的眉吧。\par
		生命该是一场饕餮,要抓紧享用。我们都要幸福,我们总是这样彼此祝福。

	\endwriting


	\writing{云。想}{2006年06月10日 ~ 16:37:36} %<<<2

		起风了,在6月,这样的风显得珍贵。

		看它们落落经过树尖,轻摇着日光依稀的午后,我的心,碎成一块一块,仿佛洁白的石子那样,向
	着季节的湖底坠去。


		云在天空,浮着那么许多,清凉的,糖果似的梦幻。

		让我们撑着伞,走过忽明忽暗的路途,有时衣裙上沾染了雨花的痕迹,有时发丝间隐藏了阳光的气
	味。

		云在青天水在瓶的从容与智慧。

		纯粹而明亮,时间嘀嗒而去,欢快的小马驹一样。

		阳台上,新洗的白衬衫被轻轻拂起,招展着,像是做着骄傲的美梦。

		我想象,衬衫的主人,那个楼上的男孩,他是怎样仔细地揉搓着衣领和袖口,又把它挂在风中。

		那一定是一个善良的人。

		有温柔的目光和语气,带着无框的眼镜,总是擦得晶亮。

		我想,他的一切,与我的恋人相近。


		看云层,厚重洁白,于是有飞翔的冲动。而我的翅膀,遗失在相爱的路途。

		不是谁说过的么,每个女孩,都是坠落人间的天使。

		我笃信着,这显而易见的谎言,正如笃信了,一切美到残忍的童话。这世界一定住着公主,王子,
	他们幸福地生活着,直到永远。


		你相信爱情吗。

		不。

		那么你成熟了。


		听到这样的对话。那么,我是幼稚的。并甘于幼稚。

	\endwriting


	\writing{痕迹}{2006年06月15日 ~ 13:30:09} %<<<2

		六月,一处光明的世界,女孩子穿上碎花布裙子,穿越正午的操场。短短的黑影子,落在她们脚下
	,无声蛰伏。

		夏天的滋味,从热的青草上蒸腾烂漫。\par
		在傍晚看看云,从树枝的空隙游过逃走。\par
		不知名的白花朵,飞散在肩膀,你小心捏起,让恋人嗅它的芳香。

		多么安静,又多么热烈的季节。

		在窗外蝉声的聒噪里,我沉沉睡去。怀抱着薄被,仿佛伏在云层的松软。\par
		人心是贪婪,却竟又如此容易满足。\par
		我微微笑,就有清凉的梦,有你的背影,在日光中,向我奔跑。

		这六月,如一年年用蓝色墨水写下的日记,成为恒久的,不退却眷恋的痕迹。

		是孩子跑过炽热的沙滩,留下的精致脚印。\par
		是昨夜哭过的枕头,深深浅浅的湿痕。

		而我,已不是爱哭的小丫头。只是,悲戚总是难免。那些时候,眼泪成为最苦,也最特效的药剂。\par
		枕巾上,残存了孤零零的睫毛。

		睡眠若是足够美好,将是一种奇异的治疗。\par
		夏天的我,总是瞌睡。梦也便丰茂,如雨后的植物。

		(最近又在读诗经,思绪也变得简洁。最近又在准备考试,头脑也转向混乱。于是,用短的字书写
	,生活的细枝末节,零落琐碎。这大约更接近日志的本意呢。)

	\endwriting


	\writing{房间。窗}{2006年06月16日 ~ 14:06:23} %<<<2

		14岁的夏天,和朱朱谈论未来。我说,我想要一扇挂着水蓝色窗帘的窗。

		后来,我如愿拥有了这样一扇窗。便时时对着那被风拂起,又悄声落下的蓝色,静静发呆。很多次
	,是在有雷电的夜晚,雨水的湿味,摩挲过土壤和植物,涌进我的窗口。

		我爱着这样的时刻,深的,浅的,泄露着幸福和忧郁的蓝色,毫无保留地充满我的世界。

		我的窗口,挂着水蓝色窗帘的窗口,总是那样默默等候一般。\par
		承载着日光,承载着星辰,开启我的想象,我的梦与幻。\par
		清早或午后,当风光顾敞开的门户,我的风铃丁丁当当,用清越包裹起小屋的自恋与孤独。

		人对于房间的眷恋,大约多数是因窗而起。没有窗的房间,不过压抑的囚笼。

		当黄永玉先生讲到自己的人生经历,他回忆的线索,便是不同居室里的窗。最困难的时期,住在没
	有窗的房间的先生,便在墙上画出一扇美丽风景。这令我不禁动容,也想到民间故事中,那马良的神笔
	。从那绘画出的窗口,先生望见的,是自己心里无垠无际的壮丽世界。感慨于大师的境界与精神,是我
	等凡夫不可效颦,惟有高山仰止,深心钦佩的。

		窗口,在我们的房间,更在不可触摸的另一端世界。

		看到一幅照片,远在丹麦的窗口。红砖的房屋,白色的木窗里,俯卧着一只睡眼矇眬的小猫。好的
	日光,把树木的影子投射在墙上,婆娑的窗口,无比悠闲。在那童话的国度,是否处处有这般的窗口呢
	。

		我渐渐开始喜欢猫的姿态。

		它们总是一副漠不关心的神情,骄傲地行走跳跃。它们依赖主人,却又随时可能弃你而去。与狗相
	比,猫大约是太不厚道的。而狗的忠诚,总有卑微屈从的意味。

		猫是孤独,是智慧,是与人群的恰好距离。它们享用感情,却从不迷失。

		于是,这样有慵懒猫咪的窗口,也显得寂寞而高傲。


		太多时候,人并不需要被理解。

		我们可以只是一只窗口里的猫,被路过的人见到,记住,或者遗忘。那将是路过者的选择,与我无关。

		空的窗口,总让你有所期待。一个一闪而过的身影,一盏忽明忽暗的灯火,一株被细心照料的植物。

		窗口,意味着一种包容与皈依。

		在萨拉霍夫的画作中,反复出现着窗口,他仿佛习惯于从窗口望出去,来观察这人间纷纷。窗外,
	是庭园,是河流,是色彩浓重热烈的一切。他饱含着热情,从窗里绘画天空清澈,草木繁茂。

		在空的窗口里,是否可以安放下,我们对于美的所有信仰与期待呢。\par
		房间与窗口,在每个人的天地里,不断制造着可能和奇迹。

		最近,人人在讨论关于搬宿舍的事。\par
		终于确定下来,我们大约是会搬到11楼的1到3层。于是,便又开始忧虑朝向和楼层的问题。\par
		超超说,她希望可以住到向阳的房间,但不要是一层。\par
		因为向阳,房间便有光亮和温暖,而一层,是过于吵闹的。现在的宿舍,是向西的,房间总是要开灯。\par
		如果还是被分到阴面,那么只能说明,我们是注定在大学四年里过着蚯蚓般的生活,我笑笑说。

		虽然这样玩笑着。我也是在期待新的宿舍,能给我一扇有日光的窗口。\par
		我想,看太阳的影子,在四壁流转,该是件多惬意的事。

		起风的下午,趴在床上,问莫:我们就要搬出这房间,会有新的主人住进来。房间将空掉,被重新
	粉刷和装饰。那些,我们曾在夜里说的悄悄话,是否也会被一并埋没遗忘呢。

		路过这里的人,会不会知道,曾经的这一扇窗里,住着梦里的女孩。

		也许,以后的我们会站在楼下,甜蜜而怅惘地辨认窗口的位置,记起一起听着广播的雨夜。

	\endwriting


	\writing{婴的回想}{2006年06月18日 ~ 21:14:19} %<<<2

		当我初来这纷纷人世。\par
		你总是怀抱着幼小的我,在心中许着幸福的祝愿,升起大大小小,晶亮的希望。\par
		也许,你也曾淡淡感伤,望着我的成长。你的目光,是慈爱,是担忧,是无以复加的爱。

		我想象着,那些时刻。

		夏天的夜晚,带着爽身粉气味的风,吹入小小的家,沉在你哄我入睡的低声吟哦。你低垂着疲惫的
	眼,看我瓷白的脸颊,疏淡的眉头和发梢,浮现着满足的笑意。后来,你说,你曾惊奇,你可以在怀中
	抱着如此完整的孩子,你创造了她。你体会着作为母亲的快乐,和艰辛。我是好动的孩子,睡眠很少,
	总要你吃力地抱着,一放下,便无休止地哭。你笑着回忆,佯装着抱怨,却又在补充中泻露了你的骄傲
	,“老人都说,觉少的孩子聪明。”你相信我的聪明,你拿了满月的照片给我看。“你瞧,那双眼睛,多
	机灵呢。”

		远去的,如隔群山的那些时刻,你在昏黄的灯下踱着步,久久凝视你的孩子。如此纯净,如此完整
	的孩子。仿佛是神的赐予一般,不可思议。你大约却没有想过她的长大,竟然是这样匆忙,不留余地的
	。

		这个六月的某天午后,百无聊赖地平卧在纱帐里。听新下的歌曲,名叫王筝的女歌手,有清淡的歌
	声。按下重复键,让一首《对你说》反复播放。

		\longpoem{}{}{}
		如果明天你就长大很多 \\
		我会不会觉得不知所措

		你不再想让我牵你的手 \\
		每天盼望从我掌心挣脱
		\endlongpoem

		长大的我,在离开的母亲视线的地方,默默安享着孤独。我不再依赖她。不再要她抱着,才肯入睡
	。不再需要她为我讲小兔子的故事。母亲的手,松开,在空中挥着,她与我道别,不显露丝毫的忧伤。
	而我,终于知道,她心中那深深浅浅的失落。她的孩子,仿佛只在一夜,便长大很多。王筝的歌声,在
	早已远去的夏夜里徘徊,从那里出发,垂直下落,直至今日眼前。

		\longpoem{}{}{}
		你也会爱上一个人付出很多很多 \\
		你也会守着秘密不肯告诉我

		在一个夜晚抑制我的 \\
		泪水止不住地流了一整夜
		\endlongpoem

		每个女人,大约都是这样长大。在母亲防卫的目光里,懂得爱情,拥有或者丧失。她不愿她的孩子
	受伤,但她只有是无可奈何的局外者。在一个访谈节目里看到,一位母亲说,她最大的悲伤是发现了自
	己女儿上了锁的日记。她有秘密了。

		而你,从不会刺探我的心事。你说:你若愿意讲自然会讲。你好像知道,在感情中你的女儿,永远
	采取着自我保护的姿态,非常安全。但我也会受伤。只是,我不曾倚住你的肩头哭泣整夜。我的坚强,
	没有你想象的勇敢。

		你总是相信我会幸福,你的女儿,会被好好疼爱。你说,要找一个好男人,不要抽烟喝酒,要顾家
	。我点头,我在期望和憧憬。

		终于有一天,我会拉着那个人的手,站在你的面前。你会安心,你会伤感。我恐惧那个时刻的到来
	,那一天,你的小女孩,便真的,无可挽回地长大了。我却依旧喜欢歪在你的怀里,一起看无聊的综艺
	节目,一起哈哈地笑。小时候,你玩笑着问我结婚的年龄,我说,我不要结婚,要一直陪着妈妈。你说
	:那不行,你长大了,我就不要你了。为这,我难过了很久。

		我躲在纱帐里听着歌发呆。夏天,不遗余力地把回忆的网织得紧密。花露水,爽身粉的气味,在我
	的床单。是同母亲一起挑选的花色,淡紫色的花。母亲的爱,令我疼痛。

		几天前,她路过学校,便带了些水果给我。艳阳下,她站在树荫里,戴着白色的遮阳帽。我跑过去
	,她絮絮地叮嘱,要多喝水,天气热,一定要吃东西,不许多吃冷饮…… 这些话,重复了许多年,显得
	无趣而琐碎。她却不厌其烦地重复重复,唯恐我忘记。她从提包里拿出水果,几只散发着香气的桃子。
	然后,她蹬上自行车离开了,在白花花的日光里。我站在远地,抱着那几只桃子,任思绪淹没了知觉。


		我照镜子的时候,你喜欢安静地坐在旁边。你看我无声息地长大了。你是不是也像那歌声中唱的:

		\longpoem{}{}{}
		和你一样我也不懂未来还有什么 \\
		我好想替你阻挡风雨和迷惑

		让你的天空只看见彩虹 \\
		直到有一天你也变成了我
		\endlongpoem

		轮回一样。你是否在这镜子前,看到了自己的青春。而年华,在你的面庞上雕刻,那痕迹,留在眼
	角额头,留在鬓角的发丝。我为你染发,你感叹着时光。人的一生,是这样匆忙得不堪一击。整理衣柜
	时,你找出原来的连衣裙,明黄的颜色。我问着曾经的故事,你就笑了。你的心底,也藏了心事吧,不
	肯告诉我。

		想到另一首歌曲。叶蓓的《孩子》。去年的自己,在日记本上抄录着最爱的两句歌词:

		\longpoem{}{}{}
		春天的花开 \\
		开在冬天的雪上
		\endlongpoem


		好多时候,感觉我好像那花朵,在母亲的雪地上绽开。一样的纯白,却带着生命的苍凉意味。一年
	年,我们的日子,这样平凡,又充满惊奇地过去。而女人的生命,总是多情婉转。她望着我的长大,就
	望见失去的自己,我看着她的光阴,就体会到经过的甜美和残忍。

		\longpoem{}{}{}
		风吹过的过去 ~ 我们从没有忘记 \\
		想和你分享 \\
		可是你已经老了 \\
		孩子孩子 ~ 我还是孩子
		\endlongpoem

		这歌声持续。

		我在纱帐里,想起母亲的怀抱。我们小小的家,那夜风的芳香,窗外散碎的星光荒凉。

		我回去了一个婴孩的状态。睡在没有安抚的梦里。

	\endwriting


	\writing{草木}{2006年06月21日 ~ 16:56:01} %<<<2

		躺在草席上,喝一瓶清水,听树上的蝉声一日日浓郁。就陷落在夏天的意味中,随着汗水,任草席
	的气味渗透在发丝。


		我在这样的下午,无所事事。安然如草木的无言,面对对街热烈的生活。

		看四海乐进进出出的食客,看站在面包店门口发呆的姑娘。这小小的街,仿佛体验着莫大的快乐或
	忧郁。我想象人们的生活,动态的,一刻不息的生活。每个人,度着各自的日月,留下清晰又模糊的记
	忆,作为报偿。

		街边的树,高大的泡桐,在春天的时候,总会由树顶坠下淡紫色的花朵,散发出香气。似乎很少有
	人注意到它们的存在,以至于几次我同人提起,他们多是一脸茫然地问,有么。


		树,不会理会人的经过,它们也从不奢望占据你那些越来越可疑的印象。它们兀自地生长,开花,
	凋零,休眠。季节在它们的树梢变换,一年年,一成不变的从容。我喜欢树,正是因为它们的挺拔和静
	默。我相信一切植物都有知觉和感情。

		它们在无声望着这悲喜交错的人间,它们知道一切。


		对于植物,我总怀着特别的爱意。很多时候,我甚至愿意,在来生做一丛开着小白花的草木。


		春天,多风的下午,我站在来园,看梨花的飘零。那时,池塘里还没有放水,于是,我只是见到了
	花树的舞蹈,而没有悲伤。只有流水,才会泄露花瓣的悲伤。整整一个冬天的封锁,花朵是用了多少力
	量,才营造出这一树树梦一样的繁华。风飘万点,它们离开树梢,飞行向还惨淡着的天空,落向没有答
	案的宇宙。草木的心,大约远比人敏锐细致,不然,它们怎么会这样精准地计算好,一切的相遇与别离
	。


		在旧家的庭院里,曾经有一架葡萄。每到夏末,总会结出硕大饱满的果实。后来,葡萄竟突然萎黄
	死掉。这原因,据说,是因为祖母在那一年出了车祸,被摩托车撞折了腿。


		祖母是去给马路住在马路对面的哥哥送牛奶。那时候,牛奶还是要凭奶证定时领取的。祖母每天都
	要穿过那条现在已经被称为三环的大马路。从来都是安全而稳妥,却就在那天,被一辆横冲直闯的摩托
	车撞倒了。一切的发生猝不及防。

		从此,祖母很少过马路,也总是紧张地提醒我,过马路要注意安全。我家的葡萄,在出事后不久,
	就一点点萎黄死掉了。这事情听来离奇,或许是巧合。但即使是巧合,也令人感到惊奇。那葡萄是祖母
	栽的。

		很多年过去,她还是常常提起,她说,那是多么仁义的一棵葡萄。我也相信,那葡萄对于祖母的感
	情。它一定是很伤心,才会这样无端地枯死。


		现在,我时常怀念旧家的庭院。

		在那里,门前的花坛中盛放着月季,南墙有父亲栽的竹子。小时候,我总是梦见,自己在那一小块
	竹子里遇见了大熊猫。在竹子旁边,还种过向日葵,收获了许多瓜子。我说向日葵是永远在微笑的花。
	在阳光很好的日子,母亲会抱了被子,晾在西房门前的空地上,有时,也会铺了竹席在地上,翻出旧衣
	服来晒。它们散发着卫生球的气味,躺在阳光里,同向日葵们一起微笑。我总是兴奋地翻弄那些衣服,
	并尝试着套在身上,对母亲说,把这个给我吧,但看看几乎托在地上的衣袖,便只有补充一句,等我长
	大了。


		终于,我长到了足够大,来穿母亲的旧衣服。我却也没有再提起当年的要求。


		旧家庭院的树枝上,挂着装蛐蛐的草笼子。它们不停歇地叫,一个又一个夏夜。我把祖母切好的黄
	瓜条,塞进笼子的缝隙里。我其实并不喜欢它们,甚至有点害怕。我更愿意看床头装着萤火虫的罐头瓶
	子。瓶子里,飞着几点小小的光亮,紫的,绿的,瓶口用细纱布小心封好。它们只能够陪伴我一个夜晚
	。而幼小的我从没有因此而感觉悲伤。我只体会到快乐,那小小的光亮,闪闪烁烁。据说,有草木的地
	方,便会生出萤火,很多年了,我却再也没有见过它们。


		我开始恐惧城市,密集的住居,令我们与土地与植物脱离。

		还好,北语有足够多的树木和花草。

		我可以在起风的日子,听到窗外如衔枚疾走的草木之声,可以撑着伞,站在梧桐树下,接住叶片脆
	碧的缝隙里漏下的阳光。有蝉声,有布谷鸟,有一塘爱唱歌的青蛙。这样的声音,调合着我们心底的浮
	躁,在热闹的人世,给你独自清凉的去处。于是有这样无所事事得下午,任由我躺在草席上,喝一瓶清
	水,在头脑里生长出这些无关痛痒的情绪。


		良说,铺草席并不能够使人感觉凉快。我说,我是喜欢草席的气味。


		那是来源于植物的气味,带着雨天里,淡绿色的清香。那滋味似乎微苦,却如茶一般,因苦而隽永
	。我闻这草席的气味,就仿佛置身于草木,仿佛听了满耳的虫声。草席带来的清凉,是靠无可遏制的,
	关于草木的追忆想象。睡醒的时候,我的脸上总会被印上草席的纹理。于是,偷偷对着镜子发笑。那好
	像是夜晚的纪念。正如,阳光灼伤我们稚嫩的肌肤,作为夏天的纪念一般。


		在草席上,安然如草木的无言。这是我度过夏天的方式。

	\endwriting


	\writing{雨中的随手留言}{2006年06月24日 ~ 21:05:31} %<<<2

		傍晚的时候,大雨如注。如我所愿,一场滂沱不期而至。\par
		我总在苦夏里,盼望着雨。\par
		愿意守住窗口,呼吸湿润的风丝,任它们游移,在鼻翼,在气管,在肺叶。


		于是浮躁的世界可以静定,在火热与憋闷里,得以喘息。


		雷声轰鸣,闪电撕裂夜的帷幕,照亮四壁。楼下有孩子们兴奋的尖叫,有穿越雨雾的列车一路飞驰。


		高楼之上,贝多芬的一曲《悲怆》,流出窗口。是谁,在这样的傍晚,听着凄恻而坚硬的钢琴。


		菊花在我的水杯里缓缓舒展,这一杯茶,贴近缄默的唇齿,留下淡漠的芬芳。杯体的温度,仿佛这
	风雨飘摇中的慈悲,在手指间散播着恰好的暖意。


		是谁,在雷电里,遇到似曾相识的许多曾经,是谁,在不言说的记忆里,有了去年今日的亡失之痛。


		我在雨声里读诗。朱湘的《夜歌》低回在浸入深蓝的空气。


		\longpoem{}{}{}

		唱一支古旧,古旧的歌……

		朦胧的,在月下。

		回忆,苍白着,远望天边

		不知何处的家……

		说一句悄然,悄然的话……

		有如漂泊的风。

		……
		\endlongpoem


		没有月亮,没有苍白的痕迹,这夜晚,我没有漂泊的心境。而茫然的穹宇之间,面对恒久时空之沉
	默,又有谁不曾战栗。


		我们,只唱一支古旧,古旧的歌,只说一句悄然,悄然的话,在历史的耳边,在命运的旁侧。所有
	的不可知,将类如宿命的安排,抛掷于人的生命。所有的荣获,和丧失,所有的眷顾,或不公。全部没
	有征兆和理由。


		性命微小,人总是没有权利责问我们的际遇和遭受。


		“在我们几十年的生命力,最不可能枯竭的就是疑难。”史铁生,他这么说。


		你若把疑难当作馈赠,那未免虚伪自欺。但至少不去质疑和怨恨。

		怨恨是徒劳的,只会加深人的不幸。


		生命的本质是悲观,因此,我愿有乐观者的姿态。当我们真正能够接受一切,无所忌惮,无所遗憾
	,大约便可以接近了幸福,可以经常微笑着,站在花开无声的阳光里。于是,生活可以回归于美,回归
	于善,到达明亮。


		这所有,原来只是我们心灵的选择罢了。


		不幸,是不肯舍弃的自我折磨,幸福,是懂得之后的主动选择。


		任何时候,都没有放弃幸福的理由。因为,有爱的存在,有爱人的目光在此处,在彼岸,在回忆,
	深深凝望。


		雨,紧扣住散碎的悲伤,落满世界。


		我想到的,只是无意中拾起的情绪。灰色的,或者洁白的,属于我的,生命的危险。

	\endwriting


	\writing{七月}{2006年06月30日 ~ 17:10:51} %<<<2

		七月,是海,是远天无限膨胀的云层。是小荷初绽时节,羞涩却热烈的气息。是漫天席地的一场滂
	沱,任雷电的光焰撕裂夜晚。


		我在等候,似曾相识的七月。在困顿却恬淡的生活中。所有的七月,掠过我的窗口,照进树影婆娑
	的日光。曾经,那么多的获得和丧失,一瞬间,都显得虚弱无力。唯有眼前的岁月,逼真清晰,伸手可
	触。而记忆的意义,从不在于真实,却恰恰是虚妄。


		因为是虚妄的,我们才有所眷恋或疼痛。


		总会在七月,画一张水彩卡片,送给朱。总会为朱庆祝生日,情绪复杂地迎接新的年纪。有时,是
	一群人吃饭,有时,只有我们两个。

		对座的朱,从未长大的模样,天真的眼睛,细弱的身体。

		她18岁的生日,在我家里,朋友们一起下厨。娴淑的静,单纯的洋,在厨房里忙作一团,我举着DV
	机拍摄,问无聊可笑的问题,造成画面抖动严重。我制作了彩色的小帽子,每人一顶,戴在头上,让人
	想起几米插画里的小孩子。朱开心地端上了她的“拿手菜”南瓜塞肉,很美味,虽然,肉似乎没有熟。

		几个女孩子,吵吵闹闹地,在那个七月里,共度了劳累又奇妙的一天。

		我时常想起,蹲在那里削土豆皮的静。想起,草莓口味的冰淇淋蛋糕,凉丝丝的甜蜜。


		18岁,如一处明亮的印迹,雕刻少女最纯粹的欢乐,带着调皮的神情,在时光的录影带里挤眉弄眼。


		那年的夏天,大街小巷在唱着那首,七里香。

		然后,毕业的聚会上,我听到他们在嘈杂的包房里唱:窗台蝴蝶像诗里纷飞的美丽章节。


		身边的人说,那MV里的女孩子很像我。我看不清她的面孔,只见到,白色的裙,淡绿的天。


		七月,好像就应该有那样一种淡绿色的天空一样,有穿白裙的女孩站在原野中央,等待着季节的流
	转,誓言的变更。

		那些属于中学,末端的记忆,是没有头绪的零乱,极美的,却永远带着虚弱的语气。只有当时的日
	记,在纸页上,袒露无疑:


		我说:

		\longpoem{}{}{}

		我是角落 ~ 是安静 ~ 是下落的灰烟 \\
		是你未及发现的一次皱眉


		你是喧嚣 ~ 是歌声 ~ 是转身后的回眸 \\
		是我促不及防的思念
		\endlongpoem


		我忘记了这些字,是怎样写出,怎样被记录,再就此遗忘。

		当时的心绪,已成往事旧迹,不容推测追悔。只是,再次读起它们的时候,心灵的颤抖,却是超越
	了所有过往的真相。


		七月,大约总是属于离别。

		看校园里穿着学士服拍照的学姐学长,手捧鲜花,面带微笑。在粉红的落霞里,听傍晚的校园广播
	,播放着《那些花儿》。

		我在这里,望见你们的离开,一切的一切,貌似不带丝毫的悲伤。也就恍如,在时光的远处遇到了
	自己,身穿黑色的学士服,站在图书馆熟悉的楼梯口,拍照留念。

		这天晚上,下着大雨,我独自在宿舍,扭开广播。窗外,是疾风骤雨。

		我想着,正在聚会的毕业生。他们会怀着怎样的心绪,面对这眼前的一切。


		我不曾知道,离别时的祝福,我不曾懂得,大学的全部意义。


		而我,仿佛是被时间,抛掷在眼前。06年的夏天。


		我坐在自己的窗口,流汗,读书,写作,默不作声。好像去年的夏天一样,于高楼之上,侧听火车
	呼啸而过,无所思想。

		病的身体,需要好好的静养。于是煲汤来喝,补一补比过往更为虚弱的肉身。

		在苦夏中,过着深居简出,类似隐居的生活。看我的狗,在地板上睡熟,呼吸均匀。依旧读诗,发
	呆,抄录心爱的语句。依旧沉迷于浴后肌肤间散发的清香,湿着头发,坐在桌前,在日记上写下生活的
	琐碎。

		因为简易,而无所欲求的日子,平淡如水,静定如水。


		去年的七月,同小鹿与莫一起去紫竹院看荷。细细的小舟,荡漾在天光水色之间。她们的善良,在
	无所保留的微笑里,令我感动。

		我身边的朋友,总是这样天真无机心的孩子。正如朱,正如静。一样的,体恤着你的阴晴冷暖,为
	你的忧伤痛哭,又为你的欢乐大笑。常常生病的我,被她们照顾,关爱,是上天莫大的眷顾。我感激着
	,却总感无以报偿。


		昨天晚上,又是大雨如注。毕业的良,即将离开学校。

		站在礼堂的屋檐下躲雨,雷声轰然。他用大手,堵住我的耳朵。

		雨下了很久。

		分别时的良,站在宿舍门口,对我挥手,你先走吧,他说。于是,那独自站立着凝望的身影,成为
	我记忆里永恒的形象。他的模样伶仃,悲伤。


		昏沉的午睡,收到静的短信。我回来了。

		七月,又仿佛是永远的相聚。

		却想到,这个夏天之后,朱便要飘洋过海,求学彼岸。想起,她说,以后就是只有你和静常常见面
	了。那海的彼岸,载着未来,和光辉。我们只有在这里,望着茫茫汪洋,朱,你要照顾好自己。

		只是,以后的七月,我们还会一起为你庆祝生日吗。我还是会画好一张卡片,等着你齐整的白牙齿
	间,那熟识的笑。


		七月,是风,是起落不定的歌声。是分别时,爱人手心的温度。是我贪恋的情感,失火在一处处,
	乱了分寸的记忆。


		我说:

		\longpoem{}{}{}

		我是分 ~ 是秒 ~ 是不曾紧握的时间 \\
		是你的疼痛与温柔


		你是爱 ~ 是笑 ~ 是洁白的天真 \\
		是我无能为力的眷恋
		\endlongpoem

	\endwriting


	\writing{自言自语}{2006年07月07日 ~ 11:58:24} %<<<2

		早上,凉风吹进房间,孤单的窗口安静。

		风飘起白窗纱,好像指尖抚过夏天的肌肤,带着草席的气味。侧卧在床,昨夜翻倦的书,掉落在地
	上。生活,在洁白的光线里被擦拭一新。

		我喜欢这样的早上,有着寸寸滋生的希望,让你觉得,生命是完好的,毫无缺损。

		在这高楼上眺望。远处,是喧哗的街,是车流的不息,是城市。

		一个多么热烈而可爱的人间。因为有喧哗,有声响,而让我感觉那人群的真实。而更高的地方,在
	云层的深处,又住着什么人,俯瞰着人们的忙碌,慌张,怀着慈悲的心肠,或暗自发笑。那是我们从来
	无从知晓的世界,一个冥冥中主宰着万物,却永远不被感知的世界。只有在心里,我们敬畏着,我们期
	盼着,等候着被眷顾和救赎。

		人的微小,在悲喜种种的表情里,被表露无疑。昼夜仿佛一本不被解读的天书,在白日里为我们展
	开,又在日落时合上。参悟天地的古人,大约最能够接近,书中的奥义。而今日今时的我们,不过双眼
	蒙蔽的盲者,只看到天黑天亮,而无从懂得丝毫自然的教诲。

		我常在想象,千年之前,庄子的大樗,那在世人眼中毫无用途的大树。庄子却说,何不树之于无何
	有之乡,广漠之野,彷徨乎无为其侧,逍遥乎寝卧其下。我于是可以望见,一个神情自在的智者,在夏
	天的原野之上,安睡于巨大的树阴下,轻轻地打着鼾。

		真正的智慧,是合乎于天地的本然,不去干预,不去改变,事物最初的本性。我们早已习惯用实用
	的眼光去考量一切,却从未觉察,这本身是多么狭隘而可笑。当我们习惯,我们也便成为被囚禁的灵魂
	,试图成为一个实用的人,而不是我们本来的模样。

		没有什么,比改变一个人的天性更为残忍。而我们每个人,又能说自己从没有屈从于世俗的价值和
	眼光,不惜以损害自身为代价呢。这是困难的事情,因为,我们不是智者,我们在真实如此,残忍如此
	的一种人间。

		与小鹿并肩,坐在学校东门新开辟的一块荷塘边。荷花含苞欲放,默无声息地生长,挺直脆嫩的身
	子,高傲又寂寞的姿态。

		我们抚摸新整理的草地,一旁的推草机还在轰鸣。

		手指陷在草叶间的样子,让我想起几米的《照相本子》里,最后一页的画面:两个人,闭上眼睛,
	并躺在草地,幸福地睡去,一直到草已长高,淹没他们的身体和面庞。

		“后来,我们在彼此的梦中,幸福地慢慢醒来。”

		我并不知道,几米在画面与文字间暗示或隐喻着什么。却感觉出时光,或生死的一些意味。时光与
	生死,大约本是一回事情。有关我们的生命,我们的沉睡,和醒来。一切的存在,流失,一切的拥有,
	丧失。这将是没有休止的思考。每一次,都在我的心房紧叩,咚咚地响。这声响,是快乐,也是悲伤,
	让人对生命有所知觉,有所感激。

		当我们这样坐着,仰头面对云朵,便想起大一的时候,一起读过的诗。

		“如今我只想静静的,躺在一个人的身边。

		任天上流云的影子,千年如一日的漂过我们的脸……”

		千年如一日。我们在茫茫时空中,奔赴着各自生涯,相遇在此处。哪一段感情的纠缠,不是三生三
	世的轮转?

		我愿意相信,我们的前生今世。

		我愿意,静静躺在一个人的身边,任云的影子流过,任青草蔓生,淹没我们的肉身。我想,我是千
	年前桃树下的女子,灼灼其华,被路过的你望见又深记。人是多么微小,却又因为微小,引着你我动情
	,落下了眼泪。


		在这里,我自言自语,思维凌乱。\par
		夏天,一年年,用相同的表情对人间微笑。有雷电,有风雨,有艳阳。\par
		路过的蜻蜓,停在谁的发髻,又兀自飞走。每个人,都曾这样飞行,用脆薄的双翅。

	\endwriting


	\writing{潦草}{2006年07月15日 ~ 10:27:16} %<<<2

		这个多雨的夏天,总是在半夜被雷电惊醒。于是便怅怅地醒在黑暗中,看闪电的光焰照亮四壁。

		世界沉静,只有雷声浮在空中,宿命的压迫一样,沉沉落入黑夜的未知。

		时间,在这样的时刻显得无力而缓慢,像雨后墙壁上的蜗牛,向前移动。多少个夜晚,我们能够这
	样静处于独自安静的角落,在风雨飘摇中,无视于动荡的威胁。这仿佛人生的隐喻,多少个日子,我们
	能够用善良和平静,安放好心魂的躁动不安。

		人总需要这样一些与天地对峙的时刻。你要询问她,一切不可解的奥妙。你要等候花开叶落的许多
	季节,看岁月如梦,繁花锦簇,又在转瞬里,凋芜腐朽。人存在于微小的呼吸之间,投身在自然不变的
	轮回中,如夏花,如秋叶。所有的安排,该是早已写定,只等一个纯白的孩子,一天天长大,把命运一
	字一句地读出。

		我愿意雨打湿我的窗口。我愿意在雨声的掩盖下,辨认夜晚里火车的呼啸。长长的铁轨,从这高楼
	的不远伸展向北方更广漠的原野。

		火车在雷雨中穿越,车上的人是睡着,还是如我一般,清醒在深夜?

		火车兀自奔驰,带着机械世界的莽撞和力量,它显得倔强,不顾一切。旅客们,被带去更远的北方
	,他们将经过飞沙走石的山岗,穿行漫长的隧道,他们将在一处处陌生的小站停留,下车吸一支烟,看
	月台边的大杨树,在夏天瓦蓝的日光下,绿影婆娑。

		我想象着那一切,美妙的,或略带忧伤的情景。想一个远行者路途上的寂寞与悲怆,想他背负着年
	轻的激动,举起手中的相机,拍摄一路的匆忙风景。会有一个吹口琴的姑娘,站在古城的废墟上,以安
	静的姿态,闯入他的镜头,以及记忆。

		我在车轮与铁轨的撞击声里,将思绪引向无限。

		雷雨渐渐停歇,一团凉风推开我的窗帘,充满了小小的房间。我把身上的单子重新盖好。

		因为有火车从家的不远处经过,我总带着旅客一样的情怀,来度过简单的生活。每个人,是这逆旅
	之上的小小过客。像所有的远行者一样,在命运的途上日夜跋涉。是这样充满风景的经过,是这样多精
	美的诱惑,在前方陌生的月台上,寂寞地等候。我想,这是多完满的安排。让我们的脚步孤独,却永远
	有所期待,有所希望,在不息的分秒流转之间。

		而时间好像一张巨大的网,把我们的爱恨捕捞。那些打捞上来的物品,有晶亮的玻璃瓶,有做工粗
	糙的塑料手镯,也有遗落的胶片,它们被一一晾晒在有阳光的院落中央,像一件件珍贵的藏品,闪闪发
	光。

		后来,我在玻璃瓶中插上淡紫色龙胆花,后来,我把手镯戴在细弱的右手,后来,胶片被装入名叫
	记忆的黑盒子,尘封保存,它再也洗不出一张,当时的画面。

		倘若忽略时间,世间许多的疑难也便迎刃而解。而时间,确实是这样一张,无所遗漏的网。网住我
	们全部的幸福,也网住我们一切不堪。

		这个多雨的夏天,滋润着北方干渴的土地。一夜夜滂沱,注满城外曲折的河流,在河床上孕育着青
	草和蘑菇。我有时醒着,有时昏睡,生命清澈,让我可以望见它最底层,那安放整齐的美丽石子。我好
	像一个赤脚的孩子,就踏在那些石子上,涉水而过,向着对岸缓慢移动。我没有火车的倔强,我是这样
	轻轻唱着一首被遗忘的歌曲,听着水花的绽放,没有喜悦,亦没有恐惧地走去。

		这所有,是安排好的情节,我认真地一字一句读出。

	\endwriting


	\writing{人世}{2006年07月20日 ~ 10:03:34} %<<<2

		打开电视,展开报纸,扭开收音机,我看到灾难,在这里,在那里,汹涌澎湃,吞噬着鲜活的生命
	。爆炸,海啸,战争,地震,台风,洪水,坠机…… 死伤人数,失踪人数,救援行动。这些字眼,一次
	次反复。一个充斥了哭泣的世界,表情肃穆地站立。


		我看到巨浪在一个下午拍向原本明媚的海滩,我看到流离失所的人群在废墟前疲惫忧伤的脸,我看
	到荷枪实弹的年轻人站在瓦蓝的天空下茫然若失。一切不可预期的灾难,在暗中谋划。人们一无所知地
	继续着生活,而毁灭,就在下一刻等待着。这是苦难的人世,因这人世的苦难,人需要圣主,需要上帝
	,需要佛陀。


		让我们在心底,期许着拯救,哪怕是一叶小小的船。

		佛说,摩诃般若波罗蜜,大智慧到彼岸。此岸,是著境生灭起,如有波浪。彼岸,是离境无生灭,
	如水常流通。这一端,是人间祸患悲喜,浪险风疾,那一端,是无念无执着,万里澄静。

		惠能说,人我是须弥,邪心是海水,烦恼是波浪。这风浪中的苦痛,任着命运起落,任着肉身沉浮
	,纠缠着我们无处逃遁的困境。此岸,彼岸,原是一步之遥,却咫尺天涯。逃不出人世万念,逃不出爱
	恨眷恋。拯救我们的,只会是我们自己,而不是佛的慈悲。


		前几天,和小蓝一起看《超人归来》。两个人都昏昏欲睡。

		超人,一个无所不能的英雄,仿佛这个世界的救星。他可以把失控的飞机拖起,安放在体育场上,
	他可以举起一块陆地,掷向宇宙。他来自另外的星球,于是,他拥有着超越人的力量。

		是谁,在最初做着这样的梦,成就了超人的诞生?因为渴望对于一切灾难的控制,这世界有了超人
	的故事,有了他披着斗篷飞来飞去的影像。越是对力量的渴望,越把人的无力显露无遗。即使在科学发
	达的今日,面对多数灾难,人还是只能无可奈何地眼见它们的发生。而所谓科学的发达,也不过我们对
	于宇宙的那些太有限,太微不足道的认识罢了。

		当我们面对一个永无穷尽的星空,便会知道,这世界不过一粒浮游的尘埃,而我们自己,是微小到
	接近于不存在。


		哭泣没有作用,悲伤没有作用。这世界没有救世主。人世的苦难,由人类一个个身躯,来经受,来
	承担。也许,这一刻,是风浪,下一刻,是洪水。不曾安宁的世界,从来都是不可预期的危险。时间与
	空间之无限,把我们抛向无限未知的迷惘。于是,在这样的时候,当我还可以安然静坐着,唯有在心中
	充满了感恩。

		因为,正如诗人说的,你也许只是想旅行,却终于不得不在终点下车。

		当我们有闲暇,有精神,在旅途中欣赏,就不要错失每一场风景的光顾。


		有人问,如果明天是世界末日,今天你准备做什么。有人问,如果生命只剩下一天,你怎样度过。

		而这些,本都是无从假设的。但如果有如果,我会在末日来临前躺在我爱的人的身边,我们将安详
	地睡去,不多说一句,没有恐惧,没有留恋,等候着毁灭从我们甜美的梦中涌入。我会在生命的最后一
	天,为自己造好一座坟墓,在周围种满花朵,在墓碑上抄写心爱的诗句,然后,对微笑着向朋友们挥手
	道别。人世不容假设,这些时刻或许会来临,却永远是在我们毫无准备下发生。于是,我的从容不迫,
	我的所有设想,也都归于空洞。类似的问题,全不过是毫无意义的提问,毫无意义的回答。


		棋罢不知人换世。打柴人,看一局童子的对弈,斧头的木柄便已朽烂,家中的亲故早已去世。神仙
	的一局棋,便成人间几世轮转。我在这个满是灾难的人世,度过着白驹过隙的岁月。人的时间匆忙,每
	一刻都是珍贵。眼前所有,却终究是才握时有,一撒手无的虚妄。而这仅有的,已足够我们用尽全力,
	去认真活过。


		愿生者珍爱生命。

		愿逝者安息长眠。

	\endwriting


	\writing{亲爱,别为我忧伤}{2006年08月03日 ~ 21:07:09} %<<<2

		隐忍住疼痛,我咬紧嘴唇,在黑夜里向下沉去。汗水湿透的床单,紧贴住狼狈的身体。剧烈的头痛
	,令人意识模糊。而也是在这意识里,我清晰感觉到,你蹲在我的床前,双臂抱拢着俯在我的身边。你
	的疲惫与担心,被我感知着,我的心轻轻地疼,轻轻地碎。我说,去睡一会吧,我没事了。你却依然在
	那里,一动不动。

		我渐渐在疼痛里睡着了。有依稀的梦,浑浑沌沌地朝我挪来。是几小时前,我在深夜因头痛惊醒后
	,是你焦急去找护士和医生的情景。是你让我靠在你肩膀,为我细心把被子围好。是孤单单站在女厕所
	门外等待的你,那伶仃的你。后来,我仿佛听到抽泣,那是我,在病痛里对于你的歉疚。怎么忍心,让
	你担心,让你整夜地守候在床旁。我开始怨恨自己的身体。如果,我从没有过什么病。如果,我能够和
	其他健康的女孩子一样。我不断假设。不断否定。不断失望。于是,在醒来的时候,我说,对不起。田
	拥有的太少了。田所能奉献与给与的,无以报偿你的爱情。

		这是我的疼痛。比身体的疼痛,更无法抵御的疼痛。

		早上,你打来热水,让我可以坐在病床上擦脸和刷牙。你的神情,像一位父亲,照顾她生病的孩子
	。中午,你摆好小桌子,掰开发糕夹上菜,一口口喂给因为打吊针而不能动手的我吃。吃完,你又去刷
	碗,好把饭盒及时送回配餐处。下午,我睡午觉,你就坐在床旁看书。怕我被吵休息不好,你又买来耳
	塞。晚上,母亲第一次把我托付给别人照顾,她信任了你。你冲好充气床,做好陪床的准备。而就是这
	一夜,我突然在半夜因头疼折磨得无法入眠。

		你说,最怕看我难受。

		在做导管检查的那天,我哭了很久,独自对着天花板发呆,饭也吃不下。那天,你没有在医院,你
	去办培训的准备。可是你的心,一直悬浮着,同我一起。第二天早上,7点多,你已经站在了我的面前
	。你一天心神不宁。你6点便从城北赶来医院。坐在床上的我,一脸绝望和狼狈。呆望着你,说不出的
	感动和矛盾。不愿你,看到这样的我。蓬头垢面,面部浮肿。我想自己,永远是那个美丽的女孩子,爱
	穿裙子的女孩子,对你调皮撒娇的女孩子。我要用大眼睛望着你看,看到你慌乱不知所措地笑。但是现
	在,你看到的,是这样不堪的我,被疾病折磨得不成样子。连自己都厌弃的一个自己。

		我变丑了。我怕见你。你说不丑,你说,田最美,没人能比。

		你捧住我的脸。你吻我的眉角,我的额头。我想哭。

		你说要送花来,你知道我喜欢花。但是医院宣传板上说,花粉对病人呼吸可能有影响。我于是有些
	失望。但是,那天你走进病房的时候,手里举着好大一朵花,微笑的太阳花。你递到我手上,我开心地
	笑了。一朵布绒的玩具花,舒展着枝叶,被插在我的床头。你送的健康云的小玻璃瓶,装着你的字,被
	小心放在柜子上。你带来的,珍藏的童年故事书,我一本本地读。还有那个长颈鹿的小木偶,你骗我说
	你会魔法,它才会动,终于被我识破机关。你说,我的一切都还是孩子一样的。孩子的睡衣,孩子的拖
	鞋,孩子的心。我说,在你面前,我永远不要长大,这样就可以一直耍赖下去。

		刚入院的时候,你还没有回北京。在傍晚,我总是一个人面对医院古老的建筑,看那些燕子在低空
	纷飞。生命,如一场狂欢。那些燕子的飞舞,总把我引向无发克制的悲伤。也或许,本没有悲伤,一切
	是我独自的幻觉。一个女孩,在古老的医院病房中,守住黄昏的窗口,等候着奇迹与转机。所有的思绪
	,都关乎命运。沉重,在越发深暗的天色里,如一口吞噬希望的井,彻骨的冰凉。恐惧,侵袭入我小小
	的,病了的心脏,如一浪浪潮水的无情。

		亲爱,要我怎么说。要我怎样,面对一切发生的,和即将发生的。我不该有怨恨与不平。我知道你
	在那里,肩膀和手掌。

		田的时光,如焰火,如光电。你知道她曾多么美。你睡了,又是疲惫的一天吧。你走出了学校,新
	的生活正在挑战。别为我忧伤,没有什么比你的忧伤,更令我疼痛。我会好好的,去坚强。

		生命中,我们都接到不同的剧本。有的平淡,有的浓烈,有的是笑,有的是泪。不管怎样,我总要
	演好,直至落幕。

		能与你同台,是我的幸福。我们一定要微笑。

	\endwriting


	\writing{短记}{2006年08月04日 ~ 14:41:34} %<<<2

		我在八月。我在清醒与迷失的临界处徘徊。

		夏天很深了。

		夜风把露水涂在草尖,作为季节的馈赠,在晨早的浓白中闪烁。

		蜻蜓,飞过我的窗口,飞过你的梦境,停在花丛,记忆起许多个被雨水打湿的日期,氤氲淡绿的气
	息,扑面而来。

		我想到一个遥远的孩子,撑着红伞,走过灰蒙的小街。

		那是多少个八月中,平常无奇的一天。暑假里的合唱队活动,每个孩子都领到新的歌谱。他们坐在
	音乐教室高高的椅子上,挺直了腰板,把嘴巴张得大大的。外边,断断续续地下着雨。是一个同样多雨
	的年份。

		校园里的几棵小槐树,在前夜的暴雨里,竟被刮倒。休息时,孩子们举起雨伞涌出教室,去看那安
	静躺着的树。它的枝叶在一片灰暗里,是如此鲜美明亮的绿。原本平整的操场,也塌陷下一大块。

		大家很兴奋。小时候,我们总是那么容易快乐。小小的变故,足以令枯燥的合唱队活动变得充满乐
	趣。卷发的音乐老师,把孩子们召回教室,于是歌声继续。

		雨声沙沙间,童年稚嫩的嗓音,混合了时间的魔药,渗透入泥土,在后来的日子,长成此刻纠缠的
	藤蔓。

		我不知道,我将在哪里停步。

		那个遥远的孩子,是拖着雨鞋的足音,一路跟随着时间,从一个夏天,走进又一个夏天。而所有的
	经过,又不容挽留地被封锁,被销毁,如泪水,在枕上的浅浅湿迹,经不过阳光下的风干。

		我们的夏天,总像是幻觉,在树的顶端,随着光影的浮动,流转变换。谁会在另一处长长的小街背
	后,想起你的八月。谁会在有星星的晚上,捡拾起光阴的碎片,拼成我们年轻的模样。

		我会在这里,在醒着,睡了的一个个昼夜,整理散失的气味和声音。经管,去年的一句喟叹,已成
	镜花水月中的一笔流水账。

		我想,我将是飞过你窗口的蜻蜓,飞过你,最真实,也最虚无的梦境。

		那里会开放着,洁白的苹果花,淡淡的甜味,落进我们的时光。

		却那么慌张,那么匆忙。

	\endwriting


	\writing{莲}{2006年08月05日 ~ 16:46:26} %<<<2

		午睡醒来,夏风穿堂而过,吹响风铃,叮叮当当的一阵清脆。坐在桌前,一粒粒剥开碧绿的莲子,
	细细咀嚼。

		洁白的身子,包裹着苦味的莲心。

		莲子,在唇齿间留下难以名状的滋味,丝丝扣扣,渗入无言的午后,散化在安静的房间,由我独自
	体尝。

		郊外的一处荷塘,每年的夏天,总要造访。那是移居北方的一家南方人,他们在郊外租下小小的一
	块田产,盖起几间简陋的矮房,在夏天种植荷花,出卖莲蓬和鲜藕。

		男主人是精瘦黝黑的小伙,赤膊坐在路旁,将荷花插在大的塑胶桶中,与莲蓬和鲜藕并列着一字排
	开。在他不远的身后,是风荷的舞蹈起落。昨夜的露水,安睡在翠碧的荷叶上,点点的晶亮。他拿了莲
	子要我们品尝,又解说着红莲子与白莲子口味的差别。他并不知道,每年的夏天,父母总会来这里看他
	家的荷花,再买回去几节鲜藕,几捧莲子。

		母亲拉着我的手,走在荷塘中央的田垄上。其实市区中有许多公园都栽植荷花,规模也不小。钟情
	着这样一块朴素的荷塘,只因爱它的不着修饰。在西山之下,在都市的边缘上,纯粹的泥土里开放出的
	花朵,保持着自然的天真。

		这山野中的荷花,不收取门票,不巧取名目迎合人们的喜恶,它们只自顾自地奉献着生命的能量,
	兀自开放,凋芜,产出莲子和鲜藕。

		种荷人的小屋前,晾晒着洗净的被单和衣裤,在平静的生活里招展,喜悦而满足的神情。

		回家的路上,对母亲说,以后我们也住到郊外去吧。她笑了,好呀。

		我们的车子,经过雨水过后草木疯长的田野,经过涨满水的小河,又回到熟悉的市区。然后,人群
	挤满了视线,然后,生活退入原有的轨迹。在高楼上,在被切割的天空下,远离着泥土和植物。

		这样的时候,总难免觉得,人仿佛被自然抛弃了,囚禁在隔绝的孤岛之上,盲目而不知所向地生存
	着。度过那么许多,不属于自己的日月。

		而现实,是我们注定在不断去接受。种荷人的快乐,在我们的眼中,却并不一定真实存在于他们的
	生活。子非鱼的辩驳,是不休的未知。

		幸福,从来只是我们的主观感受。没有一把尺,可以度量它的长短,没有一只秤,能够称量它的轻
	重。正如时间,是我们永远无从触摸与计算的秘密。

		许多时候,感觉幸福总在别人身上。好像站在田垄上,想象那一家人的生活。但真相,是我不可能
	了解的。

		与痛苦一般,幸福同样无法做到感同身受。我们只有去经历去体验,所有细小的,幸福感觉。

		幸福是主动的事。

		被疾病困顿在这漫长的苦夏。

		思绪慢慢沉淀,落入生命中最脆弱,也最坚强的底层。有一些力量,在那里滋生。这过程,仿佛一
	颗莲子,在泥土中的觉醒与挣扎。

		洁白的身子,包裹着苦味的莲心。

		哪一次黑暗的穿越,不是奇伟与壮丽,不是落着眼泪,浇灌了希望的萌芽。

	\endwriting


	\writing{无端}{2006年08月07日 ~ 20:34:19} %<<<2

		岁月静好,时间如常,变更着季节与人世,淡定如流年偷换,总在明月窥人,暗香浮动的夜晚。

		当我们睡着,当凡尘的身体,在夜色的掩盖下,隐没了形迹,当呼吸和脉搏,随子夜的秒针,指向
	平缓的节律,生命在宇宙中得到回归。

		那一处无知无觉,无妄想,亦无恐怖的世界,如此到来,在有梦的时候,在无梦的时候,永恒的,
	散碎无依的混沌。


		那是真切的生命吗。睡梦与死亡。它们如此相似。于是,身体是一种累赘。在长久的时空里,没有
	什么比身体的存在更显虚假脆弱。

		庄子说,人之生,气之聚也。这赖以存放灵魂的容器,原不过如云,如雾。我们的来处,同盛夏中
	的一场滂沱,本无差异。身体,在精气聚集的几十年时间中,收容了我们的灵魂,让它得以安身,有所
	知觉和思想。而睡梦与死亡,是我们的灵魂回归万象自然的时刻。

		我想,每一次睡眠,同死亡本无差别,是一次重生和穿越。只是,睡眠的那一端是新的日月,死亡
	的那一端,却无从知晓。或许,是彼岸花树,枝繁叶茂。或许,是雷电交加,暗无天日。那都是活人的
	想象。

		唯一可信的可能是,我们将回去夜夜相伴的那个无知无觉,无妄想,亦无恐怖的世界。在我们存在
	于这世上之前,在我们离开这里以后,所有的事情都与痛苦或幸福毫无关联。


		于是,在星光很好的晚上,我们的生命也如一颗小小的星了,在宇宙的茫然若失中,发着光热,度
	过着无来由的时间。

		呼吸是一种幻觉,视线是一种幻觉。我们是从哪里,被什么声音唤醒了,就无端端躺在了温柔的襁
	褓。我们是被谁召唤着,在母亲的身体中,萌发了生的可能,如一粒种子悄悄发芽。母亲造就了你的身
	体,你盛放灵魂的容器。而灵魂,又从何而来呢。一切的无解答,在我们意识到存在的时刻开始,便成
	为永久的叩问。人是这样不容置疑地降生了,有了呼吸,和视线,懂得了欢乐和悲伤,微笑和哭泣。然
	后,痛苦与幸福同我们产生了关系,将我们折磨或滋润。


		可以触摸到的就是真实吗。阳光无法触摸,而它的热,它的光,成全了生命的所有可能。如果没有
	光,我们从来不会去思想,不会从一个简单的细胞分裂成现在的身体。那么,睡梦与死亡也不复存在。
	世界是没有过的,宇宙是没有过的。即使时空依旧存在,但因为不曾有人的意识,而无所谓存在。原来
	,存在的,从来都是偶然。这人间是一种偶然,我们每个人是一种偶然,所有的命运,也无非一种偶然
	。这偶然,让一切的发生有了宿命的意味,你不得不承受,这样的,那样的,美好的,灾难的,种种偶
	然。生命也是因为有所承受,而获得了重量和意义。即使命运带给我们的是不幸不公,也不能够成为你
	怨恨它的理由。存在的偶然性,注定了我们不能够奢求只去获得满意的答案。


		我知道,人生中总有一些路途,是要在黑暗中蛰伏与穿行。这些过程,是我无可回避的残忍。那些
	如夜晚,不见五指的时间,总如汪洋之上的暴风,没有一盏灯塔能够真正指引向安全,风浪怒涛,要航
	行中的船员独自面对艰险。在睡眠里,在生死的思索里,生命是手心轻握的热气。这温度,在一个个夜
	晚与天地对话。在沉沉的睡梦中,我望见血液,从心脏流出,如洪流迸涌,向肢体的各处,那景象,好
	像岁月与时间的流转,平和安详,不露声色。我以为,我们的身体,毛发是森林,血液是河流,呼吸是
	风雨。


		宇宙承载了天地,天地承载了万象,万象承载了容器,容器承载了灵魂。我们无端端地来,终于也
	将无端端地离去。一切的悲伤和幸福,在这里生长。那些不舍,那些奢望,那些恐怖,只因知觉,只因
	妄想。

	\endwriting


	\writing{说}{2006年08月09日 ~ 15:32:28} %<<<2

		六月的操场上,我们看远天的云朵,那些膨胀的,流浪的棉花糖。\par
		我告诉你天空的秘密:是我用一整个夜晚,把它们绘在青蓝的画布上。\par
		蔷薇,在微风的缝隙间,侧过绯红的脸颊。我的睫毛滤过七彩日光,闪闪烁烁。

		谁在花枝上安放好季节。不发一言地,冷眼旁观。

		我在入夏的枕上辗转难眠。挂好白色纱帐,像第一次爬上这高床的那天一样,疲惫而安稳地平卧,
	闭上眼睛。

		想象玉枕纱橱的温婉清凉,读去年的日记,恍如隔世。有什么办法,可以停止情绪的流动。多数的
	时候,我的生活漂浮在半空。

		像独自嗫嚅,独自取悦的诗。散落在现实不堪的地面,却以华美决绝的姿态。那是一个任性又脆弱
	的孩子。喜欢照镜子,喜欢看自己瞳孔中尚存的清澈。

		\blankrev
		拒绝悲伤的田。总是难免悲伤。

		愿意幸福的田。发现幸福邪恶。
		\blankrev

		然后日子从脚下的影子里逃跑。全世界的风,向一个方向吹去,浩浩荡荡。让夏天快些离开吧。我
	在祈祷。夏天,总是我的受难季。

		在青蓝的天空下,发呆的人会遇见寂寞与恐惧。

		吃药,清水送服。你说,要相信好的可能。
		不要你们担心。

		我有一点勇敢,有一点坚强。

	\endwriting


	\writing{此处。陌生}{2006年08月15日 ~ 22:14:17} %<<<2

		阳光照进房间,新鲜的光线穿透窗纱,在清早的洁白里,蔓延成清醒的白日。昼与夜,被如此精巧
	分割,相持在分明的两端。分与秒,在6点30分的位置越过,如无数次重复的越过。

		我平卧在安静中。

		昨夜的梦,残存在虚弱的知觉,散发不明缘由的喜悦。那大约是一个有微笑的梦,只是,匆忙醒来
	的我,已追忆不起任何情节或场景,哪怕,浮光掠影的细节。只任心,悄悄蓬松着,溢出轻轻的快乐,
	像遥远的海上,正在生长的云朵。那些遥远的地点,海洋,原野,高山,总在无止境的想象中,将我们
	的生活带向漂浮,带向高空。

		我喜欢这样,一个人,躲在房间,不被什么人知道,不被目光遇见,独自飞向,所有只能够虚构的
	境况。

		它们总是美的,光明的,却也最令人心碎。

		而我,却从来不是一个能够甘受寂寞的人。我愿意看尘世的人。愿意他们经过我的窗口,微笑着,
	或者是忧戚的面孔。我热爱人群的生动,那让我知道,生命是多么真实的一个事件。虽然,更多的时刻
	里,我恐惧他们的喧嚣,和浮躁。我是作不得隐士的。大约能够深居简出,却终于需要能够推开门,走
	出去,便遇上一整个鲜活热络的人间。

		于是,推开门,走出去。通常,总是空阔的楼道,一个人按下下行电梯的按钮,然后等候数字一层
	层上升,再下降。今天,却迎面遇到对门的房客。我从没有遇到过那扇门后边的人与生活。

		一个高挑的,双腿修长的女孩,二十几岁的模样。她利索地锁上房门,整了整提包,转身向电梯间
	走去,一阵嘀嘀嗒嗒,鞋跟撞击地面的声响。她不曾看我一眼。我们站在一起等候电梯。数字上下,楼
	道依旧空阔,安静。

		原来,在我的隔壁,居住的是这样一位女子。这好像对于惯常平淡生活的新发现一般。在墙的那一
	边,每一天,我们经历着相同的时间和年月,却不曾意识对方的存在。

		我搬到这里5年,却是第一次遇到对面的房客。这个城市多么奇妙,让空间的距离毫无意义,一切
	人,在房门的背后,都成为陌生,赤裸裸的陌生。电梯停下来,两个人一起走进去,并肩站立着,等候
	到达地面,然后各走各的路。

		早已熟悉的街道与小区,恍然显得陌生。

		来来往往,原来也是陌生。我不知道,什么人与我比邻而居,不知道,我存在与何处,我陷落在人
	群的什么地方,不知去向。这个尘世,竟处处是慌张,处处是陌生。

		是从什么时候开始,我们开始进入这样一种生活,又漠然地屈从了这样一种生活,毫无防备,毫无
	反抗?或许,因为我们根本无力防备与反抗。城市,真正的城市,要求我们,用如此的方式生存。

		那天,和良说起童年。我说,那是我回忆中最好的时光。

		以至于,我的回忆也出现失真的印象 —— 那时的天空是淡粉红的,空气里总飘着棉花糖的甜味。

		祖母站在院门口大槐树的树荫里,穿着月白的衬衫,清瘦的脸孔上,是慈爱的笑挤成条条皱纹。我
	们住在院子里,祖孙三代人,在夏天的傍晚,喜欢围桌吃一碗麻酱面。切得很细的黄瓜丝,是满口青翠
	的滋味。

		邻居是住了十几年的老街坊,和父亲一起长大的叔叔大爷,常常提了几瓶啤酒,来喝上几杯。

		晚饭后,孩子们迫不及待地从家跑出来,成群结伙地去不远处的试验田里捞蝌蚪,捉蜻蜓。那是夏
	天里,最大的期待和快乐。我感激,农科院保留下的这一小块试验田,让我得以在儿时的记忆中,残存
	少许关于泥土的气味。

		现在的孩子们,还会在一起捞蝌蚪么。

		看到过一位父亲带着5,6岁的孩子,在学校来园的小池塘里捞蝌蚪。他用小小的网子怯懦地伸向水
	中。他快乐吗。旁观的我,没有感受到,我曾经体会过的近乎疯狂的喜悦。因为雨水的关系,今年的蜻
	蜓多了许多。我没有见到孩子们捉蜻蜓。这两项游戏,都是残害益虫的行为。也许,是现在的孩子懂事
	了。但我仍旧怀念,在天擦黑的傍晚,轻着手脚,去捏住一对蜻蜓翅膀,那一刻的颤动。

		对于蜻蜓,我怀着歉意与感激。见满天的蜻蜓飞舞,总要出神凝望许久,那些脆弱的生灵,丰实着
	我们脆弱的生命。

		此处,是彻头彻尾的陌生。城市,让我们把房门关好,不去理会其他,只过我们自己的日子。这状
	态貌似隐居。一个无人相识的地点,陌生人的存在不过冷眼旁观,与不存在并无许多差别。于是,我原
	来也是可以作得隐士的人。我仿佛也愿意是旁观者,仅仅是旁观者。我只是想看看人群。而不是与什么
	人交谈,与邻居相识,与陌生人交换眼神。这城,把我们锁入更小的城,并安心做守城的兵将,一丝不
	苟。

		然后,我开始享用生活的这些改变,并不费吹灰。

		最熟悉的,往往最陌生。正如,也许最安静的,偏偏是喧嚣。

		我把自己推向了虚构。我把时间与昼夜,安放在小小的房间。从这里的窗口望出去,几十年,不过
	如此。分秒总是越过那些相同的位置。最后无法越过的,只是我们盯住分秒的目光。

		好多时候,我感觉自己漂浮在这城的上空。我们都是旁观者。这变迁,这浮躁,这人间种种。

		海上生长的云朵,从我们的头顶飞过。一切感受,都将在那个瞬间,轻若无物。

	\endwriting


	\writing{别后}{2006年08月16日 ~ 19:29} %<<<2

		我知道。我将投入广漠的世界 \par
		没有声息。没有痕迹 \par
		我将一个人。面临海角穷涯。地老天荒 \par
		眼见。几场绚烂与哀婉 \par
		在雨之前。或雨之后。吞噬我的记忆 \par
		我知道。空白的时间。是你 \par
		我知道。窗口的沉默。是声音 \par
		我将独自。守候着许多个夜晚的无言。由安静中落下 \par
		像整个宇宙的尘埃。在瞬间里。覆盖我。年轻的身体 \par
		你会想起。或者忽略。这些无关爱恋。无关思念的过往 \par
		你的岁月。是流水账 \par
		我的生涯。是童话 \par
		灯光剪下你的轮廓。绘画在颓败的白墙 \par
		我知道。我将在陌生的奔波里 \par
		我知道。你将听到。夏日的叹息

	\endwriting


	\writing{荒诞抑或真幻}{2006年08月17日 ~ 19:47:17} %<<<2

		电影《楚门的世界》,讲述了荒诞的故事:主人公Truman Burbank被电视网络公司收养,生活在幸
	福安宁的小镇上。而社区原只是一个巨大的摄影棚,他的朋友、邻居,甚至是妻子都不过是演员而已。
	从小开始,他的一切行踪便被隐藏的摄像机拍了下来,并成为了一部受到全球观众喜爱的电视剧集。直
	到他三十岁,才开始有所怀疑,并最终识破了真相。但当他决定独自乘船离开这一切时,航程的尽头,
	竟然是作为布景的一面水泥墙。

		这结局令人感觉残忍。介于童话和寓言之间的故事,敲打着我们的思考。

		一场真人秀,虚构的主人公怎么看,都像我们每个人自己。虽然,在现实里,我们的生活中没有隐
	藏的摄像机,一举一动没有被全球的人当作电视剧来欣赏或消遣。而有谁又不是在自己的舞台,自己的
	银幕,上演着各自的剧集呢。

		一幕幕悲喜,一幕幕离合,所有的幸福,和悲伤。不曾有剧本,我们却都是努力的演员。

		有时,在别人的舞台上出现,作为主演或配角,停留一段时间。有时,不过是别人镜头下,匆忙消
	失的背景,一闪而过。这些,都是我们无可预知和把握的。遇到什么人,发生些什么故事,什么时候进
	入,再在什么时候退出。

		自然而然的剧情,好像事先虚构好的情节,一一展现,并不必去操心一般。

		没有谁不是 Truman,没有人不是无知觉地行走在一个虚构的荒诞世界。

		也许,人间本便是一场善意的虚构。是上帝的玩笑,是众神的游戏。天空是被绘画的,山川是被捏
	造的,人群是被施了魔法的。

		于是,什么是真实,什么是虚假,界限变得模糊。人的意识是可信的吗。我们能够感知的,就是真
	实存在的吗。

		我们的一生,我们的全部记忆,是否都有确实的意义。\par
		芸芸众生,在这悄悄旋转的地球之上,随它在茫然未知的穹宇中航行,度过漫长的无限时间。\par
		这样的想法,令人感觉存在的虚假。好像,所有的发生,都不过是最逼真的幻觉。\par
		我们睁开眼,它就出现,就展开,我们闭上眼,它就停止,就消散。\par
		人原本是生活在这样的荒诞之中。我们仿佛永远不可能看清,自己的确切位置。

	\endwriting


	\writing{她}{2006年08月20日 ~ 19:40} %<<<2

		听陶喆的歌,《她的歌》。简单的旋律,幽深华丽,在背景若隐若现的竹笛声中,泄漏出忧伤。

		总有一些声音,与我们的脆弱相遇,落下不可收拾的一地碎屑。好像,我们的一切经过,好像,所
	有被时间偷走的知觉。


		终于是坐在这里的自己,看着窗口呼呼吹进8月的风,想象秋天。

		天空很好,似曾相识的模样,青蓝色的辽远。写下文字,来作为生活的证据。


		你说我是纸娃娃。眼神里是心疼和珍爱。

		有时,希望自己是歌声中的“她”。

		被你唱起。

		也许,这也是我潜意识中,喜欢这首歌的原因。

		\longpoem{}{}{}
		想起她 \\
		还在等她说的那句话 \\
		忽然发现青春有白发 \\
		等待像微笑蒙那丽莎

		看着她 \\
		像开在悬崖边那朵花 \\
		回忆在一步之间挣扎 \\
		爱情让人忘了害怕

		不知吹到何年何月那阵风 \\
		不知忍到何年何月那种痛

		在我眼中 \\
		春夏秋冬在那一刻已变成了永恒 \\
		荒芜的心不要别人懂 \\
		她是我不想醒来的梦
		\endlongpoem

	\endwriting


	\writing{粘土}{2006年08月22日 ~ 19:13:33} %<<<2


		【插图】

		在网络上遇到这样一组图,两块相爱的粘土。

		难免引人悲伤的对白。关于离开,关于遗忘,和想念。这总是一些近乎于宿命的事件。

		传说里,女娲用泥捏出五官七窍,双手双脚,于是有了人。圣经里,耶和华用尘土造人,向鼻孔中
	吹入生气,于是有了亚当。

		归根结底,人是来源于土地。

		我们平凡的生,是天地间自然的繁衍。如一株花草,一尾游鱼的单纯简单。从土地中来,往土地中
	去,画我们生命的圆圈,由始至终,或清醒,或沉迷的线条。

		相爱的粘土。是多巧妙的隐喻。


		什么也没有留下。无论幸福,与悲伤。

		而所有的所有,谁能够轻易地遗忘。

	\endwriting


	\writing{夏末纪念}{2006年08月23日 ~ 20:21:31} %<<<2

		二零零六年,八月二十三日,节气处暑。处者,去也。\par
		终于,可以向炎炎苦夏道别。风里已有了秋的意味,而终究不是秋天。\par
		夏天的尾巴,托住长长的一个季节,依旧在正午的日影下故意顽抗。\par
		我看到,我们洁白的时光在楼下的树丛间婆娑。天空透出湖光的碧蓝,云朵的神情,是浅浅的。

		在这一天,和莫见面。一个纤弱的女孩子,向我一路奔跑。还是老样子,白皙的面孔,不加掩饰的
	单纯模样。

		一起去超市,每人一只购物篮,散漫悠闲,俨然已为人妇的闺中密友。这想法,让我觉得可笑。而
	这情景,又不禁令我联想,许多年以后的我们。当我们告别了年少,当此刻的单纯浸染了年华的沧桑,
	是不是还有你,有我,挎着胳膊,在偌大的超市里停留或离去。

		那时,我们的话题,还将是文学,和电影吗,还是男人,孩子,和生活呢。我无从知道。

		想起和小鹿在东门外的小店吃米线的下午。玻璃窗外,三五个超市发的员工围桌而坐,都是体态开
	始衰老的中年妇女。她们大口咀嚼,笑容满溢地高谈阔论,也不时撇撇嘴,摇摇头。我们却听不到任何
	谈话的内容。

		小鹿说,有时,她会害怕,有一天自己变成那个样子,但也许有些事是由不得此刻的我们担心的。

		每个老太太,都曾是小姑娘。每个被生活麻木的妇女,也都有过她爱做梦的少女时代。

		时间的力量,把人们推向不可预测的境况。另外的自己,在那里等待,不发一言地看你一步步走近
	。而这,大约也并不是可怕的事件。

		只要,后来的那个自己,是你对于生命的选择,只要,后来的生活,是你愿意接受并全心享用的。

		衰老,将是残忍却甜美的,如果,能够被心爱的人用一生见证。

		莫不知道,在我们散漫的步子里,我藏了这样多心思。或许莫,也在想着同样的事情,而没有多说
	一句。我喜欢类似的沉默。让我们隐秘起一些话语,在默契中。

		夏末。去年的今日,我拍下雨后的操场,拍下自己的新鞋子。良背对着我站立,白色的上衣,蓝色
	的挎包。那个瘦高的背影,印记在夏天陌生却温存的角落,被我永久地封存。

		在日记本上留下记号,于是,我能够记得,我们相遇的日期。

		从那一点上,许多的故事发生了改变,许多的记忆有了另外的可能。我们似乎走了很远的路,我们
	的旅途,从那里开始,延伸向远处。

		一年后,我们喜欢坐在一起回忆,喜欢说起小小的细节,不厌其烦。\par
		良,总是安静温和地对待,总是笑。手掌里,是明亮亮的,希望和暖。

		那天,在被推倒的11楼前默立许久。然后,日期从我们的耳侧飞逝。\par
		在这个夏末,我们的大小零碎,都搬进了新建成的11楼。\par
		站在空荡无人的走廊,只听到自己清晰的足音,一声声,仿佛分秒的越过。\par
		尽头的窗口,照进日光,把长长的走廊,拉得更长,更长。\par
		你会感觉恐惧吗。生活的改变,总是如此,恍如隔世般猝不及防。

		这一刻,夜晚把深蓝的空气注满我的房间。\par
		突然很想念学校里的长椅,想一个人坐在那里,等待一颗不期而遇的流星。像去年冬天一样。\par
		只是,这一次,我会虔诚许下最单纯甜美的愿望。

		而那,是不可以告诉你的秘密。

	\endwriting


	\writing{田在想}{2006年08月26日 ~ 21:11:38} %<<<2

		\subpart{一。}

		晴空无云的夏天。我们的回忆在长草。\par
		这些,那些,细枝末节。一幕幕忧伤的侧脸背影,一处处微笑的眼角眉梢。\par
		我在时空的淡彩里,画满了生命的茁壮 —— 呼吸,脉搏,心跳。\par
		如果,经过的花丛,开放着8月最温存的眼神。\par
		如果,离别的雨,在哭泣的夜晚过后,还有所眷恋。\par
		那么,请留我独自,守候着掌中的时间,数我们单薄的日期,一日一月。\par
		让我沉定,如一滴泪的安静。

		\subpart{二。}

		雨后的城市,是水墨的灰。唯有,树的容貌翠绿。\par
		光阴,从我们的额头上踩过,像晨早照进的阳光。凉风打窗。\par
		秋天。母亲读出它的名字。那音节,也是轻而纯明。\par
		我们善于捕捉季节的变更。我们记得,去年的梧桐树,在自习室的窗口。\par
		校园里的布谷鸟还在傍晚唱歌吗。\par
		我的日记写下:它啼着快乐,它啼着心伤?

		\subpart{三。}

		电影院里,你靠在座椅上睡着了。银幕的光影晃动,映你低垂的眼,微启的唇。\par
		身旁,是观众的欢笑。你的睡眠却安详寂静。不忍心唤醒你。\par
		后来的你说,电影院里太冷,便睡过去。\par
		原来寒冷,是催眠的。\par
		在生命的寒冷中,我们是不是也可以选择一段,旁若无人的小憩。\par
		纵使周遭,是众人为一场闹剧的欢笑。

		\subpart{四。}

		我们在幸福的边缘徘徊很久了。这孤独的,没有止境的深渊。

		丛生的欲望,是悬崖畔的花。\par
		也许,在下一刻未停留的分秒上。\par
		也许,在上一点错失的知觉中。\par
		全部的可能,都碾化成尘。来不及抚摸的,总是我们最内核的躁动不安。

		还有谁,愿意在梦外听花落的铿然,望河汉的星辉斗转。

		夏末的蟋蟀,在庭院演奏。我们在幸福的边缘,葬身黑夜。

	\endwriting


	\writing{睡前。醒后}{2006年08月30日 ~ 18:12:24} %<<<2

		有时候,会喜欢这样的夏天。灰白的天光,低垂着雨意的下午,微风吹起窗口等待风干的花睡裙。
	我总是躺在床上,蜷缩着身子,像一只猫那样,散漫地孤独着,却又仿佛自鸣得意。一本书,一句话,
	一个字,已足以消磨假期最末尾的几寸时光。在耳边放着范晓萱轻哼的《外婆的睡前吟唱》,碎如泡沫
	的声音,拨开清醒的外衣,赤脚踏在我空无一人的沙滩。好几次,我闭了眼睛,看到自己,穿着花睡裙
	在沙地上奔跑,欢笑,然后重重跌倒下去,耳畔的吟唱还在,起起落落,冰凉的潮水一样,打湿又浸没
	了我的身体。我喜欢这样,肆意于幻觉的滋味。外婆,却是多么陌生的词汇。

		我没有外婆,在我出生前她便去世,五十几岁的年纪上。对于外婆,我的全部了解只是断断续续的
	听说,以及一厢情愿的想象。没有一张照片,没有任何具体的,能够感知的线索。有人说,她的容貌与
	母亲相似。母亲是她的最小的女儿,唯一的女儿。我于是问起母亲关于外婆的事。母亲的回答总是很简
	短:她脾气不好,她爱吃鸡蛋,她在一个冬天去世,死前要了一杯水。其他的,我统统不得而知。母亲
	不愿回忆那些关于外婆的往事。我知道,很多年她都生活在年幼便失去母亲的悲戚中。也许,直到今天
	,我的追问仍然是一种莽撞的残忍。

		范晓萱的外婆会在睡前为她唱着这样安详而悠远的歌么。我想,那一定是她真实而遥远的记忆。或
	许,有许多个午后或夜晚,外婆轻摇着蒲扇,坐在她的小女孩床边,哼着的正是这样的调子,飘在房间
	,院落,和深不可测的时光。很多年,歌声稀落,星光稀落。外婆蹒跚了康健的步履,看她的小女孩一
	天天长高,笑容娟丽,如花绽放。她会记起自己的青春吧,她会记起女儿的青春吧。那一刻,全部的温
	柔,涌起在淡去又淡去的生命中。外婆,与外孙女,是只属于女子世界的情感,未掺杂入任何男人。她
	是她女儿的女儿。那种情感,大概是复杂而微妙的美好。

		如果,我有外婆,她也一定会哼着这样的歌,在我小小的床边。我想看着她一日日老去,我想让她
	看着我一天天长大。但这样的事情,是不可以妄想的。在阴阳的两侧,我们被分割,来不及见上一面。
	也许,这是种遗憾,也许,这是种完美。我只可以去想象,所以一切的境况都能够近乎纯粹地动人。

		睡前为什么要唱歌呢,无论吟唱,或是摇篮曲。或许,在人声的反复低回中,我们更容易进入甜美
	的梦境。这一生,有谁会在你的床前,用自己的声音,催化你的睡眠。外婆,祖母,母亲,还有,你的
	爱人吗。他们都是能够给予你最安稳,最无私保护的人。当耳畔有隐隐约约的歌声,当他们抚摸你的额
	头,吻你的脸颊,轻轻说,宝贝,睡吧,美梦便已为你启开了大门。你要乖乖地闭眼睛,你要带着浅浅
	的笑意来感恩生命的赐予。夜晚会很好,月光的会洁白地滑落在你的脸庞,亲人的发梢。有些记忆,就
	在这里被偷走了,也有些时间,就在你睡去的时刻里,被你的爱人悄悄记住。很多年后,你在他的日记
	里得知,自己的睡相是多么可爱,而你自己从未见过。

		当你睡熟,谁会为你掩好踢开的被子。

		半夜,母亲总会来我的房间。好多次,我是醒着的,却装作了熟睡的模样。感觉她轻手轻脚地走进
	,低下身看我的脸孔,再把我踢开的被子重新盖好。偶尔,她会在床边再站一会。我不知道,那几十秒
	的安静中,我的母亲在想些什么,我只知道,她的目光落满了我的身体和呼吸,细而柔软。我却从没有
	为母亲掩过被子。而她,也会在半夜踢开被子吧。是谁告诉我,以后要找一个会为你掩被子的男人结婚
	,因为,那才是真正爱你的人。而这世界上,怎么可能存在一个,像母亲一样爱你,照料你的男人。女
	人,总是太多奢求,太多幻想的动物。

		那起起落落如潮水的歌声。那起起落落如潮水的生命。

		在夏天的末尾,我睡在了有蓝空和幸福的梦里。我梦见自己是一颗糖果,被包裹在华丽甜美的糖纸
	里。书掉落在地,汗水渗出在我稀疏柔软的发丝,没有盖被子,音乐还在播放,我的花睡裙在窗口的微
	风里高飞。没有人知道,一个女孩,微笑于梦境,飞行于梦境,在9层的房间。有时候,会喜欢这样的
	夏天。被打湿,被淹没,去狠狠忘却,再被狠狠忘却。

	\endwriting


	\writing{Lolita}{2006年08月31日 ~ 10:35} %<<<2

		一上午,躲在房里看《Lolita》,中文译名《一树梨花压海棠》。很早的时候就在同学那里看到这
	个电影,喜欢海报上那个穿白色衣服的女孩。她的眼神是清澈而飘忽的。被划入情色范畴的影片,描写
	的也是乱伦的恋童女情感。看过包装背面的介绍,我最终没有借那张碟片。想象中,该是一部很变态的
	电影。于是,与它的再次相遇,被推迟到这个夏天。电影的原著《洛丽塔》被上海译文高调出版推出,
	很厚的一本书,封面是鲜美的柠檬黄。书的手感很好。翻看几页,终于还是受不了外文小说翻译后的语
	言。洛丽塔的名字却深深记住。后来,才知道,此洛丽塔,便是那个Lolita。

		影片从开始就揭示了悲剧的结局。满脸是血的Humbert,驾驶黑色汽车,六神无主地在乡间公路上
	行驶。然后,回忆回到他年少深爱的女孩身上,那个14岁,死于伤寒的女孩。她的笑容,在海边的小木
	屋里绽放,洁白的蕾丝裤边,和女孩的青春一样甜美而美好。她的死亡,是全部灾难的开端。美好的事
	物,似乎总隐藏着巨大的破坏力和毁灭性。

		“洛丽塔,我生命之光,我欲念之火。我的罪恶,我的灵魂。洛丽塔。舌尖向上,分三步,从上颚
	往下轻轻落在牙齿上。洛,丽,塔。” 他说。当他在花园里遇到任衣衫被花洒淋湿而依然翘起双脚看着
	画报的洛丽塔的时候。他知道,他的光,他的火,他的罪恶与灵魂,全部在那个瞬间复活了。

		后来的情节,出乎意料却在情理之中。他被这个小妖女迷惑,无力自拔。所有的疼爱与妥协中,一
	个男人,用他的全部,来试图保有这份乱伦之爱。洛丽塔的娇媚,洛丽塔的孩子气的调皮,在画面上跳
	动跳动,呼之欲出。而终于,她要离开了。但就算在受过许多苦难后,她找到他却只是为了钱。当洛丽
	塔无情出走后,再次见面的他说,你一碰我,我就死了。面对怀孕的她,他心中的声音依旧是:我望着
	她,望完又望。一生一世,全心全意,我最爱的就是她,可以肯定,就像自己必死一样肯定。当日如花
	妖女,现在只剩下枯枝回乡,苍白,混俗,臃肿,腹中的骨肉是别人的。但我爱她,她可以褪色,可以
	萎谢,怎样都可以,但我只看她一眼,万般柔情,涌上心头。

		这话令人心碎。整部影片,我从没有感受到丝毫罪恶,没有对Humberl的憎恨和唾弃。虽然,在道
	德观念上看,他的行为令人很难接受。而实际上,他只是去爱了美好的事物而已,而已。

		就像看过《断背山》之后那种深沉而真实的感动一样。我在想的只是:每一份深刻的爱情都不是邪
	恶的,都不该被诅咒。

	\endwriting


	\writing{田在想。二}{2006年09月02日 ~ 20:58:53} %<<<2

		\subpart{五。}

		九月一日,回到两个月后的学校。

		看来园里,夏荷细长的茎,头顶硕大的花朵,在风中微微颔首。

		这小小的池塘,一汪泪眼似的水面,是北语里唯一可以藏匿温情或悲伤的处所。

		在阳光清亮的早晨,在开始落花的傍晚,我喜欢在短短的碎石路上踩过,喜欢坐在凳子上等候路灯
	亮起,像一个约定那么准时。

		记得冬天干涸的池塘,孩子们在里面玩雪。我们也兴致勃勃钻过低矮的桥洞,到达池塘的另一端。
	自以为是一场冒险。

		后来的春天,三个女孩子一起在新生的枝叶间拍照。开着幸福花朵的丁香,是淡紫色的梦幻,散落
	一地并无愁怨的影子。

		那天,春风吹乱了头发。照片上的我们,笑容天真得有了几分傻气。\par
		在这小园,目睹着季节流年,也目睹相遇和离去。\par
		大一的新生,从这桥上经过。大四的毕业生,在这池塘边合影告别。\par
		一年年,它铺展开草木的繁盛与凋芜,细数你我的青春,在晴空下欢笑,又在雨天泪眼蒙蒙。

		\subpart{六。}

		15楼的女生全部搬入11楼。生活的气息,开始在空阔的楼房中滋生蔓延。

		我从楼道中经过,有意无意地窥探着别人的日子:水池边有了洗手液,梳子,洗面奶,门前挂上卡
	通图案的布帘子,房间里开始充实,也开始凌乱。活动室里,两个陌生的女孩在看电视,一旁的阳光里
	晾晒着棉被。

		经过15楼,熟悉的入口,熟悉的候客厅,却再不能走近。男生从这里进进出出。

		想象着204B的新主人们。我并不得而知他们的一切,正如他们也不会知道,那些女孩子曾在房间里
	编制过的年华。

		在相同的空间里,经历着截然不同的生涯,却一样是炽烈而单纯的回忆。

		也许,我们全部的经过,便是这样的迎来送往,平常而安详。

		\subpart{七。}

		在我水蓝的窗帘背后,又加上一层白色的薄纱。然后,可以有梦的可能在这房间酝酿。\par
		曾经说,每个女人都有一个公主梦。关于城堡,关于王子与白马。\par
		而有智慧的人告诉我,骑白马的不一定是王子 —— 他也可能是唐僧。\par
		于是,公主的梦,更重要的是公主,而非王子。\par
		懂得爱自己的女人,永远是高贵,优雅,而不失天真。懂得去爱与被爱的女人,永远是善良,幸福,且心怀感恩。\par
		公主,该住在自己的自信与尊严中,而非世俗的目光,而非虚荣与浮夸。\par
		公主,该照亮生活,该有天使的微笑,带给人间美与欢笑。\par
		公主,该有沉定的思想,纯澈的安静。她随和,并不盲从,她独立,而不孤傲,她特别,却不乖张。\par
		王子不一定骑白马。王子有时也爱骑自行车。

		\subpart{八。}

		同父母去宜家家居。新的宜家,大了许多,展示的房间也更加丰富。\par
		在一间标写着55平方米的居室中驻足。不算宽敞的房间,一室一厅的格局,却布置得温馨而贴心。\par
		这大约是宜家的魅力所在。让你仿佛走进另一种生活的可能,并被深心吸引。\par
		你会想象,自己坐在那张绿色碎花的沙发上,在一天的劳碌后,依在爱人身旁,听他发发工作的牢骚。\par
		会想象,在小却紧凑的厨房,煮两碗面,在下了雪的冬天,一起面对面贪婪地吃完。\par
		这些生活的可能,在脑海浮现。让你感觉一种希望的快乐,隐隐的,在前方等候你的快乐。\par
		我们都是依靠着希望去生活的吧。\par
		哪怕,如汪洋之上一点缥缈的灯火。便可以足够勇敢,去穿越黑暗与遭受。

		这是固执着,只愿意幸福的田,坚定的信念。

	\endwriting


	\writing{秋的可能}{2006年09月04日 ~ 20:27} %<<<2

		碎碎的语言,愿意记录在季节的日记本上,比如阳光清亮的一天,比如大朵的云飞跃你我视线。

		他们说,记录是为了不忘却。而对于我,文字是一个人的言语,像坐在墙角的孩子,独自的寂寞游戏。

		是和田自己,走进生活的真相,或者幻觉。是田触摸到自己的无虚假的肌肤,安静于人群之外。

		这种滋味,从指尖,到脸颊,到唇齿,到发梢耳际。我知觉到,生活,鲜活活的生活。

		阳台上洒满的光芒,高大的树木,碧绿色起风的下午,秋天的预感。

		生命的喜悦,让我忍不住诉说。

		坐在超超的自行车后,随机播放的音乐,是顺子的,dear friend。我们在午后的林荫道上飞过。

		“跟夏天才告别 ~ 转眼满地落叶 ~ 远远的白云依旧无言”

		秋天,我等候着,那样的时刻,随着叶片的纷飞,当沉重的地球也落进寂寞。

		像里尔克的诗,在秋风席卷了田野后:

		\longpoem{}{}{}
		谁此时没有房子,就不必建造,\\
		谁此时孤独,就永远孤独,\\
		就醒来,读书,写长长的信,\\
		在林荫路上不停地 \\
		徘徊,落叶纷飞。
		\endlongpoem

		幸福的邪恶在这里,得到永恒的安放与宽恕。我们,得以用重新擦拭的心灵,来冷静整个夏天的浮
	躁与动荡。

		田总是在九月里,神情恍惚,一半明媚,一半忧伤。

		站在风口,莫名想念。

	\endwriting


	\writing{风歌}{2006年09月08日 ~ 16:15} %<<<2

		窗前的花朵开得正好,阳光里,淡橘红的瓣,紧紧簇成微笑的一团,吐露着9月里不张扬的喜悦。

		秋天,一路奔波,在房门外呼呼喘着粗气。那便是风,是午后无形的巨手,抚起洁白的窗纱,飘扬
	,飘扬。

		看季节的更迭,如此决绝轻易,不加留恋。路人穿上厚外套,瑟缩着走在开始落叶的长街。

		校园的梧桐树,等候着欢庆,他们的幽艳将在枝头开放了,用层层叠叠的色彩与光线,迷醉我们的
	视线。

		我爱那些树下的时光,无论白天或夜晚。

		我记得,和莫读着那一句,我喜欢你是寂静的,经过的悠悠下午;我记得,良背起我,一路嬉笑地
	走在自习后安静的归途;还有,同s站在风口里听ipod的冬天。

		都是在这一条长街,落着叶,也落着梦的长街。树总是在两旁默默站立,只有风,有心无心地哼着
	歌,深情而忧伤。那一种风声,夹杂着时间给我们的错觉,在亡失的疼痛里,酿造着记忆的甜美,如一
	只寂寞中飞行的蜜蜂。

		走出宿舍,看到紧紧拥抱的母女,脚边还放着大大小小的行李。是新生吧,我猜测着。母亲的脸是
	笑着的,用手反复抚摸女孩的肩膀。

		那画面,停顿在那里,也停顿在我心里。

		母亲,永远是我们心底最温柔的一块。这一生,有多少人,我们能给过他们这样持久的拥抱呢。在
	那拥抱的时间里,你又想些什么,是幸福,还是恐惧,是欢乐,还是悲伤。

		终于,我们还是要放开怀抱,投入独自的世界。没有人可以永远陪伴,没有人,从没有。

		有时,在深夜把自己的身体蜷缩成一团,想着,大约母亲的身体中自己便是这个姿势,将自己拥抱
	的姿势。让人不再孤独的,原来只有我们注定孤独的心灵。

		要盖厚被子了,要多穿衣服,要早睡觉…… 他的叮咛不厌其烦。我安静地听,一件件照做,全然一
	幅小女人的嘴脸与姿态。

		也许,能够这样安心于简单的日子,也是一种福气。

		所有的感情,都该心怀感激地享用。所有的生命,都该全力以赴地狂欢。

		秋天,落下清静的睡眠,在田的世界,沉定纯洁,如去年的飞雪。

	\endwriting


	\writing{葬}{2006年09月10日 ~ 11:12:09} %<<<2

		当冰凉萌生在指尖,当荒原的风,翻山越岭,吹响我们的绿树,一种萧瑟的安静开始下落,落满整
	个,曾被夏天盛装的世界。秋,如雪花,纷纷扬扬,秋,用透明的日光,碾碎记忆的昏暗,碧落的天,
	是一切旧的完结,与新的发端。

		一叶知秋。夏末里,第一枚落叶,飘在谁的视线,又兀自睡去。归于土地,在这恰好的时节,每一
	棵树该都心存喜悦。

		叶与树的告别,不过轻轻的舞姿,一段下落的弧线,于茫然的天地大荒。

		诗人们感动秋叶之静美,愿死如秋叶。我也欣赏,叶的淡定与决绝,不作分别的沉哀,不必痛彻的
	涕泣,让所有的美好,融入此刻光芒下的闪烁,好像一整个春夏的狂欢。

		我们是否也会有勇气,在生死的临界之上,舞一段弧线?带着对生命那虔诚而陶醉的微笑。

		看秋天,看我们的时光,从自行车的车轮间,悠悠飞离。这早上,灰云朵被一夜吹灭,多少美好的
	,忧伤的故事又在这小小人间上演。重复着,重复着年复一年的期望,度过,与回忆。

		似曾相识的光线与气味,去年,前年,或更早的时刻,那些已如雾的陈迹,那些路过的人,在秋的
	气息里清晰如昨。你们却已站在各自幸福的彼岸,那一端是火树银花,这一端是流年似水,滔滔而去。
	流澌的水声,是此去经年,无可挽回,无从追忆的从前,那一个你,那一些他们,有多么美,就多么残
	忍。

		林妹妹含泪葬下花朵,我们亦拾起许多花期的残片,一一收敛,入忘却的布袋,深埋土地之岑寂无
	言。这是秋天里的课业。学习着,淡定与决绝,学习着,埋葬与告别。

		诗经《唐风。葛生》,一位女子,她唱着:葛生蒙楚,蔹蔓于野。予美亡此,谁与独处!(葛的茎
	缠绕着荆条,白蔹蔓生荒郊。我的爱人死在这里,谁来将他陪伴!)……冬之夜,夏之日。百岁之后,归
	于其室!(冬天的夜长,夏天的日长,百年之后,我将归于他的坟场!)

		这一首怨旷的诗,于千年之前,被什么人写下,又反复歌吟。她是不是荆钗布裙,盈盈而立,她会
	不会,在漫长的夜晚,望那河汉两侧的明星,默默不语。远方的土地,葬下爱人的尸体,不归的征人,
	倒下在茫茫的旅途。

		你牵挂着他的马,他的冬衣,还有,他的墓葬。没有人陪伴的爱人,他将孤独地沉睡,在异乡的土
	壤,任荆棘蔓生,毒花遍野。你垂了泪,你说要葬在他身旁,陪伴那一个个寒冷的日月。

		而今,千年之后,他们该是沉睡于一处了吧,尸骨也已经化入天地,滋养着万物,或许,有一棵芽
	,穿越他们的身躯,长成树,开着花。单纯的远古之音,单纯的远古之爱,连丧失,也如此芬芳动人。

		当人心被欲望与诱惑打扰,我们如何固执地守住一份情感,不染尘埃。让时间和历史,埋葬所有遥
	远的爱恋。让现在的我们,甘心于现实的人间,而不去贪念他们的与子偕老,生死相随。

		只是,在相爱的时光,一定要温柔地相待。

		\longpoem{}{}{}

		葬我在荷花池内,\\
		耳边有水蚓拖声,\\
		在绿荷叶的灯上 \\
		萤火虫时暗时明……

		—— 朱湘《葬我》
		\endlongpoem


		所有将死的,该死的美好或痛楚。都应化着秋风,葬入荷池深处。随着秋雨零落,伴着残荷的凄恻
	,想念一场,夏夜的萤火。

		朱湘的家信,总是绵细悠长,称呼是亲亲霓爱,霓妹亲爱,落款为,永久是你的亲爱,永远是你的
	恩爱丈夫,沅。而永久与永远究竟有多么久远?当诗人投江自殁,一切的爱是否也随魂魄归入不息江流
	。霓君没有怨尤抛下她与三个孩子独自赴死的丈夫,有传言,她将儿女送人,出家为尼。而更可信的说
	法是,她的后半生,以绣花缝补为生计维持生活。

		那些往昔的是是非非,那些从前的别离恩爱,会在她的心上印记着幸福吧,或者伤痕。而所有的所
	有,终于也合并葬入岁月的更迭。葬我,葬你,葬一切之丧失与获得,无一幸免。在凋败的荷塘,请风
	雨洗净沉淀后的情绪,请生命归还人间循环往复的爱与悲伤,唯留一池碎玉般的雨声,零零落落,又深
	坠入万象的未知无解。

		在秋天,等候着,风吹红山峦的翠碧,等候着,爱人把围巾缠绕在你颈上。

		一些如热水般,温存体贴的时刻,纠缠在季节里,透明闪亮的细节,在手心的温度中被记忆和感谢
	。我葬下,一段段往日,也埋入一粒粒希望。会有一棵芽,穿越我们的身躯,长成树,开着花。那是你
	我,新的生涯,又一场追随与奔赴。

		我将是站在树下,目光如蜜,细数落花的孩子。

	\endwriting


	\writing{青菜生活}{2006年09月12日 ~ 19:44} %<<<2

		和小鹿一起吃晚饭。清淡的煮青菜,翠碧的叶子半浮载清水中,模样素雅可爱,于是有点不忍下筷
	。

		想起第一次见到还种在地里的菜花。丛丛的绿叶子,紧围住一朵洁白的果实。那情形,让你觉得,
	菜花的确该叫菜花,而不是花菜。

		它全然是一幅花朵的模样,被保护着,在叶的怀抱里,等候着阳光和雨露。

		菜花,是个幸福的小姑娘。

		于是,面对眼前的青菜,我们谈论起它们的种种美好来。

		有时候,会感叹造物主的巨手,充满了力量,创造了山川天地,竟又如此精巧,在每一片叶子上都
	绘画了细密的图案。那些叶脉,那些流畅的线条,哪一件不是精美的艺术呢。

		当青菜还在生长,它们是一件礼物,采摘的人,把礼物打开,满怀着收获的喜悦。这是多么好的安
	排。

		农业也许是人类最基本,最原始,却也最富于诗意的活动。诗的趣味,不该是浓到化不开,而是恰
	如清水,淡似无味,却在舌头低下发现了甘甜。吃青菜,有同样的妙处。

		小鹿说,她发现一盘青菜若做得好,要比肉类好吃许多。

		滋味太丰富,太刺激,反而令我们的味觉麻木。老子早就教诲我们说,五色令人目盲,五音令人耳
	聋,五味令人口爽。感官的过度享受,往往打不到愉悦的目的,却走向反方向。

		吃青菜,最好也是这样用清汤烹煮。不但保持了青菜的形态和色泽,也更大程度维护了原有的滋味
	。

		只有善于体验的舌头,才能尝出它们的美味。青菜教我们去细细体味。

		这也是一种生活的方式。放慢生活的脚步,放松紧张的身心,面对青青翠翠的蔬菜,面对淡定安静
	的自己,用牙齿咀嚼,更用心灵咀嚼。青菜生活,不是小女子的矫情。是平凡的诗味,是日常的爱。爱
	我们的亲人,爱我们的朋友,爱一棵花草,一株树木,爱空气的清洁,爱凉风的舒畅,也爱我们盘中的
	食物。如果,我们是这样去爱,那么,哪一处世界不是光明,哪一种苦难,不会被救赎。心灵将被安放
	在平静与满足中。我们也会如一棵菜花,被幸福保护在温暖的怀抱。

		后来,我们终于把可爱的煮青菜吃个精光。带着饱足的肚子,快乐地离开了食堂。青菜,正在变成
	我们的身体,我们的血液和肌肤。想一想,就又是惊奇,又是感恩。

	\endwriting


	\writing{田在想。三}{2006年09月14日 ~ 08:59:54} %<<<2

		\subpart{九。}

		九月,就是这样,可以睡得很安稳的九月。蜷缩在厚棉被里,拥抱自己的身体,像一只猫。\par
		用如此温暖的姿态,编织几场属于秋天的梦吧。有关童话,有关孩子们透明的心。\par
		欢笑就会在沉默的唇角,绽放出一朵粉红的小花。\par
		想起许多年前的游乐园,父母,叔叔,还有戴红帽子的姐姐。那也是一个秋天。\par
		巨大的摩天轮,依旧满载着人们的幸福幻想,从时光的远端,缓缓旋转。\par
		爬行的蜗牛车,起落的木马,飞驰的小火车,不停歇地制造甜美的回忆,给孩子,也给记忆,给一切稍纵即逝的脆弱。\par
		那天,我哭着要姐姐的红帽子,她于是摘下它,戴在我的头上。我喜欢那顶帽子,因为小红帽的故事。\par
		我们留影纪念,在城堡前的小广场。那里,有白雪公主和七个小矮人的塑像。\par
		而今,照片已经泛出疲惫的黄色。但上边戴红帽子的小女孩,还是一脸单纯的坏笑。\par
		她正为霸占了帽子而得意。

		\subpart{十。}

		来园的椅子上,睡着一个男孩。小鹿,兔儿,还有田,扒着玻璃窗向外看。\par
		他用衣服蒙住半张脸。他从中午开始就睡在那里,直到下午。楼管阿姨说,他天天在那椅子上睡,是亚欧的。\par
		是学生呀。三个人有点惊讶。他为什么要睡在那里呢。\par
		也许,他失恋了。曾经他和恋人常常坐那把椅子。\par
		也许,他在等人。对方忘记了约定,他却一日日执着地等待。\par
		也许,他遇到了困难。在这里躲得清静。\par
		也许……\par
		谁会知道呢。只是,总觉得他是悲伤的。\par
		兔儿说,若把椅子旁边的树纳入视线,也算很美好的画面。\par
		夏末的树影,午后的日光,沉睡的男孩。可惜的是,他的模样……\par
		但或许,会有个女孩子轻轻走到他身边,拍拍他的肩膀,然后,他新的爱情就此开始了。\par
		突然发现,这有点像睡美人的故事。\par
		三个人不禁失笑。

		\subpart{十一。}

		在图书馆五层自习。第一次坐新安装的电梯。原先总是要气喘吁吁爬上去。\par
		喜欢五层的光线,充足,明亮。找一处靠窗的位子,默坐读书。\par
		看乐府吴声中有语,“风清觉时凉,明月天色高。”此刻的窗外,天空是没有情绪的灰白。\par
		梧桐还没有被浸润在秋色的妆容。世界却不动声色,偷去我们的时光。\par
		有飞机的轰鸣声,从不知远近的云层传来。于是,站在天台上看天的冬天又回来了。\par
		我们谈论远方的海,我们说,要在海滩上建造一座小木屋。\par
		那时,北方的海正飘雪。多么想见到下雪的海洋。那会是怎样的静穆和安详呀。\par
		要和你拉着手,把我们的足迹印在雪白的梦里,一个个,伸向无限的远。\par
		要看着雪花,悄悄地绽开,又融化,在你的眼前,我的眼前,我们的全世界。\par
		然后的这个秋天,田说,带我去海上吧。看秋天的海,它的温柔妩媚。

		\subpart{十二。}

		经过水果摊,新鲜的苹果被装在三个纸箱里,并列安放着。诱人的红,青涩的淡绿,一样散发出不假掩饰的光彩和香味。\par
		苹果的淡香,不需用舌头去品尝,只是用鼻子,也足够满足渴望愉悦的身心。\par
		苹果的花朵,据说是洁白的五瓣。那也是一种可爱而羞涩的花吧。会在恰好的时节,谨慎地吐露芬芳。\par
		它的花语却是“陷阱”。难道是越美好,越邪恶么。\par
		喜欢苹果,这种明亮亮的果实,完满得好像无杂质的幸福。\par
		突然,想开一家水果店。每天早上,排列好可爱的水果,等待买家。\par
		也会有女学生经过,像我一样被一枚果实的美丽打动。\par
		她会闭上眼,闻一闻我的苹果,她会问我它们的价钱。\par
		我会送给她最美妙的祝福:愿你像苹果一样幸福……

	\endwriting


	\writing{瓶子}{2006年09月21日 ~ 14:08:23} %<<<2

		我蒙住时光的眼睛,让它猜猜,幸福还有多少未尽的热情。

		九月的中午,和小鹿喝一杯甜酒。

		桃子味的朗姆酒,在舌尖上跳跃着热热的甜蜜,这粉红色的魔药。

		杨树的影子,映在新擦的地板上,摇摇晃晃。没有云朵的天,适合站在阳台上,一个人痴痴地看一
	下午。

		干杯,然后看看我们的空杯和空瓶,相视而笑。她脸上有了红晕,田却依旧苍白着面孔。

		小鹿去洗脸,我于是坐在房间,想象她的一个个日子,从这里,从那里,纷繁如梦,或百无聊赖地
	流去。

		每个人的生活,在最具体的地方,在书架和床铺间,都显得慌张琐碎。空掉的酒瓶,和所有的容器
	一样,无言地张着嘴巴,有点无辜。

		我们留下酒瓶,还有瓶盖。我们说,要在瓶里插一枝小花,用瓶盖做成项链,送给兔子小姐。

		想起用爸爸的啤酒瓶接雨水的自己。那个童年里,远远的雨天,还在原地滚着灰云朵。站在窗后的
	孩子,等着雨水装满在院中央摆成一列的瓶子。雨水淋漓,下了不知多久。长长的一个夏季,伴随着青
	草的气味,把院中的童年定格在这一副,水彩一样的画面。瓶子中的雨水,终于漫出瓶口。时间的雨水
	,也终于漫过我的心房。孩子的游戏,在雨天的背景里,像一声清澈的呼喊。

		我还是站在窗后,我还是期待着,一种满足。原来,每个人都会变成那只啤酒瓶,从你望着它,到
	你自己站在雨雾里,被一寸寸注满。

		无须记录雨水,无须记得,那些散落的天真 —— 映在玻璃窗上,黑黑的大眼睛。我像一只瓶子那样
	,继续在雨中站立。有时,也期待漂流。像那些漂流瓶一样。

		我愿意,有个女孩,把年轻的心事,把幼稚的诗句,写成一封信,装在瓶中。我愿意,她把我抛入
	大海。让那些雪白的浪,拥抱着我,把我送去未知的彼岸。或许,是另外的大陆,或许,是神秘的岛屿
	。会有陌生人,在沙滩上发现这只瓶子,展开遥远的来信,坐在沙滩细细读着,露出微笑。那将是多么
	好的一天。风会吹走陌生人的草帽,飞向云朵,飞向远方,天空正蓝得刺眼。

		我是那个女孩,我是那个幸福的漂流瓶。

		田说,把我们的爱装进瓶子吧。深埋在土地,然后,等候一千年的岁月。像许多故事里传说的那样。

		当人们再发现它的时候,我们的爱,就可以进博物馆,被陈列在橱窗,标签上写着:远古人制造的
	瓶子。他们不知道,我们在里边装了什么。

		那些爱,如果没有死,会像精灵跑出神灯那样跑出来,继续生存和呼吸。田在幻想,这一场瓶子的
	时空旅行。

		喝一杯甜酒,然后心情就成为粉红色。

		可以迷糊着胡说八道。

	\endwriting


	\writing{秋分日}{2006年09月24日 ~ 16:12:06} %<<<2

		我想,这是恰好的时刻,安放我们自己,如安放一片熟睡的叶子。


		这一天,日夜平分。这一天,我望着窗口的天光亮起,开始安静的生活。

		在唱机上播放秋天的钢琴。房间被明亮的音符充满,属于原野的金黄色由远方赶来,伴随它们起舞
	,拂动我的白窗纱,告别着我的夏天,这长长的,色泽暗淡的夏天。

		没有更多的踟蹰和等候,季节飞跑向时光的深渊。

		谁还在回忆呢。那些被歌唱过的幸福或悲伤。被年少的我们丢弃吧,要学着无情和决绝。我总是坐
	在这里自言自语的孩子。在所有人的视线之外。写着时而甜美,时而感伤的句子。正如我这个人,偶尔
	的明媚,偶尔的忧郁。

		在心里囚着不可驯服的心魔,像只幼小却凶猛固执的小兽,把自己反复折磨。如何降服其心,永远
	是每个人不可避免的课业与遭受。

		安静的时刻,却终于到来。在秋天里,在秋天之后,我们有了借口和理由,安抚心的躁动,选一个
	阳光晴好的早晨,决定一种安静的生活。仿佛今日,此刻,当日夜被齐整地切分,有理由相信,我尚有
	气力来迎接和安排更从容的生命。

		怀着爱意和敬畏,看树木脱下繁华的衣裳,看天地冷却下来,在枝头奉献一枚枚果实。收获,在农
	人粗糙的手掌,收获,在老僧静静扫过的石阶,收获,在一叶知秋的目光深处。想在一处山谷,枕着松
	涛,伴着枯灯,读一卷经书。像千年前夜读的欧阳子,听那秋声在树间的纵纵铮铮,金铁皆鸣。这一夜
	,依旧是星月皎洁,明河在天。这一夜,我与山谷和秋虫静对,洁白的月亮,升起在东山,又落入桂花
	零落的芬芳。

		多少个簌簌的秋天,任西风吹过边塞,也吹过江南,多少次木叶萧萧,在湘夫人的湖上,也在凡人
	的梦里。听谁在岸上唱送别的离歌,见爱人的身影消失在烟树重重,山川变迁,日月轮转,唯有不息的
	流水,仿佛家人的温情,万里送行舟。属于秋天的故事,总是凉凉的,婉转着清寒的滋味。这是个告别
	的季节,适合在独自的夜晚,沉沉地,轻轻思念。从远到近,从现实到虚妄,我喜欢这一切的秋天。

		% <todo: 别字: 依着船舷 -> 倚着船舷 >
		在冬天之前,在河水冻结之前,让我们去水上,划着船。陪着我,折好一只只纸船,再把蜡烛点燃
	。看我们的船,荡漾在被夜色染成流彩的水面,看小小的光亮,漂流向远处,载着你我虔诚的愿望和祝
	福。仿佛是一点点希望,被点燃在暗黑的夜里。在秋天的夜晚,我倚着船舷,萌生着许多幸福的妄想。
	我听到人们的歌声,我见到邻船人燃放的焰火。没有星光,只有人间的快乐,在深暗的水上,随时光流
	去。你开船,于是我说,船长,我们向前去。清凉的风,从耳际飞过,马达响起了,掩住了流水的声音
	。又有多少时候,我们能够忽略它的声音呢。我们向前去,放下一只只载着蜡烛的纸船,身后,是起落
	的水波,一圈圈围起,再散去。

		你问我快乐吗。我已经没有理由,给一个否定的回答。

		想起春天的花树,落在你肩头的花瓣。想起丁香树下的留影,午后斑驳摇晃的树荫。我愿意坐在这
	里回忆,任季节飞跑过我的门前,一路嬉笑,或者悲歌。我不忍丢弃,我总是学不会无情和决绝。才痴
	心一片地,握住一支笔,不停地写下句子,时而甜美,时而感伤。这一天,日夜平分。我看着日光从你
	的指缝间熄灭,开始深情的生活。

		再见了,我的夏天。

	\endwriting


	\writing{田在想。四}{2006年09月29日 ~ 16:44:39} %<<<2

		有时候,只想这样,与你在寂静中相对,让我们的沉默如鱼。

		\subpart{十三。}

		昏昏沉沉的一夜,睡前的泪痕,还挂在疲惫的眼角,却又在明亮的白日里醒来。\par
		没有梦,空空的头脑,如雪天的清寒和洁白。\par
		我坐起来,看看眼前的世界,分明是一无所失的圆满。\par
		九月,只剩最后的两天。我总是这样数着日期,一步步小心踩下去,让深深浅浅的脚印,遍布记忆的原野。\par
		却还要哭泣,还要在风中抵抗悲伤。田终于是脆弱的孩子。幸福,在时时侵袭的情绪中,茁壮或凋残。\par
		我渐渐没有言语,没有气力,来回忆和诉说。\par
		我眼见着一场场相遇与离别,一幕幕哀乐和悲喜。只是莞尔。

		\blankrev
		这是我们注定的遭受。在年轻的时候,跌跌撞撞,又迷惘不知所措。\par
		混乱,有时好像一句抑郁在心底的诗,久久地沉着,沉着,却无法读出。\par
		于是,寂寞了钢笔和纸张,寂寞了青春和时光。\par
		是在没有尽头的路上飞跑一样的痴狂。\par
		大道朝天,每个人的希望,都那么明亮,却又显得可疑。\par
		有谁真的能够辨明自己的位置。\par
		我们总是不知身在何方。


		\subpart{十四。}

		看着他一天天忙碌。想起去年散漫悠闲的日子。\par
		当我们还都在学校,当我们还有机会一起去上自习。\par
		他总是一脸专注地正襟危坐。我总是心不在焉,看看窗外,照照镜子。\par
		总是洗好两只苹果,装在袋子,等着自习回来的路上一人一只,幸福地啃食,发出咯嗤咯嗤的声音。\par
		现在,我自己一个人啃他送的苹果。想念着他雪白的牙齿。

		\blankrev
		书上写,苹果可以保护心脏,常闻苹果的香味能够缓解情绪。\par
		于是,更加努力地吃苹果。原来,快乐是简单的事情。\par
		我只要洗一苹果。\par
		我只要安静地,一口口吃掉。一口口,是珍爱的心情与姿态。

		\blankrev
		现在的良是大人了。\par
		请纵容田,赖在孩子的世界不走。\par
		给我幸福的果实。


		\subpart{十五。}

		母亲将要飞去丽江。每一天兴奋地计划着行程。我只有眼馋的份,然后祝她旅途愉快。

		\blankrev
		彩云之南,那块神奇的土地,被多少次痴痴望想。
		曾经与兔子小姐相约,一起去看大理的茶花。我说,田一定要穿白裙子。
		兔子说,我们会在开放着茶花的庭院里,说一夜的话。月亮该是细细的,从窗格子,从门缝里漏进来。
		然而,这些美丽的约定,因为田的体弱而搁浅。真的如一只船,停在了浅浅的岸,无法动弹。
		远行,成为了奢望。我只可以在心底,想念着那不曾达到,却无数次梦见的远方。
		那一次,田说,让我们去草原吧,在夏天,让我们并肩吹着坚硬的风,把面目埋没在疯长的青草。
		兔子小姐笑了:"那我们就成了草原英雄小姐妹。"
		她是这样可爱的姑娘。
		现在的我们不会再想去草原。青草却时常在各自的世界里疯长,不着边际。


		\subpart{十六。}

		田觉得,一定要早起。

		\blankrev
		早上的时光珍贵。因为安静,因为平和。可以在最纯净的分秒里,播一张喜爱的cd,缓慢地听。
		可以伸几个长长的懒腰,再扭动几下身体。对自己问早安。
		然后,叠好被子,铺平床单。退后几步,满意地欣赏它们乖乖的模样,可爱的花色。
		煮一只蛋,烤两片面包,喝牛奶。想象麦子在田地里接受阳光的营养,想象奶牛在绿野的漫步。

		\blankrev
		生活的满足感,油然而生。早起,让一天有一个从容的开始。
		也能够多在镜子前留恋一阵。
		趁世界还没有喧闹起来,独自享用一小时的甜美安详。

		\blankrev
		田总是沉迷在生活里。

	\endwriting


	\writing{缓慢}{2006年09月29日 ~ 09:45} %<<<2

		让我们缓慢地相爱。

		就算生命,已来不及,让我成为你忽略的时间。

		等候着季风,来把年轻的身体风干。

		我没有许多,我不断丧失。

		所有的嘴唇,诉说着孤单,梦,秋天。

		在没有言语的时刻,你会想起,安静的结局。

		像是从未唱起的旋律,一首哀婉的悲歌。

		我会在那些音符的背面,融化在黑影的深渊。

		让我们缓慢地遗忘,如果幸福,是种奢求。

		请允许我,用沉默,记载徒劳的眷恋。

	\endwriting


	\writing{略想}{2006年10月02日 ~ 21:20} %<<<2

		月亮亮起来,挂上窗,挂上树尖。\par
		怅怅地望着。

		想起谁说,悲伤是一种无耻的矫情。\par
		没有办法。身体沉下去,沉下去,仿佛无止深渊。\par
		任情绪,无耻捏造着生活的虚妄。\par
		我是这样。有点慌乱,孤零零地,兀自言语。\par
		自说自话,自作自受。

		让我们去海上。在十月寂寞的浪声,淹没记忆。\par
		那些,无所谓真实,无所谓谎言的美丽。\par
		拾起白色的贝壳。测量我们精致的脚印。\par
		心中升起安详,也盈满荒凉。\par
		月亮亮起来,跳出海,跳出梦境。\par
		生命的泡沫,漂流远逝。\par
		巨大的幸福,压抑着巨大的悲伤。\par
		跟随我,脆弱而敏感的知觉。

		一个人,两个人,三个人,四个人。\par
		一句话,一只手掌,一场荒唐,一梦繁华。\par
		越多的拥有,越感空洞。\par
		视线深处,总是无处安放的灵魂。\par
		疲惫,是欢乐的挽歌,睡前的摇篮曲。

		怅怅地听着。\par
		凶猛的海。恋爱的人鱼。

	\endwriting


	\writing{最深}{2006年10月06日 ~ 20:35} %<<<2

		在生命最深的地方,藏着另一个自己。\par
		她哭泣,嚎叫,她微笑,颔首。\par
		温一壶酒,却又等候着热度散去,再独自一饮而尽。\par
		时而疯狂,时而温婉。\par
		我把她藏得很深,很深。好像一句不忍说出的誓言。\par
		如此矜持到顽固的隐藏。\par
		仿佛总是带着灰色的情调,把自己封固在碧色的湖心,一块翡翠似的湖。

		这个夜晚,秋风吹起。窗外雾散,落下雨水。\par
		远天,还响着闷雷。

		想写的故事,草草有了结局。却难以下笔。\par
		在虚构与真实之间,在想象与回忆之间,文字显得无力而苍白。\par
		也许,我们同幸福,总是差之毫厘,谬之千里。\par
		那些不曾发生的忧伤和快乐,因为遥远,而神圣纯洁,令我不敢触碰。\par
		如禅的,一开口,就是错。

		深深爱着,聂鲁达的那首诗。

		\longpoem{}{}{}
		我喜欢你是寂静的,仿佛你消失了一样。\\
		你从远处聆听我,我的声音却无法企及你。\\
		……
		\endlongpoem


		适合在秋天,用心反复默读这样的诗句。\par
		特别,是在下过雨的时候。当微冷的寒气,沾在了单薄的衣襟。\par
		我喜欢撑一把夏天的伞,一步步踱过清冷的街道。\par
		让时间忘记我。让季节忘记我。\par
		让思念忽略这一切,我汹涌,或者平和的情绪,如水如梦。\par
		当人即使在梦中,仍不知幸福的所在,那才是最深的悲伤。\par
		最寂静的,总是空白的睡梦。

		一路的荒野。我们万水千山。

	\endwriting


	\writing{深白色}{2006年10月08日 ~ 16:59:11} %<<<2

		\longpoem{}{}{}
		鱼在水里哭 \\
		我握着你的手说 \\
		鱼在水里哭 \\
		你笑着说别傻了 \\
		鱼并不会哭 \\
		它们是一种没有眼泪的动物


		树在雨里哭 \\
		我抬头看着你说 \\
		树在雨里哭 \\
		你温柔看着我说 \\
		树并不会哭 \\
		它们是没有思想情感的植物


		我突然的无助 \\
		没有眼泪的悲伤没有人清楚 \\
		只能呼吸着不被了解的孤独 \\
		一个人仅仅记得一切会结束

		我矛盾着无助 \\
		很需要你能给我一点点保护 \\
		想对你说的话却总说不出 \\
		我变成了植物


		没有人在哭 \\
		你摸着我的头说 \\
		没有人在哭 \\
		我在哭 ~ 只是没有人在乎

		——《鱼在水里哭》深白色二人组
		\endlongpoem


		深白色,不是七月里远天膨胀的云朵,不是一袭纱裙的高雅素默。\par
		深白色,是安静的独处时,均匀平和的呼吸。是爱人小憩的片刻里,侧脸的温柔轮廓。\par
		我享受,深白色的氤氲。富于魔力的气氛,轻着手脚的幸福与忧伤。\par
		深白,是灰。\par
		深白,是湿润的,明亮的灰。\par
		我偶然听到,这种歌声,在落下雨水的十月。她唱。\par
		鱼在水里哭。树在雨里哭。\par
		脚步停在小园的桥头,一霎那,我的耳膜脆弱。\par
		这个雨天,湿润明亮的灰,一样的深白色,一样的清澈。\par
		他们说,鱼的记忆只有七秒。于是,鱼没有眼泪。\par
		而不是每一棵树,都在佛前求了五百年。\par
		于是,树没有感情。除非,它开出注定飘零的花朵。\par
		空气,饱和了水分,浸泡着眼前这世界的浮躁或安详。\par
		深白色,为我唱一支歌。\par
		我没有哭。\par
		因为有你心疼。有你在乎。

	\endwriting


	\writing{粉红}{2006年10月11日 ~ 21:42:31} %<<<2

		我们的拥有,滋长了无止境的贪婪。

		这一天,我们喝粉红的朗姆,吃粉红的果冻。甜美中微苦的滋味,滑过舌尖敏感的触觉,在心头轻
	轻的灼烧。

		我迷恋这粉红的色泽,半透明的天真,像是生活最热情的诱惑。\par
		我总是经不起诱惑的人。对一切美好的事物缺乏基本的抵抗能力。

		于是,可以任由自己迷恋,迷惑,迷糊。不假思索地放任,所有不知来由,又不知去向的情绪。这
	是一种自然的选择,来自于生命内核处的决定。田,就是这样了。愿意在秋天做一枚明亮的果实,挂上
	树梢,愿意幸福地傻笑着,在安静的午后,偷偷把想象装满房间。让我每一天,不厌其烦地写下日记,
	在那本被雨水打湿过的本子上。用散碎,断续的句子,记录情绪的流动。这是我每一天重要的课业。因
	为无法拒绝,生命中太多的美好,我便只有用这样的方式去珍爱,去宝藏。

		每一个简单的日期,每一处留连过的风景。你还会记得么。你还会记得我么。

		我触摸不到昨天了。只有文字,那么清晰。才让回忆的灯盏,一粒粒亮在无边黑夜,像一艘小小的
	船,载着星辉的美梦,唱着哀歌,驶到我身边来。一切的一切,历历在目。我好像又可以走进去,看一
	看当时的你们,让阳光,照旧投进窗来,暖洋洋地照在我的发上。

		记忆,在这里,因为我的眷恋,而成为最甜美的疼痛。正是,那一种甜美中的淡淡苦味,我爱的粉红。

		据说,每个女孩子都有粉红情结。这是属于女孩子的颜色,是不愿醒来的童年,最直白的表达。所
	以,有那么多的女孩子为hellokitty而疯狂。

		我的粉红色,却总是一种明亮的存在,是舌间敏锐的知觉,是精致的梦境,被小心地编织。\par
		它是幸福,是瞬间里涌起的欣喜。

		在生命无可奈何的忧伤里,在命运安排的挫败和丧失中,只须紧握,这些粉红的片刻,只须依靠,
	这些由心底流出,简单的美好,便能够安然渡过。

		几年前的夜晚,在梦中,我与沉默的陌生人相对,坐在开满粉红蔷薇的庭院。\par
		他的面目模糊。我的心底充满了湖水一般的安宁。那个人是你么。\par
		蔷薇,在微风中侧过身子,低垂下花冠。我们就那样坐了很久。你被光芒笼罩着,温和地笑。\par
		我愿意,那是我们日后的相遇。亲爱的陌生人。\par
		能够在后来的时光中,拥抱我们险些错失的幸福。这是我,莫大的满足和快乐。

		如果,我们有未来的生活。如果,有一间属于两个人的庭院。让我们种满粉红的蔷薇。让我们相对
	而坐,在午后的柔光里,默默无言。

		请亲吻我的回忆和妄想。

		请任由我任性和耍赖。在时而真实,时而又虚妄的人间。让我紧握,所有散碎,不可复制,不可追
	悔的美好,与你分享。

		这一天,我们喝粉红的朗姆,吃粉红的果冻。你的侧脸温柔。\par
		我触摸不到昨天了。\par
		却触摸到此刻,真切的存在,这些,我爱的,粉红的时间。

	\endwriting


	\writing{秋凉}{2006年10月12日 ~ 21:12:00} %<<<2

		秋凉,孩子清澈的眼神。这里,是光线依稀的十月。

		气温下降至20度。北京,终于在燥热的季节后获得了冷静。\par
		阳光在被一夜西风洗涤过的蓝空里,呈现着迷人的透明。树影子,是淡定的灰,清浅的轮廓。

		这是多么静穆,多么安详的季节。

		让我们站在时间的点刻之上,望尽快乐,也望尽忧伤。轻手拾起一片落叶,在叶的脉络里,读懂神的暗示。\par
		走过来园,看澄碧的天色,倒映在小小的水涡。荷的风华已逝,残破的茎秆寥落着神情,稀疏地生长,如水墨中的枯笔。\par
		想起七月的清早,一个人经过盈满荷香的池塘,漫天飞舞的蜻蜓,留我在桥头凝眸久立。\par
		现在的我,好像站在很高的地方,来到那个画面前。那个自己,正站在那里,苍白的面孔,却漾着美满的笑。\par
		再过几个月,等到冬天,这小小的水塘,又将被抽干,定定地在渐渐凋芜的园中,像一双干涸的泪眼。\par
		我问你:还记得吗。去年的雪后,我们站在池塘里,从桥的这一边,经过桥洞,钻去另一边。你说,自然记得。\par
		我们像两个孩子,完成了一次历险那样,得到了愉快。

		而现在,距离冬天还有时间。池塘的两涡秋水,还得以映衬天光,引人遐思。\par
		可以看云,看风,看雨和雾气,站立在水的身边,我才懂得了心如止水的境界。\par
		不需要多说一句,完全是人与天地间充满默契的沟通。

		找一个下午,慢慢地收拾衣柜。把冬衣一件件翻出,让房间顿时间充满了樟脑和羊毛的气味。\par
		将他们在床上铺展,一件件细细抚摸。属于冬天的手感,毛茸茸的温暖。\par
		它们好像我身体的壳。每一处针脚里,都藏了许多个冬天的记忆。\par
		我们的生活,原来,正是这样一次次的脱落和收藏,正是如此安静平淡,却又深情的过程。

		秋凉,问候远方的朋友,提醒一句,记得加衣。\par
		秋凉,沏一杯热茶,坐在开始落叶的窗口,温热身体和双手。\par
		我在十月,想念许多,忘记许多。\par
		在丰美的时光里,感伤,是一件愚蠢的事情。\par
		这一切,只容贪婪地享用。

	\endwriting


	\writing{记录而已}{2006年10月17日 ~ 21:25:09} %<<<2

		在那些梦境中,我们仿若拥有一切,却又空虚无比。

		最近的天空总是冰凉的神色。却又不像冬天里的灰白。

		有时,它淋淋地落下雨水,让草木散发出泥土的气味,涨满开始萧瑟的空气。于是,课间里,我独
	自站在窗口发呆。看匆忙的行人,看对面的高楼,看秋天里,寂寞着,又喧哗着的世界。

		伸出手去,接不到一滴雨水,手指却已失去温热。这个十月,冷静淡定得令人心生恐惧。

		操场边的树,每一年总是最早开始落叶。起风的下午,裹紧薄薄的衬衫,经过它们的身旁。一地散
	碎的金黄色,热烈地追风而去。那场景,是悲壮的华美。

		超说,这是她第三年看北京的秋天了,记得,刚来的那一年,还拍了照片寄回去。转眼,时光就这
	样溜走,像调皮的孩子,蹦跳着在我们身后做着鬼脸。超的家乡,叶子总是来不及变黄,便被大风吹落
	。我不能想象,那是怎样残忍的一种情形。我总是要见它们展露出最后的灿烂,再悄然离开。

		北京的西风温柔,给了叶子们机会与时间。\par
		是秋了,我们贪恋着最后的温暖,同时又想念冬日的雪白。\par
		警告自己,别再在季节的转折处,矫情这些那些自作自受的风花雪月。

		小情调,小情绪。也许,这正是我真实的生活状态。好像,千百年来,许多拿起笔,就不甘寂寞的
	人一样。

		享受着那些,轻的,明亮的欢乐或忧伤,同时,却又感觉着不安和罪恶。

		而终于,我还是那个在文字里絮絮叨叨的孩子。重复着,这些平凡的生活,一步步的安静,一步步
	的热闹,和所有的人一样。

		因为是如此渺小,我才感觉到,生命的充盈和饱满。我知道,全部的拥有,都是美妙的安排和馈赠
	。我的小情调,小情绪,也许是一种矫情。而矫情,若是到达心灵安宁与富足的途径,又有什么错误呢
	。这是我安慰自己的方式。不要因为小,而否定了存在的意义。

		去年的小鹿说,有些事,只需在梦里发生过,便足够了。然后,我们讨论梦与现实的距离。\par
		现在的田在想,如果我们连梦中的渴望也丧失,如果我们无所梦想,无所期待,该是多大的悲哀。\par
		而每一日在睡前,若都有一个梦的期待,又是多美好的事。

		中学的时候,读到过一本介绍小法术的册子。里边有一种法术讲,反穿睡衣便能够梦见想梦到的人
	。我于是兴致勃勃地试验,把睡衣翻过来穿,然后满怀着希望睡去。那一晚,我真的梦到了希望梦见的
	人。现在回忆起来,是多么可爱的自己。可以轻信这样没来由的法术。因为是相信的,所以,它是真的
	。后来的我,因为心存了怀疑,法术也失去了效力。

		世间的许多事情,正是是因为我们的不相信,才成为了丑恶与虚假。那颗可以轻信的心,玲珑剔透
	,是一去不回的天真和幸福。这一晚,我决定在睡前怀着一个期待,反穿着睡衣睡去。

		也许,会遇见你,站在飘流的小岛,向我微笑。白色的飞鸟,掠过我的头顶,没有出声。\par
		原来,我们的生活,是在不断的流失中学会了懂得,和冷却。\par
		停下来的我,自言自语。记录而已。\par
		祝所有路过花田的朋友,秋天快乐。我想栽一些小雏菊。

	\endwriting


	\writing{家}{2006年10月19日 ~ 22:26:38} %<<<2

		让幸福,在我们的原野绽放。让我,在静默的年华里,苍老成你的记忆。

		家,一个令人无法不去依恋的地方。

		也许,不过不宽敞的房屋几间,也许,不过简陋平凡的一扇灯火,远远望见,却总是心生温暖。看
	那窗口的灯火摇曳,抚摸着熟悉的门板,闻到房里煲汤的香味,我知道家正等候着我的拥抱。

		于是,总是在掏出钥匙的时刻,会心微笑。喜欢钥匙扭开门锁的声音。喜欢归来的心情,一种饱满
	的归属感,伴着家特有的气味,扑面而来。

		在这个偌大的世界。有谁不是飘零的孤独者。我们被无端中抛到人间,遇见了今生的父母。他们微
	笑着抱起你,让阳光洒在你的脸上,等候着你的成长,看你一天天茁壮。于是,人仿佛一颗种子,在黑
	暗里获得了苏醒。于是,我们有了机会,来感受所有,微小的,巨大的,幸福或悲伤。

		因为有所知觉,我们便有所爱恋,有所牵挂。

		生命,是这样简单而玄妙地开始。在混沌中,我朦朦胧胧记得,那些最初的时刻。仿佛很安静,只
	有洁白的光芒,照进老房的窗口,只有母亲轻轻的呼吸,父亲起伏的心跳。我竟能够记得,这些细微的
	感受。也许,是记忆欺骗了我。也许,是幼小的我真的理解到他们抱起我时,那自然单纯的喜悦。

		这一切的背景,是家。是无比熟悉了,却又总是恍然间陌生的家。老房窗口的亮光,在无数的梦境
	里依旧闪闪烁烁。我总是梦见自己回去那里,院中还挤满淡粉红的月季。我的家,从那里开始,我的家
	,曾经是一座朴素却神奇的花园。

		如果,老房还在,柿子树该是果实累累的季节了。父亲会把它们摘下来,在窗台上摆成一排。它们
	诱人的橘红色,总引我忍不住用手去又摸又捏。“这个软了,能吃了吧?”我一脸馋相地问。柿子很甜,
	我总是吃得满身满脸。我很快乐,只是,那时的我并不知道什么是快乐。因为,还未曾经历悲伤。生活
	是明亮亮的,我并不惊奇它的美好,只任最可爱的时光,无声息地逝去。在幸福之中,我们总是难以察
	觉到它的存在。可能,这才是幸福的真相。当人高呼着,我很幸福,那多半是一种欺骗和表演。幸福,
	是不出声的,是不知情的。现在的我,开始羡慕那个吃柿子的孩子。她不懂得快乐,却拥有了一切。

		老房被推倒了。搬家的那天,我站在空荡荡的房间里,竟没有一丝留恋。

		十二岁,还是只去期待,而不知回首的年纪。我还不曾明白,这一次告别,便是永远的丢失。我收
	拾好最后的东西,转身离开了。老房的窗口,洁白的光芒依旧。只是,这房间空了,像个无底无目的深
	渊,直通向时间的幻觉。

		孩子长大了。老人离开了。我的家,我童年的花园,荒芜了,和我的记忆一起,蔓生出绮丽的花朵
	,占据那些散碎的片断,叠错弥漫。然后,我住进新家,新的房间,拥有新的窗口。我们把墙壁粉刷,
	擦拭地板,迎接新的生活。我满怀着激动,为了一切的崭新。在高楼之上,我度过着少女的时光。不紧
	不慢的日子,在家的四壁流淌。唱着欢乐的歌,画着明媚的图画,我很快乐,只是,那时我并没有学会
	懂得快乐的可贵。我挥霍着,所有跳跃着的青春。我没有将它们保藏在最宝贵的盒子里。却任由日期忘
	记了曾经的自己。

		我们总是无法把握,近在咫尺的拥有。在还来不及告别和失去的日子,我曾多么简单地经过着,最
	纯粹的青春。

		这里是家。这里有我的书架,我的床,我的衣柜。家,因为这些物质的存在,而显得实在而安全。
	它们令我感觉有所依靠。人,终于是无法脱离物质的包围和安慰。这时的我,平躺着,感受夜晚的宁静
	。没有声响,只有火车呼啸,从楼房的不远处驶过。我习惯了,现在的家,习惯了窗口半明半暗的光线
	。在窗台上养两盆花,每一天,看它们的苏醒和茁壮。我发觉,生命的相似性。于是,我能够感受到植
	物的呼吸,能够听到醒来的深夜里,它们鼻息的微声。陪伴我的生活,充实着家的温情。我感谢我的花
	,用尽力气,开放得如此诚恳而坦然。

		母亲在隔壁房间睡了,父亲还在客厅,等候着球赛。我躲在被里,读我的书,然后,缩起身子,迎
	接睡眠。这样的夜晚,让人感觉平静安心。而我们,又还有多少时间,拥有这样的平和安宁,守在父母
	的身边。时光,令我们懂得了悲伤。时光,把我们推向不归的未来,不容你回首。一回首,便是满心的
	疼痛。痛得你甜蜜而酸涩。他们老了。不是么。你开始为母亲染发了。

		想象着,我们的未来。同样是几间简单的房间,一窗摇曳的灯火。生活,从家为基点,一点点延伸
	向这貌似无涯的世界,却终于要回归到原点。这里是家。这里,是我们的归宿。没有人不是飘零的孤独
	者。唯有家,给你我以彻底的包容。让我细数着昨日,让我任性在快乐和幸福。我依赖着钥匙扭开门锁
	的声音。那一声之后,有父母的笑,有幸福,有明亮。我也曾等待着这一种声响,那之后,是你们的归
	来,是幸福,是明亮。

		我总是声称要远行,却终于是恋家的孩子。

		在这里,我们获得一切。

		在这里,我们拥有安宁。

	\endwriting


	\writing{田的独语}{2006年10月20日 ~ 17:37:51} %<<<2

		林间的悲壮独舞。为了我们的爱,和生命。

		田开始害怕照镜子。

		田受不了自己面孔的改变。爱漂亮的孩子,仿佛只在一夜之间,便被药物扭曲变形。像是受了魔咒
	,在镜子里发现一个陌生的自己。不能接受的丑陋。美丽,原来是最脆弱的衣裳。只那么几粒药片,就
	将她摧毁殆尽。

		田经受着这样的巨变,一次,再一次。让人们惊奇,她的脸,莫名地胖起来,又消瘦下去。或者,
	让你们看见,一个甜美清秀的女孩,突然间,臃肿而狼狈。这样的改变,发生着,我也在镜子里见证一
	切。周而复始的命定一样。田渐渐以为,这是自己生命体的一种周期。

		美与丑。幸福和悲伤。健康和疾病。

		我可以淡定么。我能够坦然么。

		瞒过了全世界。让他们以为,田有多么坚强,多么勇敢。而从容与平静之下,掩盖的,是波澜怒涛
	,起落无常。

		有谁,能够在无妄的灾祸里,无所哀怨,无所悲戚?

		田怕照镜子了。田甚至怕见人。你总是捧起我的脸,说田依旧是漂亮的。这些时候,只有哭泣的心
	情,冰凉凉的,被你的手掌融化掉。只有你懂得,田的脆弱无助。你说,你知道,田受了许多苦。

		我却总是愿意你们只见到微笑明媚的自己。无心隐瞒和欺骗。我也渴望着勇气。不愿意,总是病态
	而忧伤的形象。遭受的所有,并不该责怪。全部的安排,田乐于接受。只是,还是需要你的肩头。还是
	要轻轻靠着,任泪水满过视线。只看田的眼睛吧。她没有过改变,总是明亮黝黑。爱我的双眼吧。爱我
	的心灵。

		我脱下美丽的衣裳。用丑陋的姿态面对世界。

		也许,这才更加真实,更加坦白。

		我好像洗尽铅华,任一种赤裸的,不假修饰的生命,袒露于天地。

		很多时候,这是一种无法承受的疼痛。甚至,痛过疾病所带来的不幸。

		二十岁的年纪,总渴望着镜子里,花一样的自己。

		田却在这样的年纪里,经受着花开,又凋萎的周期。

		独自的时候,当我把身体浸泡在人间之外的寂静,偶尔,也能够忘却一切。

		悲喜之间,从未有鸿沟天堑。生命的简单欢乐,溶蚀着恐惧与难过。每一个日子,都是上天的礼物
	。只是,田的生活,多了许多难题和不解。

		什么时候,我可以真正去接受,属于自己的命运,而心无怨尤。什么时候,田可以不再为了容貌的
	改变去烦心忧愁,让心如明镜,波澜不惊。

		当世间繁华都看破,当我不再迷恋于物象的表面,也许,才是真正解脱的时候。

		然而,田终于不是智者,不是圣贤。田是平凡的小女子一人。

		同所有的女孩子一般,爱漂亮,爱照镜子。而且,她似乎还比其他人更加容易自恋。现实却如此残
	忍,不留情面。

		田并不愿把这些心情表露在文字里。文字,是我生命最真诚的舞蹈。于是,我写下来,仿佛是一场
	抱怨。

		田总是隐忍着。

		在这落叶纷纷的季节……

	\endwriting


	\writing{飘过}{2006年10月20日 ~ 19:44} %<<<2

		\longpoem{}{}{}
		我感觉到,那些阴影 \\
		飘过城市,飘过孩子们,尚未清醒的梦 \\
		没有未来的可能 \\
		死亡如尖利的一声叫喊 \\
		响彻生命的夜晚 \\
		我不曾恐惧,我却只有,颤栗着前行 \\
		我感觉到,那些巨大的手掌 \\
		捏碎我们的知觉 \\
		我能够明白,睡去意味着什么 \\
		一刻的昏迷 \\
		一世的丧失 \\
		我不会哭泣 \\
		只有你们,为我拉下帷幕 \\
		回归,原始的寂静
		\endlongpoem

	\endwriting


	\writing{瞳孔之内}{2006年10月22日 ~ 20:00:40} %<<<2

		两米之外,你站在我面前,我们四目相视。我知道,透过瞳孔,我进入了你的视线。我的一举一动
	,都将成为影像,存在你的记忆,或者在转眼后瞬间忘却。瞳孔,仿佛一台摄像机,时刻捕捉着周遭的
	一切发生,再合成为我们生活的情节。这是一场鲜活可触的戏剧一般。你眼见着我,有意无意的表演。
	我目睹着你们,一场场的悲喜,一幕幕的是非。两米之外,也许,你不懂得我真实的表情,听不到我由
	衷的话语,你只是在观赏,我表面的生活,貌似繁华如梦。

		一部经典之作《楚门的世界》。看一个虚拟的人间,在大众的观赏需求下,被导演一手操控和安排
	。看Truman由浑然不知,到抗争逃离。这一个完满的,滴水不漏的世界,被人工制造。所有的命运早已
	写定,每一天的生活也被精心安排。没有任何意外的出现,也不容许有任何意外。Truman要沿着导演划
	定的道路,一步步“幸福”地前行,由他生命开始的那一刻起。每个人向他和善地微笑着,鲜花开放在他
	的花园,妻子温柔体贴,生活如此,从出生到死亡,完全是观众电视机前的消遣。Truman在这舞台中央
	,被操控着,度过一个个似乎平常的日月。如果,他不曾有所察觉,如果拍摄人员的工作没有出现疏漏
	。Truman将永远不知道这一切,直至他在全球亿万人的电视前死亡。他将用一生的长度,完成这样一部
	耗资巨大,影响全球的真人秀。而这颗星球之上,除去Truman,所有人都知道事情的真相,只有他一个
	人,活在彻头彻尾的谎言和骗局之中。这是多可悲的寓言,令人不寒而栗。

		也许,现实的生活,也正是这样。从表演,到观看,从窥视他人,到被人观看。没有人不是Truman
	,没有人不是浑然不觉的表演者,站在独自的舞台中央。只是,这里没有隐藏在四处的针孔摄像机,那
	些高科技的监控装置换成了更为无孔不入的,人群的眼睛,那一双双敏锐的瞳孔。我们被暴露在空气里
	,便意味着暴露在这人间的舞台之上,无可逃遁。我们是Truman,一个不知情的表演者,自觉,不自觉
	地扮演着自己的角色,自然,不自然地道出一句句独白。我们的生活没有导演的操控,却也终于逃不过
	命运冥冥中的安排。只要是被观看的,我们就不可能不去在意他人的目光。人群的目光,不断用多数人
	的价值尺度把你的生活测量和评价。你不得不做出选择,屈从于大众,作讨好的表演,或者,坚持自己
	的方式,作难免孤独的舞者。人生的舞台,比Truman的世界更加虚妄和险恶。故事的结尾,他终于得知
	真相 —— 所有的人都在表演,而我们,却永远无法获知,谁在表演,谁在生活,也永远无法左右,自己
	的言语,哪一句是台词,哪一句是真实。这样看来,Truman比我们幸运许多,毕竟他所生活的世界,是
	被安排好的一派纯美和安宁。


		电影中,最震惊的画面是,愤怒的Truman,决定驾船出航离开这个虚拟的小镇,经历万般险阻,最
	终却在“天边”撞上墙壁 —— 天空也是假的,海洋也是假的。Truman走下船来,影子被照见在碧蓝的,绘
	画着云朵的墙壁。他抚摸着那面墙。那画面,只令人感觉到苍凉和绝望。我们是否也有这样的一处天边
	。这偌大的人间,有可曾有过真实的天空与海洋?莫非,每个人都是如来佛手掌中的孙大圣,跳不出,
	这生活的茫茫。Truman的行动没有过真正意义上的自由,他被封锁在这小镇之上。现代人却仿佛拥有着
	从前时代从未享用过的许多自由。人们在物质充裕的城市,尽情地满足各种欲望,消费,刺激消费,成
	为时代的主题。这繁花似锦,光怪陆离的都市,是多么醉人,多么美妙的天堂。人们似乎是无比自由的
	,你可以用金钱去购买,你所需要的物质。一个物质的人间,被锻造得华美光鲜。然而,这些自由,给
	了你幸福吗,令人们快乐吗。弗洛姆在比较中世纪与现代时说,中世纪的人们是安全,却不自由的,现
	代人是自由,却不安全的。在那些远去的时代里,人们的社会关系,和角色分类十分单纯简单,人们的
	生活,充满着儿童似的盲目信仰和天真。好像我们读着诗经时的感觉,所有的情绪,无论幸福或悲伤,
	都是静穆安详的。也许,人们的物质生活并不如今日的丰富,但人却是更接近于人的本质的,对泥土,
	对山川,对生命,都充满了虔诚的热爱。现代的人,看似是在使用物质,却分明是被物质奴役了。于是
	,也才有了所谓房奴,车奴,卡奴。银行贷款给了人花钱的自由,实则,让你变成了实实在在的奴隶。
	为了物质,为了还款,你只有拼命工作。这样的生活,是自由,却充满了不安和动荡。为什么,我们不
	能只去取用我们需要的那一部分呢。为什么,人们一定要透支掉一些什么,来换得暂时的满足呢。显然
	,我这样的想法是不合乎这个时代的。所有的广告,都在刺激你,去消费吧,去满足你的欲望吧,尽情
	享用生活吧。没有人去告诉你,欲望是怎样的魔鬼,没有人向我揭露一切的真相。我们被操控着,那双
	巨手比电影中的导演更为有力,控制着我们的身体,我们的行为,甚至,我们的思想。在生活里,人从
	不是自由的。真正的自由,是人们的想象。Truman被控制的只是行动。现代人被控制的,却几乎是全部
	。

		在这充满了控制,充满了摄像机的人间,你如何从容生活。多数的时候,我们在大众的价值取向,
	和个人的欲望面前不知所措。我们被人流推挤向前,你好象没有自己的双脚,你只是这样,被动地跟随
	。从出生,到上学,到工作,再到死去。也许,最幸运的是那些从不会思考这些事情的人。他们只需生
	活,只需跟随着人流无知觉地前行便好。而每一个,在这个过程里,有所质疑的人,都将深感痛苦和迷
	惘。好像终于发觉了真相的Truman一样。如果他没有察觉,他便没有痛苦,只是浑然不觉地继续一场秀
	便好。不幸是从他的清醒开始,痛苦是从他的清醒开始。或许,糊涂的人更容易幸福?但装糊涂,却只
	会增加痛苦的程度。

		两米之外,你用头脑纪录下我的表情,我的举止。瞳孔之内,我观看你的人生,一点一滴,真实或
	者虚妄的发生。有谁不是Truman,有谁不比Truman无可奈何。我们的真人秀,天天上演,在彼此的视线
	。

	\endwriting


	\writing{菲}{2006年10月23日 ~ 20:43:24} %<<<2

		又开始听王菲了,不厌其烦。\par
		当耳朵渐渐对一切新鲜的声音,开始疲劳和厌倦,我发觉,只有这个女人的歌声,能够让我得到片刻的安慰。\par
		清冷透明的音色,飘浮在空气之上的轻,似有还无的呼吸,气若游丝。\par
		这座城市,多少寂寞的角落里,多少的陌生人,和我一般,静静听她的歌。\par
		一遍遍,让曲目重复着播放,我爱的《四月雪》。\par
		任由旋律,在耳际萦回,经久的缠绵依恋。

		当个人的往事,终于失去重量,才拥有坚强的力量。

		这一句,我近乎虔诚地相信了。\par
		往事,那些经过了,又被弃置遗忘的欢乐,终于成为我们前行的负担。\par
		曾有的一切悲伤,一切苦难,也同样令人举步维艰。\par
		往,便是离开。既然离开,又何须挂念。\par
		也许,这正是人的脆弱所在。\par
		我们总是不擅于选择记忆。\par
		我们只懂得,对于个人的悲喜离合,念念不忘。\par
		坚强,是遗忘之后。\par
		坚强,是我们放开了不断流失沙土的双手。\par
		是能够了解生命的轻,并有所感触,不再悲戚,亦不再荒废。

		难忘2046中的王菲,最不忍,也最煽情,她那低首的脉脉眼神。\par
		泪就要落下来,却终于不发一言。\par
		这一刻的缄默,不是因为语言的不通。多少次,她练习着那一句:带我走吧。\par
		无法选择离开。于是,只有无言的告别。\par
		悲剧,正是每个人的无可奈何 —— 渴望幸福,却只能够别无选择地选择悲伤。\par
		木村推开那扇门,转身而去。\par
		电影的光影流转,苍白色的日光在玻璃门上旋转。\par
		她那一滴泪落下来。

		王菲,得以全身而退的王后。足够聪明的女人。\par
		她说,大家不要记得王菲。她说,她讨厌当明星,却喜欢引人注意。\par
		这些话,坦白得毫无遮拦,好像孩子的简单天真。\par
		而敢于这样说话,也是需要资本的。\par
		她知道总会有人记得她。她知道她随便的一举一动也会引人注意。\par
		一个不甘寂寞的女人,现在,抛下所有拥护她的歌迷,过她选择的生活。\par
		这可能有点任性,却正是王菲该有的行为方式。\par
		她大约是幸福的。虽然不喜欢李亚鹏,但看上去,他貌似是一个可靠的好男人。\par
		男人最可贵的,便是可靠吧。\par
		毕竟结婚不是恋爱,面对的是现实生活的琐碎。\par
		但即使这样,我似乎依然对李亚鹏耿耿于怀。因为他,王菲不唱歌了。

		事实上,我只是迷恋她的歌声。

		有人说,喜欢王菲的人,多数都存着一处秘密的内心。\par
		而有谁,又是在人群前毫无保留地袒露自我呢。\par
		每个人都有秘密,有那些只说给自己听的言语。\par
		她的歌声,适合做你独自通向那处秘境的背景音乐。\par
		然后,这路途,仿佛并不孤单。

		有时,天空下了雪,有时,风里吹起灰白的云朵。\par
		我便想起,听着王菲的许多冬天。\par
		那年,放学后和朱一起去华人买《将爱》,也是冬天。\par
		兴奋地赶回家,把音碟放进机器,一首首地听。\par
		可能,再也不会有那样的心情了。为了一张唱片而久久等待,久久痴心。\par
		时间把我们带离原有的位置。也带走那些青春的轻狂和热情。\par
		我们,好像渐渐冷却的热水,在人生的空气里,不断散失着单纯的热量。\par
		再过一些年,我们也许连听音乐的兴趣,也消磨殆尽。\par
		到那个时候,你还会和我聊王菲么。\par
		我多希望,我们还有激动和怀念。

		\longpoem{}{}{}
		还没跟你牵著手,走过荒芜的沙丘。\\
		可能从此以后学会珍惜,天长和地久。
		\endlongpoem

		谁陪你唱一曲《红豆》。这歌词美得令人心疼。\par
		我们的今日,我们的欢歌,不是不醒的梦,却恰是不愿睡去的狂欢。\par
		我全部的爱呵,都是洁白。\par
		在心头酿着诗。\par
		在诗中藏入无孔不入的想念。扭开音响。\par
		躲在夜晚里听王菲。\par
		这个游离于柔软与坚硬,轻若无物的声音。\par
		充满我的身体,我的房间。

		我从未疲劳厌倦的声音。

	\endwriting


	\writing{自由}{2006年10月24日 ~ 21:23} %<<<2

		电影《肖申克的救赎》的海报上写着: Fear can hold you prisoner. Hope can set you free.

		读过去,沉默许久。

		自由,不是一堵实体的高墙能够阻隔。

		恐惧,和希望,将我们囚禁,或释放。

		人生的牢狱,总在每个人的内心。

		无形的囚禁,源自无休止的恐惧。从外界,到我们的精神。

		多数的时候,是不必要的担忧。

		说不必要,是因为即使担忧,也依旧无能为力。

		该发生的,终究会发生。恐惧,没有任何意义。

		只会带来不安和围困。

		人总是自己的囚徒。

		也许,也确实只有希望,给你解救。

	\endwriting


	\writing{与书有关}{2006年10月30日 ~ 15:28:44} %<<<2

		精神的家园,只容许我们保存固执的洁白。

		将沉甸甸的一袋书归还图书馆,看它们被一一放上书架。走出门,迎上透明的秋阳,光灿灿地照在
	发稍和睫毛,冰凉的安详感溢满心田。

		梧桐树庄严站立,天空高远,路过的人行色匆忙。这个偌大的世界上,原来,只有人间是热闹而慌
	张。

		想在这样的季节里,在刚好的光线中,将自己铺展。像一本书那样,被平放在微风的窗口。让风拂
	过,让空气翻阅我的身体,一页页的言说不尽,沉默着芬芳的文字。这会是毫无声息的午后,足够明亮
	;这该是忘记了获得和丧失的时刻,我的生命,成为这样的一本书,成为文字,盛开着,如一朵绯红的
	小花。我只愿是这样,无所忌惮,无所忧愁地存在。仿佛人间之外,我只被巨大的宇宙怀抱着,放在蓝
	空的摇篮。一个遥远的声音对我说着,感恩,善良,美,和爱。

		如果书有知觉,那么,它们该是最幸福的精灵。它们不发一言,却懂得所有。它们在书架上等待,
	一只手,一颗爱知识的心灵。有时,这样的等待会经过漫长的时间。在图书馆的旧书区,我遇到许多在
	等待中老去的书籍。它们书页的齐整,让我得知它们长久的寂寞,落满的尘埃,又泄漏了时间的沉淀。
	我翻过它们的书页,手指在纸页间摩挲,停留下我的目光和温度。书的封底,还插放着旧式的借书记录
	卡。日期停顿在1986年的春天。那也正是我出生的春天。20个春秋,这世界上多了一个爱书的孩子。20
	个春秋,它在书架上等待。人与书的相遇,你若用深情的心去体会,也感觉出一种美妙。

		那小小的借书卡片,记录下每一个读者经过的印记,是一个个到访者的留念和签名。想到《情书》
	中的藤井树。他在一本本书的卡片上签上自己的名字。他热衷于作卡片的第一个签名者。他签下的是自
	己的姓名,也是那个他爱着的,与他拥有相同名字的女孩的姓名。在影片的最后,多年后的女孩,见到
	了在背面绘画着自己素描小像的借书卡片,才了解了那份遥远而纯粹的情谊。这样的故事,也许从未真
	实发生过,但可以肯定的是,以后再也没有发生的可能了。借书卡片,退出了工作的岗位,取而代之的
	,是快捷方便的磁卡。关于书的记录,全部被保管在电子系统中,却让我们真切的视线中一片空白。我
	们都成为了不出声的过客。我们再无从知道同一本书,另一个阅读者的存在。

		图书馆,该是一个学校中最舒适的地方。有时,它像一处避难所,让你在喧嚣里逃出,得以藏身。

		自习室总是人满为患。大桌子上堆满了书和各种资料。我喜欢图书馆。喜欢找一个小角落坐下来,
	读我的书,写我的字,发我的呆。

		看看形形色色的人,来自不同专业,不同生活。每个人,貌似安静地坐着,却分明在这安静之下流
	动着难以平复的躁动和不安。它们属于这个时代,任何人都无以幸免。知识,多少时候仅仅成为了我们
	谋生的工具。知识,越来越少地带来快乐和丰富,越来越多地造就了压力和空虚。也许,这是别无选择
	的事。我们被特定的时间推上入了这样的境况。人,或许能够侥幸逃过现实的烦恼,却永远逃不过时代
	的控制。带着一身的欲望,而不是理想,你要怎样才能走得轻巧有力。而此时此地,正是一个充斥着欲
	望,却缺失理想的年代。听教授们回忆80年代的生活,讲那时的热血青年,他说,爱情故事大多发生在
	图书馆,他说,路过的姑娘们在谈的是美学,他说,普通的工人也读黑格尔。怪不得,80年代还有诗人
	的存在。那是一个崇尚知识,渴求书籍的,诗歌一样奔流着真情和热情的时代。20年,离我们已经很远
	了。现在,我们日复一日坐在自习室读英语,背政治,谋划着一个将来,感觉疲惫。

		这是个没有诗人,也不可能有诗人的时代。当所有的青年,都忧心忡忡地急着把自己打磨光滑,你
	又怎么敢奢求,一个单纯的心灵,一个嘹亮的声音?

		我只想躲在小角落里,我开始恐惧人群。我知道,人流会把我带离自己的方向。

		这个世界,虚妄的东西越发真实逼近,真实的东西却越发缥缈涣散。我伸出手去,只可以从书架上
	取下一本书,却摸不到,这路途上,人人奔波的方向。或许,这是属于青年的迷惑和彷徨。也但愿,这
	些体验只是年轻的偏见。但人间的热闹与慌张,在自然的静穆平和之下,是显得如此卑微可笑。天空在
	观望,星辰在观望。看你从岁月的这一头升起,不知所以地奔跑,撞上墙壁,撞上生活,时而悲伤,时
	而幸福。我们总是不辨真假。我们难以明白,人真正的需要。只是不断向外索求,一切别人也在索求的
	。却没有用自己的心和头脑去思考,那是否确实是我们想要的一切。于是,错失是难免的。而事实上,
	又有谁,能做到一无所失呢。

		大概,丧失是常态。而获得,才是异态。

		于是,我开始羡慕一本书。羡慕它的安静,它的知识,它的姿态。可以么,就任由我,平卧在微风
	的窗口,做一场大梦。让我忽略时间,忽略记忆,忽略一切。我仿佛只需等候,一只手,一颗爱书的心
	灵,穿越了重重春秋的厚度,来到我的面前。这又好像那古堡里沉睡的睡美人,睡在荆棘丛生的宫殿。
	然而,荆棘终于会开出花来。正如一切美好的事物,终于会为生命带来希望和明亮。一个声音轻轻地说
	着,感恩,善良,美,和爱。这些,或许是我们所有赖以生存的养料。如此简单,又如此困难。


		只有书从来都是单纯可爱的。

	\endwriting


	\writing{发生}{2006年11月17日 ~ 22:31:27} %<<<2

		一些夜晚,我听到心脏的呜咽,如一只受伤的蝶。\par
		那些悄无声息的发生,像是魔鬼安排下的圈套,潜入我们的生命,在全无防备的时刻。\par
		就这样,一纸结果,宣布了你将在年轻的时光中,与险恶的疾病正面交锋。\par
		长长的走廊,尽头投下一窗阳光。\par
		初冬的北京,寒意透过墙壁,涨满了空气,密布了淡蓝的安静。只有护士站的检测仪,嘀嘀的声响。\par
		这样的十一月,我被抛置在未知的命运面前,被时间等候着,或者,等候着时间。\par
		许多难以获得完满答复的问题,在许多的纸上,许多的空闲里,画着无休止的问号。\par
		头脑,是间空房。身体,是奇妙的容器。\par
		问松松,你是怎么生病的。\par
		14岁夏天的一场雨。淋雨后,便开始起红疙瘩。你呢

		我却一时哑然。我的病,却仿佛完全没有征兆。只在14岁的秋天早晨,发现自己的食指苍白,无法
	通过血液。

		如果没有病,现在的你会是什么样子。\par
		松松也许会去上电影学院。但不会遇见她的爱人,深爱她的老郭。\par
		我也许不会写字,不会沉迷,不会自言自语。\par
		没有人会说田是个多愁善感的女孩。她总是蹦蹦跳跳的,一脸顽皮。

		我们在夜晚落下的窗前听歌。许多被我遗忘了,堆积在MP3里的歌声。徐怀钰,张惠妹。轻轻跟着
	哼唱,熟悉又陌生了的旋律。与十几岁的我,耳鬓厮磨的旋律。

		回忆在音符中被触摸着,一寸寸,滑过耳膜的脆弱,隐隐作痛,却又是难言的甜蜜。那一个健康的
	自己。那一个飞奔在夏日的阳光下,影子炽热的自己。我仿佛被时光燃烧掉,如一卷诗稿,扬起漫天灰
	烬,却持握不住片刻的停留。疾病,令人有了凌空的寂寞,在如花的年岁上,懂得了性命之忧。

		你不要责怪,田的悲伤。刻在命运上的悲伤,刻在骨子里的悲伤,总是无可逃遁,无可医治。

		她是这样的女孩子。有时候,无可奈何地,成为了一棵雨天的植物。幽幽地盛开着,幽幽地接受。
	如果没有病,我不会明白这一切的真相。也许,我将是明媚的孩子,任性而肆意地挥霍。和所有的女孩
	子一样。而现在,我只觉得奢侈。

		离开医院的上午,到燕的房间与她道别。

		这个坚强而勇敢的女孩,拖着虚弱的病体,在大洋上的岛国独自战斗,几次穿越了死亡,终于完成
	学业。毕业那天,校长落下眼泪,将证书颁发给她。

		现在的燕,静坐在床,床上放着肺动脉高压的康复指南。护士把药注射入机器,她开始吸药,一口
	口,吸入生命的力量和希望。身材娇小的燕,瘦弱的燕,穿着粉色的薄棉衣,像是插在清水中的花朵,
	甜美而疲惫。

		谁也不会知道,她所经历的一切。谁也无从知道,燕的心里飞翔着怎样的翅膀。

		她微笑着向我挥手。

		燕,我想起你对我说起的梦。我们都曾梦见自己健步如飞,毫不费力。梦见爬上高楼,而不气喘吁
	吁。像从前的日子那样。也会有那样的明天吧。会有的。一切苦难和不幸,都会离开。我们的天空将
	被擦亮。你会飞翔。轻捷而自由。

		我相遇着,相似的生命。年轻,美丽,而无可奈何。与疾病正面交锋。每个人都接受了这样的发生
	。每个人,都坚强到近乎残忍。而这,不是一个适合坚强的年纪。本该穿着一袭长裙,站在落花的温柔
	里,风花雪月。本该不知忧愁,强作着诗歌的惆怅。我觉得心疼。这一种,只有切身体会,才可能感同
	身受的疼痛。我病了的心脏。我抚摸着她的轮廓。她活跃地跳,不知人间的悲喜。她不懂得我,她只有
	机械地运转,有些劳累了,有些疲惫了。你如果懂得我,会好起来。会在休息后苏醒,一日日强健起来
	。

		我深爱着,所以会疼。

	\endwriting


	\writing{独自的絮}{2006年11月19日 ~ 20:53:35} %<<<2

		十一月,想掩上故园的柴扉,守住寂寞,守住花开的可能。

		我都忘记了,那些照片,那些文字。直到这个晚上,我们一起坐到电脑前。许多花朵,和表情,从屏幕上跳出。

		拍摄的日期在文件夹上被准确标注。05年冬天,不断拍下的天空,落叶,和雪。\par
		总是随身带着相机,不厌其烦地按下快门。\par
		仰起头,或者侧过身。没有很好的技术,不懂得光线和角度,只当作目光的忠实记录。

		我却始终不是一个善于捕捉的人。

		于是,只能任由许许多多的不舍,从指缝,从睫毛,残忍地失去踪迹。

		我所有的努力终归徒劳,无论文字或者照片。好像我全部的气力,就是在执意地挽留。\par
		多么可笑,却又美好的事。

		精准的日期,丝毫不差。当时当地,细微的感触和直觉,都被机械地记载。

		想起脖子上缠住厚厚围巾的自己。站在19岁的岁末中,面对冰凉彻骨的季节,呵出纯白色的哈气。

		冬天里,就该系一条围巾吧。缠住温度,也缠住,那么许多微妙的心思。\par
		所以,在冬天织一条围巾。所以,笨手笨脚地,学着一个小女人的模样,安心生活,安心幸福。

		北风,带走树枝的余温,教室窗外的小鸟,穿上黑色的礼服,呀呀地叫。

		然后,05年的冬天下雪了。\par
		然后,洁白的梦,冻僵了世界,冰封了路面。

		和朱搀扶着慢慢走过街道,才没有摔倒。你说,我们像两个老太太。\par
		有时候,我渴望见到你们老去的模样。\par
		那并不是一件可怕的事。衰老,该是最甜美的忧伤。

		有些话,我记在日记本上。有些话,我放在目光里。\par
		在最寂静的时刻,也许,我反而说着最丰富的语言。

		它们是不容说出的。

		只能够,让时间抚摸过你我的肌肤,让彼此的热情消磨殆尽。\par
		直到生活成为一杯水。欲望和期待,全部澄静如湖。\par
		青春的故事,便像一部小说那样,可以被反复阅读,却无丝毫遗憾。\par
		那已经是别人的疼痛与快乐了。

		你说,我们的幸福,总是才握时有,一撒手无。\par
		04年,或者05年,没有什么不同。\par
		自言自语的孩子,在雪地里埋下秘密,还有,那些不分真假的回忆。

		我用寂静,安放幸福。

		温度持续下降。当务之急,是保有温度和热量。\par
		需要一杯茶。需要一个没有声响的夜晚。需要独自空白的头脑。

		蜷缩在被子里。

		原来,每个人能够占有的世界,只有这么大。小小的一条被子,就足够容纳。\par
		那么,又是什么,令我们感觉空洞无依。是什么,让我们失魂落魄呢。

		有什么是真正值得的。\par
		有什么是枉自虚构的。

		全部,不过一场场天衣无缝的幻觉吗。这样的夜晚,我开始怀疑所谓存在。

		我都忘记了。

		那些照片,那些文字。在时间里,与我逆向而行,光芒刺眼。\par
		一点点陌生了。一个个自己,一个个不再拥有的冬天。

		照片上的花,姹紫嫣红。花树下的莫一脸烂漫。那天,我们一起写诗,一起在湖边看着雨,簌簌地落。\par
		女孩子的今天,与明天,女孩子的一字一句,都是明亮的歌声。\par
		有一些碎了,有一些碎着,从不曾完整。我们只有捡拾碎片,却无法在当时当地尽情享用。

		让我拼凑所有的美好,记得你我的青春。\par
		它总是这样轻手轻脚的经过。

		你也会忘记吧。\par
		那些照片,那些文字。\par
		只可惜,我无法成为一个善于捕捉的人。\par
		我只是执意的挽留者。

		扫净阶前的落叶。天空是相似的。\par
		又是酿雪的季节。

	\endwriting


	\writing{雾}{2006年11月22日 ~ 10:47:46} %<<<2

		没有形迹的蓝,无可触碰的世界。如果,我们真的能够心如止水……

		北京的大雾,三日不散。\par
		于是,赖床的早上可以拉开半扇窗纱,欣赏蔓延的白。\par
		化不开的浓白,是天庭上打翻的一盏琼浆,散落人间。\par
		你在雾中了,我在雾中了,看灯火一粒粒稀落下去,街上的树影子影影绰绰。\par
		清早的时间,被漫无目的视线牵引着,朦胧里,回想着枕上的残梦。这是个分辨不清梦境与现实的时刻。

		那个梦里,我们在汹涌的人潮里相遇。硕大的粉红色花朵怒放在熟悉的街道。\par
		我们也许是初识,也许是重逢。只是,彼此的脸上,都没有任何的幸福与惊喜。\par
		那是一场平静得令人恐惧的相遇。\par
		很多很多年,我们计算着离别的日期。很多很多年,我们忘记爱过的面孔。\par
		直到有一天,青春终于过去了。\par
		生命不再有这样许多,迷惘而美丽的梦。夜晚可以获得安宁。\par
		那时,睡眠的时间会减少。我们会和许多老人一样,苦于失眠的困扰。\par
		可能我便不会,梦见花朵,梦见陌生的面孔,梦见种种,残忍到寂静的画面。

		这些梦,是谁的捏造么。还是谁,在临睡前的阴谋,要闯进我的夜晚。\par
		我一直惊奇,我的梦里总是开满花朵。\par
		飘零的,盛放的,一簇簇缤纷的颜色,像节日里高空的焰火。\par
		我总是失重的人,从色彩里无声下落,坠进重重的白色里,如雾的。\par
		或许,几千年来,相似的梦境,曾无数次出现。昨夜闲潭梦落花,可怜春半不还家。\par
		游子思妇的疼痛,多少年,伴随着花朵的残碎,一片片,一夜夜,碾过生命的短暂。\par
		所有的花朵,都像是隐喻。是思念么,是希望么,是悲伤么。\par
		所有的梦,都充满着冰凉的甜美。\par
		泪与微笑,在世界的这边与那边交错。真实与虚妄,在时空的两侧轮转。

		爱的前世与今生,生命的彼时与此刻,仿佛掌心的一丝温度,在独自的时刻体会着,却不能说出。

		有雾的早上,我们曾去海滩上散步。\par
		海是没有形迹的灰蓝。好像,这些值得回忆的过往,那样轻,那样不可捉摸。\par
		我们踏过沙,我们在雾中前行着,留下深深浅浅的脚印。\par
		我想象着,夏天的艳阳里,孩子们欢笑着嬉戏的沙滩,想象着海风的颜色。\par
		而此时,这里是微寒的,有雾的早上。海没有热烈,海没有凶猛。\par
		海如此静穆而宏大。仿佛沉睡着,在厚重的纱帐里,如纯净的婴儿。\par
		我为你拍下照片。然而,在雾的包围里,一切的一切,都面目模糊。\par
		正如后来的记忆。\par
		正如我们。

		后来,阳光会照进窗口。我也将懒懒地从床上爬起。\par
		雾会散去。和所有的事物一样。无论苦难,还是美好。\par
		我们,却永远是在雾色里行走的人。\par
		匆匆而去,来不及一次回首,来不及留恋。\par
		于是,我用许多的夜晚纪念和期许。\par
		我捏造着梦,我栽种着花,我相遇着,离别着你们。

	\endwriting


	\writing{碎掉的话}{2006年11月24日 ~ 17:49:50} %<<<2

		一种偶然的存在,用一种必然的方式,构成了我不可信的知觉。\par
		凉凉的空气,吸在肺里,感觉着冬天,一种凛冽而寂寞的坚硬。\par
		17岁在日记本上写下:冬天是没有被好好爱过的孩子,所以报复这世界。\par
		我任性的笔,充满了仇视。没有被好好爱过的孩子。而今,这话听来忧伤。\par
		令他寒冷的,原本是世间最温暖的东西,爱。\par
		冬天的坚硬,不是过错,不是罪恶,却是令人心疼的伤。\par
		于是,我不去责怪冬天。\par
		抱着热气腾腾的水杯站在窗前,看灰蓝的天空阴郁着,沉寂着,不发一言。\par
		你在心中承载了多少故事呢。谁与你无声地告别,谁辜负了你的善意。\par
		冬天,在巨大的天穹下,独自埋藏着所有的不安与悲伤。\par
		然后,只让你看见冷冷的神情,在枝头,在风口,在暗淡的日光。\par
		在冬天,想到很多人。从远处,到近前。\par
		大学中的朋友,小鹿,莫。两个善良的女孩。\par
		一起躲在纱帐里读着海子和摩。一起燃起蜡烛,喝一碗热汤。\par
		大一的冬天,我们写着诗,读着老庄,等候雪的到来。\par
		新年的凌晨,三个人坐在楼道里聊天,直到视线模糊,睡意朦胧。\par
		和小鹿在图书馆5层,靠住暖气,想望海边的小屋。\par
		是这样简单的朋友,不必许多繁复的言语,没有心机,没有计算。\par
		各自保持着独立,却又在心灵上依靠着。是恰好的距离,恰好的温度。\par
		让我们谈文学,谈电影,谈感情,作少女的梦,肆无忌惮。\par
		这是青春里该有的滋味,不是彻底的欢乐,却是明亮的喜悦。\par
		感激有朋友的生活。\par
		读莫写给我的字条,一字一句都是认真。她总是这样温柔地担心着。\par
		两个冬天,我记得许多,记得生活。如此真切。\par
		田好像不愿说出,却在心中默默感动。\par
		有时,感觉人便像秋阳里的一丛荒草。在原野上被暖暖照着,目睹这天地的淡定与无常。\par
		让季节的风,刺透我们的身体。\par
		让那些岁月的幻觉,在宇宙中,唤醒生命最原始的幸福。\par
		一些呼吸,一些妄想,一些梦境,真实地走近我,仿佛雨水和寒霜的降临,全无声息。\par
		星辰闪烁,万籁俱寂,我们在此等候,蔓延成连绵的土地和山脉。\par
		你问我,你见过银河么。我回答,我只见到银河的凋谢。\par
		人,如果可以如草木般静穆而从容,大约便可以懂得,这个世界隐藏下的奥秘。\par
		我却只有看到季节,听到风,闻到寒冷。\par
		在被阳光笼罩的时刻,获得散碎的安详。\par
		在街上,看到倚靠着墙壁晒太阳的老人,三三两两。\par
		老去的生命,似乎更需要阳光的能量。人终于会接近于一株植物,对阳光依赖。\par
		年轻的人们从他们身旁匆匆经过。\par
		有一天,却也会慢下脚步,站在阳光里,眯起褶皱丛生的双眼。\par
		画一个圆圈,很多事情这样周而复始。\par
		从哪里来,到哪里去,又有谁真正明白。\par
		一个冷冰冰的冬天,被我敲出声响。\par
		自言自语的话却碎了,这样七零八落。

	\endwriting


	\writing{唱歌吗}{2006年11月27日 ~ 17:59:41} %<<<2

		不断的路过,不断的错失,幸福从无终点。

		起风的时候,我躲在房间里。在安静中,听整个世界的颤栗。\par
		渐渐开始喜欢这样灰暗着脸色的天空,它容许我们去沉淀,去冷静,去勇敢地沉默,或者歌唱。

		于是,会想起一些歌声,一些日子。那么恍惚迷离的,在头脑中闪念,仿佛一场旅行的浮光掠影。\par
		所有青春的躁动与不安,还有那些属于时间的不舍和眷恋,在此刻全部化为灰烬,等候着灰飞烟灭。\par
		《且听风吟》,朴树冰凉的声音。他说,灯火已隔世般阑珊。\par
		这是一首属于冬天的歌。\par
		也许,该混合着雪后的寒意,独自哼唱着,走过荒芜的原野。\par
		看火车,看旅客们疲惫的眼睛。目睹一个懒懒的,百无聊赖,略显哀伤的冬天。\par
		该有一个远行者的勇气和姿态,踏过时光,一路歌吟,一路洒泪。\par
		歌声中,总是依稀的悲伤。属于冬天的歌,属于我们经过的路,属于奔波与流离的生命。\par
		在城市的某处把自己隐藏,且听风吟,度过漫长的冬季,看雪花落下,又匆匆融化。

		我想,许多人在相似的时刻,都曾听着这样的歌声。然后,在生活的缝隙里,被温热注满。\par
		每个人,都是一件容器。我们讨论过这样的话题。\par
		我说我愿意是白色的素瓷杯子。你说,你是玻璃瓶。\par
		两个女孩的选择,一样的朴素无所伪饰,一样的单薄易碎。\par
		我在自己的杯中注满清水。\par
		另一些人,也许是甜酒,也许是汽水,也许只是空着,什么也不去放。\par
		然而生命,总是需要我们有所选择。因它本身的空洞虚妄。\par
		于是,有了歌声,有了为我们唱歌的人。\par
		有了梦,有了诗,有了高高低低的声音,呻吟的,欢笑的,哭泣的。\par
		世间的面貌也丰富起来。那么多的幸福和伤感,那么多的甜美和痛苦,被制造了,毁坏了,忘记了。\par
		人在人间,一个无法自觉的存在,偶然到我们自己也感觉惊奇。\par
		我是如何来到了这里,来到了你的面前。

		起风了,北方的冬天临近了我们的城。\par
		看街上瑟缩着走过的人,就明白,人终于还是这样微小。被自然的大手掌控着,被天地包裹着,与一粒种子无异。\par
		那年,冬天的夜晚,和莫一起听奶茶的歌,《春光》,我们都喜欢的歌。

		\longpoem{}{}{}
		季节匆匆来去 \\
		生命不可思议 \\
		好好抓住片刻的欢喜
		\endlongpoem

		高中时代的cd随身听,盗版的音碟,那一首歌我们反复听了好久。期待着春天的心,在寒夜的深处悄然绽放。\par
		你是否记得呢。莫。\par
		很多的夜晚,因为歌声而无法湮没。很多的日子,属于大学的,总是有你,纯真的孩子气。\par
		这个冬天,你对头的床铺空空的。请原谅田的缺席。\par
		我们却依然会分享,那些深爱的歌声。

		有时,我一个人唱歌。\par
		有时,我却忘记了歌词,甚至旋律。

		多少年前,你在电话的那一头,为我唱一首歌,遥远又苍凉的声音,令人痛彻心扉。\par
		那些年少的轻狂和稚气,再也不回来。\par
		奔波的人,奔波的年轻,相遇各自的未来,我们站在对岸,遥望彼此的火树银花。\par
		然后,记忆中你的哭泣,也不再有心疼。\par
		好像我曾说的,该结束的就狠狠地遗忘掉。好像你临别时的那场大雨,淋湿了所有的昨天,变得面目全非。\par
		这是一种成全。\par
		正如许多歌声,不必要深切而清晰地记得。只是在一些恰好的气氛里,忽然想起,再迅速忽略。\par
		多少年前,我还不懂得。原来,遗忘是我们自我保护的方式。


		落下的夜晚,好像灰尘,积满我安静的窗台。\par
		风又在唱歌了。我坐在桌前,写我的诗,一句句的凌乱。\par
		我是这样经过的么。每一个平凡的日月。我是这样,用一种近乎矫情的方式,来诠释,和接受生活。\par
		文字,是田的歌声。

		那是一杯清水,那是无需等候的幸福。\par
		勇敢地沉默,或者歌唱。

	\endwriting


	\writing{阳光}{2006年11月30日 ~ 14:29:10} %<<<2

		它是会微笑的花。在阳光里,收获最饱满的果实。


		这个午后,我被阳光包裹住,像一枚糖果,被五彩的糖纸拥抱着,密不可分。\par
		白纱窗上,映着花朵们明亮的影子。这些纤弱的生命,在难得的晴天里欣欣向荣。\par
		我只是坐着,喝一杯蜂蜜水,与蜜蜂们分享花蜜的甜美。\par
		充满光线的房间,让我有机会去细微感觉,温度抚摸过肌肤的每寸。\par
		翻开旧日记,看深蓝的墨迹,在一个个冬天之后,被记忆悄然冷却。\par
		还要写下去么。记录光的变换,雨水的气味,还有,那些经过着,告别着的面孔。\par
		在青春期之后,写作大约已不是一种单纯的冲动。\par
		为了更好的记得,或者忘却,我只有不断书写。

		在我还不会写字的时候,阳光却已经照在我的头顶。\par
		那时,一家人住在北屋里。一样是充满光线的房间。\par
		我是表情丰富,爱哭的小女孩。一把稀疏的发,被母亲轻轻拢起,用红丝线细心扎好。\par
		那是被人们称作童年的时光。\par
		周日的午后,我总是被迫躺在床上睡午觉。全无睡意的我,望着窗子上斑驳的树影子,摇摇晃晃,阳光很好。\par
		院子里,是洗衣机的轰鸣,弥漫在空气里,就变成洗衣剂的香味。\par
		然后,一件件衣服被晾起来了,在风里微微拂动,让阳光照得透明而通亮。\par
		从窗口,我看着母亲忙碌的身影。\par
		当她来“检查”时,我便佯装着闭上眼。午睡,是最难熬的时间。\par
		风里的衣服,像一个个被抽空了身体的人,吊挂在铁丝上。\par
		我总是这样莫名地想象着。然后自己又感觉恐怖,便迅速把被子拉起来,盖上眼睛。\par
		却依旧是无法入睡。\par
		那些衣服在阳光里飘拂,等待着水分的离开,像等待着时间过去的我一样,迫不及待。\par
		时间,却终于以我所惊愕的速度离开了。\par
		母亲再不会要求我睡午觉。我的睡眠却多起来。困倦,好像成为生活的一部分。\par
		于是,现在的我,常常感觉是游走在梦里。\par
		现实与我的睡眠这样近了。一个不睡午觉的孩子,竟长成了时刻在睡觉的人。

		阳光里的午睡,让自己懒懒的,像一只猫。\par
		我开始后悔,小时候没有充分享用,那么多个,充满阳光的午后。

		我眷恋着光线。\par
		如果这世界没有光,眼睛便无法知觉一切美的存在。\par
		也许,人会像深海的鱼类,长出带发光体的鳞片。\par
		我只可以通过触摸,来感知,来体会。\par
		于是,在创世的第一天,神便说,要有光。\par
		我喜欢在耀眼的光里站立。喜欢看逆光的池塘里,盛放的荷花。她们硕大的花朵,好像贵妇的花冠。\par
		那是一些夏天。那是我用相机拍下的,属于7月的光。\par
		有了光,便有影子。\par
		你说,你喜欢看自己的影子,好像被画出的灵魂。\par
		据说,鬼是没有影子的。那么,我想,影子不该是灵魂,灵魂应该如鬼一般透明。\par
		只有肉身会挡住阳光。也许灵魂便是光,是明亮而无重量的。所以能够飞。\par
		光影,被我们追逐着,却找不到来源和去向。\par
		黑夜临近了。乌云生起了。\par
		光没有时刻照耀在我们的身上。于是,人需要在黑暗中穿越,在阴天里,学会隐忍和坚定。

		为了有阳光的日子。

		很多时候,梦想着一间建在山腰的房子。门前种满会微笑的葵花。\par
		那也是一间北屋,和从前的家一样,有大而晶亮的玻璃窗,有一屋子,满满的阳光。\par
		我会勤奋地写作,然后,懒懒地睡午觉。\par
		在花园里种菜,种花,养一只爱撒娇的小猫。要在阳光里晒晒棉被,拍一拍,看空气里跳舞的尘埃。\par
		这一切,都要有阳光。\par
		这一切,看似简单,却无比困难。那是我的一种彼岸。平淡却富足而宁静的生活。\par
		和植物们对话,和小动物交谈,没有奢求的世界,没有争斗的世界,如此安详。

		蜜蜂们不知道我在分享花蜜的甜美。\par
		阳光的快乐,却同时感染着万物。来源于光的能量,滋养着每一种生物。\par
		就这样,心存感激地生活吧。在光的爱抚中,体会平静和满足。\par
		我想,我是如一枚糖果了。\par
		被含在阳光的舌头里。在这个午后悄悄融化。\par
		谁望着我在午后安然入睡,谁擦拭着天空,呈现出明蓝。

		细细数,我的无数个晴天。

	\endwriting


	\writing{奔跑}{2006年12月01日 ~ 17:38:03} %<<<2

		这一个瞬间。有悄悄的破碎,悄悄的风。


		有时候,我想重温那一种速度。感觉风,贴住单薄的双耳,向身后凛冽地消失。\par
		我渴望奔跑,渴望无所约束的迈开步伐,任由能量的消散,在身体,在空间,在我们站立的尺寸之上。\par
		然而,是在哪一天,有人告诉田,你不可以。你的心脏需要休息。\par
		于是,我只有慢慢地走。经过身旁迅速流动的人群。\par
		于是,奔跑成为遥远的回忆。\par
		像飘摇着,不可触摸的所有记忆一样,如此迷离,不可信。\par
		最后一次奔跑,是什么时候呢。\par
		忘记了。我们总是记不得,那轻易就滑落在地的最后一次。\par
		只有在失去后,才不断在怅惘里徒劳地追忆。\par
		它们却碎了,碎不成形。\par
		最后一次在学校的大堂里照镜子;最后一次因为迟到在楼道里罚站;\par
		最后一次为了考试而哭鼻子;最后一次吃你买的糖果;\par
		最后一次在回家的路口挥手道别。\par
		许多场连绵的大雨,许多潮湿的青春,就这样,像一张淋湿的照片,永远地面目全非。\par
		最后一次奔跑。也许,是在寒冬的操场。\par
		那一天,有很亮的星星。气喘吁吁地跑完,然后,扑倒在你怀里。\par
		我并不知道,在那个时候,心脏已经感觉到疲惫了。\par
		它很坚强,默默坚持了很久,才迫不得已让我知觉到它的病痛。\par
		现在,我慢慢地走。数着心跳和脉搏。度过每个日子的平凡和神奇。\par
		下午,校园里举办冬季越野赛。\par
		透过窗口,看到一个个身影迅速掠过,奔向终点。\par
		风在吹着,每个人满面通红。我在温暖的室内,望见身旁镜子里自己的样子。\par
		我感觉到生命的虚弱与疲惫。那一个自己。不愿承认。\par
		然而疾病,是这样残忍的现实。\par
		不容回避与闪躲。\par
		我想起手术前,荆大夫与我的谈话。\par
		他问,是不是总觉得自己很倒霉呢。得这样的病。\par
		我强作着微笑回答,已经想通了,在学着接受。\par
		他们说,病久了你便不会那么在意,也便不会恐惧。\par
		对于生死,也有了豁达的看法。\par
		你将学会和死神周旋,骗过他的眼睛,然后享受你所拥有的时间。\par
		那些在风中奔跑着,燃烧着能量的生命啊。\par
		我看着,看着,只有羡慕。\par
		然而,他们不会知道,也不会懂得,那一种,我永久失去的幸福。\par
		健康。是这样平常的词。却成为我最深的疼痛。\par
		他喜欢在伤心的时候去跑步。\par
		他喜欢用音乐塞住耳朵,在空旷的操场上跑步。\par
		这也许确实是一项适合于孤独者的运动。\par
		如果不能够哭泣,那么便运动吧。\par
		我想象着,那样一个黄昏。\par
		让我独自奔跑在无人的沙滩,用尽全身力气。\par
		心跳超过150次,呼吸急促。我已经很久没有奔跑过,甚至忘记了奔跑的方法。\par
		然后,就让我重重地跌倒。安静地躺在沙滩与海浪之间。\par
		等待月亮升起来,再掉进海里,没有一丝的声响。\par
		我好像睡了,如一个婴儿的纯净天真。\par
		以一种孤独而寂静的方式告别。\par
		没有人哭,没有人记得,也没有人发觉。\par
		如果,我还有机会奔跑。\par
		我要你拉着我的手,穿越过最喧闹的大街小巷。\par
		我要穿着白纱裙。你要送我,最娇艳的玫瑰。\par
		也许,半路上,我跑失了最爱的鞋子。你弄丢了最珍惜的白色礼帽。\par
		我们却全然不顾。只是一路欢笑着跑去。\par
		在前方的一座花园里,有等候了许久的幸福。\par
		那些盛开的粉红色花朵。那些我,熟悉了的梦境。\par
		你说,爱情是想对一个人好。\par
		我说,让我们都相信它吧。\par
		这是故事的美满开端,而不是结局。\par
		我多想,飞奔过时光的每一种细微。记录着所有爱与善良。\par
		告诉你,最美的花朵,是怎样无言地开满生活,流溢芬芳。\par
		我多想,忽略一切的不幸和磨难。\par
		我要奔跑着,像许多年之前那样,看着影子在脚下逃跑。\par
		正如,我们的知觉。\par
		那最美的花朵。

	\endwriting


	\writing{心里那些细小的声音}{2006年12月05日 ~ 20:05:56} %<<<2

		\longpoem{}{}{}

		在我们的星星上,希望是最亮的光。

		彼岸上,蝶翅载着的微风

		变幻了,你窗口的云烟

		此岸上,眉宇间浮起的浅笑

		染红了,初夏里,一树的樱桃
		\endlongpoem


		在十二月里,随手写下这样的句子,潦草的字迹,像个顽皮的孩子,在白纸上嬉皮笑脸。

		粗粗地喘着气,寒冷里,感觉冰凉的空气从咽喉直入身体。你反复叮嘱我,出门一定要带口罩。

		然而,冬天的严酷,还是不留情面地入侵。十指凉凉的,双耳凉凉的,鼻尖也凉凉的。


		好像只有热水和文字,可以保有着温度,持续融融的暖意。


		于是,在雪白的纸上妄想着一树樱桃的甜美,妄想着晴朗里,透明的笑。

		我仿佛能够见到,那个骤然转身的女孩子,在5月的末尾,与我道别。她没有说再见。

		那是此岸的欢乐,或者,是彼岸的梦幻。


		她穿着我18岁那年最爱的碎花布衬衫,脸上没有一丝的畏惧和悲戚。

		她不曾哭,即使疼痛的夜晚,依旧是咬住薄薄的嘴唇。

		她默声地承受着,那些命运的沉痛,不发一言。


		那一年,母亲带她去栽满樱桃树的果园,是明蓝色的5月。她们拍下许多的照片。

		许多的笑,定格在原地,被细心封存,等候着怀念或者遗忘。

		我与那个女孩道别,未及道一句珍重。她穿着我最爱的碎花布衬衫不再回来。

		没有人告诉我她去了哪里,她似乎是时光中的旅人,从此之后,永远地流离失所。

		我却在这里,翻看她的旧照片,想念着纯白色的那份天真和勇敢。


		这一刻,在西风的歌唱里,我所能回想的一切,却都分明如掌心的温度一样,慢慢散失。

		我们的故事,我们的苦难,终于要选择一个恰当的时间,风流云散。

		只为了,昨夜你窗口的一片月。只为了,蝴蝶翅膀上,一次轻轻的颤动。


		我说,我喜欢满天的星星。

		那女孩说,我们就住在星星上,不是么。

		虽然,地球没有耀眼的光芒。


		她从没有恐惧和悲戚,她在生命里画下斑斓的梦,并不担心谁会将它无情地擦去。

		让风雨落下吧,任灾难降临吧,她的眉宇间,总是那无忧愁的静定,总是那清水一样的笑。

		是淡如水彩的女孩,淡如水彩的生。


		我的心底涌起温暖。飞过冬季的漫长,我把想念在山脚的樱桃树下埋藏。

		随手写着潦草的字。

		给她,穿碎花布衬衫的女孩。


		碎吧,碎吧

		这悄悄的安静,悄悄的风

		十二月,给幸福勇敢的心

		给心,平和与安详

	\endwriting


	\writing{因为}{2006年12月08日 ~ 17:23:24} %<<<2

		收藏着光明。我像个吝啬的守财奴,不容许丝毫的放弃。


		那些摇曳的灯火稀落了,一粒粒,追随风的身影,遁入黑夜的茫然无措。\par
		生命里,一粒粒,温暖的火光,却燃着,亮着,透露着倔强的执著,即使,这手掌中的全部把握,早已摇摇欲坠。

		我感觉温度,感觉声音,感觉气味。一切细微的存在,微妙的构成,都发出奇异的光彩,照亮着房间,爱,还有梦。\par
		一个人的宿舍。午后恹恹的懒散,独自听地下铁的音乐剧片断。光线昏暗,阴的冬天,阴的窗口,怀着失语的哀伤,留下静默的空白。\par
		容许我,聆听陈绮贞透明的歌声,落下眼泪。

		她的独白。

		“天使在地下铁入口跟我说再见的那一年,我渐渐看不见了。\par
		15岁生日的秋天早晨,窗外下着毛毛雨。我喂好我的猫。\par
		6点零五分,我走进地下铁。”

		然后,我听到水滴,钢琴,哼唱,一个盲女,跟随着双耳和记忆,穿梭于黑暗。她轻声唱。

		“在空荡的广场,在空荡的海洋,我学会了退后的飞翔\par
		退后在,睡睡的梦乡”

		她说,她想要记得四十七件事。\par
		夏日午后的暴雨,雨的形状。黄昏的光,光里的灰尘在飞扬。爱人如何亲吻,如何拥抱。你烦躁无奈的模样…… 许多许多。\par
		她将想念,她将记得。吃过的苹果核,发光的公路,路旁栽满梧桐树,十七岁的照片…… 许多许多。\par
		失明前,盲女努力回忆,而眼前的世界,却渐渐模糊了,模糊了。她说,她渐渐看不见了。

		“我必须全部记得,因为我害怕,有一天有人会大声地质问我。对着我看不见的眼睛,我会轻轻地说:\par
		我看不见,但是,我全部记得。”

		她全部记得。那声音坚定而悲壮,没有丝毫的感伤与怨恨。像一丛巨大的浪,从海滩的高空坠下,淹没了命运原本的平静。\par
		盲女站在原地,如此近,却又如此远,从容地亲见这一场洁白的灾难。\par
		只是,在世界终于黑下来,她还是茫然无措地大喊,开灯,开灯,是谁在恶作剧。

		“昨日的悲伤我已遗忘,可以遗忘的,都不再重要。”

		记得那些光明的画面。微笑的人群,爱人的面孔,海滩上被风吹走的太阳帽,旋转木马飞驰的欢笑。\par
		遗忘悲伤,那些徒劳无益的情绪。失明前的准备,盲女默自决定,一步步勇敢走进黑暗。好像每个人迎接着夜晚。\par
		她用如此纯净的声音,在这个下午,敲碎我的心。

		想起几米的绘本。斑斓的色彩,斑斓的想象。手执拐杖的盲女,独自穿行。无法看见,却愈加绚烂多姿的世界。\par
		她的想象在飞。那些森林,那些花园,那些月台上匆忙离去的光芒和颜色,涂满画册,也盈满本是一片漆黑的双眼。\par
		没有人能够夺走你的幸福。谁曾这样对我说。\par
		盲女全部记得。盲女已经遗忘。没有哭泣,地下铁的世界,是城市之下的净土和神秘。\par
		我们只见到拥挤的人群,麻木的过客。她却看到另外的天地。\par
		或许,真的存在另外的空间。在那里,失明的双眼,不必承受黑暗。在那里,一切有所缺损的生命,都得到完美的补偿。\par
		失去,便不该有难以释然的亡失之痛。\par
		命运总是轰然倒下,我们从来都是猝不及防。

		所以,我将用生命的全部气力,将可感知的所有,仔细深爱。

		这样,我才能无所愧疚,我的手指,我的双耳,我的眼睛,我的一切感知。当我们还有机会来享用四季的轮转,爱人的亲吻,\par
		朋友的微笑,和所有平凡却神奇的日子;当我们还能够读一本书,想念一段往事,疼痛一种伤心;要用力去珍惜,不遗余力。\par
		人们无法预知明天,我们在记忆的花园游荡,在未来的森林梦想。\par
		这些快乐,那些不安,落下的泪,爱过的面孔,错失的幸福。一一记得吗。或者,一一遗忘。\par
		我会认真地选择。

		记得光明。我默默对自己说。

		和莫站在11楼顶层的窗口前。月亮,在更高的地方照着。这即将入睡的校园,这仓促生活的城,这不断守望,不断迷失的青春。\par
		你说,你想起大一,而转眼间,我们也将是离开的人了。你说,你记得那时的我们。\par
		我也记得。\par
		那两个做着诗歌梦的女孩子。她去了哪?\par
		我们都不见了。只是几年的经历,却令自己,与那时的自己,隔世般遥远,仿佛决绝的分道扬镳。\par
		你懂得生活了吗。\par
		你学会坚强了吗。\par
		谁教会我们,不再任性,忍住悲伤,接受幸福的决定。田说,我们会是更快乐的自己。\par
		你一定要相信。

		楼下的灯火,在十二月的风中飘摇。\par
		我指给你看那一排路灯。一粒粒的光,一粒粒的温暖。\par
		你说,在这样的高处你不辨方向。\par
		我们站了很久,我们说了许多话,像从前那样。田会记得你,田知道。

		也许,没有谁不是即将失明的盲女。\par
		也没有谁,不是站在命运的海滩,静候着自己的巨浪。遭受灾难,无论甜蜜,还是苦楚,无论爱情,还是不幸。\par
		承受着,等候着,无所畏惧,安放住不安的心。\par
		在走进地下铁之前,喂好我的猫。看窗外的细雨绵绵。\par
		迎接所有安排,而无所怨恨和悲戚,从容淡定。

		这样的生命,将是骄傲而尊贵的。

		我于是决定拒绝狼狈,拒绝一切忧伤。\par
		虽然,我落下了眼泪。

		那是因为切肤般感同身受的疼痛,因为太多的深爱。

		我要记得。

		我将遗忘。

	\endwriting


	\writing{爱}{2006年12月18日 ~ 22:29:23} %<<<2

		\longpoem{}{}{}

		Gloomy sunday

		Sunday is Gloomy, \\
		My hours are slumberless, \\ \\
		Dearest, the shadows I live with are numberless \\
		Little white flowers will never awaken you \\
		Not where the black coach of sorrow has taken you \\
		Angels have no thought of ever returning you \\
		Would they be angry if I thought of joining you \\
		Gloomy Sunday

		Sunday is gloomy \\
		with shadows I spend it all \\
		My heart and I have decided to end it all \\
		Soon there’ll be flowers and prayers that are sad, \\
		I know, let them not weep, \\
		Let them know that I’m glad to go

		Death is no dream, \\
		For in death I’m caressing you \\
		With the last breath of my soul I’ll be blessing you \\
		Gloomy Sunday

		Dreaming \\
		I was only dreaming \\
		I wake and I find you \\
		Asleep in the deep of \\
		My heart \\
		Dear

		Darling I hope that my dream never haunted you \\
		My heart is telling you how much I wanted you \\
		Gloomy Sunday
		\endlongpoem

		忧郁的星期天。一首仿佛充满了诅咒意味的音乐。自其诞生之日起,有150人在听过这首曲子后自杀。\par
		后来,作家创作出同名小说。后来,导演把它展现在银幕之上。\par
		一个关于爱,尊严,与死亡的故事。

		太多的故事,总是由一个女人的美丽开始。

		风情万种的餐厅老板娘 Llona,精明能干的犹太人老板 Szabo,才华横溢的钢琴家 Andras,还有
	,那个丧心病狂的纳粹军官 Hans。

		一个女人,与三个男人的命运纠缠。有无邪天真的爱情,还有情欲控制下的无耻占有。

		三个人的情感能够称作爱情吗。当Llona,与她的两个爱人同卧在河畔的草地,当两个男人在她肩
	臂的拥抱下,幸福地闭上双眼,你是否会对他们的感情产生疑惑和不解。爱情,怎么可能是这样的。爱
	情,怎么可能分享。

		男人说,分成两半的 Llona 总比半个没有的好。她的箭已射出,一半是给她物质与安全的 Szabo
	,一半是给她精神与激情的 Andras。这好像是一个贪婪的女人。往往不可能同时享有的两种爱,她却
	在两个男人的身上同时占有了。也许,你又要羡慕她的幸运,有这样包容着她的贪婪的两个男人,分享
	着一个女人的爱情,而心甘情愿。

		在一次醉酒后的早晨,两个男人说:我们需要你,你也需要我们,Llona,我们两个。

		于是,在风雨飘摇的前夕,在短暂却令人醉心的时间里,三个人的爱情,自然而坦荡地发生了。

		他们手挽着手上街去,他们共同经营着小小的餐厅。Szabo 的经商头脑,Andras 那一首瞬间里成
	名的钢琴曲,还有,Llona 那无可遏制的美丽,令这家犹太餐厅食客云集。直到那个昔日向 Llona 求
	婚遭到拒绝的德国人的到来。而今,他已经是一位显赫的纳粹军官。

		我想,他也曾是善良的人,他也曾心无杂念地付出过炽热的爱情。不然,怎么会一脸羞怯地提出要
	为 Llona 拍照,怎么会在遭到拒绝后一心寻死,最后,又怎么会在自己80岁生日那天,依旧要回到那
	家犹太餐厅庆祝。

		然而,当他脱下为救Andras来向他求助的Llona的衣服时,一切曾付出的爱与真情,都已灰飞烟灭。

		即将登上开往集中营列车的 Andras,眼神中充满了对于 Hans 的期望,他知道,他能够救他,只
	需要一句话。但是,德国人眼见着自己的“老朋友”登上了死亡列车,而面无表情。那一刻,屏幕上光影
	流转的背后,分明有人性破碎的声音,如此惨烈。

		Llona 牺牲了肉体,却终于换不回她的爱人。

		她知道那是唯一的希望了,于是,她奋不顾身地奉献出自己,即使,她清楚地明白,那个如愿的可
	能是多么微小。

		Hans 的爱,因爱而酿成罪恶,因爱而泯灭了良知。那么,这样的爱情,只令人感觉毛骨悚然。一
	把情欲的火,充斥着这个德国人的心,从他还年轻,到他终于有足够的权利来占有他人。

		他所谓的爱,需要用占有的方式来宣告胜利,这是多么可笑又可悲的爱呢。

		Hans,当他吃下他占有了,却永远不可能拥有的那个女人为他精心安排的生日晚宴,当他的心脏在
	那首充满了诅咒的音乐下骤然停止,会不会想到,那个不能忍受尊严遭到践踏,而自杀身亡的钢琴家;
	会不会记起,登上列车前,Andras 那绝望而期许的眼神;会不会料想到,于他罪恶里出生的儿子,在
	他的死后,与母亲举杯庆祝。

		那是一个伟大的女人,伟大的母亲,养育了自己与仇人的儿子,又在孤寂的岁月里策划了一场完美
	无比的复仇。影片在她老去的背影中结束。多少的爱,多少的恨,都湮没在布达佩斯灯火辉煌的夜色中
	。

		许多的时光,许多的故事,就这样,残忍而美好地发生了,发生着。\par
		关于爱情,却永远没有人真正懂得。它仿佛是天使,却又是魔鬼。\par
		这就好像,一个女人的美丽,可能是一场灾难。

		三个人的爱情,令人感觉高尚而纯洁,一个人对另一个人的爱情,却令我们感觉肮脏和龌龊。这样
	的反差,也许是每一个观众在观看前都始料未及的。没有人会去同情 Hans 的遭遇,是因为他伪善的外
	表,是因为他把爱情,单纯等同于肉体上的占有。但也许,他的选择,是一种时间与环境对于人的改变
	。他渴望的,自然也是拥有 Llona 最纯粹的爱情,然而,当那一切都不可得时,他唯有退而求其次。
	于是,他选择了以野蛮的占有方式,来完成自己的夙愿,来证明自己的胜利。可能,从头至尾,Hans的
	爱情都没有错。爱情从不会有错。错的只是,他没有懂得如何真正去爱一个人。

		爱一个人,总该是以一种无功利的心去付出。虽然,在我们投入爱情的时候,必定是期望着对方爱
	的回报。

		爱一个人,是尽可能完满对方的生命,甚至做出自我的牺牲。而这一过程,该是没有痛苦,却深感
	幸福的。大约,便如 Andras 为 Llona 所作的一切。为了她爱情的完满,他可以与他人分享一份爱情
	。这也许是一种极端的情况。更多的时候,爱情需要对等的付出,而非单方面的付出。只有两个人生命
	的完满,才是爱情最终的幸福。

		爱情,从不该是一种煎熬,失恋者的痛苦,是来源于对于自己的爱。

		于是有人说,爱情,归根到底是一种自恋。不然,人便不会有因为爱而不得而起的悲伤,不会有爱
	人移情的难过。爱情,便是一件简单的事情。正是那一句,我爱你,与你无关。只要默默地爱着,便心
	满意足了。因这世上有一个我深爱的你,而无比感恩。虽然,你并不爱我,甚至,你从不曾察觉,我的
	爱情。这样的爱情,或许是最纯粹的爱情,却也是最不符合现实的爱情,它好像一件精美的艺术品,而
	非日常人生活里的情感。它不食人间烟火的面貌,令占着世界大多数的凡夫俗子们望而却步。

		爱一个人,可以不发一言吗。爱一个人,可以任由他经过你的生命,而不留下一丝痕迹吗。\par
		你是否能够,在多年后回忆起某人,然后带着时光的苍茫,了无遗憾地说,我曾深爱过他。\par
		虽然,那一个人,从不知道,那爱的存在。\par
		所以,Hans的故事,并不是一个特例。我们总是要把自己塞到所爱的那个人的世界里,才心满意足。

		爱,只是让我们去爱,而不会保障任何的结果。

		爱或许从来是一件不计后果的事情。曾经想,若那一天,在爱一个人之前开始细心地计算利害,便
	是真正失去了爱的能力了。

		但是,当决定了爱,便也是决定了承担 —— 一切完满的,或者残缺的结局。爱情的终点,从来不是
	童话的结局,却永远是琐碎的开始。正如生活呈现给我们的模样,所有的情节,都终于是零零落落的碎
	片,一些如愿的,一些遗憾的,一些无可奈何的。爱情,是冲动的决定吗。一个甘愿了放弃自己,被对
	方囚禁的决定。看到过最温情的一句表白便是:在你给的囚笼里,我从未如此幸福,请不要打开那只锁
	吧。

		把彼此锁在对方的生命中。这想法浪漫而形象。黄山上似乎出售这样的锁,相爱的人总会买下一把
	留在铁索上,以示爱情的恒久不变。但是,又有多少把锁,在真实的人生中锈蚀残损,甚至开裂。就好
	像,那些坚硬的钻戒一样,越坚硬,却越反衬出爱情的更变与无常。人们总是期望着恒久,却不知道,
	时间终究是宇宙间最虚妄的幻觉。我们始终在这里,从出生前,直到死后。什么都不曾改变,包括那些
	来来去去,起起落落的爱情。那是我们生命的简单相遇。像一粒尘埃与另一粒尘埃,像两条盲目前行的
	游鱼。在更大更大的世界里,爱情显得渺小而微不足道。它只是一种偶然,和其他许许多多的偶然一样
	。

		而生命,终归也不过一种偶然的存在。

		一个知觉而已。

		那首乐曲时常响起。好像老去的 Llona 在清洗着令人心脏骤停的毒药瓶子时,轻轻哼唱的那样。
	这是一首被诅咒的音乐。许多人因它而死,他们说,他们受不了它的旋律,仿佛那就是自己葬礼上的丧
	歌。每个人都在清醒里做梦,或者,每个人都在梦里清醒。当人们割断手腕的动脉,当鲜红的血液奔流
	而出,那究竟是对于生命的一种亵渎,还是超越呢。曾经爱过的人,一张张鲜活的面孔,那一切,仿佛
	才是我们对于生命的全部眷恋。因我是一粒尘埃,因你是一粒尘埃,这相遇才因平常而显得珍贵。

		多数时候,我们终于是个俗人,逃不出这天地布下的情空欲海。看不破人间因果,舍不得爱恨纠缠
	。紧握住一念之间的知觉,承受住生命,或重或轻。

		然后,在平淡琐碎的日子里,努力爱得无邪而天真,心无杂念。

	\endwriting


	\writing{雪的也许}{2006年12月21日 ~ 17:35:29} %<<<2

		无法预约的雪,一如爱情。


		听说烟台又下了大雪,好像去年的十二月。\par
		一座临海的城,有雪的频频光顾,在我的想象里,该是怎般的清寒婉转。\par
		落雪的海,还有没有细细的浪涛,涌上冷却的海滩,还有没有温凉的月光,倾注满寂寞的夜晚。\par
		海还在歌唱吗。还有顽皮的孩子,一路追逐,一路欢笑的脚印吗。\par
		雪落着,落着,积在熄灭灯火的窗口,积在熟睡的街道,全无声息。\par
		雪中的城,被洁白紧紧包裹住,宛若初生的婴儿。一样的无邪,一样的天真。

		而在这里,我好像总是等候着。让一场雪不期而至,在我全无防备的时刻。

		两年前的冬日午后,靠住教室暖气小睡。醒来时,矇眬里竟见窗外已是玉花翻飞。雪,是这样调皮
	,总是猝不及防地袭击了视野。

		天地间,只有它们的飞舞,缓慢却紧密的落。多少个冬天,我痴痴看着雪花,伸出温热的手掌,去
	承接它们的融化。

		雪会有疼痛么。雪会不会也有快乐,和哀伤。\par
		它不出声,只静静落满我们的眼,用最单纯的颜色,诉说着一切。\par
		我知道,它在下落中见过怎样的风景,我知道,它曾随着云朵,经过了怎样的山岳和原野。\par
		是隔了多少重雾色,多少个等候的日月,雪融化在我的手心,一丝丝散失不见,化作澄明的清水一滴。

		自然造就的最纯洁的花朵,也最不容任何的把玩。你只可以用目光去亲吻雪,却不能够摘下任何的
	一朵,别在衣襟发梢。

		它是如此骄傲。雪花在手心的消融,远比昙花的陨逝更加凄婉。\par
		雪只要这一个冬天,雪花只要那空中绽放的瞬间。\par
		即使零落成泥碾作尘,它依然无悔。我曾绽放了,它说。

		我想象中的雪,在人间落下,在梦中落下。开满洁白花朵的花园,被厚厚的雪覆盖着。孩子们嬉闹
	着堆起一个个雪人。

		煤球的眼睛和钮扣,胡萝卜的长鼻子。雪人总是乐观的表情,仰天大笑的模样。这个冬天于是充满
	了轻松愉快的气氛。

		圣诞节来临前,街头商店的彩灯亮起来,又是闪闪烁烁的欢乐。\par
		等候着一份礼物吗。于是,常常去翻看邮报箱,期待着一张意外的包裹单,从远方寄来。\par
		冬天,于是没有了许多枯燥和无聊。

		可以对自己许愿,为了来年,为了心灵。可以充满期待地对全世界宣布,我要更勇敢,更幸福地生
	活。

		然后,让我像个孩子一样,拉上一脸无奈的你,去堆一个雪人,或者一堆雪人。\par
		让我们拉着手,相互搀扶着走过光滑的冰面,蹒跚得一如老去的模样。

		在雪的期待里,我度过着冬天。

		想买一张火车票,去海上看雪。想看落雪的大海,在夜晚的低低歌唱。

		写信给朋友,告诉他们,今晨我的门前积了几尺厚的雪。再换上长靴子,走出门去,在上边留下深
	深的脚印。

		也许,会捡一段枯树枝,在雪地上写下一行词不达意的诗。也许,会带上一把小黄米,撒在有阳光
	的地方,给小麻雀们吃。

		我一个人走在雪后的城,看天空被冻得宛若晶体的明蓝。拍下照片,拍下自己在雪地里的脚印。\par
		找一家小店,挑一个靠窗的位子,要一碗热面,慢慢吃下。\par
		我可以沉默,只用眼睛去发现,去记忆,去经历。


		我愿意去想象,它比真实总要美好太多。\par
		所以,不必要一张去往烟台的火车票。只需一个偶尔清醒,多数混乱的头脑。

		任由思绪飞扬吧,不花费一分的旅费。想象的世界,藏在一切实体世界的背后,开了小小的门,等
	你敲开。

		只轻轻闭上眼,又是明丽温和的春,又是草色空蒙的湖岸。\par
		而雪就要落下来,落下来,抚摸我们的城。\par
		你会在哪一个窗口,熄灭了床边的小灯,合上倦困的书页,静静听,雪的鼻息。\par
		我懂得四季的流转,那些繁华和凋芜。我明白雪花的心思,它们的爱与温柔。\par
		你是否也能够,在冬天的某个时刻,记起遥远的我,那渐行渐远的人。\par
		你是否也重复过许多相似的梦境。\par
		有微笑的雪人,有一路清晰过,却慢慢被覆盖的脚印。

		雪花融化在我的手心,也融化在生命最温存的片断。\par
		终于是清水一滴。归于天,归于地,归于等待着春光的湖。

		静听雪落,每个懂得的人,都会万般深情。

	\endwriting


	\writing{草木。女子}{2006年12月25日 ~ 12:32:40} %<<<2

		是寂然,却也是多情。

		魏文帝曹丕《柳赋》曾有言:“在予年之二七,植斯柳乎中庭。始围雨而高尺,今连拱而九成。”《
	世说新语》又载:“桓温北伐,经金城,见为琅琊时种柳,皆已十围,慨然曰:‘木犹如此,人何以堪!
	’”人生代代,江流不息,看这庭中的杨柳青青,古人未尝不生出子临川上般的时间之痛。那些消逝的时
	光,飞驰的生命,在植物草木的繁荣与凋败间显现出脆弱的形迹。于是,当人们开始懂得生命的不可逆
	性时,当人们在日月轮转中觉醒地意识到时间的有限,诗歌中对于人生与时光的悲叹和咏怀便从未停止
	。而对于生命体短暂无常的这一体验,又多是由自然界草木的荣谢而引发。草木的意象,也被赋予各种
	含义而用作诗歌的起兴等用。

		这一现象,早在诗经与楚辞中便有体现。诗经常以植物起兴,而楚辞中的各种植物,更是品目繁多
	,令人目不暇接。这与当时人们所处时代的自然环境及生活方式有关,也受到中国传统文化中多崇尚天
	人合一的思想渊源的影响。推己及物,以天地万物为一大世界大关联,物我同一,浑然一体,恰是中国
	审美传统中的重要方式之一。有名的庄惠之辩,对于游鱼之乐的问题,庄子便是凭着“天地与我并生,
	万物与我同一”的观点,而做出“鱼乐”的判断。庄子的判断,是打破了物我的界限,明显有别于惠子对
	立物我的看法,这也正是诗歌中欣赏中往往不可或缺的精神基础。以我之心,融入万物之心,于是天地
	浑然一体,无分彼此,万物之悲喜,便也是我之悲喜,我之情谊,便也感染万物之情谊。是于这样的情
	怀里,诗人骚客们在一株株玉树繁花前睹木兴叹,吟咏出一首首千古流芳的诗篇。这些诗篇从很大程度
	上有不止是表达了诗人一己一时的思想情感,而是概括和总结出了人生在世的许多根本困境。这样的诗
	歌,因为透析出了人们情感深处许多共同的体验,于是拥有着巨大的活力。很多诗歌都成为代代传诵的
	佳篇。

		如汉代宋子侯的《董娇娆》:

		\shortpoem{}{}{}
		洛阳城东路,桃李生路旁。\\
		花花自相对,叶叶自相当。\\
		春风东北起,花叶正低昂。\\
		不知谁家子,提笼行采桑。\\
		纤手折其枝,花落何飘飏。\\
		请谢彼姝子,何为见损伤?\\
		高秋八九月,白露变为霜。\\
		终年会飘堕,安得久馨香?\\
		秋时自零落,春月复芬芳。\\
		何时盛年去,欢爱永相忘。\\
		吾欲竟此曲,此曲愁人肠。\\
		归来酌美酒,挟瑟上高堂。\\
		\endshortpoem

		这一首诗中,诗人以一旁观者的视角观看花树飘零与妙龄女子采桑。这本身是两件毫无相干的事情
	。花朵自然性的凋谢,和女子正常的生活劳动之间本没有特别的关联。然而,在诗人的眼中,此两物却
	有巨大的相似性。那便是,花朵的飘零是因季节的更变,妙龄的女子也如这一树繁华,终会经不住时间
	的流转而与盛年告别。而这其中二者又有很大不同:更具悲剧性的现实是,花朵在秋天零落,待到来年
	春天,又是芬芳如旧,但女子的青春却只能随时光消散,而一去不返。诗人通过花树飘堕引发春光不永
	的感叹,又进一步以花朵与女子的对比,衬托出青春的短暂,欢爱的虚空。同样是美好的事物的逝去,
	人的身躯更经不住时间的消磨,所有的光华都是转瞬即逝,无可挽回。这样的对比之下,令人不无感概
	,不无悲叹。此诗更为巧妙的是运用了问答的形式表达诗人的想法。令人仿佛身临其境,亲见一位细手
	纤纤的采桑女子,款步花树之下,聆听着岁月的脚步,一寸寸逼来。

		与这一首诗颇为相似的,是唐人刘希夷所作《代悲白头翁》:

		\shortpoem{}{}{}
		洛阳城东桃李花,飞来飞去落谁家?\\
		洛阳女儿惜颜色,坐见落花长叹息。\\
		今年花落颜色改,明年花开复谁在?\\
		已见松柏摧为薪,更闻桑田变成海。\\
		古人无复洛城东,今人还对落花风。\\
		年年岁岁花相似,岁岁年年人不同。\\
		寄言全盛红颜子,应怜关死白头翁。\\
		此翁白头真可怜,伊昔红颜美少年。\\
		公子王孙芳树下,清歌妙舞落花前。\\
		光禄池台开锦绣,将军楼阁画神仙。\\
		一朝卧病无相识,三春行乐在谁边?\\
		宛转蛾眉能几时,须臾鹤发知如丝。\\
		但看古来歌舞地,唯有黄昏鸟雀悲!\\
		\endshortpoem

		这一首诗很明显受到了《董娇娆》的启发和影响。同样是通过桃李花的盛开和凋谢来引发感概,此
	诗却在《董》的基础上有了很大发展。首先,其咏叹对象不只是女子的青春,而引申到更广的范围,即
	整个人类群体。全盛红颜子,便是指处于壮年的男子。这红颜美少年,也曾在芳树下,伴着落花清歌妙
	舞,饮酒赋诗,而今却是白发苍苍,寥落孤单。少年境遇的前后对比,女子与花开花落的相对,都具有
	极大的艺术冲击力。诗人在表达主题上,显然比宋子侯更胜一筹,给读者更大的心灵震撼。其次,刘希
	夷的表达没有止于对时光的咏怀,而进一步引申到人生沉浮无常,世间繁华如梦的更深层次。当白头翁
	回首当年的往事,那些显赫一时的生活早已灰飞烟灭。花月正春风的日子,随那万点落红,消散不见。
	只剩下被时间消磨过后残存的身躯,拖住老病中的枕席和日月。诗人又把眼光突破了人生的界限,而放
	向漫漫的历史,那古来的歌舞地,多已是门可罗雀,一片狼藉荒芜。真正是,舞榭歌台,风流总被雨打
	风吹去。如云如烟的人生,如雾如电的沉浮,自古如是。《代悲白头翁》在相似主题下,于前人基础之
	上更上一个境界,挖掘出更引人深思的生命和命运现象,令读者不禁为之唏嘘垂泪。后世人对于落花的
	独特情怀,直至曹雪芹笔下著名的黛玉葬花,大约都与这两首诗的影响不无关系。

		两首诗中,同时出现了女子的形象。一个是采桑的女子,一个是坐看落花的洛阳女儿。女子的意象
	用于表达时光流逝这一主题,在古诗中并不少见。究其原因,大约是因为女子的衰老较之男子为显著和
	迅速,这才使历代文人都难免发出“恐美人之迟暮”的慨叹。女子又往往与思念和远征的主题相关联。女
	子的青春,与不知何时息止的战争,在时间上形成对比。青春是短暂的,而战争却漫漫无期。对于远征
	的丈夫思念,杂合着不安焦虑的心情,在诗歌史上反复出现,历代不止。这一主题,也被人们称为闺怨
	诗,成为比较独立的一个主题门类。“何时平胡虏,良人罢远征。”女子们的心情,通过诗人的纸笔,娓
	娓道出,流传千古。采桑女子的形象在此中也常常复现,这其实可以作为劳动妇女形象的一种概括。被
	征收去战场的多是最底层的人民大众,这些诗歌往往同时体现出人们在现实生活中对于和平安定生活的
	向往。张仲素曾写《春闺思》:

		\shortpoem{}{}{}
		袅袅城边柳,青青陌上桑。\\
		提笼忘采叶,昨夜梦渔阳。\\
		\endshortpoem

		渔阳是唐时的征戍之地,这首诗是一首思念远方丈夫的作品。柳树与桑树,在诗的开头引发兴叹。
	然后,画面转入一位提笼采桑的女子,这与《董》诗颇有类似之处。不同的是,这位女子凝神迟疑,时
	常停下手中的劳动若有所思。诗的最后点处原因,昨夜她梦见征战在外的丈夫。“提笼忘采叶”一句,令
	人联想到诗经中《卷耳》所描述的相似情形。“采采卷耳,不盈顷筐。嗟我怀人,寘彼周行。”筐篮本不
	难装满,却总是空空。原是因这采叶的人,心魂已飞越万水与千山,追随心爱的人一路而去了。这个采
	叶而不顷筐的形象,几乎成为了诗歌史上的一个经典。女子那凝眸沉思的模样早已深入人心,感动了无
	数读者。短暂的人生,易逝的青春,欣欣的植物,一年年的柳色如旧。只有归人未归。让一个个月光独
	照的晚上,一处处寂寞寒冷的楼台,做着闲潭落花的梦,咏唱着可怜春半不还家的惆怅不安。闺怨诗在
	表达时光流逝,人间无常的主题上,比其他的诗歌门类更具有感染力。这大概也与人们疼惜美好事物的
	心理有关。读者比较容易进入一种感同身受的欣赏状态,也便拉进了诗人与其的心灵距离,使诗歌所表
	达的感情更易接受,引起共鸣。

		对于时间的叩问,永远是诗人们咏叹的不变主题。只要有人生,有消逝的美好,这些疑问和感叹就
	不会停止。而落花的风依旧,时光的流逝依旧。天地无言,没有任何改变。但有谁能够真正做到无所动
	心,无所感触?这人间是否真如李白所言“草不谢荣于春风,木不怨落于秋天。”?当我们把心投入到这
	个世界当中,当我们愿意以自己小小的存在与万象同一悲喜与呼吸,就是选择了去追寻一种诗意的生活
	方式。用审美化的态度来观照万物,而不是把物我对立起来,冷静地观看。所以,在面对一棵大树的时
	候,我们也会像古人一样,引发了岁月流转的遐思,深深地感慨一句:“木犹如此,人何以堪?”所以,
	独立的落花下,微雨的天气里,燕子们的双翅会载上我们的心灵,飞入时光,飞入梦境,飞入一切飘零
	着的情绪。这些因草木而起的情绪,因时间而生的感叹,都是真实的,不只是那些纤细的女子,而是每
	一个人所面临的问题和困境。但它们又是美的。因为匆忙的消逝,而更加美好。

	\endwriting


	\writing{无奢求}{2006年12月26日 ~ 14:02:04} %<<<2

		给自己一颗糖果,来宠爱沉默的舌头。

		\blankrev
		也许,人从不该奢求生活。只应平静地观看,享用,或独自默饮。\par
		像品一盏茶,一杯酒那样,用唇齿轻触,让或淡薄或浓烈的滋味,越过舌尖,滑过喉咙,坠入身体的深处,而无丝毫声响。\par
		又好像,吃一只棒棒糖。要一口口,细心地舔过每一点甜蜜,却不能够一口吞下。\par
		这全部过程,仿佛是一场孤独的行旅。\par
		只是,我总是忘记了行李,一个人盲目上路。直至发现自己站在人来车往的站台,不知去向,才茫然失措。

		\blankrev
		却从未慌乱。\par
		因为渐渐懂得,没有人不是这样经过着生活。一切平凡的,却波澜壮阔的日月。\par
		看盛大的日出,安详的日落。听海浪漫过生命的沙滩,抚平凹凸的痕迹,留下年龄的光润。\par
		我们被反复打磨,如一粒石子。我们在各自的期许和挣扎里,慢慢获得着智慧,足以拯救我们,或者毁灭我们的。

		\blankrev
		心是间空房子,装满了回声。跺一跺脚,抖落一路上的风尘仆仆,求一处安宁的处所,安放自己,这灵魂,这身躯。\par
		我全部的工作和努力,原来只是去寻找这房间的钥匙。一枚晶亮的,插着翅膀的钥匙。\par
		当我找到了,便推开那扇门。于是,之后的生涯,我不再是奔波的旅人,而是无言的草木。\par
		自开自落,一场场春光,一年年秋风,听任自然的安排。让我心无杂念地盛开,再全无悲戚地谢落。\par
		我的心中,是花瓣坠地的铿然,是年华流水,鸣如环佩。\par
		天地在这里老去,时光在这里破碎,没有哀怨,只有寂静的轮回,绚烂如梦,开到荼靡花逝。

		\blankrev
		人不该奢求生活。所拥有的,都该心怀感激。\par
		因这世间,没有谁一定要善待你的义务,也没有一种获得,是我们可以轻易领受,而心安理得。\par
		没有什么,不是恩赐,没有什么,不是意外。\par
		所以,忧伤是一张矫情的脸,面目可憎。既然选择活着,就必定有所承担。\par
		那些无谓的情绪,怎么有资格出席生命的狂欢?

		\blankrev
		于是,田说,要主动地去感觉幸福,而不是徒劳等候着,幸福的事件来将我们袭击。\par
		那些事件的发生概率太小了。

		有时候,想寄一封信。收信人却是十年后的自己。问候天空,问候窗口,问候不失约的风季。

		又想寄一只棒棒糖。告诉她,慢慢品尝吧,你的生活。\par
		像舔一处无法愈合的伤口。

		那滋味,一定是甜的。

	\endwriting


	\writing{寂寞}{2006年12月29日 ~ 17:26:35} %<<<2

		月如何缺,天如何老。

		\longpoem{}{}{}
		园中野草渐离离 \\
		托根于我旧时的脚印 \\
		给他们披青春的彩衣 \\
		星下的盘桓从兹消隐。

		日子过去,寂寞永存 \\
		寄魂于离离的野草 \\
		像那些可怜的灵魂 \\
		长得如我一般高。

		我今不复到园中去 \\
		寂寞已如我一般高 \\
		我夜坐听风,昼眠听雨 \\
		悟得月如何缺,天如何老。

		\begin{flushright}
			——戴望舒《寂寞》
		\end{flushright}
		\endlongpoem

		冯至在诗里写道,“我的寂寞是一条蛇”。它把爱人的梦境衔来,像一只绯红的花朵。\par
		望舒的诗中,寂寞却如野草,在空阔的世界,听风,听雨,与天地一并荒老。\par
		诗人坐在原地,仿佛默语着天地玄机的哲人,了却了尘世,了却了纷扰种种,独与天地往来。\par
		一首没有人间烟火色的诗,在素白的纸页上开启着现实之外的境地,却不是想象,或不止是想象。\par
		正如诗人所说,“诗是由真实经过想象而出来的,不单是真实,亦不单是想象。”\par
		那是一件游走于现实与梦境边界的事。那是不可以轻易说出,却又无法在头脑中久居的话语。\par
		诗是流动的水,诗情是霎那里飞溅而起的水花。

		\blankrev
		让诗人们说起寂寞,为了美,为了爱,或者,单单为了生命本身。\par
		这是逃不去的真相。每个人的寂寞,如此分明地展露在一个个看似恒久,却转瞬消散的日月。\par
		我无法触摸风,我无法留存雨,我只有读这些远去的人们的诗,摸索着散发出墨香的纸页。\par
		书写着寂寞的人,已安眠在广漠的大地,皈依于自然的怀抱。\par
		阅读着他寂寞的人,从文字的缝隙里,侧身走入了,他那渐行渐远的世界。\par
		而光阴如旧,一寸寸移动着日影,一天天消磨着尘世。\par
		几十年的时间,隔绝着生死,却也令人对字句的寂寞,有了更深的体味。\par
		诗人长眠。这一年春天,我手捧纯白的雏菊立于他的墓前。\par
		每一夜,这小小的坟墓中,他是否仍然听着夜风吹去。\par
		每一个白日,他是否还在睡梦里,感受到细雨的缠绵。\par
		那个雨巷里的女子走远了。一把油纸伞,却撑起了多少人的愁思。\par
		诗人大约忘却了这个纷繁芜杂的人间,却在身后的世界里,引发了多少,对于生命的思索。\par
		寂寞。你说,寂寞永存。\par
		墓中的你,永恒地享用着无尽的安静。从人间到彼岸,你是否早已悟到,生死的全部奥义。\par
		只是,你没有说出。只用手掌,抚摸过你深爱的土地,只离开了园子,任由寂寞离离如草。

		\blankrev
		我们原都是这样如草的生命。生长在岁月的园子。等候着一种荒芜,或者,一种终归于寂的繁华。\par
		只应守寂寞,还掩故园扉。\par
		是谁对我说,一个人,首先要学会的便是耐住寂寞。在生活那些冰冷里,我们要有独行的勇气,一个人只身穿越。\par
		寂寞,不是一种羞耻,寂寞,是人生的常态。\par
		能够在寂寞中处之泰然,甚至有所收获,才是真正懂得生命的开端。\par
		轻掩上小园的柴扉吧,独自守住风雪的夜晚,让野草在院中萌生,让落花铺满了小径。\par
		一个个日子,我们度过自己的心,度过着困境。一个强大的灵魂在生长,知道我们看清,那些青春的彩衣,是怎样的迷离。\par
		该消隐的,已经消隐得毫无踪迹。园中的草,已如我们一般高。\par
		寂寞,恒久如不变的日月。它坚韧地生存着,郁郁葱葱,一派生机。\par
		寂寞,却也恰是最高的清静。\par
		夜晚的枕上,我不只听风,亦听那空山的松子落下。\par
		月出惊山鸟,我披了寒衣,独步在曲折的花径,数天外的星斗,一颗颗,像智慧的眼。

		\blankrev
		寂寞,是这样的草,用碧绿的纤纤身躯,淹没我们的心,我们的身躯。\par
		人在寂寞里,悟得天地的教诲,懂得月轮盈缺的轮回,和天地的深情。\par
		一并与天地荒老吧。\par
		诗人说,“我躺在这里,让一颗芽穿过我的躯体,我的心,长成树,开花……《致萤火》”。

		\longpoem{}{}{}
		把我们自己归还给世界。\\
		把寂寞的真相归还给世界。
		\endlongpoem

		% <todo: 别字: 随心的 -> 随心地 >
		在诗歌里,我们随心的导航,无拘束地遨游。想象与现实,是如此近了,却又仿佛如此远。

		\blankrev
		是谁把你从前的梦境衔来,夹在我今日的书页,如一朵绯红的花朵。\par
		在此刻一样野草离离的园中,我听到雨中的山果,灯下的草虫。\par
		幽人应未眠。\par
		诗人,却含着微笑睡了。

	\endwriting


	\writing{新开辟}{2007年01月03日 ~ 11:13} %<<<2

		很久了,遗忘掉这样一块田园。像遗忘一个人那样,决绝,却无法彻底。\par
		在msn上写字,不停地写,有点疲惫的我,却不厌其烦。\par
		正如我所做的许多事。仿佛一种暗中决定的规律和惯性,我只不顾一切地去做。\par
		还是听Faye的歌,同一张唱片,同一个排序。\par
		熟悉的,是她凉到透明的歌声。陌生的,是每一年的自己。\par
		站在河水两侧。对岸的自己,早已形迹模糊,不可辨认。\par
		看去年参加婚宴的照片。妈妈说,这一张真好,笑得多天真。\par
		是啊,那笑容,和身边上小学的小女孩,竟无二致。\par
		今年呢。\par
		我在哪里呢。\par
		我总是不知道自己的位置。\par
		也许,是在人生的某个点刻。也许,是在别人的梦里。\par
		也许,是在我自己,己渐荒唐的回忆。

		\blankrev
		效法小鹿,在这里说话。\par
		可能都是无关痛痒的东西。却是最真诚的。

		我在这,等待我的朋友们,听我说话,对我说话。

	\endwriting


	\writing{田愿意是桃汁,只愿意是桃汁}{2007年01月04日 ~ 19:39} %<<<2

		在mop找到许多水果的桌面。最喜欢的苹果,草莓,还有樱桃。\par
		原来,自己喜欢的都是有鲜亮色彩的水果。\par
		小时候学美术,颜色盒里最爱的也是柠檬黄。\par
		画在白纸上,是一抹亮而新鲜的感觉。好像可以闻到柠檬酸酸的清香。\par
		把它与湖蓝(也是田喜欢的颜色)调和,便是特别诱人的绿。\par
		属于阳春的草色。

		\blankrev
		前些天小鹿对我讲,她为每个朋友选了一种代表饮料。\par
		她自己是绿茶。而我是桃汁。\par
		田窃笑不已。桃子,又是我喜欢的水果……\par
		关于桃子的滋味,最先想到的是很早以前喝的一种叫做摩奇的软包装饮料。\par
		似乎只有几种口味。桃子口味的包装是淡粉红的,上边画着大大的蜜桃。\par
		那是我童年里很迷恋的饮料。\par
		带有芳香的甜味,现在回想起来,喝着饮料的自己,脸颊也好像桃子一样了,粉扑扑的。\par
		是多可爱的回忆。\par
		还没有柜台高的我,举着两元人民币,兴奋不已地说:阿姨,我要一个摩奇,桃儿的。\par
		这个故事,竟然让我想起一首歌:长大后,我就成了你……

		\blankrev
		现在,我成了桃汁。\par
		在桃子丰收的季节,田一天能吃掉七八个桃子。\par
		虽然,在吃的时候,时常也想起,自己的一颗乳牙当初就是被这种可爱的水果硌掉的。\par
		那是很远的夏天了。和小学的好朋友在院子里做作业。一边做,一边得意忘形地吃桃子。\par
		结果,一颗活动的牙齿,便被大大的硬桃硌掉了。(于是后来倾向于吃较软的)\par
		掉牙的故事,其实完全够写一本书的。\par
		书的名字可以叫《乳牙自传》。\par
		如果我是名人就好了。\par
		那样一定会有许多出版社排着队央求我把这本书出版。

	\endwriting


	\writing{钢琴}{2007年01月05日 ~ 10:38} %<<<2

		有点失眠的晚上,反反复复听George Winston的《四季》。流水一样的钢琴。\par
		对应着每个季节,唱片由不同的专辑选出,是一张精选。\par
		《辞冬》,《夏》,《秋》,《十二月》。最喜欢的,是《辞冬》这一张。\par
		专辑的封面是一片明黄的花田,背后是远天的蔚蓝澄碧。\par
		辞冬,英文的原文是Winter into spring。似乎是贴切的翻译。\par
		从冬到春,大约是一年中最冷静的一段时光。要我们用守望和期待的心去慢慢体会,慢慢度过。\par
		最早听这张唱片,是大一那年的冬天。\par
		一月星,二月海,海浪,倒影,雨…… 美丽的曲名,让人未闻其声,已是浮想联翩。\par
		在一个寒冷的冬天,躲在厚厚的棉被里,按动CD的播放键。让钢琴如水,灌注我的双耳,我的夜晚。\par
		聆听那些手指触动下的叮咚,清越而不失温存。\par
		好像,春天就这样在安静的乐声里绽开了。\par
		一月的星星是冰凉的。\par
		二月的海是沉默的。

		\blankrev
		很长时间,希望有一位会弹钢琴的朋友。\par
		若是女孩子,该会有披肩的发,柔和的口吻。若是男生,便有嶙峋修长的一双手,温文尔雅的举止。\par
		想和他们并肩坐着,让他们的手指抚过黑白的琴键,让我轻闭了眼睛。\par
		喜欢会弹琴的女孩子。喜欢她们春风一样的微笑,带着音符的韵律。\par
		更喜欢会弹琴的男生。喜欢那样的手,仿佛带着魔力的。\par
		于是,有一段时间对李云迪颇有好感。因为他弹钢琴,又常常演奏李斯特,而且,模样也与我想象中的钢琴家比较符合。\par
		该是有他那样的一双眼睛吧,有点犀利,却又是深情。

		\blankrev
		可惜我不是会弹琴的女孩子。更可悲的是手脚笨拙,协调能力极差。\par
		我也没有长头发,没有足够柔和的口吻,足够温暖的微笑。\par
		这个世界,在自剖的时候,总会惊异地发现自己如此多的缺憾。\par
		但也许,这也才是我们每个人真实的生存状态。\par
		期望着一种自己,接受着一种自己。\par
		后一个自己似乎永远追不上前者的步伐。\par
		对于自己的不满,是不是每个人都有的呢。\par
		这大概是人本性中反叛自我的一种体现。

		\blankrev
		钢琴作为乐器是比较大的了。不方便携带。\par
		于是,它成为了一种沉静的乐器。它总是坐在那里,不是么。\par
		要让演奏者走过去,带着些庄重,坐下来演奏。\par
		钢琴也是很骄傲的乐器。不像一支芦笛,一把提琴,可以随时奏响。\par
		你总须安静下来,仔细去聆听。\par
		有时候它是欢歌,有时候又是低泣的呜咽。

	\endwriting


	\writing{小心感冒}{2007年01月07日 ~ 21:02} %<<<2

		最近感冒的人很多。报纸上几乎每天都有关于流感的报道。\par
		大约都是说,医院人满为患,请市民们注意添加衣物,少去人多的地方。\par
		小小的病毒,竟可以轻易放倒一批又一批人。\par
		怪不得,有人推测病毒是地球未来的主人。\par
		那人类去哪呢?\par
		和病毒的战斗,是由来已久的了。人们发现了如抗生素一类的武器来对付它们。\par
		但这些武器的威力却渐渐减弱。\par
		病毒适应环境的能力,远远超出我们的想象。\par
		很多药已经不那么管用了。病毒的更新换代,与时俱进,实在令人惊叹。\par
		越微小,却恰恰是越强大。\par
		用病毒的例子来诠释这一道理是再合适不过了。\par
		天地的奇妙就在于,它有那么无限的微观世界,又有漫无边际的宏观宇宙。\par
		都是我们无法到达,无法穷极的领域。\par
		人类原来只是占据了一小块有限的领地。\par
		在人能够感知,能够把握的范围内,自以为是地生存。\par
		而事实上,我们并不比病毒强大,不比山石恒久。\par
		于是不喜欢听人把自己称为万物之灵。\par
		也不赞成有思想的动物就比没思想的生物高级的观点。\par
		众生平等,不是人类出于慈悲而发的一句宏愿,而是事实的真相。

		另外:要注意身体哦,别被病毒打败…… \verb|^_^|\par
		最近的田在流鼻涕。

	\endwriting


	\writing{思想是一个美人}{2007年01月08日 ~ 18:52} %<<<2

		那天乱翻书,读到废名先生的诗《十二月十九夜》:

		\longpoem{}{}{}
		深夜一枝灯,\\
		若高山流水。\\
		有身外之海。\\
		星之空是鸟林,\\
		是花,是鱼,\\
		是天上的梦,\\
		海是夜的影子。\\
		思想是一个美人,\\
		是家,\\
		是日,\\
		是月,\\
		是灯,\\
		是炉火。\\
		炉火是墙上的树影,\\
		是冬夜的声音。
		\endlongpoem

		读罢,立即用钢笔像个初习文字的小孩子一样,一笔一划细心抄录,丝毫不敢怠慢。\par
		喜欢那遥远冬夜的炉火,喜欢诗人静默于天地的玄思。\par
		爱这一句:思想是一个美人。

		眼前于是是高山流水,月夜星空,是寒意袭袭的星光,是繁花锦簇,是鱼龙潜跃。\par
		思想的光芒,如一炉寒夜里的熊熊火光,映亮面颊,温暖整个世界的冰凉。

		这里是安静吧,是独自的嗫嚅吧。\par
		诗人该是面朝着灰蒙潮湿的窗口,任精神的飞马驰骋,上天入地。

		思想是那红袖添香的美人,美目流转。

		在这个俗世之外,觅得一处清寒的境地,读一卷书,写一帖字。\par
		让梦里有松涛阵阵,有落花轻盈;与仙人下一局棋,看手中的斧头锈蚀腐烂,日月如梭。

		人却如何逃开,尘嚣烦恼,滚滚红尘?\par
		在思想的怀中藏匿,在思想的照耀里沉醉吧。

		文字的脚步是这样轻。\par
		一步步,踏过雪,踏过生命的微风。\par
		乱翻几页书,读几句安静的诗。\par
		在冬夜,我的世界便也有炉火,有夜的声音。

	\endwriting


	\writing{这一杯苏打绿}{2007年01月09日 ~ 16:18} %<<<2

		在06年的夏天,遇到一个清爽的声音,署名Sodagreen。\par
		好像饮料的名字。一杯冒着泡泡的苏打水,透明的绿,散发薄荷的淡香。\par
		没有更多的了解,就下载了全部的曲目。\par
		然后,坐在6月被阵雨澄净的树荫下,一首首细心聆听。\par
		明亮的旋律,独特的嗓音。是属于夏天的歌声。\par
		适合在混合了青草味的傍晚,伴随轻快的脚步。\par
		让心情清清凉凉,瞬时间离去一切灰暗与低落。

		\blankrev
		喜欢他们的名字,苏打绿。\par
		喜欢绿。\par
		让我想到远处的原野,满山的树木,一片片舒展的叶子。\par
		想到许多一样是下过雨后的夏天,想到蜻蜓们轻盈的双翅,掠过我的肩头。\par
		那是一些与歌曲全然无关的事。却在歌声里肆意蔓延开来。\par
		就要淹没我,淹没双眼,淹没世界。

		\blankrev
		秋天的时候,苏打绿出了第二张专辑,小宇宙。\par
		还是那一杯苏打绿,清亮明晰。\par
		反复听一首叫做《小情歌》的歌。记住那句歌词:

		你知道就算大雨让这座城市颠倒

		我会给你怀抱

		\blankrev
		大约是太善感的人。眼前竟又是大雨滂沱的城市,是被围困的窗口。\par
		想象那一种大雨中的寂寞与无助。\par
		也想起,许多年前的夏天午后。想起自己告别的身影。\par
		一样是在雨里,一样是颠倒的城市,令人不知所措。\par
		在许多歌声里,我们好像遇见了从前的自己。\par
		回忆却总是零散的碎片。

		去聆听明亮欢快的声音。感觉着无处不在的拥有和幸福。

		其实,我只是喜欢沉在歌声注满的湖水中,安享平静的心。

	\endwriting


	\writing{关于田}{2007年01月10日 ~ 18:05} %<<<2

		突然发现田是个可爱的字。

		《说文解字》上说:田,陈也。树榖曰田。象四囗。十,阡陌之制也。

		田,原本就是一块地,种上庄稼,一天天日出而作,日落而息。田,是农耕生活的形象图形。一幅
	俯瞰的劳动画面。这四口之间,好像可以望见农人头戴斗笠的身影,可以想象禾苗稻谷的青翠模样,一
	派生机盎然。

		有一种生活,叫做晴耕雨读,在我的理解便是晴天耕种,雨天读书,不知是否有偏差。这一种生活
	,有田园的青草气息,又饱含了书卷的墨香。与世无争的隐者一般,躬耕于田,却又洞悉着天地奥妙,
	体味着人间悲喜。或许会种一丛菊花,在黄昏的斜晖里与同道的好友把酒推杯,让暗香流泻在柴扉篱门
	,浸满就要初升的月色。


		田,又是古诗里那清丽单纯的一句:莲叶何田田。

		千年前,水畔的女子吟唱着,采满一船的莲子,满载而归。千年前,莲叶间的小鱼,自由嬉戏,鱼
	戏莲叶东,鱼戏莲叶西,鱼戏莲叶南,鱼戏莲叶北。

		天是澄碧的,水是澄碧的,人的心是含着相思,惆怅而甜蜜的,鱼是自得其乐,不知烦忧的。

		田田,是莲叶的姿态,是舒展的神情。

		田,还是一扇窗。小时候画画,小白兔们住的房子上总会开这样一扇窗。它很简单,却是最形象的
	表达。它很朴素,却是孩子们眼里的真实。现在的儿童画上,还是开着这样的窗子吧?我们心中的小白
	兔,都是住在有田字窗口的房间。

		田,是四个格。是学写字的时候,我们作业本的图案。那种叫做田字格的本子,你已经多久没有用
	过了?为了写端正的字,我们握住铅笔,一笔一划地,在田字格里描画。

		田,发音时用舌尖轻触上颚,田,便有了轻巧晶亮的滋味,像一颗水果糖。

		于是,我愿意有这样一个名字,只这一个可爱的字,田。

		我喜欢被朋友们这样叫着,喜欢在信的结尾,署上这样一个字。

		田,是横横竖竖组成的图形,却又是一张画,一首诗,一种淡远的生活,一个甜蜜的声响。

		田,还是我。一个爱写字的女孩子。

	\endwriting


	\writing{理想的爱情是哑的}{2007年01月12日 ~ 10:26} %<<<2

		高圆圆新片的海报《男才女貌》贴到了知春路上。据说,她在片中饰演一位聋哑女孩。

		从《十七岁的单车》到《青红》,我所知道的高圆圆都是出现在王小帅的镜头里,于是,都是青春
	成长的主题,一副青涩涩的模样。

		喜欢《青红》里不停下着的雨,喜欢被偷偷穿上的皮鞋踏过那条灰蒙蒙的街,喜欢吹口琴的男孩子
	伶仃的身影。

		青红,好像就是青春的色彩。微微泛起的悸动,被羞涩的心包裹住,不轻易露出一丝痕迹。

		那是一部悲剧。\par
		故事的结尾,男孩被送上死刑犯的刑车。一个年轻的错误,成为他再无机会挽回的罪恶。\par
		而年轻的爱,却从没有过错。只是,他们并不懂得,那终久是怎样的一件事。\par
		是还没有来得及懂得吧。就这样在颠簸的山道上悄然结束了。\par
		青红,是一个女孩懵懂而矜持的青春,是一个男孩痴情万般,却无从宣泄的挣扎。

		簌簌的雨花还在下。从80年代的故事,落到银幕的光影流动里。

		多少人的爱情,在我们还不懂得的年纪里已无声埋葬,多少人,不曾真切地爱过,就匆忙地老去了。

		如果有一天,我们能够骄傲地对什么人说,我爱过了,真正地爱过,大概也是一件无比幸运的事。

		这一次,影片里是一位聋哑女孩。

		似乎许多电影都喜欢把女孩子设计为聋哑人。很早以前,看过一部陈小春演的电视剧,女主角便是
	每天比划着手语。

		前不久,看韩国的《悲伤电影》,其中的妹妹也是在一场火灾后成为了聋哑人。

		她没有在成为聋哑人后而自暴自弃,却依旧是活泼可爱的女孩。

		女孩的脸在大火中留下了疤痕。她便在游乐园担任扮演卡通人物的工作。每天戴上可爱的娃娃头,
	蹦蹦跳跳,给人们带来欢乐。

		她爱上一位在游乐园为人作画的画家,却没有勇气以真实的面貌面对他。直到画家即将离开的那一
	个雨夜。

		那天,她终于摘下了头饰,答应他为她画像的请求。所有的灯光都亮起了。眼前,是一个美丽的女
	孩,虽然她不能说话,只能够那样静静地坐着,默默注视着你的双眼。那一道疤痕已经不再重要了。

		最动情的,大概便是这无言的凝视吧。也许,这也正是导演们的良苦用心。

		爱情,本便是这样一件安静的事,无需言语,更无需喧哗。\par
		让人从目光里读懂一切。而不是从口舌间获得爱的消息。\par
		理想的爱,绝对是心有灵犀的,理想的爱,使诗歌和誓言都显得苍白可笑。\par
		爱,从不必矫饰。于是,一个不会说出一句话的女孩,仿佛成为了爱情的精妙代表。\par
		她是这样美的。又恰是如此安静。

		\blankrev
		原来,理想的爱情是哑的。

	\endwriting


	\writing{不是为了生气}{2007年01月15日 ~ 20:02} %<<<2

		最近看到一则小故事:一位禅师很喜欢兰花,平日里花费了许多心思培养寺院中的兰花。

		一次禅师要出外云游讲法,于是嘱咐寺中的小和尚,要悉心照料好兰花。不想,一日风雨骤至,十
	几盆兰花全部凋残。

		小和尚很害怕,整日担心怎样向师傅赔罪领罚。禅师回到寺院,闻知此事,非但没有责怪,反而召
	集弟子说:“我种兰花,一来是用来供佛,二来是为了美化寺内的环境,不是为了生气而种兰花的。”

		我不是为了生气而种兰花的。只一句,便拨云见日,碧空清朗。禅师明慧的心,比兰花的清芬更令
	人怡神爽气。

		智慧,是存在于最日常的点滴中。智慧,从来是在如此微小的不经意间,显露出真实的行迹。

		是怒发冲冠地责怪和怨恨呢,还是洞见到事理的分明,让负面的情绪瞬息里风流云散?这是一个莽
	夫与智者的分别。

		能够气定神闲地面对无妄的灾祸,能够心存正念地体会眼下的生活,需要一颗通透晶亮的心,去解
	悟,去了然这世间众生的悲喜哀乐。

		不是为了生气。多少时候,我们能够在情绪冲撞的尖峰处,保持足够的清醒?多少时候,我们任由
	了自己的心,陷入无止境的悲伤或愤怒?

		年少的我们,总是无法降服心中的小兽,让它尖利的角横冲直闯,刺伤了我们自己。

		在情绪溢满的当口,问自己一句,我是为了生气而做这件事的吗。然后,大约便能够平复了躁动和
	不安,把自己从愤怒的悬崖拽回。

		那一头的乌云,也便无声消隐,惟余朗照的青天,悠游的流云。

		这是一件看似简单的事,却很少有人能够真正做到。只是做到孔子所言“不迁怒”,便已困难,更何
	况“不怒”。而心中的火舌,一喷薄便是火山风雷,情绪的海水,一涌出便是滔滔汪洋。灼伤的,最终是
	我们自己原本清澈明净的心境,覆没的,永远是平静安宁的生活。

		不轻易愤怒和纵容无谓的情绪,不代表麻木无知,无所表情。这是在穿越了层层云雾后,望见了生
	命的慧心,这是在纷繁的世界里,认清了人存在的真实。

		要记得,我们不是为了生气而生活的。

	\endwriting


	\writing{流萤}{2007年01月16日 ~ 09:52} %<<<2

		喜欢杜牧的那句:轻罗小扇扑流萤。喜欢凉如水的夜色,还有幽幽发亮的痴情星座。

		提及了流萤,便总会神思飞扬,仿佛眼见了那一粒粒微光,飞舞旋转,拥着夏夜的跫响,拥着一场
	千古的梦。

		玉溪生一首《隋宫》,有句言:於今腐草无萤火,终古垂杨有暮鸦。

		又复是萤火。世传隋炀帝曾命人捕捉萤火虫,待夜晚放飞,以代烛火之用。扬州还建有放萤院。

		荒淫的帝王,总有许多取乐的花招,令人目不暇接。

		杨广大约也是个心存浪漫,充满天真的人。不然,怎么想到开凿运河,乘舟游玩,又下令在两岸遍
	种垂柳。

		不然,怎么会爱那漫天的流萤,与星辰的光芒交相辉映。\par
		他是不是也喜欢坐在如水的夜色里,喜欢吟一首诗,吹一支幽曲。

		据说,隋炀帝确是个自视甚高的才子。可惜的是,他做了皇帝,于是,浪漫遭遇了政治,终于酿成
	身死国灭的悲剧。

		倾覆的王朝,相似的故事,要么是美人惑国,要么是帝王暴虐无道。\par
		更多的时候,也许是史官们与后世者开了个玩笑,用漫画似的典型嘴脸,描画出亡国之君的面貌。\par
		而事实上,他们不过是失败的政治家罢了。

		坐在龙舟上的隋炀帝悠悠荡荡,诗情画意的滋味,早已淹没了“水可载舟,亦可覆舟”的训诫。\par
		于是,千里荡漾的温柔春水,瞬时里变幻成洪水惊涛,势不可挡。\par
		土崩瓦解的隋帝国,埋葬在一片腐草遍生的荒原。\par
		隔江陈后主的玉树后庭花还在被轻轻唱起。亡国的历史,一次次重演。\par
		张丽华死于刀下,多少年轻美丽的生命死于刀下,一起做了政治的陪葬。\par
		从艺术的角度观看,隋炀帝是一场悲剧,陈后主是一场悲剧,张丽华是一场悲剧。

		流萤漫天,许多的风云,一夜流散无迹。\par
		亡国者的疼痛,最深刻的莫过于在李煜的词中。\par
		多少恨,昨夜梦魂中。还似旧时游上苑,车如流水马如龙,花月正春风。\par
		许多梦,就这样在春风的爱抚里悄然消逝了。安静的悔恨或者安静的悲叹,都化作一杯酖酒,要你独自默饮。\par
		一转眼,便已是百年。而星光依旧,山川依旧。\par
		你不会知道,很多很多年以后,有人会在如水的夜色里,迎着电力的灯火,读你的故事,你的诗。\par
		生命莫不是一粒流萤。

	\endwriting


	\writing{宴会后的芜杂感受}{2007年01月22日 ~ 21:21} %<<<2

		生活总是仓促,来不及细细咀嚼细枝末节的滋味,便已是风流云散。\par
		一月,我读着日期的名字,语气轻轻。

		昨日的烟霞,却已苍茫如梦。

		总是一场场喧哗欢聚的筵席,又一场场人去楼空后的寂寞。\par
		谁陪伴谁,倚住怎样一幅肩膀,默默看日升月落,流水落花。

		玻璃窗上是满布的雾气,用手指画一朵小花。想很多年前的自己,一样的窗口,一样的双手。\par
		画着同样的图案,却怀着别样的心。\par
		告诫自己,不要敏感于时间的痛感。而那一把无形的刀刃,依旧切割着,我难以把握的知觉。\par
		听到分秒的呼唤,在钟表上发出。就这样,我们不断地跋涉在无边际的世界。\par
		它是如此辽阔。又如此狭小。\par
		我只拥有,这呼吸的片刻。比秋叶的飘落更为安静的一生。

		被水雾模糊的小花,像那些渐渐模糊的面孔一般,沉入这个冬天的最深。\par
		如冰下暗中穿梭的游鱼。是自由,又是百般迷惘。\par
		有时,也想在一个冬天,潜泳入湖水的深处。做一只从不会声响的鱼。\par
		你只须想象,却永也无法捕捉我真实的行迹。\par
		我要躲起来。\par
		在所有人的视线之外,在所有故事的记忆之外。

		于是,我可以是无所牵挂的人了。

		像一个云游的行者,观看世间的风景,嘴角隐着恬淡的笑。\par
		让我时不时停下来,低头默想,如有所思。\par
		让我偶然路过桃花灼灼的村落,看山泉泻落,如九天银河。

		生活,应该是无哀伤的一首诗。\par
		是与爱人并肩而坐,数远天的云朵,夜晚的星辉。\par
		是独自迎着如豆的灯火,读着,写着,让思想盛开成一座花园。\par
		它却总是仓促。\par
		你刚刚举起这一杯温热的酒,手间便已消散了温度。\par
		我们的日日与年年,从这一端注入生命的容器,却又悄悄从另一端流失散尽。

		“今日良宴会,欢乐难具陈。”

		遥远的一桌筵席之上,诗人唱起人生寄一世的哀叹。\par
		而原本这都是不足以我们去为之神伤感怀的。\par
		宴会之上,推杯换盏的醉意,是人间的繁华演出。\par
		真正的世界,在醉意朦胧的眼前,早已寂然落幕。

		有谁不是生命的饮者,有谁不是醉这一场春秋大梦。\par
		所以,喜欢这一句:醉笑陪君三万场,不诉离伤。

		一月,我走失在自己的人间。

	\endwriting


	\writing{望}{2007年01月24日 ~ 19:07} %<<<2

		有时,我情愿是永远的观望着。只肆意去想象,而不迈出半步渴望的步伐。\par
		像一个站在河畔的游人,久久地站立,看水的凶猛或温柔,却不寻觅一叶摆渡的小舟。\par
		是怕对岸的风景,远没有遥望的灿烂,是怕走近了,反而惊扰了那隔岸的雾色。\par
		是不愿让远处的火树银花,在瞬间里化作了灯火阑珊。

		观望的距离,让风景成为风景,因想象而获得完美的可能。\par
		一位著名的汉学家一身却未到过中国的原因便是,他希望在心中保留对于这个遥远国度的原始想象。

		很多地方,大约是不必亲身到达的。\par
		也许,只是去凭空地想念,反而比真实地踏上那一片土地来得美好。\par
		正如,往往越是深爱的,越是不忍触碰一丝一毫,越是小心翼翼,万般谨慎。

		05年的春天,和母亲一同到扬州旅行。\par
		现在想来,却只是对于扬州的伤害。那个诗词里妖娆曼妙的扬州,永远地不见了。

		二十四桥的风月,不如留在玉人的箫声里,桥边的红药,不如开放在冷月的无声。\par
		瘦西湖的瘦,原本是无尽风姿的遐想,却被游人如织的嘈杂搅扰得唯余艳浮的繁华。\par
		后来,我只想忘记真实的扬州。不是它不好,只是,它与我梦里的扬州全然是两个模样。\par
		我想那烟花三月,孟浩然的广陵。我想那枕着云烟,欧阳子的平山堂。\par
		如果,我只是翻开一卷卷书册,去品读,去想念,一个未曾亲临的扬州,大概便不会有今日的遗憾。\par
		我梦里的扬州亦不会破碎,它会完整地,飘着脂粉的芳香,落着春江的花瓣,一夜夜造访我的睡眠。\par
		它现在是一座欣欣向荣的城市,至少以现代化的角度来看,它是在喜人地发展。\par
		但在我的世界中,它与扬州却是格格不入,它的繁华,却也恰是它的荒芜。\par
		这世上于是有两了两个扬州。一个在火车的终点,另一个在想象的起始。

		如扬州,太多被我们诗化过了的城市和古迹,是不宜走近,只堪思念的。

		喜欢那些能够引人遐想的地名。却也难免失望。\par
		樱花西街,没有樱花树的落英缤纷。名为百花深处的,也不过是一条平常无奇的胡同。

		地图上充满色彩与风景的名字,跳到现实的面前,活生生地向你展露出的面貌,往往正如我们生活
	一般平淡。失望是过多期待后的一种正常结局。

		却依然愿意固执地相信,在遥远的时代里,它曾拥有不凡而美丽的过去。

		双榆树,是我从小便熟悉的地名。一条乏味的小街,几栋普通的住宅楼,和北京许多的地方没有分
	别。

		偶然却读到,清代词人纳兰性德的父亲纳兰明珠,曾在这里为爱子建造书楼,以供其进学研读。\par
		而双榆树的名字,又是源自村口的两棵大榆树。

		貌不惊人的双榆树,却掩藏着词人的足迹,那一个才情过人,妙笔生花的纳兰,曾在这里轻轻走过
	,驻足,提笔写下清丽的辞章。

		他书楼上的灯火,曾照着而今的土地吧。他曾见那两棵大榆树的绿荫,在夏日的蝉声聒噪里,散下
	一方清凉吧。

		榆树早已不见,书楼也已消失,动人的诗词却永存世间,被后来的人们陶醉着,沉迷着,赞叹不已。

		因为是消亡得不留丝毫踪迹,才让人能够穿越历史的长度,去遥想到当年的风貌,仿佛见了他走过
	的身影,和那流溢墨香的时光。

		现在,经过双榆树,我便总想起纳兰,好像我们是百年前相识的朋友一般,那样亲切而熟悉。\par
		在向榆关的路途上,他写下:风一更,雪一更,聒碎乡心梦不成。故园无此声。\par
		故园情结,在飞雪里漫卷天地。\par
		此刻,恍若旧地重游的我,却又像带着百年前的心,来探望故园的风雪。

		这些埋没入历史,埋没入记忆的美丽。

		樱花西街,该是曾栽满樱花的。每到春时,必是花树璀璨,粉云似梦。\par
		百花深处,应有幽人居住,他喜种兰花,在闹市中深居简出,过着心远地自偏的大隐生活。\par
		人心是最神奇的魔法。任你去想象什么,便有了什么。

		于是,我愿作永远的观望者。我愿沉入诗的扬州,一觉不醒,不止十年的酣梦。\par
		让美丽的,保有它在心灵中最初的形貌,如一枚不改变的朱印。\par
		让平凡的,突破现时的双眼,在时间的彼端,在幻想的此岸,觅得另外的存在。

		从不去搭乘一叶摆渡的小舟。\par
		只为对岸的火树银花,美得令人销魂。

	\endwriting


	\writing{空山}{2007年01月26日 ~ 21:17:38} %<<<2

		空山,一场幻觉。

		\blankrev
		也许,每个人都是一座空山。\par
		没有猴子,没有大王,只有我们自己,或者,连我们自己也被溶化在山风凛冽。

		\blankrev
		每个人都是一座不言语的山。你想你是溪流,便成为溪流,你梦你是落花,就化作落花。\par
		任月光来去,将连绵的身躯照成苍茫。让天地这样肃穆地存在着,一场场雷电,一夜夜雨雪,漫漶了生命的碑刻。\par
		你在那里静卧,从春山的笑靥,到冬山的睡梦。看山坡上的青草绿得忧伤,让空翠的树木沾湿了衣裙。\par
		每个人把自己在这山中安放,如一粒白石,一阵松涛。\par
		没有人语,唯有尘世外的花树,自开自落。那便是我们心底的安详。\par
		你必得有这样一座空山,一个永是寂定的去处,才不致在人的行走中茫然若失。\par
		山上没有路,没有白云生处的人家。\par
		山上是草木的蔓生,是灵魂,在悬崖与陡壁上的生长。\par
		而所谓生,有哪一刻又不是崖上的徘徊,我们在危险中望见风景,在恐惧里懂得欢乐。\par
		最美的花,总是开在险恶的岩石。

		\blankrev
		也许,人间也是一座空山。\par
		幽篁里的山鬼,彷徨在芳菲的寂寞。孤云在唱歌,孤云在飞逝。\par
		人是山中的风丝,高飞的众鸟。一些是高傲的,一些是谦卑;一些是从容的,一些是仓皇。\par
		空山无言,时光无言。\par
		一季季的风烟,亲见着日升月落的轮回。玫瑰色的霞光熄灭了,如一个个曾经美丽过的姓名。\par
		好像山的默声,山的无情。\par
		人间在这里,一样的无言,一样的寂寞。看顽石磊磊,葛生蔓蔓。\par
		子规的啼唱,是泣血的思念,是一场遥远,不可触摸的梦。\par
		夜晚泻入山林,多少人睡了,多少人却在山风的呼号中独自失眠。\par
		写一首诗,寄一枝梅给远方的友人。告诉他,此刻的窗上,星辰朗朗,竹风清丽。\par
		许多的多情,需要这样一座空山,一个独往又独坐的世界。

		\blankrev
		也许,宇宙亦是一座空山。\par
		上下四方曰宇,往来古今曰宙。月亮是山中的池潭里的小洲,群星是散落山野的花朵。\par
		它们开放,它们明亮,它们迎着永新的山,在漫无边界的未知里盛开。\par
		此时,人已微不可辨,人成为一粒比尘埃更细小的尘埃。\par
		生命与非生命之间,又还有什么分别。当这座山大至无形,一切的区分都显得可笑。\par
		美与丑,得与失,聪慧与愚钝不过是人无知的判断。\par
		宇宙是这样一座空山,容纳了时间的长,空间的广,无所遗漏。

		\blankrev
		我总与空山对坐。\par
		相看两不厌,或许转瞬便已百年。\par
		那里从无具体的人,一个个自己,随于大化,归于万象。\par
		空山,这样容纳了我,却始终未发一言。

		\blankrev
		我想我是溪流,于是,我是溪流。\par
		我想我是落花,于是,我是落花。

		\blankrev
		我选择沉默,在只能独居的山中。

	\endwriting


	\writing{狡兔三窟?}{2007年01月29日 ~ 19:05} %<<<2

		这两天发现msn已经基本恢复了。试着贴了一篇日志上去。\par
		还是比较喜欢msn的界面。可以用好看的图片,风格也比较简约。\par
		但愿不要再出现什么问题。\par
		这样才能保证我的花田不至荒芜。\par
		写博客的生活,从04年夏天开始,也便是高中毕业的那一年。\par
		渐渐成为了一种依赖。这种依赖,让我离不开网络和文字。不知是一种幸福,还是悲哀?\par
		日子在这里留下踪迹,让回忆丰盈起来,而不单单是恍惚的一些色块,形象,或气味。\par
		花田恢复了,却不会因此空置下这块亲爱地。\par
		想在这里写一些生活的外壳。\par
		用简单的字,为自己记一册流水账。

		欢迎来花田做客。
		%http://sui-liang.spaces.live.com/

	\endwriting


	\writing{爱不会错}{2007年01月31日 ~ 17:13} %<<<2

		杨乃文的新歌《女爵》,一贯冰凉苍茫的歌声。\par
		要在冬天聆听的声音,要迎着凛冽,一个人陷落在每个音符的细节。\par
		在路上,或者在独自的房间,她的声音伴随着一颗冷却的心,慢慢呈现出透明的光泽。

		最早听杨乃文,还是在高中。从音像店买回她的cd,之前却未听说她的名字。\par
		只因被封面的歌名吸引:爱上你只是我的错。\par
		爱会有错么。我不懂得爱情。却固执地认为,爱从不会错。

		她唱:

		\longpoem{}{}{}
		爱上你只是我的错 \\
		爱情是一种无止尽的痛 \\
		离开我现在就走 \\
		我宁愿寂寞
		\endlongpoem

		从决定爱的那一刻起,我们就该做好接受全部后果的准备。\par
		无论,那是悲伤,还是幸福。\par
		因为,那同样是我们的选择。

		正在播放的,是《女爵》中的《今天清晨》。

		\longpoem{}{}{}
		不忍心叫醒你 ~ 不应该触碰你 \\
		现在你的世界 \\
		有没有我?\\
		等光线更清晰 \\
		照亮你我身体 \\
		我紧闭着双眼 \\
		忧郁坠落

		我们的幸福 \\
		我们的承诺 \\
		我们的明天 \\
		都隔成两个梦

		不忍心叫醒你 ~ 不应该触碰你 \\
		我想你的今天 \\
		将没有我 \\
		你和他的黎明 \\
		是美丽的风景 \\
		我睁开一双眼 \\
		冷静沉默

		我们的幸福 \\
		我们的承诺 \\
		我们的明天 \\
		都隔成两个梦

		今天清晨你爱我吗?\\
		今天清晨你在乎我吗?
		\endlongpoem

		身体虚弱,头疼,昏睡了整天。

	\endwriting


	\writing{生日快乐}{2007年01月31日 ~ 17:35} %<<<2

		亲爱的,生日快乐。\par
		祝福你的生活平静而明亮。

		\blankrev
		感谢你的温柔和痴呆。\par
		感谢恰恰好的我们。\par
		让田是幸福的。

	\endwriting


	\writing{由企鹅想到的一些事}{2007年02月02日 ~ 13:56} %<<<2

		那是一个被冰雪覆盖的世界,有狂暴的风,冷冻的海水,和漫长的极夜;那也是一片婴儿般纯洁天
	真的大陆,有如诗如幻的极光,有飞行的雪花,洁白的峰峦,还有帝企鹅此起彼落的歌声。

		\blankrev
		它们说,企鹅就一定要会唱歌,不会唱歌的怎么算是企鹅?而它,偏偏是一只不会唱歌的小企鹅。\par
		故事就这样开始,关于一个被命运捉弄的小可怜虫的传奇。

		\blankrev
		它不会唱歌,却有一双会跳舞的大脚。这却并没有给它带来好运,反而被视为是一种奇怪的行为,而遭到禁止。\par
		小企鹅只有在无人的角落里,偷偷做自己喜欢的事情,一个人孤独地跳舞。\par
		有多少人是如它一般,不被自己所在的环境所理解的呢?\par
		没有人看到小企鹅身上的特殊才华,因为所有人都认为,只有唱歌,才是一个企鹅该去做的事情。但是,帝企鹅们不知道,就在它们生活的不远处,另一个种群的企鹅却恰是以舞蹈作为衡量能力的标准。\par
		当因不会唱歌而被同学们冷落的小企鹅来到它们面前的时候,当它迈起自己的大脚,欢快起舞的时候,没有人不为之赞叹。\par
		它们认为眼前的小企鹅一定是个万人迷,却不会相信,它竟然会在自己的族群受到漠视和排挤。\par
		不一样的评价标准,使小企鹅遭遇到两种截然不同的眼光。

		\blankrev
		企鹅的故事是人类想象的一个寓言。人们渴望在它的身上,找回自己世界的失落。一样多的误解,一样多的非议。

		企鹅真实的世界里大概会简单许多,单纯许多。它们在寒风里站立,紧紧怀抱住蛋,把全部的温暖都留给即将降生的孩子。\par
		那一种执着,那一种期待,在漫无边际的寒冷和黑暗中发出令人动容的光芒。\par
		企鹅是这样一种貌似笨拙,实则坚韧无比的动物啊。\par
		看到它们那逆风挺立的姿态,我的心被缩得紧紧的。就在这个星球上,就在海的那一岸,生存着这样一群勇敢的生物,在与冰冻的世界抗争中生生不息。

		\blankrev
		动物园中从未见过冰原的企鹅不会懂得生存的欢乐,不会懂得一条鲜鱼的美味。它们生来便在这人造的“家园”中了,看惯了壁画上的冰川和白雪,却从未感受过一丝寒风的气息。\par
		企鹅们无知无觉地度过着并无惊奇的生活,吃下饲养员投下的食物。\par
		它们衣食无忧的生活,却正是人类制造的巨大悲剧。\par
		于是,常常觉得动物园是一个悲伤的地方。\par
		不愿看到疲惫的北极熊趴在夏日的艳阳里,不愿见仙风道骨的丹顶鹤在肮脏的铁笼中狼狈地悲鸣。\par
		只是为了人们的取乐,只是为了周末的安排多一种选择,就要牺牲这样多的动物离开自己的家园,和天然的生存方式,以囚禁的方式被展出。\par
		在动物园里,我只感觉到难过,却没有丝毫的快乐。

		\blankrev
		这个世界的许多不幸都是因为把一种意志强加于他人之上而造成的。\par
		比如不会唱歌的小企鹅,比如动物园里混沌度日的动物们。\par
		用一种简单的标准去量化个人的价值和能力,所以造成太多个性的磨灭。以自己的兴趣和愿望出发,去粗暴地干涉他人的生活,所以使很多处于弱势的人群陷于无奈的困境。\par
		从动物与人的身上,我们看到更多的是自己。\par
		人却不是无知的动物,人有敏锐而发达的神经,于是,人有更多的悲伤和痛苦。对于命运和人生的捉弄有着更加深刻的痛感。

		\blankrev
		你是否也曾是一只会跳舞的小企鹅呢?你是否和它一样遵循着自己的心灵,踏出坚定而欢乐的足音呢?\par
		还是,你早已放弃了那毫无用处的特长,而去苦苦研习唱歌的技巧,并最终成为了一名平平的歌者呢。\par
		人们赞叹你的刻苦,你的努力,而你,原本是可以成为一个杰出的舞者的。可是,到这个时候,连你自己也忘了,自己曾多么迷恋那足底的节奏,你认为做一个三流的歌者已经很好了,毕竟,你没有什么才华。\par
		许多上帝赐给的礼物,就这样被轻易地遗忘和丢弃了。因为这个世界的衡量标准,因为别人的眼光。\par
		人好像总要活在一种基本的肯定里,才会感觉安全。能够特立独行的人,敢于特立独行的人,是绝对的勇士。\par
		如果没有足够的勇气,那么就至少做一只懂得自我欣赏的企鹅吧。不要放弃真正的自己。\par
		因为据说,是金子总会发光的。\par
		企鹅和人,都要坚定地相信。

	\endwriting


	\writing{蛾与蝶}{2007年02月02日 ~ 15:45} %<<<2

		病中的日子总是百无聊赖。头脑昏沉,缩在棉被里,时睡时醒。难得的清醒里却又开始胡思乱想。\par
		原来生病是用来整治我这般迷恋于想象的人的。\par
		却似乎也恰是生病,使我有了这样多的闲情逸致,去放任自己的思绪。

		它们是这样绵延不绝的,像五彩的丝线,从房间的窗口同阳光一并照进我的世界,将我重重缠绕。

		我乐于被缠绕着,如一只蛾那样,作茧自缚,却又乐此不疲。

		蛾是一种很神奇的生物。\par
		现在,飞蛾对于光明的追求不会有生命危险了。\par
		但试想电灯发明以前,夜读的诗人看到扑火而亡的飞蛾该有多么伤心。\par
		大概也才会有了“扫地不伤蝼蚁命,爱惜飞蛾纱罩灯”的说法吧。

		人心总是柔软善良的。

		飞蛾扑火的精神令许多人动容,而那却不过是它的一种生命本能罢了。\par
		它不懂得光明,不懂得爱和温暖。只有人才能够体会那一切。\par
		人分明是把自己当作了飞蛾,扑向了那一团火光,而义无反顾。\par
		Faye唱,爱到飞蛾扑火,是种堕落。\par
		蛾的执著,是人的执著,蛾的疼痛,原是人的疼痛。\par
		人总是从外界的世界中找到了自己,感受到了相似的遭遇,于是默默落下泪来。

		林妹妹葬花,簌簌泪下。只缘这一句:侬今葬花人笑痴,他年葬侬知是谁?\par
		春色与红颜,落花与人亡,一样是无可挽回的消散。\par
		黛玉是在葬花,也是在葬自己,这一如花爱娇,也如花短暂的生命。

		对于美好事物的亡逝,人们为他们寻找着同样美好的归宿。\par
		如焦仲卿的故事结尾,夫妻化作孔雀,如梁祝的最终结局,相爱的人化为翩翩蝴蝶。\par
		蝴蝶,被人们当作花间穿梭的美丽使者。它们不像蜜蜂那样,是勤劳劳动者的代言。\par
		蝴蝶的存在,仿佛便是为了展示它们的美。蝴蝶是天生的艺术家。它的翅羽便是它最得意的作品。\par
		蝴蝶的姿态是悠闲从容的,大概也是因此梁祝才会选择化身蝴蝶,而非蜜蜂。\par
		试想他们若化作蜜蜂,整日忙碌,嗡嗡不迭,该是多么缺乏美感的画面。

		有人说,蜜蜂是孤独而忧伤的动物。\par
		也有人说,蜜蜂在酿蜜,也是在酿造生活。\par
		在人们的眼睛里,小小的昆虫们有了各式各样的形态和性格。\par
		而飞蛾,原本只是飞蛾;蝴蝶,原本只是蝴蝶;蜜蜂,也只是蜜蜂。\par
		是我们把自己的心装进了它们小小的身体,随它们翅羽的扇动,一路飞行。\par
		这世界于是变得很可爱,很多情。

	\endwriting


	\writing{明亮的}{2007年02月05日 ~ 09:21} %<<<2

		似乎是从05年的夏天,在自己的文字里开始反复出现这个词,明亮。\par
		一个充满光芒的词。它的到来,直刺入田的生活,田的身体。\par
		她通过文字读出那名字。\par
		她知道,是要改变一些什么了。\par
		是告别一些什么的时候了。

		于是,田不断重复,明亮,明亮,明亮。

		我开始喜欢颜料盒里的柠檬黄。\par
		喜欢聆听甜美清新的歌声,大迈步走向前。\par
		喜欢细心为一只只水果洗去表皮上的尘埃。\par
		喜欢安静地在心底埋下一颗糖果,偷偷地欢笑。\par
		幸福从不是谁的恩赐和赠与。\par
		幸福是我们自己,最重要的决定。\par
		那么多细小的机会,要你去体会,去发现,去察觉。\par
		明亮,是拒绝少年们故作的忧伤。\par
		去尽情享用飞驰的青春。

		田爱植物。爱夏末的雨水,秋天的晴朗。\par
		打开一扇窗,翻开一页旧日记。\par
		我们已经没有那么多时间,让生命陷落在自造的灰暗中。\par
		田的决定,充满光芒。

		常常想象,自己像一只明亮的浆果,挂在阳光铺满的山坡。\par
		那会是一个微风的下午吧。\par
		让我轻轻摇曳,如一场欢歌。

	\endwriting


	\writing{时光的三个碎片}{2007年02月05日 ~ 11:11:15} %<<<2

		鸟儿斜光里的侧翼。季节散失的温度。我们,忘记了,又拾起的昨日。


		像鸟儿在冬天的余晖下抖落季节的余温,我们将时间投入漫长而虚假的记忆。\par
		冬天,是冰凉的,鸟儿高飞。时间,是冰凉的,我记录它的痕迹。\par
		虽然,这从来是一件荒唐的事。

		夜晚是一位高明的魔法师。他制造我的睡眠,又塞入记忆的袋子里,令我迷失。\par
		很多次,我回到旧家的院子,我透过那熟悉的窗子,看到里面的自己。\par
		我认得那半明半暗的光线,认得空气里的气味。是一个雨后的傍晚。\par
		天空澄碧得如一汪湖水。远天依旧漂浮着刚刚逝去的灰云。\par
		去看彩虹吧。那一道七彩的桥梁,架上了我们的屋顶。\par
		蜻蜓飞过奶奶的月季花丛。我看它们轻捷的薄翅,快乐的飞舞。\par
		生命是这样一场雨水,是这样一场狂欢。\par
		很多次,我就躲在那童年的院子里,望着从前的自己。静静地,一言不发。\par
		我看到她的哭,她的笑,她的怯懦和顽皮。\par
		我看到奶奶对着那面老镜子梳起花白的头发。看到年轻的母亲,在阳光充满的下午,擦一扇玻璃窗。\par
		叫贝贝的小黑狗,还是卧在门口。西房前的石榴,开满一树红艳艳的花朵。\par
		很多次,我是这样近地,仿佛触摸到那已不再的时光。但是,它们终于是遥远的。\par
		我只能够这样安静地望着,却不能够喊你们的姓名。\par
		只任一树绯红的石榴花,默默飘零,坠入夜晚的深谷,寂然无声。

		后来,一队穿红裙子的女孩向我走来。\par
		她们在齐膝的青草中穿行,一路的欢声和笑语。是夏天,让草木这样丰茂地生长。\par
		我知道,我也走在她们中间。却无法辨认出。\par
		14岁的六月,穿制服的我们,走过圆明园荒芜了,又草木丛生的土地。\par
		你曾为我拍下照片。那年,我们的笑是纯白色的,一尘不染。\par
		谁会和我一样,在昏昏的睡眠里,遇见那个夏天的青草。\par
		谁会在醒来时,想起一位失去联络的朋友,想起少女们的天真梦想。\par
		那是一个丢失在时间里的人形了,那是一个陌生的名字了。\par
		现在的你,是否坐在王子的马车上。\par
		现在的你,是否还保有着少女那水晶似的心,期待着许多糖果一样的幸福,闯进你的生活。\par
		我们都该是永远的小女孩。\par
		更加智慧,却不失梦想,不失天真地生活。\par
		失去联络的你,大概也会做这样的梦。会在城市的某个角落想起那一个穿红裙子的自己。\par
		然而,我们是如此轻易地失散了。\par
		好像我们同回忆的失散一般。只有凭借着残存的片断和画面,来填补,来完全。\par
		搬家时,红色的裙子丢失了。\par
		我再不愿穿起,那样一片鲜明的颜色。

		很久,没有一起骑自行车。很久,我总是孤单地奔波在路上。\par
		一个人倚住一扇窗,看夜色吞噬城市,看霓虹亮起,人们的神色匆忙。\par
		有时,我在梦里又和你一道骑车。还是天光微蒙的早上。我们相约着一起去学校。\par
		我总是在出发前给你打电话。\par
		“我出来了,你过一会也出来吧。”每天,重复一样的话。\par
		骑到你家的路口,总会看到你那纤弱的身影,停在路边等候着。\par
		然后,我们一起向前骑。然后,我们把车子放在一起,让它们也像两个好朋友那样肩并肩地站立。\par
		我们是这样的好朋友。\par
		现在想起来,感觉不可思议。我们可以每天一起上下学,一起吃饭,一起上课,一起买零食,而不觉得厌烦。\par
		晚上回家后,还往往要打一个小时的电话。\par
		如今说起来的时候,我们都笑了。这一份友情,令我们引以为豪。\par
		很久,我没有骑自行车。你的自行车也不见了踪迹。\par
		那骑车的日子,这样远了,远了,像一片淡粉红的海,澎湃着细小的浪花,抚摸过我们的记忆。

		人们说,记忆是我们心灵的捏造。\par
		发生过的一切,在时间的拉扯中,早已面目全非。\par
		记忆呈现出的,不过是人们一厢情愿的谎言。它却因此是美的。\par
		美得比一场五彩缤纷的梦,更令人着迷。\par
		我在许多个夜晚,被困入记忆的口袋。\par
		我知道,我是一只冬天的鸟儿。我知道,我是时间的指缝间,那漏掉的光芒。

		我记录它们。\par
		却终于是徒劳的。\par
		耽于回忆的人,总是同样地耽于遗忘。\par
		于是,鸟儿高飞。\par
		于是,我伸开空无的双手。

		这一切,都无比美好。

	\endwriting


	\writing{一个人的旷野。一个人的眼泪}{2007年02月07日 ~ 21:20:50} %<<<2

		每个人的旷野。风声鹤唳,草木皆兵。


		如果有泪水,让我独自哭。让世界安静下来,只留下呜咽的风,流浪在无人的旷野。\par
		这一晚,像打翻的深蓝色墨水瓶,浸透我洁白如纸,也脆薄如纸的生命。\par
		泪水,滴落在衣衫和手腕。任它们奔流,任它们沸腾,再在月光里冷却,化作明早的露水。\par
		会有穿白裙子的女孩,赤脚走过那片草地。\par
		是我梦里轻盈的脚步么。是遥远的,那一个完好无损的我么。\par
		她捧起一颗水晶一样的心,在清晨的阳光里许着愿望。\par
		她轻轻地歌唱,她甜美地微笑。\par
		她知道,她口袋里装好了满满一袋的幸福。像一颗颗糖果。等待着她去剥开那美丽的糖纸,一一品尝。\par
		多么好的早上,多么明亮的开端。只有碧绿的草尖上,挂着一滴悲伤。\par
		她可以忘记么。她可以忽略么。\par
		女孩赤脚走过那片草地,脚心冰凉。\par
		我掏掏口袋,那曾满满的糖果被偷走了一半。谁说,谁在我的耳边说,生命的残忍,与无可奈何。\par
		是我惊慌失措的脸孔么。是眼前的,这一个残破不全的身体么。

		如果有泪水,让我独自哭。亲爱的,不需要倚住你的肩膀,不需要一句安慰。\par
		让时光冰凉,青春冰凉,让我的日子,一朵朵无言地绽放,像十二月的雪花。\par
		我想纷飞,我想痛哭,我想一个人站在旷野的中央,质问命运。\par
		大声地呼喊。你听得见么。如此渺小的我,在竭尽所有地渴望着生存的力量。\par
		让我独自哭。就这样承担起一切无妄之灾。我要独自地面对,像一个真正的勇士那样,坚毅的心,坚信的意志。\par
		不要说,不要问,我将何去何从。\par
		不要抱怨,不要叹息,我的遭遇,我的命运。不幸的故事太多,我并不是特殊的一个。

		很多时候,我看到自己。看到她坐在书桌前,在莫测的未来面前茫然无措。\par
		很多时候,我离开自己。我说,忘记时间吧,然后忘记自己。\par
		她于是独自坐在那。她于是只是一台出现故障,难以恢复的机器而已。\par
		悲伤和恐惧,那样微小了。我离开自己。我知道,身体,不过我栖身的屋宇。\par
		你会消失吗。那么,我将流离失所。\par
		所以,我才如此珍爱你,因你的故障,而有了悲伤和恐惧。\par
		如果有泪水,让你独自哭。\par
		让身体站在呜咽的风里,让生命独立在人间的旷野。\par
		这洁白而脆薄的生命。

		我剥开剩余的糖果。\par
		我一颗颗细细品尝着滋味。还有多少时间,还有多少力气,容许我享用它们的甜美?\par
		一切未可知的长度,一切未可知的后来。\par
		这不该是一件感伤的事。\par
		我在口袋里摸索。每个人都分到不同的糖果。\par
		有些丢失了,有些遗落了,有些没有被发觉。

		这一晚,你说有雪。\par
		我于是等候着。仿佛等候着另外的自己,在天地间纷落。

		这样的时刻,让我独自哭。

	\endwriting


	\writing{三个人}{2007年02月08日 ~ 09:21} %<<<2

		三个人,一起吃热乎乎的拉面。\par
		昨天的北京不太冷,却是灰暗着脸。\par
		看朱朱试衣服,挑选夏季穿的短衣,才感觉到她离开的日子,已经如此迫近了。\par
		之后,三个人在三座城市。\par
		之后,天各一方的朋友,只有彼此怀念。\par
		朱朱说,今年不能陪你过生日了。\par
		我忘记了没有遇到你之前,那些生日是怎样度过的。\par
		那还会是一样的春天吧。\par
		海棠花会开满枝丫。我会画她们的模样,会一个人偶然地伤感,偶然地幸福。\par
		只是,身边没有了你们。\par
		没有了陪我欢笑与哭泣的你们。\par
		三个人,被我们自己笑称为“中关村铁三角”。\par
		三个人,留下彼此的故事,那些眼泪,那些快乐。我们埋藏了回忆,各自向前走去。\par
		海洋,山川,还有嘀嗒而去的时间,多少的美好在这里了,多少的盼望在这里了。\par
		每个人都会更好地生活,对么。\par
		田现在要做的,是停止悲伤和哭泣,慢慢坚强起来。

	\endwriting


	\writing{情人}{2007年02月12日 ~ 11:11:44} %<<<2

		终于,我渐渐了解,一切的始末。

		\blankrev
		情人,是脸颊的一抹绯红,酝酿在等候花开的季节。\par
		情人,是一个长长的故事,要你听老去之后的我,娓娓诉说。\par
		我常常想象,许多个遥远的午后,远到我们开始怀疑记忆的时候。\par
		那时,你有了斑驳的发,昏花的眼。不再是此时挺拔的少年。\par
		你会安静地读一份报,饮一盏茶。你会翻开纸页发黄的日记,你会想起某个春天里的欢笑。\par
		那时,你会不会依旧记得我。你会不会在心底轻轻地疼痛了,在昨日的情怀中惘然若失。\par
		情人,我们会在各自的故事里,如此决绝地悄然老去了。\par
		让我向什么人说起,那些年少的轻狂和执迷。让我怀抱着甜蜜和哀愁的心,无声地经过你的生命。\par
		我愿意,把所有相爱的时光,编织成一张温柔的网。\par
		我要将一切不忍丢失的细节捕捞,令全部的往事一无所失。\par
		我愿意,把相逢的快乐,书写成一首漫无边际的诗。\par
		我要你在后来的岁月里,读起这些我最美丽的文字。

		情人,是要我轻轻拾起一颗曾经冷却的心,细心擦拭。是要你握住一双冬天的手,讲绿树和花朵的希望。\par
		情人,是相爱中点滴的温存和珍惜,是每个人不必说出的心事。\par
		你会知道我的目光,你会知道,那一汪湖水的幸福和惆怅。\par
		一朵孤单的魂,一场寥落的梦。你看到我哭了。你看到天空的灰色。\par
		你说,让我将你陪伴。\par
		情人,在哀伤的时刻,在不知所措的迷失。情人,捧起我的脆弱,让光芒洒满凋敝的花园。\par
		任时间将我带去何方,任风雨飘零。我不会有恐惧。\par
		你用粉红的胭脂,擦在我的双颊。我该是坚强的爱人。\par
		我该重新生长出勇敢的心脏。

		情人,在这里,我们享用青春的明丽。\par
		我们相信一只指环的决定。\par
		听我诉说,一个长长的,没有结局的故事吧。\par
		让我有机会,用生皱的双手抚过你松弛的额头。让我看你的步态蹒跚,听你的口齿不清。\par
		那遥远的,远到我们开始怀疑记忆的时候,你是否还能够找到我。\par
		我们是长久地失散了,还是再不曾别离。\par
		也许,我将是你模糊的一个回忆。也许,我将是你今生全部的故事。\par
		那是我们所不知道的未来。\par
		情人,在这里经过,我们都是生命里的匆忙房客。

		爱,从不是一枝玫瑰。\par
		爱,是如此悠长,又如此短暂的深情。\par
		要你记得,要我记得。

	\endwriting


	\writing{练习者}{2007年02月12日 ~ 11:55} %<<<2

		重读史铁生《记忆与印象》。\par
		他说,最后的练习是沿悬崖行走。\par
		他写下徘徊在头脑中的短诗:

		\longpoem{}{}{}
		梦里我听见,灵魂 \\
		像一只飞虻 \\
		在窗户那儿嗡嗡作响 \\
		在颤动的阳光里,边舞边唱 \\
		眺望就是回想。
		\endlongpoem

		轮椅上的史铁生,他的地坛,他的笔,他的灵魂,给多少迷失的人以勇气。\par
		生命,是这样一次不可思议的旅行,我们从不知道下一站的风景。\par
		也许,是风和日丽的一马平川,也许,是雷电缴交的荒山野岭。\par
		遇到怎样的路,便有怎样的经过。有时,是信步徜徉,有时,是艰难跋涉。\par
		他在对未来充满无限期待的年纪上坐在了轮椅里。他不再能够行走,他眼见着命运,如此残忍而决绝地重重倒下。\par
		现在的他,又行走在生命的悬崖,更加险恶的疾病令他的肉体受尽磨难。\par
		在偶然清醒的时候,却依旧不忘生命的思考。他写下一本本书,一字字都是心灵深处最纯澈的感悟。\par
		其中,有一本书被命名为《灵魂的事》。\par
		生命,大概从来不是肉体的生命活动,而是灵魂的觉醒和成长,是灵魂的事。\par
		不然,人又何以为人,不过一具行尸走肉罢了。\par
		活着,便该有所知觉,有所思考。\par
		活着,不是简单的呼吸和心跳,而是在生命的有限时间中,勇敢地独行。\par
		每个人,都是练习者。\par
		练习着行走,练习着爱恋,练习着承担人生的重量。\par
		史铁生在沿悬崖行走。\par
		他没有恐惧,他知道,这同样是一种练习。\par
		面对命运的无常和残酷,有些人足够坚强,摆出战斗的姿态。\par
		他们向命运宣战,面无惧色,笑对刀锋。\par
		有些人,如一局外人,静观命运起落,风轻云淡。\par
		他们把一次次的挫败和不幸看作练习的过程。他们知道,没有人能够扼住命运的喉咙。\par
		擦干泪,不要问为什么。\par
		战斗的人,他们的精神值得人们敬佩。练习的人,却给生命更多的自由。\par
		不要让自己像一只困兽。给自己豁然的心,接受命运,并学会从中有所获得。\par
		于是,我们能够不只是失去,不只是命运脚下的失败者。不必用仇恨的眼,看它的无情。

		史铁生坐在轮椅上,心灵却走了比常人更远的路。\par
		灵魂在病中行进,灵魂在磨难里升华。这是他肉体的不幸,也是他灵魂的幸运。\par
		让我们练习去真正地生活。\par
		而不只是简单地活着。

	\endwriting


	\writing{气球}{2007年02月15日 ~ 19:33:02} %<<<2

		是漂流,是随遇而安的心。

		\blankrev
		星期二,突然起风的早上,透过玻璃窗看到一只漂浮在高空的气球。\par
		它也许飞行了很远。它即将在凌空的寂寞里不知去向。\par
		它在转瞬里飞离我的视线。与我目光相遇的全程,不过短暂的几秒钟。\par
		一只高空的气球,一只被风吹向未知的气球。\par
		是谁,为它的身体注满了气体,是谁,又任它被狂风卷走,去面临一场独自的旅行。\par
		这个蓝到忧伤的冬天。\par
		我想象着气球的生涯,一路的飞行,一路的无目。\par
		或许,它曾是孩子心爱的伙伴。妈妈用五彩的丝线,把它系在书包的拉链。\par
		或许,它曾是恋人的礼物。戴蝴蝶结发卡的女孩,把它挽在修长的手指。\par
		而这个时候,它是孤单的气球。\par
		它在高空的气流里,被驱赶,被推挤。它不知道它的方向。\par
		气球等候着,气球默无声息地接受着时间的安排。

		\blankrev
		想起许哲佩的歌,想起MV画面上,飘进云里的白色气球,还有她悲伤的侧脸。一个人独自的画面。\par
		她唱:我不在意,不会在意,放它而去,随它而去。\par
		十六岁的春天,坐在飞絮漫天的日光里,和你一起听这一首《气球》。\par
		你说,这是多么不开心的一首歌。不许再听。\par
		我却依旧任性地按下重复键,沉迷其中。\par
		那是多远的四月了。我都忘记,日光的深浅,忘记你的容貌,只剩下恍惚轮廓,像一句含混的道别。\par
		只是偶尔记得,短暂的花季,记得你讲的笑话,和春日午后纯白的欢笑。\par
		那是匆忙的过往。你说,你无法忘记。\par
		后来,就像一只气球那样,我们都飞走了,飞向各自的方向。\par
		吹着似曾相识的风,分别去相遇另外的情节。\par
		十六岁,连天空都是青涩的,如一只淡绿的苹果。\par
		一切的一切,都这般小心,像一场谨慎的花开。

		\blankrev
		我们都是这样经过,许多的风景,一树树繁华,一树树凋零。\par
		放飞一只气球,任它们漂流高飞,永远地,永远不知去向。

		\blankrev
		星期二,突然起风的早上,一个人听如咆哮,如怒涛的风歌涌过屋角。\par
		是寒冷的冬天吗。是寂静的冬天吗。\par
		我总是飞行着。\par
		默无声息。

	\endwriting


	\writing{旅}{2007年02月17日 ~ 09:53} %<<<2

		\longpoem{}{}{}

		倘若,这世上从未有我 \\
		那么,又有什么遗憾 \\
		什么悲伤

		生命是跌撞的曲折 \\
		死亡是宁静的星辰

		归于尘土 \\
		归于雨露

		这世上不再有我 \\
		却又无处不是我
		\endlongpoem

	\endwriting


	\writing{妈妈}{2007年02月17日 ~ 11:49:33} %<<<2

		我惟有,不知如何表达的感激。


		\blankrev
		早上醒来,手机的震动提示:2月20日,妈妈的生日。忘记了是什么时候设定下这样的提醒。\par
		本是一个无需提醒的日子。怎么可能忘记或忽略。\par
		每一年,这一天,都令我疼痛地感知到,她又老去了一些。

		\blankrev
		古人说,父母之年不可不知也,一则以喜,一则以惧。\par
		时光是如何如一汪春水的碧波,浮去年月的花瓣。父母的发,如何成雪,散落入你我的转眼。\par
		我曾是那襁褓中的婴孩。我曾是你手心里盛开的一朵生命。\par
		你望着我长大。像你的感叹,不过弹指,便是人间的一次更迭。\par
		好多次,我们一起翻看旧时的影集,你对我说,我儿时的乖戾和顽皮。\par
		你告诉我,哪一年,我们去看腊梅,哪一年,我们去观赏了灯会。\par
		照片上的妈妈,纯澈的脸孔,纤弱乌黑的发。\par
		远处的灯火闪烁,你的笑,在方寸间定格,竟这般杳无了,阑珊在我记忆的湖底。\par
		那一切,已恍如隔世。\par
		喜欢父母的一张黑白合影。父亲的手,轻放在你的肩头,你微微侧身,坐在春天的石阶上。\par
		身后是如笑的春山。看不到斑斓的色彩,却有两个人温和的四目,暖似熏风。\par
		我是在这样的目光间,萌生又孕育。\par
		我出生在一个春天。你说,你从病房的窗口望出去,树木刚刚生出细而黄的幼芽。\par
		我没有记得,那个最初的世界。我却仿佛能够见到,你怀抱着幼小的我,走进一片明媚。\par
		花朵在怒放,你的爱在怒放。我只静静地睡,缓慢却匆忙地成长。

		\blankrev
		这个时候,我已经是二十岁的人。\par
		这个时候,我却仍无法令你放心。你无法不担心着多病的我。\par
		这时常令我感到不安和愧疚。

		\blankrev
		零六年的六月。一个多云的天,忽明忽暗,风吹入我的房间,又逃走。\par
		午睡的倦意未消,起身下楼。母亲站在楼前的树下,白衬衫,白帽子,被日光照得发亮。\par
		她送来水果,一包散发着香味的桃子。她叮嘱,要多喝水,注意休息,不要熬夜。\par
		母亲总是不厌其烦地重复相似的话。\par
		我停在原地,怀抱着那一包桃子,看她骑上那辆旧自行车,离开了学校。\par
		她的身影一点点缩小,我的心也被收得紧紧的。\par
		回到宿舍,一个人打开水龙头,细心地将桃子一只只清洗干净,又一只只吃掉。\par
		整个下午,躺在竹席上,看树影婆娑摇曳。不知不觉里,竟泪流满面。

		\blankrev
		夏日在蔓延。大雨总是滂沱而至,令人猝不及防。\par
		暑假的夜,和母亲睡在一处,紧握住她的手。很久了,我没有这样依偎在她的身边。\par
		是在得知了自己的病情后,我才发现,对于妈妈的眷恋和依赖,原是如此之深。\par
		也许,我所有的坚强,都是因为妈妈。为了她,我才有勇气,去面对我的命运。\par
		她为我扇蒲扇,她说不要开冷气,那对身体不好。\par
		她对我说,不要怕,她劝我多吃下一些食物。\par
		而我,常常对着饭菜发呆,一个人默默在深夜饮泣。她擦我的泪。\par
		我知道,她的心在碎。\par
		感觉着母亲的呼吸,感觉着她的心跳。我决定要有斗志地生存下去。\par
		我不可以轻易地放弃,我要陪伴在她的身边,至少到我照顾她的时候。\par
		我不可以留下孤单的妈妈。\par
		我怎么可以,怎么忍心,让她的后半生没有了我,她唯一的孩子。\par
		这样地想着,于是,泪水又蒙住了我的双眼。\par
		我不敢让她看见。

		\blankrev
		只有妈妈知道,她心中的忧伤。她从不让我看出,她的难过。\par
		她鼓励我,她微笑,她的眼神传达着明亮的希望。\par
		妈妈总是说,一切不幸都终于会过去,只要你敢于经过。\par
		医院的傍晚,黄昏中有低飞的燕。它们飞舞,它们鸣叫,它们狂欢。\par
		我们并肩站在窗口。我已比你还高,却依旧倚住你的肩膀。\par
		妈妈。我只轻唤你,便已泣不成声。\par
		我心中全部是对于未来的悲观和绝望。你抚摸我与你年轻时一般乌黑纤弱的发。\par
		你不发一言。我们这样看黄昏中的燕。\par
		一场生命的飞舞,生命的鸣叫,生命的狂欢。\par
		恐惧是一张网,这个夏天里,我被它困住,不得自由,不得呼吸。\par
		夜夜的梦魇,却又失望于黎明的到来。\par
		我对你说,我怕着白天。在白天,我要真实地面对一切。\par
		夜晚,却不过是梦。梦,即使是险恶可怕的,也终于会醒。而现实不是。\par
		现实是这样清醒,这样真实得,一览无余。\par
		妈妈说,如果能够再次孕育你该多么好。\par
		你仿佛是在怨恨自己,将我生成多病的身躯。\par
		你遗憾没有给我一副强健的肉体。\par
		你觉得,是自己造成了我连绵的苦难。

		\blankrev
		妈妈,我却时常感谢,你给我的生命。\par
		即使这身躯,有许多不如意。但生命,从来是独一无二,最可宝贵的礼物。\par
		我感谢,今生是你的女儿。感谢,能够依偎在你的身旁,能够开放在你的手心。\par
		妈妈,不幸的部分,是我们共同的命运。我深知,我的疼痛,在你那里总要加倍。\par
		幸福,却是更深切的主题。\par
		从这世上有了我,你便呵护着我。从我得到了知觉,便对你万般依恋。\par
		这人间,据说百年才能修得同船渡。那么,母女的缘分,该有千万年的修行。\par
		我是经过了许多的漂泊和艰险,才投入你的腹中吧。\par
		是你收容了我游移的灵魂,给我温暖的家园。\par
		这缘分,是该令我们感激一世的。\par
		让我们并肩地站立。看落下的雪花,落下的风雨。你在这里,你在我的身旁。\par
		我于是不肯放弃,决不放弃,丝毫生命的力量。\par
		我将飞舞,我将鸣叫,我将狂欢。如那黄昏下的燕一般。

		\blankrev
		这一天,你对着镜子将白发染黑。\par
		我远远看你。妈妈,你又老去了一些。\par
		甜蜜而疼痛,交织在一瞬。\par
		农历新年的鞭炮已经响起。想起去年的烟花,我们一起在烈烈的寒风里观看。\par
		今夜,一样会有盛大的焰火,一样会有缤纷的色彩,流溢在深暗的空中。\par
		我们说,在午夜前去放流光棒。我喜欢那手上的火花,喜欢它们的光芒,如星如电。\par
		一年年,经历着多少的爱,多少的辛酸和欢乐。\par
		它们都将在夜空里盛开,繁花之上,又生繁花。\par
		妈妈,让我们一起去看。\par
		妈妈,让我紧握住你的手,容许我有时间,望你的老去,如你望我的成长。\par
		这不是一件悲伤的事,而是淡而深长的幸福。

		\blankrev
		提前的祝福,生日快乐。

		妈妈。

	\endwriting


	\writing{不可求}{2007年02月20日 ~ 22:31:08} %<<<2

		让我相信么,那一切美好的可能。

		\blankrev
		读聊斋,总会有感于花妖狐魅的真情。\par
		那一个亦真亦幻的世界里,有报恩的狐,有惩恶的妖,有助善的仙。

		《香玉》一篇,写黄生与牡丹花妖香玉的相爱。\par
		当牡丹花被移植,以致萎悴而死,黄生作《哭花诗》五十首,日日临穴涕洟,独对冷雨幽窗,辗转床头,泪凝枕席。\par
		他哀吟:

		\shortpoem{}{}{}
		山院黄昏雨,垂帘坐小窗。\\
		相思人不见,中夜泪双双。\\
		\endshortpoem

		花神感其至情,终使香玉复生。后黄生入山不返,两人恩爱相待,一如人间夫妇。\par
		他每指璀璨似锦的白牡丹说:“我他日寄魂于此,当生卿之左”。\par
		黄生实践了自己的诺言,临终前,他笑对其子:“此我生期,非死期也,何哀为!”\par
		他亦化作一株牡丹,生与香玉旁侧。那是一株不开花的牡丹,默默伴随在爱人的身旁。\par
		院中的小道士,却因其不曾开花,而将他砍去。随后,白牡丹便也憔悴而死。

		\blankrev
		这是一个近乎于童话的故事。虽然,没有公主和王子,没有华丽的舞会,没有美满的结局。\par
		但它所描述的爱情,已远比童话的甜美更令人动容。\par
		这是美到不食人间烟火的传说。生与死的阻隔,人与物的分别,被一一打破。\par
		因为是爱,因为是真情,死可以复生,人可以化身为花。正如蒲松龄所评:“情之至者,鬼神可通。”

		\blankrev
		爱情,是否原本应该是如此的模样?爱情,是否就该有生死相许的勇气?\par
		古人的心中,有那一份情的敬畏,于是,有这样美丽的故事,有人们联翩的浮想。\par
		有多少读了这故事的人,开始妄想化身一朵牡丹,安守在爱人身旁,静静度过山中的岁月。\par
		可遇不可求的美,却惟有想象,惟有等候,惟有听任缘分的安排。

		\blankrev
		人与妖的相爱,凄丽得荡气回肠,摧人心肝。人间的深情,也同样令人下泪。\par
		《瑞云》写贺生同名妓瑞云相知,却无财力为其赎身。\par
		后有客过,以一指按女额曰:“可惜,可惜!”瑞云额上便有如墨的印黑,并逐渐扩大,最后竟使原本光洁的面容丑状如鬼。\par
		门前仰慕者络绎不绝的车马绝迹了,媪母拿走了她先前所穿用的首饰和衣物,将瑞云驱使为奴。\par
		曾有的光华,瞬时间消散,孱弱的女子,不堪繁重的劳活,日益憔悴。\par
		正是这时,贺生货田倾装,为其赎身。\par
		瑞云自惭形秽,面壁自隐,贺生对她说:“人生所重者知己:卿盛时犹能知我,我岂以衰故忘卿哉!”\par
		这一句,真是振聋发聩,令人深叹。\par
		以色事人者,色衰而爱驰。这仿佛是美丽女子们自古以来的悲惨命运。\par
		从“无与士耽”的告诫,到美人迟暮的悲伤,总是诗人低唱的哀歌。\par
		而颜色不再的瑞云,却幸得一位有情的贺生,对于已丑状如鬼的她,仍痴心不改,不顾世俗的讥笑,而情深宜笃。\par
		故事的结局是完满的。贺生巧遇了当初按女前额的秀才,终为瑞云洗净面容,复成艳丽光洁的佳人,一如当年。\par
		秀才说:“天下惟真才人为能多情,不以妍媸易念也。”\par
		真情,是不该因美丑而有所改变的。\par
		他当年在瑞云身上施下法术,也是因为“惜其以绝世之姿而流落不偶。”

		\blankrev
		瑞云是幸福的,幸遇“怜才者之真鉴”。而又有多少人,能够拥有这样的幸福?\par
		年轻的时候,我们总是难免爱上彼此美丽的面孔。\par
		然而,如果爱只停留在这肉皮的光艳之上,它该有多么脆弱。谁也不愿接受色衰爱驰的结局。\par
		年华流逝,我们有多少美丽,可堪时光的消磨,我们又有多少爱,经得起青春的告别。\par
		女子总会问他的爱人,你爱我的原因。\par
		女子希望自己在情人眼中是美的,却又担忧他不过爱她的美。这是无法改变的矛盾。

		\blankrev
		若当我失去了美丽的面孔,你是否能够依旧,将我温柔地对待,小心地呵护?\par
		瑞云,是女子们的一个梦,恒久不醒的梦。多少的故事里,总是难遇有情郎。

		\blankrev
		在茫茫人海,我该如何在最美的时候遇到你,我是否也需在佛前求下五百年,而得与你凄婉如落花的姻缘一场?\par
		深情厚意,生死相许,也许都不过书页上的荒唐梦。\par
		爱,不过一次燃烧的炽热,不过两双渴望的眼,一种牵挂的心情。\par
		但爱,在我们的心中有过多少奢望,便会有多少美丽的故事,被想象,被流传,令人辗转,令人反侧。\par
		聊斋,在那个如烟似雾的世界里,成全着我们。\par
		善与美,惩戒着恶与贪婪,有情的人,感动着天地,终成佳侣,心生邪念的,果有业报,大快人心。\par
		一切人间的理想在这里实现,有情的鬼怪,比无情的人,可爱也亲切许多。\par
		一只狐,一枝花,一只鸟,全部是重情重义,一段奇遇,一次行旅,一场爱情,亦都是波澜壮阔。\par
		这此处与彼岸,人间与虚构,竟叫人不辨真伪,不分虚实。\par
		哪一处,是我们真实的寄托?\par
		一侧是触摸到的生活,一侧是不消失的想象。蒲松龄神游的世界,在字里行间,引人陷落。\par
		我向往那里的明亮与纯澈。\par
		我愿化身为花,我愿相遇一个有情的你,许我一世的深爱。\par
		所有的情节,却无法被我们自己撰写。\par
		一切的一切,只可遇,不可求。

	\endwriting


	\writing{两棵树}{2007年02月21日 ~ 11:20} %<<<2

		看一个心理咨询的节目。被丈夫离弃的女人戴着面具,语气愤然,却是哀伤。

		心理医生让她想象并肩生长的两棵树在春天的沃野,她问,你看到些什么。\par
		女人说,只见到树身上一个巨大的洞。心被掏空的感觉。\par
		心理医生让她继续想象,一棵树离开另一棵树,继续在沃野上勃勃生长。\par
		她说,树身上本没有洞。在遇到另外那棵树之前,它是完好的,独立的,它可以独自承受风雨和阳光。\par
		现在,它一样是完好的。那一处洞,只是想象。

		女人若有所思,一言不发,面具遮住了她的脸孔,我见不到她的表情。\par
		节目就这样匆忙地结束。女人后来的生活,我们不得而知。\par
		她是否能够回到故事的起点,重新做自己那棵完好独立的树?她身上的空洞,能否真正地消失,了无痕迹?\par
		也许,每个人莫不是沃野上的一棵树。\par
		从前,我们生出自己的叶片,伸展在四季的流转。后来,我们与另外的树并肩而立,我们彼此依靠,共撑一片郁荫。\par
		可能,就这样并肩地站立,直至地老与天荒。可能,其中的一棵树会在中途退场,留下空出的一块土地,无以填补,无以充实。

		这是人们感情的真相,温暖里永远潜伏着危机。\par
		爱人,不会是爱情的保险柜。你的心,不能够存放在那里,你的心,只能由自己保管,才不致丢失。\par
		戴面具的女人,丢失了自己,丢失了心。所以,她只能见树身上显现一个巨大的洞,一处不知如何愈合的伤。

		为什么,我们开始需要另一棵树的陪伴。\par
		为什么,我们不甘于固守住自己的世界,感受独自的风雨和阳光?\par
		唐君毅先生的书中写,人在童年时代,所关心的是食物,在中年时代,所关心的是地位,而青年时代,所关心的恰是爱情。\par
		是因为年轻么。是因为我们还没有度过情欲的河流么。\par
		人的苦难,总是来源于这种种的关心,因它遮蔽了其他,令人的视野狭窄,稍不称意便陷于难以自持的困境。

		如果,我们能够突破人生里的太多在意,如果,我们能够平复一颗欲望丛生,永不满足的心,是否
	便可以静如止水地生活,是否便没有了离别和哀伤,获得和失去,也便没有意义,没有分别。

		做一棵树,一棵充满生机的树,无论身旁是否有令一棵树的存在。\par
		倘若注定要孤单地生长,那么便更加努力地抽出新枝,怒放花朵。\par
		当你长高,便会有更辽阔的视野,你的世界将不再只是两棵树的并生,而是天地万物的欣欣向荣,饱含深情。

	\endwriting


	\writing{与书二三事}{2007年02月25日 ~ 10:46} %<<<2

		\subpart{畅销书}

		和静到图书大厦买书,才发现,居然有这样多在写字的人,居然有这样多的小说被印刷出来,摆上
	书架。\par
		这样的新发现,并不是因为我很少光顾书店,而是很少光顾如图书大厦这般的大书店。

		常去的自然只是海淀这里的小书店,光合,万圣,风入松一类。\par
		它们的规模较之大厦自然逊色。书籍的品种也多侧重人文科学。\par
		摆上书架的,也多是在茫茫书海中挑拣过的精品。文学,哲学,宗教,一例素雅的装祯,安静地陈列。

		于是,我竟不知道,有如此多的小说,花花绿绿地被印刷出来,竟不知道,那样多光怪陆离惹眼招
	摇的书名和作家。

		也许,是我挑剔,也许,是我自命清高,或者权当是我无知的偏见吧,对于一本本穿着花俏外衣的
	读物,我实无任何阅读兴趣。但是,它们既然被出版了,就说明还是有一定的阅读者吧。而且据我所知
	,其中的很多书目,还成为了畅销书。

		有人说,这已不是一个阅读的时代。当视觉的冲击一次次刺激人们的瞳孔,还有谁能够沉定下一颗
	心,甘心捧一本书,就度过一个下午。

		小时候觉得,识字了,能读书看报了,是成为大人的一种标志,一种自豪,虽然,小孩子并不愿意
	去读字书,他们更青睐各种画册。

		现代人却正走在向儿童回归的大道上,人们开始对文字抵触,而对画面充满兴趣。\par
		看一看各位大导演今年的作品便会了解,这是一个多么注重视觉的时代。

		多少人都好像视觉饥渴一般,涌向电影院,花上近百元,只为去享受一顿情节和主题离奇模糊“视
	觉的盛宴”。

		可能,现代人过于渴望儿童时代的无邪与天真,才在无意识中做出了这样的选择。不然,怎么有中
	年妇女自称女生,不然,怎么电视上蹦蹦跳跳的主持人都装得很幼齿,弄得娱乐选秀活像少儿节目。

		然而,就是在这样的时代里,有一种叫做畅销书的东西诞生了,而且深入人心。

		书店里,还常常有一面叫做畅销书排行榜的墙,来指导读者,最近什么书在畅销。仿佛又是提醒众
	人,千万别误了潮流,现在大家都在读某某小说呢,你不读,岂不落伍。

		读书,有时候可能成为了与潮流接轨,不被时代淘汰的手段。\par
		虽然,或许你并不喜欢那一个作者,那一本书,但是大家都在读啊,你不读读显得很没文化似的。\par
		毕竟,读书好像还算得上是有点文化的人做的事。

		畅销书,就这样畅销了起来,有点起哄架秧子的味道,一片要吆喝喝,有时还搞个签售什么的,让
	本来该乖乖躲在文字后边的作家出来亮个像,仿佛明星,被粉丝们的欢呼和尖叫声淹没。

		当作家们把自己装扮得好像偶像,闲来无事还拍些自恋照片贴在博客上,我不知道,这是文学的一
	种进步,抑或堕落?我总以为,作家的灵魂是文字,而决不是脸蛋儿。当然,这也只是个别作家的个别
	现象,不足够以偏概全。

		读书的人少了,真正读书的人更少了。

		书,本该如朋友,是心灵的选择,而非时尚的追逐。读书,该是独自的事,不需要谁来指导,横加
	干涉。书店,不同于商店,更有别于菜市场,不该有喧哗和吵闹,它是一处净土,你所听到的只是手指
	翻阅的纸页沙沙,漫步于书店,该是一件惬意而温馨的事。

		在书籍的海洋里,寻找你的朋友,在人生的迷途里,辨识方向。\par
		当然,每个人的阅读趣味不同,我所不屑的书,也许在他人看来却恰是甘泉。

		读书的原则,套用一句话便是,读自己的书,让别人说去吧。


		\subpart{中学的书店}

		中学的时候,家附近有一家中等规模的书店。放学后,我通常是先到那里看看书再回家。\par
		时间久了,每一架上有什么书,我都了如指掌,比店员还要清楚。\par
		或许因为位置不佳,书店的生意很是冷清,很多时候,偌大的书店里,顾客的数量比工作人员还要少。\par
		那里便好似成了我的私人图书馆,事实上,我也经常妄想拥有这样多的书架,这样多的书。\par
		看到那一架架一排排的书,我总是心生敬畏,感觉到它们的神圣。

		如果,有一天我能够写出一本书,放在这样的架子上,与那些我喜爱的作家的作品并肩,该多么好。

		年少的我,痴望着一本本书,细细摩挲着雪白纸页,生出了作家的梦。\par
		这个梦,像个晶亮的水泡一般,在我的世界里越升越高,又渐渐扩大。\par
		有时,我竟怕它因为扩充得过于饱满而在一瞬里碎了。于是,我总是悉心地将它保护,唯恐稍有疏忽,它便化为乌有。\par
		我一直相信,有梦想总是美好的,没有梦想,青春该多么苍白。

		写一本书,被放到书店的架子上,这便是我的梦想。我仰望着这枚水泡,满心的激动不安。\par
		现在,我却对这梦想不以为然了。

		当我知道,只要有足够的钱给出版社,任何人都可以把自己的文字印成书,放到那个架子上,我的
	水泡从天而降,却砸得我生疼。

		去年,和梁老师在学校的来园里聊天,他掏出一本寄给他的书,对我们说:你们以为这本书里的文
	字已经达到了出版的水平吗?其实,不过是父母给钱,印出来的。

		我看到,这本书序言作者的署名,正是梁晓声。\par
		梁老师于是补充:也是父母托了许多层关系找到我,让我作序,人情而已。

		许多书,就是这样被印刷出来的么。

		那一刻,我竟有种悲凉,又好像羡慕这个有父母去出钱,去拉关系的女孩。但很快,我竟又庆幸,
	我没有这样的父母。不然,印一本书,不过是浪费些纸张,或者用金钱做一次纪念罢了。

		书,也便成了日历或照片一样的东西,随你去印,随你去自娱自乐。\par
		我心中的书,却全不如此。\par
		我要写的,是一本真正的书,对他人的心灵有所帮助的书。

		家附近的那家书店终于倒闭关张,虽然在此前,它也作过一些积极的挣扎,比如搞了类如诗歌之夜
	的活动(请了许多自称为诗人的人当场朗诵自己的作品),但它的关门,还是成为了残酷的事实。

		在我看来,那些活动非但没有起到增加客流的目的,反而使许多顾客对于这家书店有了反感。\par
		至少我,在那一夜的“诗歌盛会”后,每每走进书店便浑身不自在,甚至一阵阵感觉干呕。\par
		关于诗歌这件事,那又是另一篇文章的主题了。

		也可能是书店老板故意的,用这一场活动把顾客们都吓跑,也好彻彻底底地倒闭,不给自己一点后
	路。

		若真如此,老板确是个很有手段的人。


		\subpart{美女与书}

		却也会怀念中学时那些泡在书店里的下午。

		是在那家书店里,我培养了阅读的习惯,有了写作的意愿。\par
		现在,书店变成了一家小商品批发市场,生意自然红火了许多,也没有任何要倒闭的迹象。\par
		看来,书籍还不是我们生活的必需品,人大约可以不读书,却不可以不穿衣服。

		虽然,不读书会令我们的面目可憎,人们却通常无知觉于这张脸,以为涂抹上粉底,画上眼影,便
	可以艳美如花了。

		殊不知,艳美的脸,不过是一张画皮,好像聊斋里的故事,那张皮下,还是一个厉鬼。

		读些书,不为了记住什么名人名句,在人前炫耀,不为了参加选秀,而狂补某部名著,好在台上说
	,我自幼熟读什么什么梦。这些都是读书的悲哀,书籍的悲哀。

		书不是一件衣裳,不可以任人套在身上,就能上街招摇。

		前些天,一位以大眼睛著称,出演琼瑶剧而成名的女星为大众开出了一张推荐书单。\par
		于是,一石激起千层浪,许多网友质疑:她有时间读书吗?更有人直白地说:装什么文学女青年。\par
		大眼睛女星是否有时间读书,是否在装,这我们都无从查证,也无需查证。

		该星虽然确实曾作过一些缺乏文化和基本民族精神的事,但乐于读书,或者乐于装作乐于读书,还
	是好的。这说明,读书还是一件好事,不是令人不屑,甚至遭人唾弃的行为。

		也许,大家是误解了大眼睛女星,那么,怪只怪她出演了许多粗俗的角色,又把这些角色塑造得深
	入人心。从另一个角度看,这也是她艺术生涯的一种成功。

		另外,若说美女爱读书,很多人也会有这样的第一反应,装。美女怎么有心情,有时间去读书呢?

		读书,从来是那些无人问津的丑女的消遣。无事可做,无人邀约,只有啃啃书本,沉浸在文字的幻
	境里去,好逃避眼前残忍的现实。

		和仇富心态一样,这是一种仇美心态。\par
		读书又怎么关乎美丑?如果有关,也该是越读越美。\par
		正所谓,腹有诗书气自华。这气,简单些理解便约等于气质。\par
		真正的美,不是第一眼的惊艳,而是令人回味的温存隽永。\par
		视觉大约是最容易对神经造成刺激的,却也是最不可靠,最不持久的刺激。\par
		一个外表光鲜,谈吐粗鄙的女子,你会觉得她美么。我以为,相由心生,若心是污浊的,面孔自然也不会清澈。\par
		美女读书若是装出来的,只会显得无知可笑,破坏了脸面的美感,最终脱离美女的队伍。\par
		美女若真爱读书,只会越读越美,吹气如兰,你去嫉妒,去诋毁也无济于事。\par
		仇美的心态大可不必。还是那句话,读书是独自的事。

	\endwriting


	\writing{原来}{2007年02月27日 ~ 18:50} %<<<2

		\longpoem{}{}{}
		原来,我只是一缕 \\
		绵薄的呼吸 \\
		气息微弱 \\
		如一枚秋叶,一片冬雪 \\
		原来,我在命运的掌中 \\
		沉眠似婴孩 \\
		却又耐不住,梦的寂寥

		我想要的,一切 \\
		也许,只是生活 \\
		生活的一切,以及 \\
		一切的生活

		原来,我是一个我 \\
		原来,我不只是,一个我
		\endlongpoem

	\endwriting


	\writing{来临的三月}{2007年03月01日 ~ 10:54:59} %<<<2

		面对,接受,学习着从容淡定。如一株沉默的花树。

		\blankrev
		北京被阴郁覆盖,雾色苍然。流动的灰云,潜伏在一汪无色的天空。\par
		已是三月。我看日历上赫然的日期,才知道,季节又一次轻易地改变了。\par
		想到湖上缱绻的柳色,想到愁怨的丁香花。三月,是时候让我们期待,这一切的温存和明丽。\par
		春天。读它的名字,只容用轻而又轻的口吻。\par
		好像,在恋人耳畔的细语。

		\blankrev
		我开始想望着,一个草长莺飞的日子。\par
		让时间停止在开满紫色花朵的山坡。\par
		植物的种子,在日光下飞行,散播着春光的秘密。\par
		去年,我们在那里拍照留念。\par
		青草漫过我的鞋子,风拂过一片片花瓣,吹乱我的发。\par
		你不断按下快门。我总是来不及做好一个恰当的表情,就匆匆被定格在你的视窗。\par
		那是一些自然的照片。仿佛自然而无矫饰的生活,一样是平淡,琐碎,略显仓皇。\par
		我好像懂得了一些,始终被蒙蔽了的真相。\par
		也许,影集里那些甜美而端正的笑容,不过是一场真假难分的表演。\par
		纵使真实有一千一万种悲伤,面对镜头的那一刻,我们还是选择了幸福的模样。\par
		因为,你知道,这一个时刻即将成为回忆的线索。\par
		去年,我们翻越了那座开满紫色花朵的山坡。\par
		我们站在山顶,看远处的湖水。风晴日暖的天气。\par
		我说,我想睡,想沉醉在青草连绵的绿。\par
		你笑了。你没有言语。四处安静。

		\blankrev
		去年,我还有足够的力气,去登上一座小山。\par
		我还能够,与你并肩,看湖水的波涛,与岸纠缠。

		\blankrev
		又将是春。我一半惊慌,一半期待地等候。\par
		写信给小鹿。说起楼后的那株海棠。洁白的花,缀满挺拔的枝条,几乎遮蔽了小园的天空。\par
		我从未见过,如此高大,如此盛丽的花树。\par
		我们抬头仰望,望得神思焕然,痴心一片。\par
		没有雨,它便默立在那里,筛选着阳光,投下斑斑点点的影子,婆娑婀娜。\par
		落雨的天,它是低泣的诗人,一树的碧色,唱着沙沙作响的悲歌。\par
		那是一株美丽的花树。那是一颗多情的心。\par
		我撑伞走过它的身旁,看到淋湿的一路落花,无瑕的身子依偎在泥土,等候着轮回。\par
		你相信轮回吗。你相信人的前生和今世吗。\par
		我拾起一朵萎落的生命,安放在手心。也许,植物懂得这世间的一切奥秘,却从不说出只字片语。\par
		它们只是兀自地生长,开花,兀自地生与灭,信守着天地的约定,淡定从容。\par
		若真有来世,我愿做这样一株花树,默默地开放,守住几尺泥土,不断地向上,去触摸流云和星空。\par
		树比我们更了解宇宙和生命。\par
		你的前生,是否也曾是这样一株花树。那么,我便是另一个春天里走过树下的女子。\par
		我想,前生里,我或许不曾读书,不曾写下日记和诗歌。\par
		我只是粗布荆钗的女子,在你的树影下盼一封烽火里的家书,在你的落花里,缝一件寒衣,寄去边关,又识得流年偷换。\par
		有时,我倚住你的身躯哭泣。有时,我把絮絮的心事讲给你听。\par
		我想,那是一个寂寞的前生。一个女子和她的花树,几十年的时光沉默。\par
		所以,当今生遇到这一株海棠时,我才会亲切莫名。\par
		我相信这些看似荒诞无稽的前世与今生。虽然,它们虚无缥缈,无从验证与捕捉。\par
		来生,我愿做你门前的花树,默守一世的深情。

		\blankrev
		拉开窗帘,没有阳光刺入。灰暗的天,仿佛酿着雨。\par
		我从种种想象里抽身而出,站在窗前。三环路上,依然车如流水,马如龙。\par
		我小小的房间,在这偌大的城,不过一方窗口的灯光,不及萤火虫在黑夜的明亮。\par
		这窗口,总好像沉入深海的渔火,一撒手,便是希落,便是无可寻觅。\par
		我们都是躲在这样微弱的光芒中,阅读着人间,在纷繁里修行。\par
		还有许多的问题没有解答。还有许多的危险,没有被消减。\par
		没有人不是如履薄冰地行进。\par
		我感受着世界的冷暖,世界也体会着我的悲欢。\par
		我这样想着,撕下了一页日历。原来,每一个日子,都是独一无二的。\par
		三月。\par
		春天。\par
		我读这些名字,轻声细语,如一句句情话。

	\endwriting


	\writing{文字}{2007年03月02日 ~ 14:58} %<<<2

		文字,是一种生活。

		是不是,从仓颉造出它们,开始记录下狩猎,祭祀和占卜的时候开始,人们就有了区别于原始人的
	思维。他们开始参悟天地,开始书写下思念和离别。

		如果没有文字,我们如何知道千年前的一丛桃花,灼灼如火。如何听见,那蟋蟀在堂的鸣叫。\par
		于是,我时常感谢那个开始记录生活的人。\par
		最早的那一部诗集《诗经》,只用四言的简单语句,却将一个静穆典雅的远古呈现眼前。

		文字成全了人们抒发情感的要求,亦满足了我们去触摸时间之苍茫的希望。\par
		一种遥远的生活,以文字的面貌,纯白而赤真地袒露着无伪饰的真情。

		喜欢《邶风。燕燕》,喜欢短短几句:
		\shortpoem{}{}{}
		燕燕于飞,差池其羽。\\
		之子于归,远送于野。\\
		瞻望弗及,泣涕如雨。\\
		\endshortpoem

		这使我想到,小晏的“落花人独立,微雨燕双飞”。\par
		一样是湿润缠绵的天气,一样是燕儿们飞翔的羽翅。\par
		这是一幅被千古构想了多次的图画:离别的情境,飞去的燕子,原野上草色凄凄,落花万点,随风而逝。\par
		有忧伤吧,却是淡而无形的一缕青烟一般,只在两处的心上氤氲着相思。\par
		离恨是恰如春草的,更行更远还生。\par
		这一点,是你独守的窗口,寸寸柔肠里望去,那一端,是平芜无际,行人更在春山外。

		是被《燕燕》所描画的诗境所感染吧。才有不断的离别,被锁入精丽的文字,用相似的春天,抒发
	着字里行间的深情。

		\blankrev
		诗人们握着沾好墨汁的笔,走过春天的微雨,登上烟锁的重楼,写一篇流芳的辞章。\par
		他们的形象,在我的心中,总是一个硬瘦的背影。他那样缓慢地经过,在历史的青石板上踏过,与所有的人别无二致。\par
		诗人之所以被人们记得,是因为那些美丽的字,那些他在心中构想了,又吐露在纸页的情境。\par
		后来漫长的时光里,或许不会有人记得他的容貌,也无从记得。\par
		他的诗,他的文字,将成为他留给人们的全部印象。也就是说,他的字,最终成为了他,他溶化在他的文字。\par
		于是,我们读杜甫,会觉得他是暮年的老者,读李白,却以为他是永远的青年。\par
		而事实上,在同一时代里,李白的年纪却比杜甫大上许多。\par
		是文字所表现出的风貌,使我们产生了错觉。\par
		其实,李白也有衰老,杜甫也曾有少年的轻狂。

		\blankrev
		文字,是一种魔法。它不只是记录,不只是记忆力的补充。\par
		文字,幻化了这个棱角分明的世界,让它温柔多情。\par
		臆造了太多,无可到达的境地,容你去遨游飞翔,上天揽月,入海探宝。\par
		今天,我们读千年之前的诗。千年之后,我们有什么留给后人赞叹饱览?\par
		我无从回答。\par
		难道是粗制滥造的小说,难道是一场场炒作的荒唐闹剧,这些,都是我们时代的盛产。

		\blankrev
		静穆典雅的时代,自然有静穆典雅的文字。\par
		如今的我们,怎敢有所奢望。\par
		文字,是一面分明的镜子。

	\endwriting


	\writing{田的职业测试}{2007年03月04日 ~ 08:49} %<<<2

		Psytopic分析:您的性格类型是“INFP”(内向+直觉+情感+知觉)

		理想主义者,忠于自己的价值观及自己所重视的人。外在的生活与内在的价值观配合,有好奇心,
	很快看到事情的可能与否,能够加速对理念的实践。试图了解别人、协助别人发展潜能。适应力强,有
	弹性;如果和他们的 价值观没有抵触,往往能包容他人。

		INFP 把内在的和谐视为高于其他一切。他们敏感、理想化、忠诚,对于个人价值具有一种强烈的
	荣誉感。他们个人信仰坚定,有为自认为有价值的事业献身的精神。INFP 型的人对于已知事物之外的
	可能性很感兴趣,精力集 中于他们的梦想和想象。他们思维开阔、有好奇心和洞察力,常常具有出色
	的长远眼光。在日常事务中,他们通常灵活多变、具有忍耐力和适应性,但是他们非常坚定地对待内心
	的忠诚,为自己设定了事实上几乎是不可能 的标准。INFP 型的人具有许多使他们忙碌的理想和忠诚。
	他们十分坚定地完成自己所选择的事情,他们往往承担得太多,但不管怎样总要完成每件事。虽然对外
	部世界他们显得冷淡缄默,但INFP型的人很关心内在。他们富 有同情心、理解力,对于别人的情感很
	敏感。除了他们的价值观受到威胁外,他们总是避免冲突,没有兴趣强迫或支配别人。INFP型的人常常
	喜欢通过书写而不是口头来表达自己的感情。当INFP型的人劝说别人相信他们的 想法的重要性时,可
	能是最有说服力的。INFP 很少显露强烈的感情,常常显得沉默而冷静。然而,一旦他们与你认识了,
	就会变得热情友好,但往往会避免浮浅的交往。他们珍视那些花费时间去思考目标与价值的人。

		您适合的领域有:创作性、艺术类 教育、研究、咨询类等

		您适合的职业有:

		% 特殊的 · 手动水平左移吧,无奈,,,
		\indentenv{4\ccwd}{0\ccwd}{\hspace{-1ex}}
		· 心理学家 \\
		· 心理辅导和咨询人员 \\
		· 人力资源管理 \\
		· 翻译 \\
		· 大学教师(人文学科) \\
		· 社会工作者 \\
		· 图书管理员 \\
		· 服装设计师 \\
		· 编辑 \\
		· 网站设计师 \\
		· 团队建设顾问 \\
		· 艺术指导 \\
		· 记者 \\
		· 口笔译人员 \\
		· 娱乐业人士 \\
		· 建筑师 \\
		· 社科类研究人员 \\
		· 教育顾问 \\
		· 各类艺术家 \\
		· 插图画家 \\
		· 诗人 \\
		· 小说家
		\endindentenv

		如果感兴趣,你也可以来试试,我觉得还是有点准的吧。

		%http://www.psytopic.com/mag/post/820.html

	\endwriting


	\writing{春天的雪}{2007年03月04日 ~ 12:53} %<<<2

		一场春天的雪,不期而至。\par
		昨夜的我枕着窗上淋漓的雨声入睡。早起的睡眼里,却盈满了洁白。\par
		雪,无声息地落下,落满就要花开的世界。

		在这样一个日子,读冰岛诗人的诗,一样带着雪意的清寒和明澈。

		\longpoem{}{}{}
		像一件柔软的长袍 \\
		我把我的沉默放在你的肩上 \\
		像一个明亮的梦 \\
		我把爱抚编织成我们的缄默 \\
		……
		\endlongpoem

		我爱这样的句子,爱这样怀着温存的轻轻口吻。\par
		像一朵包裹着温暖的雪花,散落入发芽的土壤。\par
		又像一个说出口,就被遗忘的约定,即将融化在明日的晨光。\par
		有许多的约定却正是因为无可到达,而愈发迷人。\par
		正如,你对我说,我们去海上,霸占一座小岛。\par
		正如,你答应为我造一只船,在船体上绘画上我设计的图案。\par
		那些,是我们美好的约定。某一个夏天,某一个走失在记忆的八月。\par
		云朵膨胀,一场场雷雨,淋湿我们的双眼。

		\blankrev
		有时,我想起渔夫和金鱼的故事。\par
		贪婪的人,终于一无所获。有时,我怀疑自己也同那妇人一般,贪得无厌。\par
		所以,才有了悲戚,有了不平,有了那么许多扰乱了宁静的情绪。\par
		我无法是静定的湖水,却是汹涌的汪洋。\par
		我说,我想要一只船。这话里的意思,或许没人真正懂得。\par
		我需要的,是渡过。我独自的海,独自的困境。

		\blankrev
		我喜欢仰卧在床上,看窗外发射塔上闪烁的灯光。\par
		我把自己当作船,远处的光,便是海上的灯塔。\par
		也许,我始终是在航行。也许,没有人不是在波涛里沉浮。\par
		我愿意自己是沉默的。\par
		好像一片皎白的月光,照得澄江如练。这样的夜晚,我会架小舟出航。\par
		水总是多情的,无论陆地海洋。\par
		我愿意在所有人熟睡时,到一处峭壁下,去和天地悄悄地谈话。\par
		你会告诉我,宇宙最初的模样么。\par
		你会知道,末日的太阳,是如何沉入黑暗么。\par
		你大概也只有沉默,同我一般不发一言。\par
		可能,这便是我们谈话的方式。唯一的方式,也是最恰当的方式。

		\blankrev
		用铅笔,在白纸上匆匆画下草图,被雪被覆盖的屋顶。\par
		因为雪,屋顶是可爱的,显得稚拙天真。\par
		我用线条记忆下它的轮廓。也用心,记忆下一个落着雪的春天。\par
		很多时候,我是这样独行的人,像海上的渔夫。\par
		我看月起与日落,标记星辰的改变。\par
		时间在掌心滑落,如一尾顽皮的小鱼。\par
		我永远无法将它囚禁在鱼篓,它却一次次误入我的网中,再诡异地逃走,踪迹全无。

		\blankrev
		雪花在日光里融化。\par
		它们是微笑的,从未有过悲伤。\par
		悲伤,是人们愚蠢的发明。

	\endwriting


	\writing{离开}{2007年03月11日 ~ 22:05:01} %<<<2

		不愿道一句珍重。相逢的时光,我倍加珍爱。


		\blankrev
		静走的那天,整座城市再次降温。躲在房间里,看窗口明媚的几尺阳光,起风的下午,天空是碧蓝的安静。\par
		仿佛没有说一句告别的话。醒来的早上,打开手机,收到静的短信:我要走了…… 十点多就到沈阳。\par
		那个时候,睡意朦胧的我知道,她正坐在北上的飞机上,正在大朵的云层间穿梭。\par
		我的确,没有说一句告别的话。\par
		我和我的朋友,就这样轻易地,被生活的洪流冲散在两个截然不同的城市。

		\blankrev
		这是一个离开的年份。几天后,朱也将飘洋过海,求学他乡。\par
		真正的天各一方。从来未曾料想,每天在同一间楼房里度过着青春的我们,会相隔万水与千山。\par
		而且,这一切又来得如此突然。\par
		没有人伤感。年轻的时候就该这样,憧憬着无限的未来,而不是回忆旧时的光阴,不忍离别。\par
		会有更美的明天在等待吧。这想法,像一丛未开而将开的春花,隐隐地,在心头发着芽,萌着亮晶晶的希望。\par
		一起吃饭,静为我们盛汤,和这么多年来一样。\par
		总调侃,她是贤妻良母,下辈子若作男人定要娶她回家。静一脸得意的笑。她该是个幸福的小女人。\par
		高一军训的夏天,静提着行李包站在队伍的后边。日光充盈,绿树的影子投在焦黄的土地上,一点点散碎的印象一样,无法完整。\par
		我扭过身看她,注意到她双膝下的两道疤痕。后来,我才知道,就在那一年,她经历了一次手术。\par
		刚刚缝合并在愈合的伤口,如此坦然地暴露在日光之下。\par
		静是坚强的,她没有因为刚刚完成手术便放弃训练,申请免训,而是和其他同学一样,出晨操,站军姿,走队列。\par
		由于免训,而总是在训练场边休息的我,默默注视着那个身影:白色的体恤衫,清爽的短发,一双尚在恢复的伤腿。\par
		那时候,我觉得,我们是截然不同的两个世界的人。一个逃避,一个面对,一个脆弱,一个刚强。\par
		静的存在,在那个夏天,时常令我感到羞愧。\par
		后来,静的腿完全恢复了。她奔跑在篮球场,拥有一双健美而光洁的腿。\par
		只是,那一次手术,稍稍改变了她的行走方式,她弯曲膝盖的幅度较常人略大。这使我们能够看到她鞋底的颜色。\par
		静不信,于是满是把握地问我们:那你们到说说看,我的鞋底是什么颜色的?\par
		几个人异口同声,橘红色…… 静才相信,原来,她的脚真的抬得足够高,让我们看清鞋底。\par
		她笑了,又来来回回走了几次。撅起嘴,佯装着生气,丢下一句:讨厌。\par
		有时,我记起中学的点滴,总忍不住笑出声来。\par
		静,你都记得吧。许多的欢乐,单纯得如一页白纸,却又丰富得像一幅蜡笔画。\par
		我不再用文字去回忆。\par
		因为,与我们昨日的一切相比,所有的描述,竟都显得如此无力而苍白。

		\blankrev
		我们唱歌,唱那些被我们烂熟于心的歌。\par
		我一杯杯喝着橙汁,甜蜜的汁液,从舌尖滑向喉咙。我并不清楚,此刻的滋味,是甜美,还是酸涩。\par
		我们在小小的包房,被音乐包裹着,共度了整个下午。\par
		我看着你们。两张熟悉的侧脸。\par
		突然感谢,我短短的青春中,有你们作伴。这不是一句煽情的话。\par
		我只是在往日的浩瀚中迅速检索,却只捕捉在你们。没错,只有你们两个人。\par
		许多的朋友,仓促地经过了,又仓促地消失了。\par
		我说,我相信缘份,相信宿命。\par
		不然,人与人的相识与相知会有这样的区别。一些是亲密,一些是疏离,有的被轻易遗忘,有的却无法忘怀。\par
		感谢,让我们能够有所依靠,像姐妹一样,相互牵挂和惦念。\par
		这是上天多大的恩宠。\par
		田为此骄傲和幸福。

		\blankrev
		离开。我想象着飞往澳洲的飞机,不见在我的视线。\par
		问朱归来的时间。答案竟是08年的一月。感觉遥遥无期。\par
		我常常恐惧,这一年将有多少的发生和改变。\par
		也许,这便是我最近在仰望天空时,总莫名不安的原因。\par
		在海的那一边。在世界的另一个方向。\par
		朱说,她每到一座城市便会寄回明信片给我们。\par
		我于是想起陈绮贞的歌。也想到,一个人漂流在异乡的朱,会逢着怎样的阳光,和雨雪。\par
		那些雪花,是否和北京的一样呢,你会不会需要撑一把伞。\par
		你会想念么。\par
		想念我们,想念一碗热面的温度,一只煎饼的香味。

		\blankrev
		我依旧不愿意说告别的话。任何的一句,告别的话。\par
		因为,明天是一丛将开的春花,满是希望。\par
		你们会发现,田在这离别的季节里,异常沉默。\par
		你们不会知道,无声的田,心头承载着怎样的重量,在独自的隐忍中,默默承受。\par
		那是一些,不需要被发觉的事。\par
		我希望世界是安静的,即使有动荡飘摇的不安。\par
		我们,在回忆,也在未来。\par
		这几天,反复读海子的几句话:

		\longpoem{}{}{}
		永远是这样 \\
		风后面是风 \\
		天空上面是天空 \\
		道路前面还是道路
		\endlongpoem

		我以为我终于懂得了一些什么,但也许,什么也没有。\par
		你们只需明白,不道一句珍重,是因为倍加珍爱。

	\endwriting


	\writing{醒。之外}{2007年03月12日 ~ 10:54:19} %<<<2

		隐忍里,梦中的独行。

		\blankrev
		沉沉地睡着,像一个婴儿在母体中最初的混沌,像一切生命的开端,无所知觉,无所牵绊。\par
		梦一场色彩斑斓的大梦,只等到晨光熹微,照亮了窗口,才微微睁开双眼,去感觉这个满身光明的世界。\par
		我醒着,坐在清晨淡淡的色泽里。我回忆昼夜的更迭,竟在分明里望见,一个生命的圆圈,如此完满。\par
		原来,我所经历的,不过这圆圈上一段短短的弧线。\par
		由睡梦里诞生,再由睡梦里消亡。生命的安排,是这样巧妙,没有丝毫破绽。\par
		沉沉地睡着,像一粒种子在泥土中的蛰伏,像人间的种种穿越,由苦难中剥离出意义,懂得了跋涉的艰辛。\par
		要发芽,要用尽能量,冲破头顶的冰冷土层。要独自走过,许多个无望的路口,要在黑夜里摸索着潜行,不落一滴脆弱的眼泪。\par
		耐住冬天的寂寞,耐住磨难的时光。\par
		我好像那一粒种子,又仿佛,梦中的独行者。\par
		分不清睡梦与现实的界限,那无法丈量的距离,是一道迷题,由不得我们去苦苦求解。\par
		也许,有时,我在梦蝶,有时,蝶在梦我。如千年前庄周的梦一样,混沌成一片生命初始处的汪洋,浩浩汤汤。\par
		白日里的一切苦痛,如果不过是幻象一场,那么,又有什么值得悲戚和哀叹。\par
		就好像,那些夜晚里甜美的梦境,若不过是梦,醒来的世界里,又何必有不舍眷恋。\par
		苦痛是幻象。于是,我愿意这样相信着。于是,能够有更多的勇气,去冲破头顶的硬土,在绝望里生出希望的花朵。\par
		耐住命运,耐住跋涉。\par
		春寒的天气里,站在小园的腊梅树下。鹅黄的花朵已绽放枝头。\par
		碧色的蓝空,是一只深情的眼睛。清香四溢,我仰头望着一树喜悦的生命。\par
		浅浅的欣喜,如一泓春涧的溪水,漫过心田。我感觉到阳光的温度,我感觉到万物的力量,膨胀在这个茫然的宇宙。\par
		它们经过了冬天。它们渡过了轮回中惨淡的磨难,终达彼岸的春光。\par
		这也许,也是一切生命的必经之路。\par
		植物的轮回,在四季的流变,月光的轮回,在晦明的变换,人的轮回,在生死的更迭。\par
		这所有,又都如同日夜,光的来临,与光的消隐。\par
		是谁的巨手,绘画着这一个个圆圈。\par
		是谁在主宰,星球的运行,梦的起始?\par
		是谁,先在自己的梦里,安放了这个世界。又是谁,把一个个灵魂从母体的睡梦中唤醒,如一道晨光。\par
		不要问我从何而来。或许,在这个人间诞生之前,我们早已长久地存在了。\par
		那时,我们是一个分子,一种矿物。那时,我们在星际间漂流,彼此陌不相识。\par
		我们不知道,在千万年之后的相遇,这些爱恨情仇的缠绵与发生。\par
		画一段短短的弧线。一些重叠了,一些却渐行渐远,分道扬镳。\par
		你我皆在梦里么。\par
		一只翩翩的蝶,一场华丽的演出,不忍谢幕。\par
		这些真实的,抑或虚假的知觉,在夜晚来临,在白日来临。我惟有承受。\par
		像听一个古老的故事那样,幸福与苦难,都是精彩的情节。

	\endwriting


	\writing{记录}{2007年03月12日 ~ 11:08} %<<<2

		你对我说,不要不开心。\par
		你要我相信,一切的一切,终于会好。\par
		早上,我继续吞下药片,继续期待,一颗坚强有力的心脏。\par
		而脆弱总在吞噬。\par
		夜里,依偎着母亲的手臂睡去。\par
		有时,我渴望回去母亲的身体。如果,能够有再一次孕育,是否许多的苦难都能够得以避免?\par
		也许,命运是这样画出了残忍的路线。要我一步步跋涉而去。\par
		春天,在不远的日期上招摇。\par
		我记起许多个春天。有飞絮的,雪白的,温暖的春天。\par
		我的脸上缠着那条红色的纱巾。\par
		我坐在春天的院子里,被日光炙烤着,像一棵等待萌发的树。\par
		我记得那种气味。记得母亲的侧脸,温柔的弧线。\par
		我喜欢生命中最初的这些记忆。这样轻这样轻。\par
		我愿意在独自的时刻里想念。\par
		然后明白,自己是多么富有的人。\par
		不该有悲戚,不该有抱怨。\par
		怎么可以如此贪心,要将全部的美好占有得干净。\par
		想到这些,看窗外的春光,开始平复了心境。\par
		想沉默地思考。\par
		想奋力地生存。

	\endwriting


	\writing{《空房间》观感}{2007年03月16日 ~ 16:45:20} %<<<2

		一无所失,如同,一无所有。


		\blankrev
		电影《空房间》,一部适合独自锁在深夜观看的片子。\par
		幽灵般的男子,沉默坚忍的女人,绝望暴戾的丈夫。\par
		寂静的画面,了了的对白,在几十分钟的光影流动间,用另一只眼,洞穿现实身后的世界。\par
		荧幕前的观众,不必有欢笑,不必有泪水。影片一祯祯播映,你所感受到的,不该是情节带来的快感。\par
		这是一部淡如清水的影片,甚至因为空灵而显得诡异,亦幻亦真。\par
		你只应默然地独品,每一种眼神的微妙,每一处精妙的隐喻。

		一个在城市中无人居住的空房里四处寄居的男子,在又一次的寄居生活中遇到长期遭到丈夫虐待的女人。\par
		他带她离开那所囚笼般的房屋。女人跟随男子继续他寄居的生活。\par
		把传单塞入住宅大门,若几日后传单仍在,便撬开门锁,洗衣做饭,沐浴清扫,修理电器,一如在自家般悠然居住。\par
		两个人,在城市中的空房里寄居,烧一壶水,冲泡一杯茶,安静的生活,没有一句言语,目光里的默契却又仿佛早已相识千年。\par
		男子在音响里放一片音碟,让清越的歌声充满房间。女人沉默依旧,目光里却有如刃的锋芒。\par
		男子给她轻轻的亲吻。一切的一切却依然是寂静的,不需要一句对白。\par
		在漂流寄居的生涯中,他们遇到突发疾病死于家中而无人发现的老人。两个人将尸体用心清洗,又庄严埋葬。\par
		却也因此暴露了寄居的行为。男子被捕入狱,女人回到从前的房屋。\par
		女人几近发狂的丈夫不断实施着暴力,他咆哮着追问:你是不是在等他,你是不是。\par
		女人从未改变的沉默,仿佛冰封的湖水,没有丝毫波澜。一张失却了血色与表情的面孔,在镜中苍白如纸。\par
		直到有一天,她在镜中看到了男子。他在监狱中修行了神奇的法术,得以在他人的视线之外,自由游走而不被发现。\par
		只有女人能够看到她。只有女人真切触摸到他的存在。\par
		女人的脸上第一次有了笑容。她面对丈夫说,我爱你。令丈夫幸福到慌乱得不知所措。\par
		他却不知道,在拥抱时,女人真正亲吻的是身后那个他所感知不到的男子。\par
		在别人看不到的地方,男子以诡异的方式存在,两个人得以在现实之内超乎任何人想像地相爱。\par
		影片的结尾,他们一同站在体重秤上,显示却为零。\par
		世界在那一刻归于空无。\par
		像所有无人居住的房间。也或许,每个人的世界都是一所空的房间,本无所有。\par
		最后的一句话,出现在屏幕:\par
		It's hard to tell that the world we live in is either a reality or a dream.

		莫说人间如梦。梦又为何物,你可能说清。


		\subpart{空}

		《空房间》,由一所所无人居住的房屋,到最后归于零刻度的体重秤,所展现的种种意象,莫不令
	人想到空这一字。

		空,仿佛是一种缺失,是一种不存在。但却又恰恰是完满与充盈。正如零,本是一无所有的数字,
	却偏偏最为圆整无缺。

		想到卞之琳先生的一首诗:

		\longpoem{}{}{}
		我在散步中感谢 \\
		襟眼是有用的,\\
		因为是空的,\\
		因为可以簪一朵小花。
		\endlongpoem

		我在簪花中恍然\par
		世界是空的,\par
		因为是有用的,\par
		因为它容了你的款步。\par
		因为是空,方才有所用处。因为是空,方能有所容纳。空貌似是世间最大的缺乏,实质却是最无限的丰富。\par
		影片里隐藏在城市各处的空房间,给寄居的男子以安身之所,因其空,而得其用。\par
		男子淡定地居住在这样的空,没有陌生之感,比原本的主人更加安闲自在。\par
		一家旅行归来的人打破了房间里原有的温馨和宁静。男子在其中洗衣,晾晒,拍照的盎然生活情味,竟在一个瞬间里踪迹全无。\par
		男子仿佛不是一个闯入者,而是这空房间的真正主人。\par
		归来的主人反而令人心生厌恶。\par
		男子给这样一所所空房以另外的可能。他像是潜入了他人的生活,其实却是令空的房间获得新的生命。\par
		空房间不空,空房间接纳着寄居者,也接纳着一种奇异却美妙的存在方式。\par
		空房间,又是城市人空漠内心的隐喻。\par
		从表面上看,男人和女人的行为令人感觉怪诞。但是也正是这一种怪诞的行为,与其他人的“正常”形成了分明的对比。\par
		死在房间的老人,他无情而虚伪的儿女,在父亲去世多日后才得以察觉。孤独的老人,不知在人世忍受了多少冰冷的时光。\par
		男子和女人为老人细心擦拭身体。比他的亲生儿女更加精细。\par
		那一个画面令人在疼痛里升起温暖的热流。\par
		寄居的两个人,以一种异类的形态,游走在城市的缝隙,那一所所空房。\par
		所到之处,留下的是生活最平常也最动情的画面,女人在阳台晾晒一件件水淋淋的衣服,男子蹲在卫生间搓洗出肥皂泡沫。\par
		所到之处,令人感到的是生命的美感,轻而浅淡的一抹红晕一般,在泠泠的人世,如林的城市,露出温煦的笑意。\par
		一种空间的空,折射出城市生活的空,却有人在这空的世间布下反璞归真的简单。\par
		许多个画面里,我被他们寄居生活中的细节所打动。\par
		不过是一缕水壶的热气,一双等待落座的碗筷,竟如此真实地敲击着心灵深处最柔软的一块。\par
		也许,在每个人的潜意识中都期望着一种朴素生活的可能。\par
		与爱人默坐对饮,不去担忧住房和贷款,不去劳心失业或升迁。只于此刻,安享生活的一盏茶,一钵饭,一个吻。\par
		这人世本如一间间空房。每个人何尝不是如寄的过客。\par
		于是,影片便成为一个人生的微缩寓言。\par
		只去淡定地生活吧,而不要有恐惧,有忧愁。洗你的衣服,烹你的佳肴,爱你的爱人。\par
		打开一扇扇门,在空的房间营造生活之幸福,而不是去用空间囚禁什么,如那暴戾的丈夫,为了占有,而终一无所获。\par
		迎接一所所空房间,迎接种种未知的可能。在空的世间游走。\par
		因为是空的,而能够游刃有余,能够容心灵与肉身。\par
		空而不空,谁又不是那个四处寄居男子,谁又能如那个男子一般。


		\subpart{梦}

		很难讲,我们所居住的世间,是真实,还是梦幻。

		影片的结尾是童话式的,相爱的人获得美满的结局。这又是梦幻化的结局,男子独特的存在方式近乎一种梦。\par
		于是,我竟会害怕,女人会在某个清晨恍然醒来,发现所有的一切不过昨夜的虚幻。\par
		还好,影片就这样,在甜美的梦幻里结束了。\par
		也许,这的确是一个梦,也许不是。他们失去了身体的重量,如两朵灵魂。\par
		人的视线范围只有一百八度。可能,这也正是男人得以逃脱常人视线的原因。\par
		而人的身前与身后,是又两个一百八十度的。\par
		也便意味着,在这个世界上,总有我们的视线无法到达的地方。\par
		人对于世界的认识便很难全面。当你向前,再转身观看,这两个世界已经不共时了。\par
		而时间,是最莫测的力量,无时不刻不再改变着一切。\par
		海枯石烂,沧海桑田,莫不是时间的巨手在不知觉间的游戏。\par
		人的认识是有限的。现代人以为梦是现实世界的附属。\par
		梦却又为何不可能是另一种世界存在的形式。\par
		或许,现实世界才是梦的附属。我们的梦,才是生命真实的存在。\par
		在这世上还未有你,到这世上没有了你,你莫不是在一种无知觉的混沌与黑暗中。\par
		现实生命的知觉与之相较却好像沉沉睡眠中一次偶然的眨眼罢了。\par
		真实的,无限的,永存的,不是眼前的光明,却恰是漫长的混沌与黑暗。\par
		它与梦的存在方式更加相近。\par
		不要因为只看到眼前那个一百八十度的世界,便全然否定身后的那一个不在视线之内的真实。\par
		梦,让我们懂得人生存的意义并不只是活着。\par
		如果,人世是无限的轮回,那么梦便如死亡,是这一次次轮回上的衔接。\par
		梦,是一次练习。由白日到黑夜,由清醒到混沌,由喧嚣到寂静。\par
		梦游的人在梦中行走。\par
		清醒的人自以为清醒,其实仍不过是梦。\par
		影片里的男女看似梦幻的经历,却可能是世间最可望不可及的美。


		\subpart{重量}

		让我失去重量。好像不存在那样,却与你永生。

		这是我想象的一句对白。其实,在这部电影中增添任何的一句,都会显得多余。\par
		只需要画面,只需要无声里的发生。空,无,静。\par
		它好像一句禅语,不得说,一开口便是错。\par
		只容你去体悟,画面背后所要传达的意念。

		世间最重的是什么?\par
		你说是人心,你说是感情,你说是欲望。\par
		我看到被占有欲而逼迫入绝境的丈夫。那是一个也许被忽略了的,不幸的人。\par
		他的暴戾是因爱而起。爱本无错。只是他的爱中,只有占有。占有,成为爱的全部意义和手段。\par
		你是否注意到他的绝望。\par
		一种无可消除的重,为了“有”,而压于他的生命,令他不得解放。\par
		整部影片,绝望的丈夫,是真正的悲情人物。因为害怕失去而紧紧握紧的双手,终于一无所获。\par
		仿佛执意要抓住一捧清水那样,徒劳无功。\par
		女人是水。爱人是水。\par
		不该,也不会被一双不懂得的手握住,而彻底占有。\par
		世间最轻的又是什么?\par
		是这一捧清水。\par
		是相爱时,浑然忘我的两人。是情愿失去自己,交付于另外的生命。\par
		那一刻,人间是空的,世界是无的。\par
		万物无所重量。\par
		这虚妄人间,因人之有情,方显珍贵。不然,人又何异于山石?\par
		肉体是有重量的。\par
		灵魂却是轻的。\par
		相爱,不是两幅身躯的交融,而是灵魂的合一。

		《空房间》,是童话,也是寓言。\par
		用空容纳了,非常的丰富。我所看到的,只是不足一百八十度的局限。

	\endwriting


	\writing{失去暖气}{2007年03月19日 ~ 09:50} %<<<2

		房间停了暖气。缩在棉被里,重温冬天的气息。\par
		呼吸,一声声,一缕缕。感觉心跳,感觉我自己的存在。

		昏睡着,或者,怅怅地醒来。\par
		生活显得孤单零落。\par
		画杂乱无章的铅笔画,孩子们模糊的面孔,七月的温柔海滩。\par
		享受一个人的安静时光。这样洁白,无底色的时光。\par
		任我来来去去,拖着缓缓的步伐,走在自己的花园。

		田在这个初春里蛰伏。\par
		田想起许多个之前的春天。在梦里,在昨天,在日记的谎言。\par
		天空是灰色。\par
		每一年,你从这里走过,却从未知晓所谓人间。\par
		花落,花落,落一肩的粉红,仿佛,青春那惨淡而明媚的过往。\par
		我曾经不知忧愁。

		我没有难过,只是,寒冷令人沮丧。\par
		于是,听一首甜美的歌。\par
		有冰淇淋,巧克力,恋人,和电影。

	\endwriting


	\writing{明天}{2007年03月21日 ~ 19:49} %<<<2

		偶然间,又听到这首歌。孙燕姿《相信》。\par
		于是,想起01年的春天,同样的恹恹天气,料峭中有些茫然的前行。\par
		那一年,我15岁。\par
		那一年,为准备体育中考,在北航的操场独自奔跑。气喘吁吁地停在终点,大口呼吸。\par
		八百米,对于我,已经显得过于漫长了。\par
		但至少,那时,我还有气力跑完全程,拿到勉强及格的分数。\par
		从操场回来的途中,我用那台卡带的单放机,反复听这一首歌。

		\longpoem{}{}{}
		突然觉得我只是一个人 \\
		有点孤单浅浅的忧郁 \\
		我不知道明天会不会很美丽 \\
		虽然今天天很蓝 \\
		而云很白 ~ 风很凉
		\endlongpoem

		是有些疲惫吧,或者,在莫名的情绪里,感觉到生命中最初的孤单。\par
		原来,有这样多的挑战,要一个人去勇敢面对。没有选择,没有退路。\par
		感动于一句:该是我的总会来,就算挑战我不走开

		有多少注定的风雨在等待。\par
		在人生的许多路口,我们从不知道会遇到些什么。\par
		也许是欢乐,是幸福,也许是苦难,是遭受。路途崎岖,唯有微笑着,一路承受。

		我不知道明天会不会很美丽

		未来,是一个明亮的希望,是鲜美的诱惑,却又充满着未知的恐惧。\par
		只是淡定地走去吧。随遇,随缘,如风中的一粒种子,让命运的手轻轻安放。

		朱打来电话,明天便要飞往澳洲。\par
		田说,她最好的时光是中学。朱自豪地说,那是因为有我。\par
		两个人一起笑了。那情景,却分明有些伤感。\par
		许多日子,十几岁的日子,真如如一季春光。\par
		吵过的架,闹过的别扭,噘过的嘴,也显得分外亲切可爱。\par
		初二开学那天,你穿着白衬衫,站在讲台,怯怯地自我介绍。\par
		未曾料想,这个戴着圆圆眼镜的转校生,成为我最亲密的伙伴。

		我们说珍惜。\par
		我们说,难得遇见一个和自己一样没心肺的小孩。

		明天,明天,你乘风而去。\par
		在海的那一端,要有快乐和幸福。

		田在这里,在相信与怀疑的两侧,迟迟而行。\par
		有太多疑难等待着解决。有太多的梦,不忍心任它破碎。

		她会是坚定的。去生活。

	\endwriting


	\writing{逃}{2007年03月22日 ~ 20:44:44} %<<<2

		原来,是无路可逃的困境。

		\blankrev
		如果生命于我,是一场残忍的战役,那么,渐渐地竟希望,自己能有幸成为一名成功的逃兵。\par
		不愿见冲锋与杀戮的悲壮,不愿这个世界瞬时里,风声鹤唳,草木皆兵。我想逃走。\par
		逃出号角和呐喊的激昂悲壮,褪去那些虚假的荣誉或耻辱,还原平静的家园,躬耕于田,看日升月落的平凡。\par
		这也许也是所有生于乱世之人的梦想。\par
		但当战火摧毁了昔日的山河,当千里的路途之上,唯见枯骨,不见人烟,逃,又向何处而逃。\par
		如果生命本身亦是如此的真相,身陷狼烟之中的我,也只有无路可逃的困境。\par
		没有退路,眼前是荆棘丛生,是艰险重重。我看到一些勇士,对着命运的刀锋嚎叫狂笑。\par
		我看到他们赤脚踏过那荒芜的原野,一路高歌地经过。生命的残忍,在这一双双脚下,显得苍白而可笑。\par
		我敬仰他们,也赞美他们的无畏。但是,我终于无法与他们并肩。\par
		如果生命的确残忍,为什么又要伪装出高傲的坚强呢。\par
		如果痛苦是真实的,为什么不承认苦难,而强作欢颜呢。\par
		那些勇猛,或许是真实的。但有谁,能够真正在不幸面前谈笑风生。有谁,不想彻底逃脱这一场残忍的征战,恢复安详。\par
		从未有一场战争,给人们带来幸福。\par
		只要是战争,无论正义与否,都无疑是灾难。

		\blankrev
		因为疾病,我成为乱世上,流离失所的游民。\par
		因为疾病,我被迫荷枪实弹,时刻戒备地,与命运交锋。\par
		我感觉劳累了,这个无所附着的世界,在我的上空飘着,充满不安。\par
		像初春里恹恹的天光,灰蒙的云,了无生趣的窗。我一个人面对,忍住眼泪,却终于没有办法。\par
		我的身体被吞噬掉。惨淡的日月,再也无法掩饰我的悲伤。\par
		我承认,我脆弱。

		\blankrev
		若你问我,究竟要怎样的生活。我又该如何回答。\par
		或许,不过是一幅自由行动的身子,一个清醒的头脑,一颗无风浪的心。\par
		如果,容许我有无尽的奢求,容许我有不羁的幻想。那么。\par
		想把房间漆成青苹果般的淡绿色。想让它漂流在温暖的海洋上。\par
		打开我的窗,见到的,是洁白的水鸟,在水天一色的汪洋翱翔。\par
		半夜,我听到人鱼的歌声,月光里,礁石上有她们朦胧的倩影。\par
		你会在我的身边么。你会读一首古老的诗,或者讲一段英雄的传奇给我听么。\par
		也许是你,也许,只是我自己。任潮湿的海风吹乱我床前的书卷,让一个个朴素或华丽的句子,泄露给天地的沉默。\par
		物换星移,我小小的房间,漂流在无目的时光。我想,我将在这里了,我将恒久在这里了。\par
		仿佛,小人鱼化作的泡沫一般。

		\blankrev
		我想逃走。不是失魂落魄地逃,而是从容淡定地遁逝。\par
		而这分明的人间,哪里有一条路,任由我去拒绝现实的征战。\par
		命运将这一场战役安排入我的生涯。也许,这是必经之地的一处隘口,也许,这便是我存在的一切经历。\par
		即使,注定是不幸福的,却依然倔强地,愿意幸福,只愿意幸福。\par
		那是长长隧道里,仅有的光亮。\par
		我不去伪饰苦难的真实,不去否认内心的焦灼。如果,终于无处可逃。\par
		虽然,我是胆小的战士。\par
		流着眼泪,却咬住嘴唇,一路冲锋陷阵。

		\blankrev
		而战斗,永远是为了结束战斗。

	\endwriting


	\writing{南风}{2007年03月26日 ~ 15:01:04} %<<<2

		南风知我意。如梦方醒。

		\blankrev
		阳光和暖的下午,推开紧闭了一冬的窗。\par
		徐徐的南风吹来,叮叮咚咚敲打沉寂许久的风铃。泥土的气息,在饱含了温度的空气,不再寒冷。

		我坐在这里,知道风的奔波,是如何急迫得一日千里。

		吹去封锁的阴霾,吹去黯淡无光的天色,一汪碧空,澄明如此,宛若清泉一泓,润泽草木的干涸。\par
		春是希望,是天真的不安和期待。\par
		春,万物都积蓄着盛开的力量,只待一个恰好的时刻,瞬间迸发。

		一棵小草,一片新叶,都在春光里舒展着生命的欣喜。它们的神情是绿,是鹅黄,是草色遥看近却
	无的朦胧。

		南风,经过多少寂寞或喧闹的城,抚去谁镜台上积落的尘埃,看红颜如花,看四季流转,来我的窗
	前,唱一日温煦的歌谣。

		你从未失约,一个个大陆上的风季,你由海上来,你由南方来,褪去北国的冰冷沉默。\par
		这风是多情温婉,是水的造化,水作的骨肉。

		南朝的女子在思念中轻轻唱起:
		\shortpoem{}{}{}
		海水梦悠悠,君愁我亦愁。\\
		南风知我意,吹梦到西洲。\\
		\endshortpoem

		多少爱恨离别,在这风丝缱绻之间了。\par
		多少千古未了的哀愁,在你落花如粉的梦中了,在我零落一世的情怀间了。

		我感受到风,我感受到时间未曾改变的气息。南风,大约自古便是如此的味道,清苦的土,柔润的
	水,缠绵于一处。

		在这气味里,我们最早知晓了冬季的告别。让我有了一夜夜盛放了桃花的梦境。

		风晴日暖的天气里,采桑陌上试春衣。灼灼的花朵,更映红了新擦了胭脂的面颊。

		我好像是那片桃园的主人。溪水从我的柴门前流过,青山在我的屋后苍翠。\par
		花在风中飘落,飘落,用尽整个繁华又寂寞的春天。

		过路人,你何时经过我的门前,讨一碗井水,来解路途漫漫的饥渴。\par
		过路人,你何年重到我的桃园,看如旧的春花摇曳生姿,写一首流芳的诗词,待我用一生默念。

		在有花的梦里,我仿佛永是一个安静无言的形象,是日光里孤单的一条影,却怀着无限惆怅的深情。

		那是我所喜爱的夜晚。\par
		那是我所喜爱的梦境。如一首古诗的绵绵深意不绝,质朴洁白的人,赤子般的心。

		南风敲打我的风铃,我痴痴听,就荒度一个下午的光阴。\par
		清越的声响,飞越了千里冰封,那些曾被白雪覆盖的山原,那些荒蛮的田地和村庄。

		叮叮咚咚,这其中有孩子手中风筝的快乐。

		想和什么人,在倾斜的草坡上,放飞一只纸作的蝴蝶。\par
		想看它五彩的翅羽在蓝空下的翩翩,想看一个春天,在风里上升,上升,载满了幸福的可能。

		当它飞入云里,让我们并肩躺在初生的青草上,像两块不曾获得呼吸的泥土那样,任流云的影子抚
	过我们的双眼。

		那时,世界离我们很近了,生命好像回归它本来的模样,恬淡如婴孩的睡眠。

		南风渗入我们的发丝,我们的肌肤,我们的血液。

		春天,是一夜间解冻的河水,破冰的湖,奔涌而来。\par
		我站在原地,推开我的窗,闭上双眼,迎接一切。

	\endwriting


	\writing{思念。很玄}{2007年03月28日 ~ 13:11} %<<<2

		朱画的麦兜和麦嘜。看着就不禁会心笑起来。\par
		她说,她想家了。在海那边的秋天,朱,你的窗前在落叶么。\par
		北京的阳光很好。田的生活像一只懒猫。

		翻看你空间的文字和列表。读到你的五道口饮食杂记。熊家,桃屋,还有会议中心的寿司店。\par
		你忘了,还有学校正门对面的那家一心。都是我们爱吃的地方。突然想和你一起去吃,却知道,我们已隔了那样远了。\par
		又想起,中学时候放学我们总要在外边逗留一阵才肯回家。我们去吃DQ,去吃拉面,去吃煎饼。\par
		有一段很好的话被我记下来,你问,什么是幸福?\par
		我一本正经地回答,幸福,就是每天吃一个煎饼。\par
		幸福,就是这样简单琐碎,看似可笑的点点滴滴吧。\par
		一起啃煎饼的快乐感觉,混杂着鸡蛋和面粉的香味,这样朴素却真实。\par
		春天来了,知春路上的桃花却因为拓宽道路而被移走。\par
		曾经,放学的路上,在桃花开放的日子里,你总是逼迫我站在道旁的树下,唱一首《在那桃花盛开的地方》。\par
		有时,我们也一起唱。唱些儿童歌曲,或别的什么。总是扯着破嗓子,唱得难听,却不以为耻,反以为荣。\par
		那样的春天,不再回来,那样的我们,也被封锁入记忆。\par
		朱,我要怎样感激上天的安排。拥有了如此纯撤天真的青春。\par
		那时,我们总是笑。笑到肚子疼,笑到直不起腰。\par
		你临走前的最后一通电话里,我说,现在的田,感觉很压抑。\par
		多数的时候,我的世界是如此安静了,远离人群,远离喧嚣。\par
		在这样的安静里,有恬淡的心境,也有寂寞。时常怀念叽叽喳喳的那些时光,是吵闹的,也是最可爱的。\par
		希望自己永远是那个蹦蹦跳跳,不知疲惫的孩子。却不可以了。我们终于成了大人。\par
		但我依旧喜欢,和你在电话里想入非非地胡说八道,说些不着四六的事,乐作一团。\par
		田本是这样爱笑的。

		那年,你写了一篇《给田田》。今天,我又仔细地读,满心是说不出的滋味。

		“田在SPACE上写下了“给朱朱”,使得朱坐在电脑前泣不成声。其实朱在很早以前就想给田也写些东
	西,写原来那些快乐,那些傻事。可是朱的文笔不好,只能闷在心里,存在脑子中,记一辈子。朱朱在
	想,也许田是上帝给朱最好的礼物,一个一生陪伴的亲密朋友。在田那里,朱是透明的,没有秘密,因
	为朱知道,有什么是不可以和田讲的呢?近7年的友情,会继续到17年,更久更久。在朱的世界里,可
	以没有漂亮的名牌衣服,踩上去就很舒服的NIKE鞋,爱朱的男朋友…… 一切一切的身外之物,但是田,
	不能缺少。因为田早在朱这里刻骨铭心。

		记得田曾经问朱,在朱的心里,是用什么来形容我们之间的关系:沙石,珍珠,黄金,木材,玉石
	,贝壳,树叶还是湖水?朱没有思考就选择了贝壳,于是答案说,我们是有缘的朋友。2000年的夏天,
	让朱认识了田,大眼睛,聪明的女孩子。天真的笑容,有时的表情也很无辜,让人心疼。朱在想,起初
	我们并没有如此亲密的关系,又是游戏,让我们从此密不可分。一切都是缘分吧,一个在东城生活了12
	年的孩子,转校来到了海淀,迷迷糊糊的进了中关村中学,偶然的被分到了初二10班,就这样,缘分左
	右着我们,从陌生到熟识,再到亲密无间。就是这样不可思议,又仿佛冥冥之中命中注定。

		朱朱是幸福的,不管是开心还是难过,一直有田在分享和分担。虽然有时,朱是爱生气的,会耍性
	子不和田说话。但是,那是朱的小孩子气,朱在撒娇,在田面前。田说过,最不喜欢那样的朱。其实田
	不知道,朱经常会在赌气时偷偷的哭,不和你说话,只是不知道要如何开口,死要面子活受罪。但是,
	以后不会了,朱长大了,懂了好多,再没有小性子,臭脾气。

		有时,朱会觉得,我们很像麦兜和麦麦。两只肥肥的小猪,在春田花花幼稚园,呆头呆脑,说些傻
	里傻气的话,做些荒唐可笑的事。嘻嘻,我想任何人都不会把穿白色羽绒服的人想成北极熊,红色的是
	大苹果,蓝色的是蓝精灵,绿色的是绿豆蝇;也不会站在马路上,想象着所有人都穿着肚兜在骑车,散
	步;更不会因为吃了一个绿豆煎饼而幸福的笑上半天。。。也许这个世界上,脑子进如此多水的人并不
	是很多,而且又难得凑到了一起。

		田说不愿意看到朱伤心难过时的样子,在田眼中,朱总是七分天真,三分狡黠地笑,已成定式。希
	望正如田说的朱会幸福,田也会幸福,最终我们都会幸福。答应你,你最美好的心愿。”

		我给你留言:突然觉得,不管发生什么,想到我们拥有这样的友谊,还有什么能让我们畏惧的呢?

		很多时候,也许,正是你们,身边的你们,给了田许多许多的勇敢。不然,我有什么能量,能够如
	此坚韧地去相信幸福呢?生命为我安排了苦涩,也同时赠与我太多太宝贵的美好。于是,田没有怨恨,
	没有不平。惟有感激。

		朱,现在的你好吗。这一刻,我也有了想念。有些东西,竟然真的能够如影随形。

		真的好喜欢你画的麦兜。

		在播放的歌曲,像我们的约定:

		\longpoem{}{}{}
		明天心也要做伴 \\
		也要勇敢 \\
		不管是否天涯两端……
		\endlongpoem

	\endwriting


	\writing{辛夷。花开}{2007年04月02日 ~ 11:22:08} %<<<2

		四月。一切美与光明,归入崭新的希望。

		\blankrev
		一个四月的晴天,一个日光充沛的早上,无声息地由黑夜中绽开,袒露着无邪的心。\par
		我在安静里看春的发生。\par
		铁路旁的杂草丛已是绿意茸茸,孩子们在南风里追逐一只气球,老火车缓缓从盈满了欢笑的背景中驰过。\par
		远天是涤洗一新的蓝空。\par
		一些似曾相识,又全然陌生的画面,在我的窗口显现。\par
		我透过玻璃窗,满心的惊奇和欣喜。\par
		四月。在日历上标注痕迹。一场绵绵的雨水,一日骤起的风沙,一屋明媚的光线。\par
		它是这样多变。忽而风,忽而雨,忽而晴。\par
		一刻是满脸淋漓的泪水,一刻却又是天真顽皮的笑。\par
		四月,永远是长不大的孩子,是站在青草丛里穿着花裙子的小姑娘。

		\blankrev
		母亲说,楼前的玉兰花开了。我却不愿去看。\par
		我知道那几株玉兰。总是零落消瘦的模样,在春光里,疲惫地开出几朵惨白色的花,却又摇摇欲坠。\par
		那情景,令人疼惜。不知今年,它们的神色有没有改变,或许已是繁花满树。我却仍不愿去看。\par
		记忆里的玉兰花,开在幼儿园的院子里。一样是四月,一样明朗的天。\par
		那是两株高大的树,洁白的花朵缀满枝条。每一年春天,母亲总把我抱上椅子,让我站在上面,来和它们合影。\par
		那一张张属于四月的照片,有玉兰花的纯明,有戴着粉色毛线帽的我,有刚刚脱落了乳牙的口腔,毫无遮拦地笑。\par
		那些照片,被母亲插入一本本相册,又放入抽屉。它们就沉睡下去,在时光的彼端,把今日的我等待。\par
		从前的四月里,躺在睡眠室小木床上的我,也曾面对窗帘上忽明忽暗的云影发呆,也曾想象,十年,二十年后的自己。\par
		我想,当我二十岁,该有了一双晶亮的高跟鞋,像所有走起路来声声作响的阿姨一样。\par
		我想,当我二十岁,就可以留一肩长发,再学着公主的模样,穿一袭白纱。\par
		我并不懂得未来。我只是知道,那些遥远的,远到不辨真假的渴望,终于会在某天变得清晰无比。\par
		像一首轻轻哼唱的摇篮曲,渐渐熄灭了声响,却在梦境里真实起来。\par
		二十岁,我曾在四月的天光里想象,二十岁,我如何能够到达的世界,却在这一个瞬间里成为此刻。\par
		二十岁,又将成为不再的符号。是谁在我的生命刻下这样许多刻度,来把光阴丈量?\par
		二十一岁,我望着即将到来的生日,一时间,竟茫然若失。

		\blankrev
		想回去幼儿园,看看那两棵玉兰。\par
		两年前的夏天,和苏经过那里,透过粉刷一新的栏杆,看到孤单的秋千,在浓密的树影里摇晃。\par
		还是那一架秋千。还是被称为娃娃城堡的乐园。\par
		只是,架子的油漆不知厚了多少层。只是,曾经高大的城堡,在我眼中已如玩具。\par
		开着粉红绒花的合欢树把枝条伸向更远的天空。\par
		还会有孩子在树下争抢一朵小花吧,还会有谁握住它在午后的阳光里甜甜入睡吧。\par
		笑声仿佛尚在耳际,阳光也如旧时温煦,却已无可触摸。\par
		我站在那栏杆外,眺望教室的窗口。有钢琴声传来,然后,是孩子们高高低低的歌唱。\par
		我的记忆,好像与他们的今日重叠。\par
		有一天,他们中的某人,是不是也会站在这栏杆外,痴痴地如我般,想起些什么。\par
		玉兰树该是依旧茁壮。它是否能记得那个戴粉红色毛线帽子的小女孩?\par
		是否记得她脱落了乳牙的笑,还有,她一年年拍下那些照片的年轻母亲。\par
		多么远了,又多么贴近的春天。\par
		柳絮在飞,幻化了一座城,幻化了我们的昨天,雾失楼台般,如烟如尘。

		\blankrev
		十七岁的春天,去看颐和园的玉兰。\par
		那一天,我们绕着湖水走了很远。解冻的春水,撞击着石垒的堤岸。\par
		花瓣在斜飞,柳丝在斜飞。\par
		我仰头看那几株惊人的花树。成千上万的花朵密布在我的视线。\par
		也许,这便是所谓皇家园林的气派,连玉兰树,也如此惊天动地。\par
		数不清的花朵,如星辰在夜空的散落。原来,花开可以是一朵的孤芳自赏,也可以是盛宴般的眩目。\par
		而我,并不喜欢后者的热闹。仿佛宁可是孤芳自赏,也要觅得安静。\par
		一场过于华丽的盛放,只令我在惊叹之余无所适从。\par
		玉兰不是艳丽的芍药,不是惹人的牡丹。玉兰是着素衣的女子,回眸轻笑,凌波而去。\par
		在植物园,有一处木兰园,总是从园门经过,却从未进入。我不知道木兰是怎样一种花。\par
		想象里,大约是低矮的草本花卉,且不免大红大紫的色泽。\par
		直到读到李商隐的《木兰花》,才恍然,原兰木兰便是玉兰。

		\shortpoem{}{}{}
		洞庭波冷晓侵云,日日征帆送远人。\\
		几度木兰舟上望,不知元是此花身。\\
		\endshortpoem

		这一首诗,更是有美丽的身世和传说。据宋人笔记记载,竟传是由玉溪生的鬼魂所做。\par
		在南方那一片充满传奇的湖上,在茫茫的烟波里,小小的木兰舟,荡漾其上。\par
		木兰舟,同桂舟一般,都是诗人眼中美妙的载渡工具。很多时候,或许它们已不止是一架小船,而是一种诗化了的象征。\par
		这令我对玉兰有了更深的好感。\par
		又再读书间得知,紫玉兰还有另外的名字,辛夷。\par
		王维辋川别墅中建有辛夷坞,那一首诗,更是充满了花开空山的禅意。

		\shortpoem{}{}{}
		木末芙蓉花,山中发红萼。\\
		涧户寂无人,纷纷开且落。\\
		\endshortpoem

		木芙蓉,这清澈如水的名字,原来一样是指玉兰。\par
		玉兰,多么平凡的花,却原又这样多不凡。\par
		看它的花开,在还显坚硬的风里,在我斑驳散碎的记忆,在一首首芬芳的辞章。\par
		玉兰,属于这个四月,属于每一个四月。

		辛夷花,我愿意读你古老的名字,愿意听你空山里,静静的开放和陨落。\par
		那一切的发生,都一如春光,美得悄无声息。

	\endwriting


	\writing{田野}{2007年04月04日 ~ 18:49} %<<<2

		那些坏天气,终于都会过去。

		这是阳光遍洒的一天。坐在车里,看飞驰的世界,看人们脸上洋溢的希望。\par
		世界很匆忙,世界被照耀着,蒙受着一种复活般的喜悦。\par
		于是,在一些时刻里,我忽略了自己的疑难和悲伤。\par
		听这一首不厌其烦在播放的歌。\par
		多少次,我默默去坚忍地相信,默默记得这些歌词,又默默念起。

		你的生命她不长,不能用她来悲伤。

		都会好的,我对自己说。坏天气终于过去。好像我曾经也曾经历的那样。\par
		风雨与阴霾,在回望处,已风轻云淡。\par
		只是记住那些挣扎在病痛和失去中的时光。一个人的独行和穿越,于隧道中循着微光匍匐。\par
		一些春天,一些夏天,在本该风和日丽的年纪里,一霎时天昏地暗。\par
		有时,我看到一个个身体孱弱的田,站在原处。她有怨恨,有哀伤,她写下安慰自己的话,却又终于流下眼泪。\par
		因为残破的身躯,她倍受折磨。她渐渐开始明白,无处不在的幸福,渐渐懂得珍爱,平淡的生活。\par
		一场雨,泥土的气味,一把伞,恋人的手掌,一个早上,窗棂上渐起的日光。\par
		原来,生命是这样琐碎而细小的片刻,是这些不经意中涌起的淡淡喜悦。\par
		也许是充满波折的生涯,我却依旧无比热爱它。

		在田野上。我常常想象那样一片田野。让我站在中央,也仿佛站在了天地的中央。\par
		青草蔓生,光阴仓促。好多时候,我以为,人该如一株植物那样,随遇而安地存在,抛弃自我的意识。\par
		听任安排,享受拥有的一切,雨露,阳光,氧气,而不生出妄念,来搅扰心灵的平静。\par
		坚定地向上生长,去接近晴空。\par
		然而,又有谁能够真正做到心静如水,淡定从容。\par
		无可降伏的心,在撞击心的牢笼。我们被自己重重围困。

		该如何面对我们自己,该如何安放我们自己,这是所有人面临的问题。而不止是我的疑难。

		田,要有更勇敢的心。田,要做强大的人。\par
		我对自己说,一次次下定决心。特别是这个阴云渐远的春天。\par
		忘记,去学着忘记。记住,去学着记住。\par
		你还有微笑吧?哭的时候就去照镜子。悲伤的模样是多么丑陋。\par
		总是这样,便止住了哭泣。哭泣没有用处。若反复觉得自己可怜,只会令可怜加倍。\par
		即使是困苦重重,田依然倔强地,只愿意幸福。

		明天早上,当太阳再升起来,便是21岁的自己。\par
		21次的花开与花落,还没有看厌春天的轮回么。我依然这样喜欢春天。\par
		喜欢四月的自己。你问我为什么喜欢过生日。\par
		我说,因为可以收到很多礼物,我喜欢被礼物包围的感觉。\par
		也许,骨子里,没有不物质的女人。\par
		我说,为了每年都收到你的礼物,我要用力地健康地生存下去。\par
		这话说得悲壮,却令我升起一个晶亮的希望。\par
		良的礼物我还没有打开。我说,要到生日的那一天再打开。\par
		好像揭幕一个全新的自己那样。\par
		好像那淡粉色的包装纸下,包裹的,是一个无比甜美的未来。\par
		我相信。\par
		我这样坚定地相信。

	\endwriting


	\writing{春山}{2007年04月07日 ~ 15:47} %<<<2

		春山如笑,因为有花烂漫。\par
		杏花在微风里怯生生地开了,缀满一树枝条。\par
		我总是难免感动于如此的春光。\par
		看远天的云朵在嶙峋的山腰飘浮,看原野尽头那一列萌发的树木在晴空下舞蹈。\par
		仿佛是时光的纪念册中最光明的一页,被一支彩笔图画得清婉动人。\par
		每一个春天,这样似曾相识地经过着,如一只缓慢的爬虫,漫不经心却又万分谨慎。

		看去年拍的照片,一瓣落花停留在你宽厚的手掌。\par
		这使我想到每个人的经过。想到许多出现了,又渐渐消失的人。\par
		由陌生到熟悉,再由熟悉回归到陌生。有多少人,是以这样的方式交错入彼此的生命。\par
		后来的陌生,是一种顺理成章的遗忘。\par
		某一个人,在某一个时刻,成为了一个模糊的剪影,一个恍惚的名字。\par
		某一段经过,在某一天,幻化为一个遥远的故事,一串真假不辨的对白和动作。\par
		如那一瓣落花,曾经年少的温柔,坠入谁的往事如烟。\par
		如那一双手掌,从前年少的晴朗,捧起谁的昨日,再轻轻放下。\par
		那是我们所无从预见的明天。\par
		好像当初的你我,无从知道今日的离别与陌生。\par
		人与人的交错,纯属意外。人与人的失散,又如梦幻一场,惊恐中醒来,已失向来烟霞。\par
		来不及挥手道别,昔人已远。\par
		乘风万里,远走高飞,遗忘的终于被丢失干净,记住的终于也一丝丝消磨殆尽。\par
		如果没有记日记的习惯,我大约不会有这样好的记性,不会有闲心去记住一场雨,一个荒废的下午,一张忧郁的侧脸。\par
		那样多散碎的细节,会被我毫不留情地抹去。抹去了,也便如没有发生过一般。\par
		好像,我在许多人的世界所遭遇的那样。也许,会有同学在看毕业照时,竟叫不出我的姓名。\par
		当然,如今的我,在翻看那些旧照片时,也会对一些面孔哑然。而且,这些面孔的数量明显有逐年增多的趋势。\par
		没有人不会被遗忘。没有人能够霸占谁的记忆,赖着不走。\par
		如花的年纪,自然也该有落花的姿态。是这样轻的,离开了,满怀深情与不舍,却又伪饰得仿佛淡定从容。

		这一切,原来都是时光的把戏。\par
		这一切,莫非是我们辗转奔波中的真相?\par
		亲爱的朋友,如果明天我们便要失散,我该用些什么去保管今日的欢乐,用什么去封存不多的青春。\par
		如果你将远行,如果你后来的故事里不再有我的出现,你是否也会有所留恋。\par
		这些,是我不该去问,也无需去问的。\par
		这一路上,谁不是匆忙仓皇的过客。\par
		我们从未停留。我们不忍停留。\par
		我惟有感谢,全部曾有的懂得与珍爱。\par
		这些年少的温柔与晴朗,是你我不灭的春天。

	\endwriting


	\writing{腔调?}{2007年04月09日 ~ 15:51} %<<<2

		[亲爱地] —— 这个在歪酷的博,最早是想用来写一些随心所欲的话的。\par
		不经过大脑,任性任意地脱口而出,不去在意谁在听,谁在读,谁会有什么表情。\par
		但不知道为什么,仿佛是写字时间久了就进入一种无法摆脱的模式。\par
		想说得直白简单的话,非要用奇怪的句式去表达。\par
		很坦率的情绪,非要含含糊糊,欲说还休地写出来。

		原来自己写的字已经有了一种腔调。\par
		好像那些掐起嗓子说话的主持人,已经忘了怎么正常去说一句话。\par
		难免感觉造作。有时候开始讨厌自己不厌其烦写生活里这些那些琐碎。

		人不能一辈子都在回忆里沉湎,在未来中畅想,在幻想里飞翔。\par
		我承认,也许多数时候,我的生活不切实际。\par
		而这,又恰恰是我所选择的生活方式。\par
		一个小情调的人,过一种小情调的日子,或许也无可厚非,自得其乐。\par
		但这一过程里,我又在不断反省和质疑。

		开始只想说简单的话。所以句子也越写越短。\par
		不再用一个个繁琐的定语去修饰什么。力求用最普通的词汇,精准表达出句意。\par
		这好像是个很高的要求了。\par
		因为难以达到,而难免对自己失望。\par
		创作中的挣扎,是否所有写字的人都曾经历?

		不知为什么,天气暖起来便有写小说的冲动。\par
		一些人物整天在我头脑里打架。我看到他们的面孔,听见有一句没一句的独白或对话。\par
		要写下来。可好多时候坐下来,一切故事却又烟消云散。

		灵感总是很调皮,喜欢捉迷藏。

	\endwriting


	\writing{四月天的幻想}{2007年04月09日 ~ 17:39:38} %<<<2

		我爱,请原谅,田的世界不着边际。

		晴朗的天,仿佛年少,仿佛飞驰的四月,眩目灿烂,从我的指缝漏下一地光芒。\par
		谁不曾有如蓝空般的青春。谁不曾痴狂,不曾张皇,不曾在独自的夜晚抬起头,看亮起又熄灭的灯火。\par
		影影绰绰的一些光亮,在身前,在身后,一处处埋伏,一处处躲藏。\par
		谁抄录一首诗给我,《你是人间四月天》。\par
		听林徽因轻轻在说:你是一树一树的花开,是燕在梁间呢喃,你是爱,是暖,是希望……\par
		湖上的春水已一夜溶逝。它的脆弱,它的温柔,在细小的波浪里起伏。\par
		我说,我愿在湖岸上筑屋。如瓦尔登湖畔的隐居者,去度过一个个寂寞的日月。\par
		我想有一条船,好让我仰卧在碧波上,去目睹星辰的改变。\par
		“斗柄东指,天下皆春。”我神往那些夜观天象的古老岁月。\par
		那时候,我们与天地是在一处的。人一如鸟兽草木,无所损害,取我所需而已。\par
		人间的四月天,繁花满目,蝶舞莺飞。\par
		独坐小园,看樱树缤纷,又如何不引人情意绵绵,多愁善感。\par
		想着的,不是林妹妹把锄的一双纤手,却是武陵人误入桃源的小舟。\par
		有多少的船,载我们进入世间隐藏的奇幻。\par
		许多的传说故事里,更有人乘舟漫游了夜空。\par
		在古人的世界里,江河与天空相连。误入奇境的人,总是架了一叶小船。\par
		蓝空是海,不然海如何望不见尽头,而与远天相接,不见界限?\par
		头上的这一片汪洋,又远比海洋辽远。那纵深无底的夜,星座的迁徙,诉说着时间与空间的无限神秘。\par
		是杳无的一个梦一般。我好像从未知的另一端睡去了许久,才又在世界的这一侧醒来。\par
		于是,好多时候,我依稀记起踏过奈何桥之前的自己。\par
		我仿佛恍惚里辨认出你前生的面孔。倘若我没有喝下那一碗孟婆汤,是否今生便能在人海茫茫中将你找到。\par
		也许,我们早已在这里相遇了,却全然不知。原来的我们,都忘记了那一端的故事,那些或悲或喜的曾经。\par
		一定也是一只小船,载你我到这此岸人间。\par
		我感激,降生之日正是这风晴日暖的四月天。\par
		我在母亲的子宫中登上渡口。我又一次醒来。船在我身后消失,等候下一次轮回的旅途。\par
		我睁开沉睡了不知几个世纪的双眼。晴朗的天,光芒刺目。\par
		开始新的故事,开始新的年少,这些如春的晴朗,要尽情享用。\par
		去飞驰,去幻想,唱一首嘹亮的歌。去爱,去经历,写一篇飞扬的诗。\par
		纵使后来的我们还是终将忘记。我终于无法在来世的人来人往中将你认出。\par
		依然让我去相信,去付出所有,只属于青春的热情。\par
		我看到光亮。碧绿的湖水,枉然的岁月,一样如波,一样有起伏风浪。\par
		我说,我愿在湖岸上筑屋。\par
		我说,我要去静静生活。仿佛在无声中参悟着天地的玄机,却又好像什么也没有。\par
		我抬头看蓝空的清澈,低头涤荡开一片心灵的涟漪。\par
		栽几亩荷花,种一畦蔬菜。乐天知命地去安享生命的恩赐。\par
		你会是湖上荡舟而来的访客吗。远远在烟波浩淼上唤我的姓名。\par
		四月天,容许我扫开积落的花径,与你把酒东篱。\par
		我等候着你,又仿佛是等候着我自己。\par
		一树一树的花开,才是整个春天最要紧的事。\par
		要被宠爱着,然后固执地孤芳自赏。

		\verb|^_^|

	\endwriting


	\writing{灿烂过}{2007年04月11日 ~ 21:19} %<<<2

		\longpoem{}{}{}
		思念是一种自由 \\
		如此的赤裸裸 \\
		倒影心的轮廓 \\
		回忆是一种寂寞 ~ 只能接受 \\
		变迁的时代中 ~ 心情却依旧 \\
		是不是我们都太渴望了 \\
		那一次花季后 ~ 蔓延了 \\
		你和我 ~ 面对各自生活 \\
		和整个世界的寂寞 \\
		在美好季节后 \\
		拥抱我 ~ 在你幽幽梦中 \\
		我们之间除了思念 ~ 还缺了些甚么 \\
		桃花曾灿烂过
		\endlongpoem

		袁泉的专辑《孤独的花朵》。干净的封面和声音。清澈如水,纤尘未染的纯粹。\par
		适合在春日里静静聆听,心无杂念,独自沉默。\par
		那些关于女子的微妙心境。暗恋,光阴,落花,诗歌,梦与错觉。\par
		这个寂寞的世界,充满回忆的色彩,光移影动的生涯。\par
		喜欢这样的歌声,喜欢这样的袁泉。\par
		不只是一幅美丽的面孔。她在用心灵去诠释生命的种种喜悦与悲伤。\par
		如此安静,看季节的离别,人的失散,在如花的年华。\par
		桃花曾灿烂过。\par
		我听着听着,竟泪流满面。

	\endwriting


	\writing{自己与世界等等}{2007年04月13日 ~ 19:43} %<<<2

		\subpart{一。}
		是谁说,当一个人过度关注与自己的生活,便忽略了外围许多存在的真实,变得容易不快乐。\par
		就好像,将自己封锁,包裹,远离人群,往往不能解决内心对于社交的恐惧。\par
		我不知道,自己处于一种怎样的状态。\par
		只是渐渐乐于沉默。也许,我过度关注了自己。在小房间里安放了太多的镜子,去反思和省悟。\par
		人类多数的烦恼,是毫无意义的。\par
		却总是为明日的种种不确定性,透支了悲伤和不安。\par
		人仿佛总是动荡地生活。


		\subpart{二。}
		和一个病友的妈妈通电话。她推荐给我一首歌《自如》。是一首佛教歌曲。\par
		她说,她在困苦中,反复默念那几句歌词,便得到一种心灵的解脱。

		用感谢心去付出,以欢喜心来受苦。\par
		用大智慧去领悟,以大慈悲来祷祝。

		我也静静去听这样一首歌,心如明镜,一时豁然。她说,人生莫不是苦,但每一次磨难都是修行。\par
		什么时候,我们能够不去对灾祸心存怨恨,能够面对不幸一笑了之,大约便是真正获得解脱的时候。\par
		宗教的力量是惊人的,有时,那种冥冥之中的东西,真的可以指引着我们穿越生命的一些艰难时刻。\par
		要懂得惜福。她说起她的女儿。那个坚强的女孩子,在与死神多次擦身之后,依旧乐观坚强地与疾病斗争着,并鼓励与她有许多相似遭遇的人们。\par
		死神没有带走她,是上天的眷顾,她告诉女儿,这便是她的福。\par
		不要把眼光注目在自己缺失的部分,而要去学着收集和发现身边的幸福。\par
		我想,这些都需要智慧,需要一颗越挫越晶亮透彻的心。\par
		在命运面前,我们或许都不是强者,但至少要做止住绝望与悲伤。

		\subpart{三。}
		和朱朱视频,见她瘦削的脸孔微微胖了起来。看来澳洲的伙食不错。\par
		她呵呵地在那一头傻笑,做着鬼脸。\par
		我照照镜子,我的脸才是胖得离奇了。\par
		都没有了自嘲的勇气。田,总是像个气球,被吹起来再瘪下去,再被吹起来,再等待瘪下去。\par
		药在我身体上起的作用是如此明显。\par
		现在,我是气球。\par
		有时候,真想找根针把自己刺破。

		我终究看不破世间色相虚妄。\par
		讨厌镜子里的丑八怪。\par
		那不是我。

	\endwriting


	\writing{去年夏天}{2007年04月17日 ~ 20:36} %<<<2

		为了和朱视频,去买了摄像头。
		能够看到彼此,两个人都很开心。不停地挤眉弄眼扮鬼脸。
		我问,原来两个人都在北京时,也不是每周都会见面,为什么就不会想念。
		当地理的距离拉开了,怎么就突然体会到朋友的可贵了呢。
		现在想起朱的感觉,是酸酸的,甜甜的,像娃哈哈果奶。
		朱说,她看到mp3后边的大头贴偷偷哭了。
		然后,我们就不让对方再说下去。
		否则,情绪大约会失控。

		在北京的春天里。田站在教室的楼梯上看窗外的飞絮漫天,载着一身光明,它们飞行。
		想起远方的朋友。静静和朱。
		人生是如此聚散无常。

		在良的相机里找到去年夏天的照片。
		学校东门的马路上,那个天空有灰云的下午,和朱疯狂拍照。
		我觉得,那些没有矫饰的笑脸,就是我们最真实的青春了。


		朱,你也许不知道,那一天田的身体已经感觉到疲惫。
		在一心寿司店,因为冷气的缘故,我换了好几次位置,才最后选定一个比较“温暖”的角落。
		田始终在咳嗽。但我们还是吃得很开心。
		后来,我再没有去过一心。
		等你回来,我们再一起去吧。
		你说,你六月会回国一次。那么,又将是夏天了。
		希望田的身体可以健壮起来。
		一定会的。
		为我加油。


		照片上的田笑得很开心。
		现在,只有田自己驻守的北京,有些小小寂寞。
		但这绝不是一座死寂的空城。
		田在努力安排好自己的生活。
		享用眼前的幸福,并为迎接更大更多的幸福做准备。

		喜欢超载乐队《魔幻蓝天》里的歌词:

		荡尽尘埃
		继续信赖
		幸福仍然队列在等待
		我们的到来

		看看天空,听那一个苍茫却嘹亮的声音,瞬间里,好像望穿了世界。

	\endwriting


	\writing{收藏。一}{2007年04月26日 ~ 17:24:17} %<<<2

		我爱这春草如茵的季节。

		\subpart{一。诗}

		在独自的清晨,看洁白的日光缓缓漫过我的窗棂。\par
		听一曲如水的钢琴,叮叮咚咚的深情或寂寞。\par
		又一个平凡的日子,在我眼前真实地展开。

		如果每个人的生活都是一本书。\par
		那么,我的纸页上永远只是些散碎的句子,仿佛是诗的,又仿佛未成诗的,像我的日记。\par
		也许,我不懂得它们,不懂得一句句零落的词语。正如我不懂得生活,只是盲目地执迷坚定。\par
		我希望它是一本诗集,然而,它终于无法是。\par
		我散碎的句子,深情或寂寞,略显冰凉。\par
		它也许依旧是美丽的,却难免凌乱。像一幅信手的速写画,是真实,也是潦草。

		\subpart{二。梦}

		人大约可以做许多美妙的梦。美梦却不是可以夜夜造访。\par
		梦如雪花,无法预约。\par
		我在睡前,许一处甜美的梦境,仿佛我在烛光中,许一个明亮的愿望。\par
		有一个梦里,你驾了南瓜马车,来载我到远方。\par
		有翡翠屋顶的城堡,有彩绘玻璃的教堂,彩云上有天使的歌声。\par
		我忘了后来的我们,我忘了颠簸的旅程上,有没有忧伤和疲惫。\par
		远方,在无法触摸的梦里。梦,在无迹可寻的远方。\par
		你说,睡眠不好才会常常做梦。\par
		总在深夜醒来的我,回味着一场场接踵而至的梦,以为那是夜晚的礼物。\par
		梦却是永远无法两个人做的。\par
		但或许,我可以在你的梦里设下埋伏和陷阱,好让你在无端里,跌入我的梦境。\par
		梦如雪花,转瞬间,便是融化。

		\subpart{三。春野}

		我喜欢那种小野花。\par
		喜欢漫山遍野,被它们纤弱的身躯织就成淡紫色的海洋。\par
		二月兰,这或许是它们的名字。\par
		我们从盛开的花朵间走过。春光温煦,透过初生了绿叶的枝条,照在身上。\par
		我跟在你的身后。你的背影,在我的眼前,是这样清晰真实。\par
		我惊讶地想,是谁,是如何,将你安排在此时此刻,我的生命中。\par
		人的相遇,从来便是小概率的事件。\par
		如果,我们不曾遇到。好多次,我们一起这样去设想。\par
		那么,这个世界该是怎样。\par
		那么,我们会不会为错失的幸福而悲伤。\par
		也许,会有另外的女孩,跟在你的身后,走这一段盛开了花朵的小路。\par
		在另一处春野上,我们陌不相识地擦身。蝴蝶飞过你的头顶,扇动着彩翼,翩然而去。\par
		我们彼此的幸福毫不相干。\par
		我将是寂寞的,将一个人走过春天,无人诉说,对一朵小花的喜爱。\par
		没有人将我如孩子一般宠爱娇纵,没有人痛心我的哀伤,任泪水浸湿他的肩膀。\par
		也许,也会有另外的人,出现在我的生命,也许,他会如你对我呵护备至。\par
		我却依旧会遗憾悲伤,因那个人不是你而悲伤。\par
		你转过身看我,你无法知道,那一刻,我心中的波澜。\par
		你拉起我的手,你总是如此温存地对待。我们走过这一天盛开花朵的春野。\par
		你将是寂寞而悲伤的么,如果,那一天,我们错失了彼此。

	\endwriting


	\writing{闹剧}{2007年04月27日 ~ 08:42} %<<<2

		整理一冬的衣物,把它们打包放入衣柜底层。\par
		夏季的布衣短裙则被安放在方便拿取的位置。\par
		每一年换季,我们总要做相似的事情。\par
		不断地脱落,然后不断地整理,目送我们的轮回,如此淡定地流变。\par
		一些心情,某种气息,在这些花花绿绿的衣衫上,烙印下生活的痕迹。\par
		在那里被整齐地折叠好,等候你下一次的展开,仿佛记忆。


		什么时候买了这样多的衣服呢?以至于衣柜已显得不堪重负。\par
		只怪女孩子的衣柜里总是少了一件衣服。\par
		才使人永无休止地去填充,却又终于永无满足。\par
		女人是贪婪的,女人也是最懂得伪饰的。\par
		所以,这个世界有这样多的橱窗和时装,这样多的珠宝和香水。\par
		城市是属于女人的,只要去最繁华的街市上走一走,便足以了然一切。

		男人也许的确已经沦为女性消费的附属品。\par
		从“家有好男儿”的选秀开始,“男色”的概念进入人们的视线。\par
		在先秦,中国的男子以腹部饱满为美,于是,我们看到兵马俑无一例外隆起的小腹。\par
		后来,人们喜欢皮肤白净,风度儒雅,美须髯的男子。\par
		于是潘安出游,总是满载一车瓜果而归。古时女子们表达爱慕的方式,实在热情可爱。\par
		男子之美,自古便不是一种罪恶,不是人们咬着牙根说的小白脸儿。

		《世说新语》有容止篇专门来品论人物相貌风度。\par
		但为什么今日论及“男色”,总有人不屑一顾,甚或义愤填膺呢?\par
		大概,因为在电视机前支持收视率的多为女性。\par
		男人好像成为了消费品,在心理上,又将自己联系到古时的男宠之流。难免不生出异样感受。\par
		仿佛这个世界上只有女人可以花枝招展。\par
		仿佛男人展示美便是无耻的卖弄风姿,令人作呕。\par
		这对女人是不公平的,对于男人更是不公平的。\par
		欣赏“男色”没有问题,有问题的,是一些人的心态。\par
		最近,郑钧在“快乐男声”的海选中,与杨二车那姆发生争执,斥责她是选“男色”,而非“男声”。\par
		仍然抱着吉他坐在台上的选手表情无辜。他也承认自己不是帅哥,所以不会去参加“加油好男儿”。\par
		杨二的表现,只令人感受到她的粗鄙,而毫无她大约想标榜的特立独行。\par
		“男色”与“男声”之争,令两位评委都丢了面子,却无疑起到了宣传节目,提高收视的目的。\par
		这个时候,坐在一旁的张东,显得聪明许多。\par
		一样是没有“男色”的东子,同样把自己的事业走得很好。\par
		“男色”是好的,谁不爱美呢。“男色”却也绝不是万能的。\par
		想凭一幅脸蛋吃饭的人,永远只能是个可怜的跳梁小丑。

		这个永无满足的世界,总是光怪陆离地迷人眼目。\par
		有人群的地方,从来不缺少新闻和混乱。\par
		我们无法脱离媒体编织的这张大网。正如人逃不开自己的时代,我们逃不开媒体。\par
		那么,就以局外人的身份,微笑看着。\par
		反正,种种喧哗,无非一场场闹剧和玩笑。

	\endwriting


	\writing{尘}{2007年04月30日 ~ 16:10:47} %<<<2

		吹落,谁多情的眼中。

		\blankrev
		像一粒尘埃闯入一束光线,许多时候,我们是这样轻易地闯入别人的故事。

		谁回忆那炽热的沙地上,有了你一串或慌乱,或从容的足印。\par
		如尘埃的起落,如夏日随风飞扬的沙粒,有些什么,在一个秘密的地方把那些琐碎,积成山丘。\par
		从前,你只是个赤脚跑过的孩子,却不懂得那一座山的生长。\par
		但它就在那里,无声息地一日日积累,直至连绵成一列嶙峋的山岭。\par
		它是沉默的,它不发一言,它不让任何人知晓它的存在。\par
		它是只属于你的,你的山,你的一粒粒,千万粒沙。

		回忆,是一片沙漠。\par
		读到一位法国老人的回忆录,其中写到他晚年罹患痴呆症的妻子的最后岁月。\par
		他的妻子,一日日丧失了记忆,到最后甚至不认得自己的丈夫。\par
		她的目光中充满了疑惑和陌生,眼前的世界,令她感到恐惧不安。\par
		老人在每晚睡前,为妻子用拉丁语朗诵圣经。\par
		只有这个时候,她才平静下来,仿佛回到了遥远的童年。\par
		在她的童年,使用的正是拉丁语。\par
		睡熟的妻子好像婴儿,她不认得这个世界,她的记忆只残存了生命开端的那一些模糊的语音。\par
		当我们失去回忆,原来,无异于失去了生命。\par
		我无法想像老人妻子所承受的痛苦。\par
		因这一段人生的旅途,我们的全部意义从不是终点的到达,而是路上的风景。\par
		是那些散碎的尘埃与沙,是那片,我们难以穿越的沙漠。\par
		她向后看,却是没有过程,没有痕迹的空白,于是,她选择向前 —— 到上帝那里去。\par
		宗教给了她最后的救赎。

		你看到我们的沙从你指缝间流散了,被风吹向这个空旷的世界,一刻不停。\par
		你有恐惧么,有丧失的疼痛么。\par
		我喜欢那些飘动在光线里的尘埃。只有在光明里,我们在能够意识到它的存在。\par
		喜欢陈绮贞的一首歌,《小尘埃》。\par
		高二那年,在深夜的广播里,我第一次听到她的声音,和这一首歌。\par
		黑暗中,我安静地聆听,她唱:“我在这里,手提着沉重的行李,迷失在我和你未完成的旅行”。\par
		我不知道,自己为什么会被这样一句歌词感动,不经世事的我,在17岁的年纪上,一时间,在黑夜里悲伤莫名。\par
		也许,我想像着自己,站在长长的,无人的月台,如尘埃般漂泊在生命。\par
		我们永远无法知道,明天,我们更无从知道,明天的明天。\par
		也许,这是我年少里,最单纯简单的悲伤。\par
		每个人都在旅行么,在各自的路上,一路跋涉,又一路欢歌。\par
		其实,现在的我更愿意相信,生命是一次飞行。\par
		好像一颗星,一朵云,甚或,只是一粒沙,一点微尘。\par
		让我是细小的,却是飞舞着的。

		你闯入了谁的故事。在谁的心中埋下一列山岭。\par
		你懂得幸福吗。你相信幸福吗。\par
		有时,我想知道这一切的答案。\par
		一个早上,我独自走在学校的梧桐道上。我仰起头,看一树树新生的绿叶。\par
		我想起许多,一首诗,一个人,一句话,一处沉默。\par
		只是一个刹那,我发觉了自己的改变。\par
		你究竟是谁呢。你从哪里来呢。\par
		我好像一个疲惫的旅者,跌倒在沙漠的中央,不辨方向。\par
		树在不断生长,每一年更接近天空,让地上的人知道,生活是多么美的,枝繁叶茂。\par
		我深深地呼吸,然后有了微笑。\par
		在这个四月的末尾,突然想穿上白色的裙子,跑过你的视野,像一粒沙那样,迷住你的眼,世界的眼。\par
		你会因我而有眼泪吗。\par
		如果有,一定也是因为幸福。

		我是这样闯入你们的故事。\par
		然后,等待着成为一列山岭,或者,等待着归入万千尘埃,再也无从辨认。\par
		我无权选择这结果。\par
		我总是相信,宿命的安排。

	\endwriting


	\writing{雨。生活}{2007年05月01日 ~ 22:46:03} %<<<2

		城市的某端,是否有你,如我一般,凝望雨雾的双眼。

		\blankrev
		四月末尾的傍晚,雨水淅沥沥淋湿我的窗台。\par
		白色纱帘被忽而扑入的风抚起,夹杂着水汽的清味,飘在寂静的房间。\par
		我站起身,合上书页,去关一扇被吹开的窗。\par
		独立在窗前,透过眼前已隔水雾般的玻璃,从九楼之上望去。\par
		风与雨水,在光线渐暗的时刻,将天空溶化成一片水墨。楼房的缝隙间,隐约的绿树苍翠欲滴。\par
		有人打起了伞,有人撑起夹克在头顶,一律是急行的脚步。\par
		其实,不过是如毛细雨,只是风略显猛烈。\par
		没有人漫步在雨中,亦没有人独自望这一夕悄然的飘洒,如同从前的那个我一般。

		我喜欢这样的雨天,在春末,在仲夏,在一些干渴的,需要水分去汲满的日子。

		我会取出那柄透明的雨伞,一个人走去有树木的街巷和花园。\par
		雨水沙沙,击打在头顶的伞面,一粒粒浑圆的水珠在那里,又在瞬间滑落。\par
		那时,我便听到水的喜悦与哀伤。\par
		那细小的声响,是簌簌的情话,是耳畔的爱语轻柔。比雨打荷叶的碎玉之声,更加深情动容。

		有时,我走过一棵棵开着花,或者落了花的树。\par
		有时,我只是站在那里,望眼欲穿般,不发一言地站立,渐渐好像有了树一样的姿态神情。\par
		微雨的天气,该是如此寂静的,静到容不得一丝杂质,唯有密密织起的雨声,如一张网,将我们的种种情愫,一一捕捞。\par
		在雨中酿一首诗,在雨中默念一个句子,然后在天晴的早上,用蓝色的墨水,写在展开的纸上。\par
		从前的那个我,是这样度过许多个雨天,在十几岁的年纪里。

		现在的我,却只在雨天的窗后出现。\par
		我透明的伞,遗失在一次匆忙的聚会,再也无从找回。好像许多消失踪迹的过往一般,永远地不知去向。

		那一年,在一个灰暗的雨天,我穿上心爱的裙子,到对街的花店为自己买了一束龙胆花。\par
		淡紫色的花,纤弱的枝条,没有香气,却美得令人心碎。\par
		洗净一只透明的花瓶,注满清水,将她们浸泡其中,放在书桌前。\par
		痴痴凝视着,这一束花朵,满心是那一句花语:爱上忧伤的你。\par
		那天的我,在莫测的时光里猜测着,或许,某一天会有一位男孩送一束龙胆花给我。\par
		那么,我该是如龙胆般忧伤纤弱,令人心碎的女子。\par
		我爱如龙胆花般的女子,却不愿自己的世界有太多的忧伤。\par
		但也许,在年少的某段光阴中,当我们还无从懂得生命的悲喜,是注定要怀着哀婉的心绪,一半恐惧,一半享受地走过。

		忧伤是年轻的事。忧伤是我们对于痛苦的演习。

		后来,我们渐渐会有平静的心,如漂过流云的湖水,虽有风起,也不过微澜。\par
		后来,我们会忘记曾有的怨恨和遗憾,任昨日之虚妄,都似烟雨飘散。

		我们终于懂得,我们终于不再是轻狂的少年,一路莽撞,令自己遍体鳞伤。\par
		会有那一天,你问起遥远的一次离别或伤害,而我只是淡然地摇头,再也记不起丝毫。\par
		也许,在最后,我们只能保藏好欢乐与幸福的片段。\par
		那么,忧伤的龙胆花,也会有生出甜蜜的花蕊。

		这一晚,我在窗上的雨声中睡去。梦里,我以为雨就这样迟迟绵延了一夜。\par
		早上,却被明亮的日光刺醒。钟上的时间,是五点三十分整。

		雨是在什么时候停歇了。

		蓝空一望无际,没有一片漂泊的云。又一处天晴的早上。\par
		我展开纸,用蓝色墨水写着昨夜的呓语,好像从前写下雨中的诗句一般。

	\endwriting


	\writing{晴天}{2007年05月03日 ~ 20:23} %<<<2

		持续的好天气,阳光灿烂。\par
		陪妈妈到紫竹院散步,坐了一程没有荷花的荷花渡。\par
		暗红色的荷叶,蜷缩着伏在水面,零零落落在空阔的水面。\par
		我知道,在这之下,正酝酿着一场华丽的盛开。\par
		每年夏天都去坐荷花渡。\par
		看木桨在水中划出一串串水痕,看流光的影子照在游人的侧脸。\par
		当然,荷花才是主角。最好是微雨的清早或傍晚。\par
		像今天这样明媚的日子,在夏天是容易令人生倦的。\par
		即使是在春末,我依然被日光曝晒得深感疲惫。\par
		晴天是好的,但好天气一样会带来麻烦。\par
		而所谓好天气的标准,也是不一定的吧。\par
		最近一直在期待雨。\par
		像一株干渴了一冬的植物那样。

	\endwriting


	\writing{伪艺术青年798之旅}{2007年05月05日 ~ 16:15} %<<<2

		798工厂,一直想去,却又一直没去的地方。\par
		主要是自惭对于艺术知识所知甚少,不敢贸然前往。\par
		怕在艺术家的大作前弄得一头雾水,一头汗水。\par
		不想以亵渎艺术为代价去附庸一把风雅。

		五一放假,松松盛情邀请我试乘她家新购之宝来轿车一同出游。\par
		蹭车蹭饭的好事,就这样如一枚色泽金黄诱人的馅儿饼,瞬时里从天而降,砸到我头上来。\par
		遂欣然规往,并鬼使神差般,想到了我想望已久而不敢亲临一睹真容的798。\par
		最后,798真的成为了本次出行的目的地。\par
		当我把这一决定转达给活动参与者之一的大熊同学时,有如下对话,堪称经典:

		大熊问,去哪啊?\par
		798。\par
		去酒吧?!!\par
		不是……是798……(解释798的含义,约300字,略)\par
		哦……(恍然状)在哪啊?\par
		在大山子那儿。\par
		什么??大山??!(语气万分迷惑)\par
		……\par
		之后是田近一分钟的沉默。


		沿京顺路走走停停,一车人发现前方一路牌指示之方向为密云怀柔和承德。\par
		似乎已经开过了…… 莫非,我们要一路向北,开到承德避署去?!\par
		即使已经感到异常,司机老郭(松松的相公)仍然坚持:再开开看。\par
		终于,我们开到首都机场了!\par
		一架巨大巨大的飞机从我们的车顶飞过。\par
		我从没有见过这么大(或者说这么近)的飞机在飞!\par
		松松也很激动,我们俩一阵雀跃。\par
		并感叹,就冲刚才看见了这么大的飞机,今天也算没白出来。\par
		其他人貌似愕然。\par
		终于,我们又从首都机场开回来了。\par
		终于,我们找到了传说中的大山桥和798工厂。\par
		松松又一阵感叹:哎,你看人家大山,在北京真没白混啊,连桥都有他的名儿了。\par
		田说:岂止是他啊,连他儿子的也有啊。大山子环岛……


		就这样,跌跌撞撞,外加走了不少冤枉路,交了10元冤枉高速钱,总算进了798。\par
		顿时,伪艺术青年的面目就暴露出来。\par
		还没下车,一行人就饿了。于是,吵吵着找饭馆吃饭。\par
		不好意思一下车就问人家:您知道附近哪有吃饭的地儿么?\par
		你看艺术家(当然多数是落魄时期的),不都是饥一顿饱一顿的么。\par
		有时候,几个干馒头,就着指头上的油彩一块吃下去,一天的饭就解决了。\par
		哪有天天寻思着吃吃喝喝的艺术家呢?\par
		艺术家如此,艺术青年至少也不该如我们这一车人这样贪吃。\par
		所以,眼睛盯在四周的路牌,找听上去像能吃的。\par
		终于,我们吃到饭了。\par
		然后,一吃就吃了2个小时。\par
		我被现实的镜子照得原形毕露…… 一幅画还没有看,已经吃得满嘴是油了。\par
		我感觉惭愧啊,惭愧。


		吃饱后,有力气了,开始欣赏艺术。\par
		在安静的展厅是不宜大声讲话的。\par
		于是,在这里,欣赏艺术的过程,也便还是归于安静的好。

		作品是好作品。\par
		拍下一些喜欢的画,和雕塑。

		观众却不是好观众。\par
		多数时候,田还是落入了一头雾水,一头汗水的下场。\par
		在作品面前却又不敢显得太迷茫,也不好意思和大熊交换意见。\par
		用通俗一点的语言讲,基本上没怎么看懂。\par
		但艺术,作为梦的另外一种形式,又何来懂与不懂呢。\par
		只要是它在你站在面前的时刻里,给了你心灵的振颤,我想,那便是一件好的艺术吧。\par
		虽然,也许很多时候连你自己也不知道,那振颤是些什么。\par
		梦是不可以许多人一起做的。梦是私密的,艺术也是。\par
		这可能是一个不懂艺术,却又非要热爱艺术的伪艺术青年可笑的借口。\par
		但这大约也没什么可责怪。\par
		一个人热爱一些什么,是他自己的选择。\par
		有你所热爱的,也是一种莫大幸福。

		回到家,看到传入照片的编号,才想起是五四青年节。\par
		798,艺术的工厂,也是梦的工厂。\par
		这天,伪艺术青年度过了一个不错的节日。

	\endwriting


	\writing{与静}{2007年05月07日 ~ 21:48:14} %<<<2

		坐在我对面,静已是如此娴淑端庄的女子。


		我觉得这样很好,和你聊聊天,点一盅芳香的花草茶,一匙匙喝下。\par
		让甘甜中微苦的滋味,漫过唇齿,流入喉咙。\par
		像一场我们共同经历的故事那样,没一丝声响,却又分明曾是惊心动魄。\par
		真正是如人饮水,冷暖自知。

		窗外的五月明媚光鲜,熟悉的街道,匆忙的车流,令人有恍若隔世的错觉。\par
		多少次,我坐在这样的窗畔,一个人,或者和你,和你们。\par
		多少次,细细想起一段时光的温暖,说起今日的欢乐或忧伤。\par
		在这家店,有分别,有重逢,有像现在这样,两个人微笑的对坐。\par
		有一些人消失,有一些人被忽略。偶然,我们说起从前,已是全然的风轻云淡。\par
		十几岁,那个时候,谁懂得认真。\par
		今天的你我相视一笑,生活告诉我们许多真相。\par
		我们渐渐懂得,感情的路,有多艰,多长。

		什么时候,我们能够遇到那一个人。真正令你不再有悲伤和孤独的人。

		我曾经喜欢点的芒果水晶绿茶,早已换了名字。\par
		我曾经喜欢喝的茉香珍珠奶茶,也被替换下了菜单。\par
		那一年,和朱坐在二层的秋千上,摇摇晃晃一个下午。\par
		说了太多顽皮的话,笑到脸部肌肉发紧。\par
		那一年,无邪的心思,容不下烦恼和忧虑。\par
		今天和你坐在这里,秋千依旧摇摇晃晃。\par
		下楼的时候,我想到了一个人,想到一滴没有流下的眼泪。\par
		是多么远的事了,在一个雷雨天。\par
		而那,分明标志着我天真年代的结束。

		田也许改变了许多。\par
		田在这几年中,有挣扎,有失望,有慌乱不知所措。\par
		我不知道别人的生活,是以怎样的步态向前。\par
		田只是任性在幸福,却又时常低落感伤。\par
		有时,我以为自己拥有了全世界。\par
		有时,却又在瞬间里看清,原本的所有,也终将是一无所有的结局。\par
		爱情无法愈治孤独。那骨子里的清冷和疼痛。\par
		现在的我,依旧写字,不断地写,在纸上,在电脑上,在心里。\par
		一些贴在博里,一些藏起来,一些丢失在时间的延迟中。\par
		我忘了很多句子,写过的,还有准备写,却未写出的。\par
		好像我忘记自己那样,属于回忆的自己,还有还未到来的自己。\par
		只认得此一刻的我。\par
		因为只想用心度过每一日的生活,一半清醒,一半梦幻。

		和你聊聊天。然后,陪你选耳钉。\par
		看镜子里日渐贤淑的你,记得那年去打耳洞时的情景。\par
		三个女孩,只有我没有耳洞。于是,我没有首饰,也便不关心首饰店的新品。\par
		其实,我喜欢那些有装饰效果的东西。\par
		我热爱一切有美感的东西。\par
		但是,我终于没有打耳洞,也不习惯戴项链或戒指。\par
		也许,我注定是要轻装上阵,素面朝天的人。\par
		不化妆,却情迷香水。气味是摸不到的,我以为,它是如魔法一般的东西。\par
		好的香味令人心神愉悦。\par
		妆是化给别人看的。香水却是喷给自己的心情。

		你送我特百惠的杯子,你让我自己选。\par
		我最终还是选择了那款母亲节的样式。如儿童手绘的图案,一派纯真气息。\par
		想用它在夏天来临的时候冲泡薄荷茶。\par
		那时,我会是在怎样的窗前,读怎样的一本书,我会怀着什么心情想起你。\par
		会有清凉划过舌尖。\par
		傍晚的时候,大约有一场雷雨。\par
		空荡的校园里,我将一个人,拿起笔,在泥土的湿味里,给远方的你写一封久违的信。

		希望重逢时,你说起的,是如蜜的幸福。\par
		再无遗憾感伤。

	\endwriting


	\writing{五月。零散情绪}{2007年05月11日 ~ 18:37:57} %<<<2

		我在这里默默诉说,只因如天色般,宁谧的喜悦。


		五月的天气,总是湿润而清亮。细小的雨水,安静地落了一夜。\par
		槐树花飘散,一粒粒洁白的光点一般,闪烁在忽明忽暗的天光。\par
		天是淡远的灰蓝。\par
		没有晴空炽热的欢乐,没有雪天冰凉的悲伤,这一刻,它只是一派宁谧的喜悦。\par
		梧桐道上行人来往。裹住头巾的西亚女子,蓄长发的美国男人,穿短裤戴毛线帽子的日本学生。\par
		北语是如此多姿的地方。\par
		你看到在同一个空气安详的早晨,各种各样的人,神色匆忙地奔赴生活。

		我坐在莫的后车座上,在飞驰的车速中,穿过了人群。\par
		道旁这些比我们年龄大上许多倍的植物,用它们的手掌撑起碧色的凉棚,营造着一条翡翠的长廊。\par
		我喜欢这一条梧桐道。喜欢它们沉默而多情的姿态,和如此挺拔的生。\par
		这一天早上,莫很开心。在数次失败后,她终于学会了骑车带人,并顺利将我送到学院。\par
		因为身体的原因,好多时候,我需要以这样的方式到达教室,才不至劳累和气喘。\par
		莫本不会带人。瘦小单薄的她,却下决心学会,好每天骑车带我上课。\par
		莫很开心,当车子在学院门口稳稳停下,她高兴得像个孩子。\par
		我们相互做着胜利的手势,旁若无人地欢呼,噢耶,成功了。\par
		这个微凉的早上,莫的额头却渗出了汗水。“我害怕摔着你,特别紧张”,她笑着说。

		她是这样善良的女孩。总是为别人考虑着,为别人疼痛着,默默地付出。

		冬天的时候,我的身体状况很差,上过课回到宿舍便已疲惫不堪。\par
		莫便每日把饭打回来,和我一起吃,然后再回去自习室学习。她在寒冷里往返奔波,只是因为我。\par
		莫不愿我说起这些,不愿听我的感谢。\par
		她说,她是快乐的。“为你做些什么,我是开心的。”\par
		莫用单纯天真的心,去体会这个世界,去爱身边所有的朋友,像一株明黄色的小花,照亮哀伤的眼。\par
		虽然,她的眼里也有哀伤,她的生活也有艰难和困惑。\par
		莫依旧是用尽全力去温暖着他人,而不计较什么得失,什么回报。\par
		也许,不过是一些小事。但真挚而细微的帮助,却最令人深记。\par
		不过初夏,宿舍中却已经有蚊子出没。\par
		莫说,田,我帮你挂上帐子吧。\par
		于是,她爬上床去,一处处精心地调整,将纱帐高高挂起。\par
		莫坐在床上问我高度怎样,我说很合适了,她才放心地顺梯子爬下来。\par
		我没有说一句感谢的话。有些话,是不宜说出,亦无法说出的。\par
		只让我深记吧。\par
		我善良的朋友。

		和莫一起站在来园的池畔。朗诵外文的声音不绝于耳。\par
		在北语这样一所语言学校,这样的情景是太合乎情理的。\par
		我却不愿,将晨早最洁白的时光奉予他国的语言。\par
		纵使,那是多么美妙,甚或多么优雅的一种语言。在我的心中,在中文的魅力前,都显得脆弱苍白。\par
		在莲花睡眠的池上,在绿树掩映的石子路,我们该读起诗的,不是么。\par
		诗经的静穆庄严,汉乐府的天真朴拙,唐人的天马行空,宋词的低回婉转,哪一种声音,配不起眼前的晨光洁白。\par
		或者,只是读我们自己写的短句吧。\par
		一些散碎在生活缝隙里的沙粒,一些还未消减的少女情怀,让我读着,让你读着,像一棵树对自己的沙沙细语。\par
		对莫说,以后早上我们来这里读诗吧。\par
		莫欣欣然地说,好啊。\par
		两个人会心地笑了。

		莫开始准备考研。而我则考虑觅一份稳定的工作。\par
		我愿意做学生,却不愿承受考研的压力和风险。我仿佛从无法是一个全力以赴的人。\par
		我只是跟随着自己的心,做好该做的事情,然后任由随心所欲。\par
		莫却是认真的人。我相信,她能够通过自己的努力,实现目标。\par
		莫担心考研失败,所有认得她的人都说,若你都考不上,那便没有人考得上了。\par
		她的笔记整齐细致,一丝不苟,从不缺课,甚至从不迟到。\par
		在现今的大学中,如莫一般的学生,还有多少个呢。\par
		莫的认真,有时,令我感到羞愧。

		这一周,我回到了宿舍居住。不再在家与学校间往返。\par
		宿舍是陌生的。搬到新宿舍楼后,我便几乎没有在学校过夜。\par
		这一周,我回到了朋友们的身边。\par
		和她们聊天,嬉笑,互相吹捧和诋毁,毫无顾忌。\par
		一起去吃菜场里的小笼包和烧卖,再挑一条顶花带刺的黄瓜,半斤酸到胃疼的杨梅,拎回慢慢品尝。\par
		我们在宿舍煮面,分吃一包武汉的特产,分喝一瓶奶茶。站在阳台上看楼下拥抱的情侣,看路边等候中显得不耐烦的男生。\par
		用一个下午的时间,重温泰坦尼克号的爱情故事,又一次落泪。\par
		小小的房间,充斥了女孩子们的天真欢笑。\par
		我的生活回归到如此简单,心无旁骛的地方。只有那些真诚的友谊,只有我们定然会在来日想念的如此青春。

		小鹿在凌晨1点30分由五层的通宵自习室走下。她写,夜很美,很静。\par
		我想,她穿着白色高跟鞋的脚步声,该是踩过了我昨夜甜美的梦境。\par
		有时,也想和她一起上到五层去,点一盏蜡烛吧,若没有别的人夜读。\par
		让我们对坐在烛光的柔光里,说起那些已经真假不辨的往事,让你说起你第一次深爱过的人,而没有一丝悲伤。\par
		然而,我终于不能。\par
		我的夜,而今,只能够属于梦境,属于睡眠。\par
		也许,这是一个莫大的遗憾。

		我只有想象着,小鹿在深夜独自走下楼梯,见一窗雨水,安静地下落。

		我不曾奢望,在身边围绕着你们,如此可爱的朋友。\par
		而我,却这样幸运地拥有了这些,与你们相处的光阴寸寸。\par
		田没有一句感谢。\par
		田却在悄然绽放,幸福的笑意。

	\endwriting


	\writing{饮食趣味}{2007年05月11日 ~ 20:35} %<<<2

		如果按灵魂和肉体的标准来区分,饮食趣味大约算得一种典型的低级趣味。\par
		不要过多地满足肉体的欲望,许多智慧的哲人这样教导我们。\par
		寡欲清心,无欲则刚,这样的话,都是我们所熟知的。\par
		在人的诸多欲望中,口腹之欲又是其他欲望产生的开始和基础。\par
		食不果腹的人,必然先为肠胃的充实奔波,而后才可能产生其他的要求。\par
		吃,是生命体得以存活的最基本要求。\par
		于是,节制欲望的修炼,便多由此作为开端。\par
		宗教中多有关于饮食的规定,甚或有节制饮食的节日和仪式。\par
		管住一张嘴,不去贪恋食物的美味,只满足于维持正常生理活动的饮食,说来简单,实则很难。

		饮食作为一种文化,却又是不容忽略的。\par
		饮食对于一民族之气质及思维方式的影响,虽并无确凿定论,但也是无法否定的。\par
		至少,在饮食结构上的差异性,一定程度上影响了各人种体格体质。\par
		饮食是一民族文化不可或缺的组成部分。\par
		饮食,不只是简单的果腹。特别是在物质足够满足大多数人正常生理运转的现代社会。\par
		正如同性,由原始的生殖繁衍功能转向近似与娱乐的需求,饮食逐渐成为一种对于更高品质生活的追求。\par
		对于饮食的要求亦不断提高,从单纯要求烹饪的品质到对就餐环境的考究,从对菜色品质的品评到对餐具的艺术化欣赏。\par
		饮食文化,其实是从一个最世俗的层面,反映小至一个家庭,大至一个国家的生活品味和品质。\par
		从这种优劣高低的区别中,我们又可以窥见一个国家的经济发展水平,文化精神的状态,甚至未来发展的前景。\par
		这不是耸人听闻。细节上暴露的往往是真实的整体状态。\par
		从国民的一张嘴上,可以洞见的内容,却是包罗万象。

		美好的食物能够令人愉悦,这一点大约无人质疑。\par
		但过度的饮食,同样会产生负面的情绪,这也是众所周知。\par
		饮食的趣味,各人有各人的判断,他人无权干涉。但有一点该是努力提倡的。\par
		那便是安全健康,节制有度。\par
		用过多的精力去侍奉肉体的欲望,确实会阻碍我们灵魂的生长。\par
		先哲的教导不无道理。\par
		去享受饮食的乐趣,而不使它干扰我们的清醒和思考。我想,这大约是最好的状态。\par
		记得在书店见过一本书,名曰《生活的高潮 —— 半饱》。\par
		托翁对我们说,要向动物学习,只在饥饿的时候进食。\par
		这也是有利于我们沉湎于饮食无法自拔的方法。\par
		要有智慧的头脑,首先要有懂得满足的胃。

	\endwriting


	\writing{伞}{2007年05月15日 ~ 23:08:32} %<<<2

		我们的伞下,藏了多少美丽而哀伤的秘密。


		\blankrev
		一柄遗失的伞,一个密布着铅灰色云朵的雨天。\par
		谁在回忆里踟蹰不前,在长长的街巷。忘记昨日之惆怅,静静地,看一片误入烟云微茫的风景。\par
		每个女孩的青春里,都该有如此的一幅画面。属于青涩年华里,那些若隐若现的情感。属于一场没有结果的执迷和守候。\par
		你是否,曾遇见那样一个人。\par
		令你如一棵初夏里绽放的洁白花树,几分寂寞里,满心欣悦地将自己精心妆扮,无言吐露一心的温柔。\par
		一柄遗失的伞。你在多年后忆起,那样一个雨雾里悱恻的画面。\par
		你庆幸,你青春中的故事,充满了遗憾。你以为,年轻的时候,该这样写下忧伤的诗,才不枉年少。\par
		那一柄伞,是否仍然立在他已时过境迁的窗口。\par
		如歌中唱着的:我忘的伞还倚你的窗,望着窗外,那悠悠春光。\par
		每一次,听到这一句,你总禁不住,记起分别的夏天。\par
		花树绽放的时节,落花在他的肩头,你的发上,一朵朵,凄清的洁白。\par
		你深心感谢,这一场因热烈而失真的情感。你知道,你们是太年轻的,于是,难免是如此不得善终的结局。

		那时,也许你只是爱上爱情。或者,只是爱上爱情这个词。因不懂得,而能够跌撞着勇往直前。

		临行时,开始落雨,你把手中的伞递给他。就这样,那个经过你青春的人,消逝在你空空的窗口。\par
		没有一次回首,甚至没有片刻的停留,离开了,便不再回来。如一捧吹入风口的纸屑,不知去向,零落四方。

		你的青春,在雨天里被标注成灰云样的色彩。\par
		你却依旧如那株花树,固执在美丽的年华,骄傲到孤独。

		爱情,是容不得说出口的。后来,你这样对我说。

		只让爱着的人远远望着,而不去打扰他的生活,不去图谋在他的心里安营寨扎。用无功利的心去守
	望,也许,这是爱情最美,最安全的模样。

		爱情,是多么奢侈的。你用尽青春,也难得荣获一个完满如你所愿的故事。于是,只是安静地望着
	他吧。不要有任何野心。你笑容恬淡,细长的手指掠过乌黑柔软的发丝。我听你说这些,竟有辛酸的滋
	味。

		难道,那不是我们每个人都有权享用的一种幸福么。\par
		什么幸福?携子之手,与子偕老吗。\par
		我只是不想,让我们的爱情尘封在记忆的底层,成为一桩真假不辨的悬案。\par
		我只是想在发苍苍,视茫茫的年纪,坐在摇椅上,听我的爱人说起曾有的轻狂。\par
		听过最动听的一句情话是:我想在老了时候,吻你光光的牙床。

		岁月,时光,这些都是不由得我们去细细思量的东西。\par
		青春,红颜,这些都是世间最动容的疼痛。

		你说起一个词,红粉白骨。在男人眼中,红粉白骨意味着时时的提醒:再美的女子,也不过一张完美的画皮。\par
		如波德莱尔笔下所言,我的爱人,我的天使啊,你也终究将成为一具腐败的尸身。\par
		在女人读来,却是悲伤,是叹惋,美丽总也敌不过岁月的变迁。不过十几年光景,便已纵使相逢应不识。\par
		女子,如此美的生灵,亦如此脆弱。圣经中讲,对待妻子,要如对待脆弱的器皿,确实不无道理。

		青春就该有故事。似乎这也是一首歌的歌词。\par
		你庆幸,你的青春充满了故事,遗憾的故事。你写下许多诗,你读给我听。一句句,零落如雨的哀伤。\par
		那年,你是雨巷里,握一柄油纸伞踟蹰,丁香一样的姑娘。你寄来信,告诉我现在的生活。\par
		我欣慰,后来的你,终于可以甘心于一场无风暴,无狂涛的情感。\par
		终于能够用一如既往的温柔去体会爱人细微的冷暖。\par
		你却依旧不相信,白头偕老的爱情。\par
		只是,那只是因为,你不敢奢望一个遥远的结局。你不敢相信,你能够有足够的幸运,去荣获一个完满的故事。\par
		那柄遗失的伞,那个在雨天一去不回的身影,你还记得么。\par
		我知道,这已经无关紧要。\par
		你说,你懂得了如何去享用爱情,在你美丽的时刻,在你不再美丽的时刻。

		而我,大约仍旧不能够如你一般,心无波澜地去从容于爱。\par
		那是年轻的原因,所谓爱的代价。难免狂乱的心,难免迷惘的脚步,错失的,慌忙的,不知今夕何夕,不辨春秋冬夏。\par
		你说,把一切交付给时间吧,去经过你的故事,跟随自己的心。

		吐露着一心温柔,如一株初夏里绽放无言的花树。\par
		落花一夜,落雨一夜。\par
		我读你的信,提起笔,又几次放下。最后,只是写下了了的几句:

		你在明日,我在昨天。我等候你,却有无以到达的恐惧。

		空白大半的纸页,被封入信封,投入邮箱,寄去一个远到令人担忧的远方。回来的途中,又是雨雾绵绵。\par
		看着手中的伞,想起岩井俊二的《四月物语》,想起那个骑着单车经过青青河堤的女孩,和她那一柄伞,一段没有结局,淡如水彩的青春。\par
		想起西湖断桥上的白娘子和许仙。\par
		想起多少的雨声缠绵里,共剪一窗烛火的夜话。\par
		那些遗失的伞,那些借出的伞,那些归还了,和未归还的情感,在这小小的一块无雨的天空,织就着长长短短,纷纷扰扰的故事。\par
		被我们记录,或者丢弃,想念,或者拒绝。

		爱情,我不敢说出口。\par
		爱情,我只放在一双深深的眼神。\par
		要他去读,如读一首隐讳艰涩的诗。要他去听,如捕捉蝴蝶彩翼飞过清风的声响。

		如果,今夜有雨,我想,我会在梦中相遇,那个为我撑伞,走过青春的男子。\par
		是你吗,是你吗。\par
		我已不忍问起。

	\endwriting


	\writing{夏天。水果等}{2007年05月20日 ~ 20:25:42} %<<<2

		夏天来了,水果摊上的水果开始丰富起来。\par
		忍不住让人想买回一袋又一袋,放在玻璃碗中,用清水浸泡起来,当作一件装饰。\par
		有时候,我可能只是想欣赏它们鲜美诱人的模样,胜过一口口享用的贪婪。

		\blankrev
		用相机拍下它们可爱的模样。青红相间的荔枝被放在不锈钢的盘子里,盘底映着天光的色泽。\par
		午睡时落了几滴雨,现在却是一方澄净无染的蓝。\par
		坐在窗前,剥开一只只荔枝,洁白的果肉绽露如笑,甜蜜的汁液浸润在我的手指。\par
		带着午睡后的几分慵懒,在荔枝的滋味里,一个初夏,这样不清醒,如醉地飞逝在我的身旁。\par
		此刻,我只体会到幸福。

		\blankrev
		樱桃是一种神情。\par
		属于小女子的爱娇和温柔。或许,又带三分的稚气天真。\par
		我总以为,樱桃是初夏时节最好的水果。\par
		也是最适宜用一碗清水浸泡着,放在桌前,却久久不忍吃下的水果。\par
		该是读一本薄薄的诗集,倚在被树影照得斑驳的白墙,随手捞起一枚,细心地咀嚼。\par
		那一年,和母亲去一处樱桃园采摘。\par
		第一次见到被樱桃缀得弯弯的枝条。我们照了许多照片,是日光充沛的一日。\par
		我穿着满是花朵的一件衬衣。现在看来,不仅失笑:还是小孩子呀。\par
		樱桃好吃,树难栽。若有一园樱桃的丰收,园主人该是花费了多少的辛劳。

		\blankrev
		两周前,从花卉市场买回两包花籽,兴冲冲地栽到花盆里。\par
		一盆是二月兰,一盆是牵牛花。\par
		然后,便是漫长的等待发芽的时光。\par
		上周三,母亲发短信来:牵牛花长出叶子了。\par
		周末回家,果然见一盆破土而出,样子纤弱却倔强的绿叶。\par
		另外一盆,却是没有丝毫动静。也许是气温过高,错过了二月兰生长的时机。\par
		牵牛花,小时候我叫它喇叭花。它们总是早上开放,太阳出来便闭合。\par
		在一本书上读到,日语中的牵牛花写作:朝颜。\par
		这名字令人见到的不是一株蔓生的植物,而分明是一个黑发齐眉的少女的笑靥。\par
		童年的院子中,祖母在门前栽下牵牛花。\par
		绛紫的,雪白的,衬在弯弯曲曲的枝叶条蔓上,还沾着晨早的露水。\par
		现在,我等待着花开。

	\endwriting


	\writing{有月亮的晚上}{2007年05月27日 ~ 21:07:32} %<<<2

		月的温度,是我此刻,飘忽的目光。


		\blankrev
		有月亮的晚上,我们面对面盘坐在地,让清辉的洁白撒在额头发梢,仿若暮年的霜雪。\par
		轻抚你宽阔的肩,嶙峋的骨,如一列列山陵的冷峻。这一夜,盘坐的你,令我想起月照山苍然的诗境。\par
		没有一颗星,唯有一轮月,在窗上流泻下一地温凉。\par
		夏风清澈,这一刻,我的镜台本该绽放一盆茉莉,我该是读一首远古的诗歌给你听。\par
		我说,我喜欢这样的晚上。让我们陷落在寂静里窃窃私语。

		你没有多说话。无意的低首间,我却看清你轮廓忧伤的侧脸。\par
		这时,我们本都不该有泪。\par
		这时,我们却分明在沉默里,有了莫名的辛酸。\par
		漫漫的路,生活,荆棘和鲜花丛生。背负着各自的重量默默潜行。\par
		也许,当我们能够肩负起这些重量,并站直了身躯,昂首向前,才是真正蜕变了的自己。\par
		而现在,我还是如此脆弱。我还是有悲戚,有抱怨,只能够靠住你的肩膀哭一场。\par
		但我知道,有一天,我们终于会拥有强大的心,不再怨天尤人,不再茫然若失。\par
		我们将学会,如何安放好一颗躁动的心,如何抚慰一只不安的小兽。\par
		相信我,这只是时间的问题。

		高二那年,反复读顾城的诗集。平庸的装祯,被包上绘着水彩花朵的书皮。\par
		那一本书,以一副花般的姿态,平躺在我的书桌里,接近两年。\par
		总是在课上翻起它,于是便能够逃离晦暗无趣的现实,去他的童话里遨游。\par
		抄写他的诗句,在生长着瑰丽木纹的桌面,用一支相伴许多年的铅笔。\par
		“告别绝望 / 告别风中的山谷 / 哭 / 是一种幸福”\par
		好几年,我的心,如一间空房,被这样一句诗充满,像一室温凉的月色。\par
		于是即便有泪,也不是悲哀。\par
		现在,我又在默念这一句,被抄下无数次,又无数次擦去的诗。\par
		我知道,我是需要在文字间取暖的人。\par
		那一年,教室窗外的灰天空下有摇曳的树,绿得伤心。\par
		十七岁的我,在一个将雨的下午昏昏欲睡,任一桌敞开的书页,被风翻乱。

		有月亮的晚上,我想对你说起这些。关于我的从前,关于那个小女孩,那个女学生的一切,一切。\par
		也许,你愿意去听。也许,你宁可站在那门外,对于曾经的我一无所知。\par
		其实,那门内的所有,此刻,连我自己也已感陌生。\par
		我只是喜欢,喜欢在一片梦一样深的月光里,触摸一些消逝的虚无。\par
		我只是喜欢,那个小女孩,那个女学生,和关于她的一切,一切。

		很远很远的地方,一样是有月亮的晚上。\par
		我一样喜欢看月亮,却是看月中深深浅浅的阴影。想着祖母的故事,那些月中的传说。\par
		嫦娥,吴刚,桂树,还有捣药的白兔。他们真的住在那一轮明黄的光里么。\par
		幼小的我,仿佛对那一个神仙世界深信不疑。所有的怀疑,都是成长的悲哀。\par
		后来,我知道了高处不胜寒。后来,我担心起月宫的寒冷,想着嫦娥如我一般冰凉的十指。\par
		碧海青天夜夜心,月的世界,原来如此寂寞。\par
		月色间,不宜有欢歌,不宜有笑语。或许,只应对着清影凭吊心魂,饮一盅酒,诵一首诗。\par
		应寂寞如嫦娥,独对万象冰雪,时光纷落。不宜有红粉的笑颜,管弦的喧扰。\par
		若有声,也只玉人的箫,或梅边的笛。\par
		古人说,明月楼高休独倚。怕的是思念缠绵,怕的是月冷情伤。\par
		有月亮的晚上,属于多情的人们。有月亮的晚上,纵有绵绵的话,也要轻轻地说。\par
		于是,这一晚,我们多数沉默,只有零零的言语,散落在空白的时间。\par
		我们面对面盘坐,如一对参悟天地的禅者。\par
		空山花落,月影徘徊,你我有没有相看两不厌的默契。\par
		嶙峋如山的你,在我的天地,静候着一场场日后的霜雪。\par
		染白我们此时的乌发青丝,褶皱我们此时的青春如花。

		这一天,有月亮的晚上,我站在小小的阳台。\par
		我伸出手去,捧起一掌的月色如水。\par
		时光,在刹那间,竟如此洁白,安静。

	\endwriting


	\writing{这些。那些。六月}{2007年06月03日 ~ 15:09:56} %<<<2

		六月,日光多情,明亮刺目。


		\blankrev
		六月一日的早上,电台里一首首播放着那些熟悉而陌生的歌曲。\par
		花仙子,蓝精灵,黑猫警长…… 旋律弥漫小小的房间。主持人言语激动地回忆着自己的童年。\par
		于是,我也记起,许多令我痴迷的卡通片。想起雪孩子溶化时的悲伤,想起大盗贼欢乐的歌声。\par
		于是,我也记起,一条梦寐以求的公主裙,一双晶亮的红皮鞋,还有,夏日午后从树缝间漏下的阳光。\par
		那糖水一样的阳光。

		童年,已落入往事。归纳入一个个名词。

		\subpart{小号手}

		记忆中,所有的儿童节都有鼓号队的喧闹,和插满操场的彩旗,在风里飘散招展。\par
		带着桐树花浓烈的香气,空气被晕染成一片淡紫色的底。\par
		那天,女孩子都穿着白色的连裤袜,红裙子,头发上扎起了大大的蝴蝶结。\par
		男孩子穿着新衬衫,蓝短裤,和那走起路来啪啪作响的塑料凉鞋。

		我忘记了,我在哪一个位置,做着怎样的表情。\par
		我只是被淹没的一个声音,像所有的孩子一样,只顾唧唧喳喳地说着话。

		鼓号队的演奏开始了,大家望向同一个方向。\par
		小号手们的脸憋得通红,还不纯属的技巧,令他们感觉费力。

		那只是一只简陋的小号,上边甚至生出了锈斑,侵蚀掉原有的金色光泽。\par
		但即使如此,男孩子们还是会因成为一名小号手而感觉自豪 —— 这资格是需要经过选拔的。\par
		被选中的男孩子,每人得到一枚号嘴,大队辅导员,那个留着时髦卷发的女老师告诉他们:吹响了号嘴,才能够正式开始小号的练习。\par
		于是,这些男孩子,每天带着几分得意又几分焦急地吹着那些号嘴,这几乎占用了所有的课间。上课时,号嘴就放在桌子上。\par
		邻座那个未被选中的男孩,总是一脸羡慕地望着那生了锈,并不漂亮的小东西。

		后来,号嘴被一只只吹响了,虽然,发出的是奇怪的声音,却依旧令他们欣喜若狂。\par
		男孩子一个个飞奔向办公室,迫不及待地去领取一只真正的小号。

		他们都很努力地练习,由一位高年级的男孩带领着,一次次重复着单调的曲子。似乎却没有人厌烦,他们总是带着激动而神圣的神情。\par
		也许,他们知道,就在花墙的背后正有另一群男孩偷偷地看着这一切。

		在高年级的男孩中,有一个人是很小便开始练习小号的。据说,在他成为鼓号队的小号手之前,便早已学会了许多高超的技巧。\par
		他有一只皮箱子,里面装着属于他自己的小号,一支金光闪闪的小号。\par
		那小号与学校的小号不同,多了几个按钮,显然高级许多。同班的男孩悄悄告诉我,那是三音号,可以吹出更多,更美妙的旋律。\par
		大家都对那支小号神往不已。不必听它动人的音色,只是看它晶亮的模样,已经令人感到无限神秘。\par
		那个男孩,总是提着那支皮箱子,经过之处无不引起一阵议论。

		学校的不远处,有一块农科院的试验田。那时,田还没有专人看守,是可以随意出入的。\par
		一个麦子成熟了的六月早晨,我经过那块试验田去上学,听到了小号圆润而嘹亮的声音。\par
		远远地,我望见一个身影站在金黄的麦田中央,正是那个高年级的男孩。\par
		他雪白的衬衫被晨光镶上淡粉的轮廓,金色的小号闪烁着和那乐曲一样嘹亮的光芒。

		那天,那个安静的早晨,在起伏着麦浪的田野旁边,我站了很久,聆听着那个就要钻入云霄一般的声音。

		后来,我才知道,每天他都会到那里练习,已经坚持很多年了。\par
		而那一年,他也不过一个不满十二岁的孩子。

		现在,我还经常从那块试验田经过。大门被紧紧锁上了。麦田被棉花取代。\par
		我透过重重冰冷的栏杆向里看,棉桃被包裹在叶中,还没有长成。田野空阔,不再有孩子在吹起一支骄傲却孤独的小号。

		那支三音号,是否业已生了锈迹。\par
		曾经的小号手们,还能否记得,号嘴吹响的,那奇怪的声音。

		\subpart{白裙子}

		我依旧清晰记得,第一次见到祁老师的情景。

		她站在大队办公室里,背对着门口,穿着一条长长的白色连衣裙。

		孩子们挤在办公室门口,却只敢露出半个脑袋,或者一只眼睛,偷偷地往里看。\par
		初夏的绿树,在窗口荧荧地闪烁。她就安静地站在那里,仿佛等待着什么,一言不发地,任洁白的裙子也沉默地垂着。

		同学们都激动不已,小声地议论着这个美丽背影的来历。\par
		“你们还不知道?她是我们新的班主任呢。”“她刚从师范毕业的,好像才20岁吧。”\par
		我们不断听着这些听来很可信的传言,心中满是期待。孩子们大约总是喜欢一个年轻漂亮的班主任。\par
		我们已经厌倦了学校里太多的老年女教师。

		她却始终背对着门口站着。我想,她一定能够听到孩子们的推挤声和议论声,也许是羞涩,令她没有回过头。\par
		后来,她真的如传言所讲,成为了我们新的班主任。\par
		还是那一条长长的白色连衣裙,她转过身来,立在讲台上,一脸纯净的微笑。\par
		她也确实刚刚毕业,大概不过20岁的年纪。与其说她是老师,不如说更像一位亲切的姐姐。

		除了语文,她还教写字课一类的副科。\par
		我记得,她的字很漂亮,粉笔总在黑板上吱吱地划出有力而不失优美的线条。她教我们使用钢笔。\par
		那时,我总希望把字帖写好,一笔一划地练习着,却由于用力太大将钢笔用分了叉。\par
		为了写好字,我的字帖上的空白处也被练习的字迹占据了。规定练习5次的字,我却愿意写上10次,20次,还乐此不疲。\par
		我太希望能写出和她一样漂亮的字了。从那时起,我就对钢笔水划过白纸那蓝色的线条痴迷不已。\par
		写好的字帖交给老师批改,她会在写得好的字上画上圆圈。渐渐,我获得的圆圈越来越多了。\par
		而今看来,我曾经写下的那些练习中的钢笔字一定非常稚拙。但老师却看得出,每一个字,每一个笔画,都包含着孩子认真的心。\par
		她于是常常鼓励我。所以,我一直相信,自己也可以把字写得很漂亮。

		到现在,那个绿树荧荧闪烁的窗口,那个日光充沛的初夏,还仿佛近在眼前。\par
		但穿着白裙子的祁老师已经成为孩子的母亲。\par
		我们的小学,那只有几排简陋瓦房的小学,在城市改造中早已被拆除。一行行缀满花朵的槐树,也被移走,或干脆砍掉。\par
		好像是夜晚的星星带走了那些星光一样的小白花。站在树下唱着歌的孩子们不见了踪影。\par
		老师也离开了,调往周边的学校,继续他们的教师生涯。\par
		她书桌的玻璃板下,会不会压一张旧时的毕业照片。那一年,我们还是天真的孩子,那一年,你还是穿白裙子的女孩。

		很多年,没有了她的消息,不知道她去了哪一所学校。

		一个不经意的念头,让我在网上搜索她的姓名,令我得知她目前大约的工作单位。\par
		那一所小学,有一位和她同名的语文老师,也许是她吧,也许不是。我忐忑着在留言板上留下我的联系电话,然后等候回音。\par
		这之后的第三天晚上,我真的收到了老师的短信。

		她说,此刻她激动而意外,她说,她心潮澎湃。

		十几年的光阴。我们都各自穿越,又在另一个端点上再次重逢。我记起的,是她纯净的微笑,洁白的裙。

		老师,你好吗。我们都在一夜之间长大了,如我们初见你时,你的年纪。\par
		初夏,仅能联系上的几个小学同学,相约要去看看她。\par
		绿树依旧荧荧,在那张毕业照片上,你还能认出谁的脸孔,叫出谁的姓名,想起谁,那时的调皮,那时的可爱。

		\subpart{伙伴}

		其实,儿时的记忆多半可疑。我于是感激,我所记起的,总是些明亮的欢乐,而将晦暗的部分全然忽略不计。\par
		关于我的伙伴,我记得的,只是一起嬉笑着走在阳光里的片段,只是舌头在冰凉凉的小豆冰糕上感受到的甜蜜滋味。

		我第一个伙伴,是我的哥哥。比我大四岁的哥哥。\par
		上小学前,我每天在家里等着他放学回来。上小学后,又有很长一段时间与他一同上下学。\par
		哥哥不曾欺负我,我却是要听他的指挥。他不是学校里出风头的好学生,却喜欢在我面前把自己装扮成那种模样。\par
		哥哥当上了小队长,带回一个画着红杠杠的牌子,用别针别在袖子上。\par
		他告诉我那是小队长的标志,除了小队长还有中队长和大队长,分别是两个杠和三个杠。\par
		还未上学的我,好奇地问:那哪个长比较大呢?哥哥毫不犹豫地回答:当然是小队长了,他们都得听小队长的。\par
		我心中于是对小队长充满了敬佩之情。直到我上了学,才明白事情的真相。

		那时,我也羡慕哥哥的红领巾,羡慕哥哥可以去上学,背着小书包,很神气的样子。\par
		后来,我终于也戴上了红领巾,背起了小书包,和他一起上学去,一样很神气的样子。

		我一年级,哥哥已经五年级了。于是,他有更多可以支配的零花钱。\par
		他不是小气的人,总带我去小卖部,买糖果,和那些小零食给我吃。我喜欢那种站在玻璃柜台前,眼花缭乱的幸福感。\par
		虽然,那时我们买回的多是一些一两毛一袋的萝卜丝一类的小食品,却能够快乐地在回家的途中快乐地吃一路。

		前几天,在QQ上遇到哥哥。远在大庆的哥哥告诉我,他就要结婚了。\par
		哥哥要结婚了。怎么会,分明的,昨天我们还是孩子,还是那个那对在小卖部高声叫着:“我们要五毛钱水果糖”的兄妹。\par
		前年的夏天,你对我说:时间老人真坏。\par
		我笑了,我真想撅着嘴埋怨,责怪他的匆忙。老人为什么还不走得慢一些呢。\par
		你的脚步太快。

		同学里,有更多可爱的伙伴。大眼睛的静,长头发的卉,和我同桌几年的稳。

		静住在街对面的胡同里,她的胸前总挂着一枚钥匙。她梳着短短的小辫子,于是有了“小已巴”这样的外号。\par
		她有一个穿着粉色裙子,能够转动,并发出八音盒一样音乐的洋娃娃。\par
		去她家玩,常常是上满了发条,两个人就静静看那公主一样的娃娃,一圈又一圈优雅而缓慢地旋转。

		卉的家里有一架风琴,这令许多女孩都羡慕不已。\par
		曾有一个女孩对我说:如果我也能考100分,我妈妈一定会给我买一架像卉家那样的钢琴。\par
		我们都叫它钢琴。长头发的卉,和她的钢琴,是这样完美的结合。让所有人都知道,她是一个被父母无比宠爱着的女儿。\par
		大家都喜欢和她在一起。很多时候,她是孩子们围绕的中心。

		与我同桌的稳,和我一起在课堂上画日记。\par
		我们用彩色铅笔在那些笔记本上涂画出帆船,树木,花朵,小兔子,和冰淇淋。\par
		我们写下一些歪歪扭扭的字,编出一两个离奇古怪的故事。\par
		我们总是忘记带手工课上要用的剪刀,于是,一同在课间以百米冲刺的速度跑回家去取。\par
		多数的时候,总能顺利地取回,顺利地上课,而不至于因为没有带剪刀而被罚站。\par
		这令我们有种难以名状的,胜利的喜悦。\par
		上中学后,我从未失去联络的同学只剩下稳。几次搬家,也不忘事先互相通知,留下新的地址电话。\par
		生日的时候,她从郊区的家跑来,把礼物送到我手上。\par
		我计算了一下,我们居然已经是15年的朋友了。\par
		她是我最“老”的伙伴。

		还有一些男孩子,是全然失去了踪迹。仿佛只在那一段记忆里出现,仿佛他们只是记忆中小小的演员,而从未真实存在。

		好像和我坐在教室最后排玩着拔根儿的梁。

		那是一个小眼睛的男孩,他说他因此喜欢大眼睛的女生。\par
		他喜欢开玩笑,也会讲许多笑话。有一段时间,我们会一起放学回家。我曾经弄碎了他挂在脖子上的玉坠。\par
		记得,他似乎是生气了,连玉坠也丢下不要。我回家将那碎成三块的玉用透明胶条黏好,第二天带给他。\par
		我早已忘了,后来我们是如何和好。我也忘了,后来我们是如何疏远,又再次熟络起来。\par
		小学毕业后还常常接到他的电话。直到有一次,他到中学门口等我放学,我却匆匆地骑车跑开,装作没有看到他。\par
		似乎是那之后,他没有再找我。我也松下一口气来。而我,不过是害怕同学的闲话罢了。\par
		却就此,失去了他的全部消息。\par
		最后一次见到,是高中的某天,在学校的后门。他已是一幅社会青年的模样,和一群人坐在一处,香烟的雾,模糊了他的脸。\par
		不知道梁现在怎样,也许他已不再轻狂,而有了静定和沉着。

		曾坐在我座位前的岩,在中学六年依旧同我一所学校,只是在不同的班级。\par
		然而,我们似乎却再没有说过话。

		岩常常是一个人,背着硕大的书包,默默地独行,从我的眼前经过。我开始不敢与他打招呼。\par
		他那沉默的神情,甚至令我怀疑,他是否还记得我是他的小学同学。

		小学时,他是快乐的孩子,他的学习很好。\par
		岩的姥姥开着一家小文具店,我们常常去那里买一些橡皮或者圆珠笔之类的东西。\par
		一次分角色朗诵课文《草船借箭》,他扮演诸葛亮,我读周瑜,于是后来,他便常以孔明自居,把我叫做公瑾。\par
		那一段时期,我们给每个人都起了三国中的名字。\par
		我一直对周瑜充满好感,大约也于此有关。何况之后我又读到“曲有误,周郎顾”这样美丽的故事呢。\par
		岩做了许多小纸人给我,是周瑜在演奏各种乐器。我也在白纸上画出羽毛,做成了一把羽扇送给他。\par
		岩大概早已忘记这些陈年的事。我却时常记起,并不禁会心一笑。

		听卉说,现在的岩留了长头发,还有些卷。\par
		我无法知道,他的生活有了怎样的改变。我无从了解,他的心里有怎样一个关于诸葛孔明的回忆。

		在四年级转学离开的军,有一双漂亮的眼睛,长长的睫毛,乌黑的瞳孔。\par
		他是个有些脾气的男孩,有一阵,坐在我的右边,我们常常争吵不休,却并不知道为些什么。\par
		一次,情急之下,我伸手拿起桌上的橘汁泼在他的头上。两个人一时间同时哑然了,好几秒。\par
		我有些后悔,看着那橘红色的汁水从他的头发上流下来。\par
		出乎我意料的是,他竟然没有生气,也不再和我争吵。他刚刚将汁液简单擦干,上课铃便响了。\par
		我始终坐立不安,偷偷往他那里看。他傻傻地冲我笑笑,摸摸因为糖分而被粘黏立起来的头发说:跟发胶似的。我于是也笑了。

		我借给他的一支笔他始终没有还。我几次催他向他要,他总是说忘记带了。\par
		直到他转学走后,另一个男孩告诉我,军和他说,他是想留下一些纪念。

		再不曾见到那双漂亮的眼睛,再不曾有任何关于他的消息。他的出演大约就此落幕。\par
		军只是属于那一年,属于童年里无邪的吵闹,和孩子们各自越发迷离的记忆。\par
		军还会记得么,那一支笔还在么。

		军一定忘记了。


		\blankrev
		坐在这里,一个下午,我的回忆无法遏制。竟写下这样零碎的许多。\par
		我仿佛一只小老鼠,把藏起的粮食,在一个晴天搬出洞口晾晒。\par
		这些人,那些人,这些被别人遗忘了,或者以另外的版本存在的故事,被我在这个六月重新整理。\par
		还有太多,沉在昨日的湖水之中,不及打捞。\par
		我想,这些时光的果实,该在充沛的日光下被我们在一个恰当的时刻采摘。

		我穿上一双有蝴蝶结的小皮鞋。\par
		我像个孩子那样,吹起一只气球,用细线栓在书包上。

		有时,我距离曾经的自己很近。\par
		有时,却又很远。

		现在,我只是幽幽地,在今天的风里记忆起这些,像一朵墙角的小花,幽幽地独自开放。\par
		没有人知道,我心中的美好。\par
		只有我自己懂得,那一切的甜。

	\endwriting


	\writing{枕上。一}{2007年06月04日 ~ 21:09:08} %<<<2


		今夜,请听我轻声道一句,温存的晚安。

		\blankrev
		我喜欢在睡前的枕上触摸自己的脉搏。一个数字一个数字地细数。\par
		感受着血液的升降起伏,如暗夜的河流,汹涌的寂静,温柔的水。

		我愿意以这样的方式去参悟生命故有的节律。

		当天地都沉默,月色泠泠,我在小小的床铺,如身在孤舟一叶,航行在无际汪洋。\par
		睡意朦胧,涛声清越,就任肆意的幻觉淹没我的夜晚。\par
		就让我是今夜的渔夫,撒一片网,打捞童话中金鱼。\par
		向他许一个愿望,不要木屋,不要城堡和宫殿,只要一处开满茉莉的花园。\par
		来亲手栽植树木和青草,编一圈稀疏的篱笆。\par
		等着早晨的阳光,透过云层,将我的脸孔染上玫瑰的颜色。

		睡前的枕上,我总是这样,漫无边际地想像。

		有时,想你戴一顶草帽,经过我垂着竹帘的门前。艳阳高照,田野葱茏。\par
		有时,想你在远方,寄来窗前的寒梅一朵,夹在泛黄的诗稿。\par
		那是从未存在的你。却无数次,来我似真似幻的知觉。

		你好吗。我轻声在问。\par
		声音刺破我的纱帐,刺破夜空,飘去了谁的耳畔。

		如同我触摸脉搏,如同我聆听心跳,我触摸没有行迹的你,聆听你的一切。\par
		仿佛是陌生,却又亲切得像熟识多载的密友一般。

		你在这里。你在那里。你在我梦里的梦里。\par
		你与我,用同一个姓名,怀同一种情绪,同一种喜悦。\par
		却在相异的时空,各自漂流。\par
		唯有夜晚,汇流于一处水上小洲。

		你是那撒网的渔夫,是茉莉花园的主人。你穿紫罗兰色的衣衫,你有云霞一样漂泊的眼神。\par
		你从不是梦,梦只是虚妄。\par
		你是精灵,是落下的一地风花,如星的光辉,明净的荒凉,却无忧伤。\par
		你会听见我的呼唤,在血液中苏醒,如我从生命的开端苏醒。\par
		你伸出修长的手指,触摸我的脉搏。我们聆听,这跃动不息的生。

		我闭上双眼,在深暗的夜世界,感受你。\par
		仿如在一面镶嵌满魔石的镜前,望见另一个自己。

	\endwriting


	\writing{枕上。二}{2007年06月13日 ~ 15:31:10} %<<<2

		也许,真正的幸福,从来便只能是不为彼岸,只为海。


		\blankrev
		在一个阴天沉睡,忘记白天与黑夜的界限,停止爱恨,无论幸福悲伤。\par
		轻闭双眼,听墙外树声沙沙,云影聚散。\par
		用尽一天的光阴,静默不发一语,沉睡,如沉入暗黑的深海,如重温睡美人的秋冬春夏。\par
		荆棘蔓生床畔,时间被封锁在某刻,以完美的姿态保存。\par
		多少年,风花雪月早已凋谢,唯你的容颜未老,一如往昔。\par
		通往城堡的路,崎岖坎坷,充满险阻。谁能在一个恰当的时刻,将一个恰当的吻,及时送达。\par
		马蹄声响起,隔了几千重的山水,谁会骑这一匹白马,谁会用锋利的佩剑,斩断锈蚀的锁链。\par
		睡美人的梦境荒凉,时光消散,如风似沙。\par
		爱情的玫瑰,开放在百年后的唇角。她的唇是冰凉,一如海上的月光。\par
		睡美人在亲吻中醒来,后来她是否会迅速衰老,像所有平凡的女子一样。\par
		时间的魔瓶,一旦开启,便再无法收起其中的魔鬼。

		我爱这一半荒凉,一半繁华的童话。\par
		好像爱着一个冬季的寂寞,又爱一场炎夏的喧嚣。\par
		站在满目洁白的雪原中央,与观看傍晚骤然而至的雷雨一样,令灵魂激荡。\par
		而现在,在六月的一个阴天,当灰云朵吞噬了晴空,我只想沉沉地睡去。\par
		去你的海底,寻找属于我的一只贝,相遇在星夜里落了眼泪的人鱼。\par
		一朵泪花,便是一粒珍珠。她有多少的忧伤,让这深海,缀满了珠光的华美。\par
		想轻抚她的发,想听她诉说,那些古老的爱情,关于歌声,关于双脚,和消逝的泡沫。\par
		想靠在她的肩,在巨大的礁石,看月光怎样冰凉,如睡美人的唇。\par
		在远方,你说,你的窗口能够望见大海。\par
		这一片漫无边际,令我忧愁的汪洋,在你的窗前,也许寂寞,也许喧嚣的窗前。\par
		我在我的枕上想像。\par
		我飞越半个地球,去看你的灯火,去听你的浪涛,整夜不息。\par
		亲爱的人,我依然在荒凉的古堡,封锁在荆棘丛生。\par
		亲爱的人,我却不曾有容颜不老的魔法,来有足够的可能,目送时光的离去。\par
		我只是平凡的女子。\par
		如所有平凡的女子一般。

		多少守候的心,在故事中,故事的故事中,望穿秋水。\par
		此时的夏天,园中小莲初绽。无言洁白,无言芬芳。\par
		池塘中的湖色天光,仿佛谁随遇而安的心境,任四季花开花谢,云卷云舒。\par
		读一本写满心事的书。\par
		在睡前的枕上,在梦的开端,幸福别人的幸福,悲伤别人的悲伤。\par
		然后,渐渐淡忘。然后,六月的蝉声在雨中熄灭。\par
		我沉入深海,我将自己藏匿在荒凉的古堡。\par
		不要问,什么时候,你才能够走出自己的世界。\par
		不要问,什么时候,你才能够停止这一场场枕上的,纸上的荒唐。\par
		如果,我无法是海上的月光。那么,任我做池中淡定微笑的小莲。\par
		在枕上,在纸上,度我的春秋冬夏。

		你永远不会懂。\par
		我的心,是一只沉默的贝。

	\endwriting


	\writing{亲爱}{2007年06月16日 ~ 11:37} %<<<2

		来sohu,希望系统足够稳定,不至像歪酷那样,时时出现故障。

		选了紫色的模版,又在电脑里找到一张紫色花朵的图片。喜欢紫色,淡紫,饱含着水分,没有欣喜
	,亦无忧伤的模样。

		有时,我也问自己,为什么要在msn那个花田半亩之外又建一处博客呢。把文字写在那里,大概也
	该足够。

		却仍然固执地想在另外的地方写一些什么。从半亩花田(中国博客)到花田半亩,04年的夏天,到
	现在。对于花田里的字,我似乎越来越苛责了。希望它们是美如花的,才容许在那里开放,渐渐又希望
	它们不只是美的,还要是静定的,令人心灵纯净的。我开始回避过多激烈的情绪,文字不再是我宣泄的
	场所,而是疗养院一样的地方,有洁净的空气,有安谧的花园,任我去懒散和疏放。

		因此,有许多情绪被隐藏的起来,许多琐碎的生活片段也不愿提起。花田里的文字,不是生活真实
	的热闹,是飘在半空的寂寞。我爱那凌空的寂寞,因寂寞而冷静,而人却不能够不用双脚去行走。生活
	终究是热闹的,世俗,喧嚣,甚至肮脏。但这就是我所眷恋的生活,我不该拒绝它真实的面目。

		于是,在这里写字吧。写大大小小的情绪,那些琐碎,那些日后不辨真假的经过。

		亲爱地。我这样命名它。

		喜欢亲爱这个词,满是温情的爱意。我想,对于每一个平凡的日子,都该怀着这样喜悦而感恩的心
	吧。

		这两天,没缘由地发低烧,吃过药后蜷缩在床上,等着汗水渗出发丝和皮肤。然后,终于稍微轻松
	,起身试表,体温恢复了正常。过几个小时,却又再次烧起来,于是又要重复刚才的步骤。出汗,让我
	更深切地相信,人是水做的骨肉。

		生病,已渐渐不令我沮丧。有时候甚至觉得,是病,令我的灵魂更加纯净,懂得了怜悯和慈悲。

	\endwriting


	\writing{医院}{2007年06月18日 ~ 20:41} %<<<2

		到医院复查,大厅中挤满了人,门外的挂号处更是人山人海,路边躺着一脸疲惫的患者,很多人都
	是在这里熬上几天几夜,才挂上一张专家门诊。

		我从人群的夹缝里挤过,坐在诊室外候诊。来来往往的患者,天南海北的口音,全国人民上协和,
	丝毫不假。

		被父母轮流抱着的小男孩,不过七八岁的模样,因为身体消瘦而显得格外大的脑袋无力地耷拉在父
	母肩头。他还穿着学校的校服。这个时候,他本该和其他孩子一样奔跑在夏日的艳阳里,痛快地踢一场
	球吧。然而疾病,剥夺了这些别人看来太过寻常的快乐。

		很多事,在健康人眼中是平常甚至无趣。在病人看来,却是莫大的,也许无法企及的幸福。

		好像在夏天畅饮一瓶冰水,好像看一通宵喜爱的电影,好像走很远的路,去看看山顶的风光。太多
	这样的事,对于特定的病人,是不可以去做的了。

		去年夏天在协和住院。临床是一位三十多岁的东北大姐,因为病情较重,她同时在服用四种免疫抑
	制剂。一天,她突然说想吃桃子,而她并没有家属同来北京陪她看病。于是,她便换下病号服,自己去
	医院对面的巷子里去买。大家都劝她不要去,让别人的家属帮着带点就好。她却执意不愿麻烦别人。桃
	子买回来了,她却也开始发烧,整日躺在床上动弹不得。那一袋桃子就放在床边,一直没有吃。到我出
	院时,她的烧还没有退。

		这样的事,健康人是无法理解的。但往往就是这样细小的事,在病人身上却是致命性的。

		那时,由于肺动脉压力很高,大夫给我开下医嘱,只能够在床边活动,去厕所也要有人陪伴,而且
	不可以洗澡。本来病房中是有浴室的,每晚定时开放,洗澡还算方便。我于是提出异议。也觉得上厕所
	也要人陪同,太过夸张。护士听过后,只是淡淡地说,大夫的医嘱一定是有道理,对你好的。后来,我
	才知道,就在这间病房,同样是肺动脉高压的患者,曾经在厕所中昏倒,在浴室中因为缺氧而心衰。

		有时,想起一起住院的文文,瘦瘦弱弱的女孩子,有总是微笑的眼睛。为什么疾病会降临在这样美
	好的女孩子身上呢。他们是真正无辜受难的人。也许,这便是所谓命运。令你只得去接受,而无权提出
	质疑和反抗。

		文文来电话,可爱的辽宁话,一直没发现东北话也可以说得这样婉转动听。她问,你和你对象咋样
	了?我一一作答。我们谈各自的病,谈感情的事。有许多话,是病友间才能够沟通,才能够听懂的。

		不去医院,人不会深切地发觉生命有这样多的疼痛与苦难。好多时候,又往往是无妄之灾。仿佛一
	个瞬间里,世界就倾倒了一块。压住你,让你不得不坚强起来,咬破嘴唇也要挺直身子爬起来。坚强多
	数是这样逼出来的。

		有时,我宁愿坦白承认自己是脆弱的。却不得不去坚强。

		为了好好地生活。

	\endwriting


	\writing{K}{2007年06月21日 ~ 22:17:41} %<<<2

		“K到村子的时候,已经是后半夜了。村子深深地陷在雪地里。城堡所在的那个山冈笼罩在雾霭和夜
	色里看不见了,连一星儿点显示出有一座城堡屹立在那儿的亮光也看不见。K站在一座从大路通向村子
	的木桥上,对着头上那一片空洞虚无的幻境,凝视了好一会。……” ——《城堡》弗兰茨·卡夫卡

		这一段来自《城堡》开头的文字,反复读来,都感觉像卡夫卡对于自身生命状态的一次形象化概括。\par
		冷峻,苍凉,如照片上卡夫卡双眼中洞射出的含义复杂的光芒,那里,一半是无所不在的恐惧,一半是旁观者般的镇静。\par
		在漫无边际的黑夜,在空洞的幻境面前,土地测量员K在原地凝视。\par
		1922年,已罹患肺结核的卡夫卡,在生命的黑夜里,在现实的空洞中,写下了这部后来被视为他代表作的小说。\par
		没有谁不会去联想:K莫不是卡夫卡(Kafka)的缩写。\par
		这样的疑问不会有回答,但可以确定的是,K所遭遇的荒诞情节,在无数的生命体上曾无数次上演,并在持续上演。

		K的遭遇,是人所遭遇的众多困境的一种概括。

		也许在卡夫卡看来,每个人都是土地丈量员。他曾在笔记中写到:

		“道路上没有尽头的,无所谓减少,无所谓增加,但每个人却都用自己儿戏般的尺码去丈量。……”

		因此,与其将K简单看作作者自己的简称,倒不如将其看作整体人类的概称。

		K之存在,其意义大约早已超越卡夫卡本人创作的单纯目的,而在广泛的人群中得到共鸣,引起了灵魂的颤动。\par
		也正是因此,卡夫卡的作品,才能够在他去世后独立于作者之外,用自己的心脏跳动生存,经久不衰。

		站在雪地上的K,去苦苦寻找进入城堡的途径,一次次的尝试,一次次的失败,却义无反顾地继续向目标进发。\par
		城堡,那仿佛无可到达的地方,在阅读过程中令读者感到无限的焦虑和绝望。\par
		卡夫卡似乎是在刻意将这一种焦虑感在文字中扩大,令它笼罩住整部作品,紧紧揪住读者的神经,压抑你的呼吸。\par
		20世纪的人们对于这样的焦虑感到熟悉莫名。透过那或许略显艰涩的文字,人们在K身上见到的分明是一个同样挣扎于其中的自己。\par
		当人在一夜之间变成了虫子,当人开始与曾经的那个人的世界格格不入,显得异样,而充满不安起,这个世界的焦虑便开始了肆意的蔓延。\par
		比瘟疫蔓延的速度更快,比瘟疫更加无影无形,且无孔不入。\par
		卡夫卡用一只甲虫点醒了世人的异化趋势,又用一座无可到达的城堡,揭露了人生的终极困境,和残忍真相。\par
		这样的冷峻无情,他有怎样的勇气,来直面这看得过于透彻的一切。卡夫卡因此是孤独的,卡夫卡因此是痛苦的。

		他说,只应该去读那些咬人和刺人的书。

		“如果我们所读的一本书不能在我们脑门上击一猛拳,使我们惊醒,那我们为什么要读它呢?”

		毫无疑问,卡夫卡的作品便是这样咬人的,刺人的书,使我们惊醒,醒来在浑浑噩噩的生活里,让痛感使我们清醒。\par
		卡夫卡把写作当作一项神圣的使命。写作对于卡夫卡的意义,远远超过了一般作家,虽然,直至他去世他也只是一个默默无闻的业余写作者。\par
		他在信中写到:

		“倘若我不写作,我就会被一只坚定的手推出生活之外。”

		写作从不是他谋生的手段,却是他生命的依靠。在卡夫卡看来,人生的意义绝不在于延续肉体的存在,而在于寻找到精神的家园。\par
		于是,我们是否可以对于城堡做这样一种解读:无可到达的城堡,正是人们所追寻的精神家园,而到达精神家园的过程,亦如去往城堡一样,一样的令人焦虑,绝望,充满了痛苦与折磨。\par
		然而,即使如此清醒地意识到这一条道路的真相,卡夫卡依然不曾屈服或者放弃 —— K倔强地继续着他的寻找。

		和卡夫卡许多的小说一样,《城堡》也是一部没有完成的作品。\par
		但或许没有完成,正是它最好的“完成”方式。好像关于人生的太多发问,好像宗教世界的太多悬疑,是不可解,亦无解的。\par
		K是否最终进入了城堡?K是否完成了他的工作任务,丈量好土地?这些,都已无关紧要了。\par
		重要的是卡夫卡让土地测量员K凭空来到这部小说里,接受着煎熬和折磨,荒诞地经历着一次次的尝试和失败。\par
		卡夫卡让我们想到同样是凭空里来到这个世界,接受着与K相似经历的自己。\par
		没有人不是那个无辜的土地测量员,没有人没有一座自己的城堡。我们都在渴望进入,却又无从进入,焦虑与恐惧在这其中生长。\par
		有多少人能够坚持如K,倔强如K,固执地去寻求那一条路途。这是永远不得而知的事情,没有回答,没有结局,像这一部貌似离奇的小说。

		孤独的卡夫卡,在病痛与感情的折磨下思索,在恐惧与压抑中走过短短41年的生命。\par
		他有瘦削的脸孔,窄窄的肩膀,一双因冷峻而显残酷的眼。他用一支寂寞的笔,震惊了后世的灵魂。\par
		他不曾停止的是思考,不曾停止的是追寻。\par
		布拉格,卡夫卡的故居,朴素的墙漆着蓝色,那是天空的颜色。门前的小巷貌不惊人地延伸,多少人从这里经过。\par
		一定会有人,恍然间记起他的那句话:

		“目标确有一个,道路却无一条;我们谓之路者,乃踌躇也。”

		读起这样的话,我只想说,也只能够说:哦,我亲爱的,残忍的卡夫卡……

	\endwriting


	\writing{雷声}{2007年06月24日 ~ 16:53} %<<<2

		隐隐的雷声,在窗外徘徊,低沉的天空,下午四点。

		母亲说,会有雨吧。风飘开一扇半掩的窗口。

		未开花的牵牛,枝蔓已攀上最高的竹条。

		我盼望着花开,如同此时,盼一场酣畅的滂沱。

		夏天就该有雷雨,有彩虹。好像去年,那一个个骤然雷鸣的傍晚。

		六月的末尾,和良被雨围困在礼堂门口小小的屋檐下。

		看闪电划过浓黑的云,听雷声在耳畔炸开。多少年,我都不曾目睹那样一场雨。

		如此酣畅地宣泄,声嘶力竭一样的呐喊。

		我的心一时间紧紧的,缩成一小团。那个时刻,我突然觉察到人的微小。

		哪怕只是一场雨,已足以令人战栗。

		在这个宇宙里,屋檐下的我们,比两粒尘埃更加微不足道。

		但因为知觉,我们能够体验到幸福。

		此时,小屋被凉风注满。我闭上眼,等待那雨落的片刻。

	\endwriting


	\writing{夏天的碎语}{2007年06月25日 ~ 19:51:11} %<<<2

		花都开好了。我的夏天,苦涩的甜美。


		\blankrev
		纵使夏天于我多数时候是一种苦涩,我依旧无法停止对它的喜爱。

		六月,七月,只是简单地读去,便有了夏的气息。

		是朗朗的晴日,远天膨胀的大片云朵,是刺目的光线里,一件被晾起在风中的白衬衫。是雷雨的傍
	晚,和雨后弥漫的青草香,是一抹彩虹,一只蜻蜓,一篇在雨声里写下的日记。

		小甲虫爬过我的窗纱,门外的墙角处幽幽地开着一丛明黄色的花。这些不知名的小生命,生气勃勃
	地在夏天里享用着丰沛的阳光和雨露,热烈地生活着。

		我一样是如此的小生命,于是,看到日光,听到雨水,会满心的欣喜与光明。深深地俯首,叩谢造
	物主的恩赐。令宇宙中有了世界,令世界有了生机,令万物有了灵性,令我有了知觉。

		在午后听一树的蝉鸣,并不觉聒噪,反是尘嚣中难得的宁静。\par
		一只只蝉,是穿越了千万黑暗,才到达了这个明丽的季节。每当想到这些,心中总是感动莫名。\par
		多少蛰伏于苦难的人,大约便能够有所安慰与鼓舞,相信希望的存在,收起悲哀,去守望破土的明亮。

		夏天,因为生命,因为许多爱,与懂得爱的眼睛,而分外深情。

		读古人的诗,于是灼热的天气里,也有清凉境地,全无食欲的暑热中,也能唇齿生香。端坐桌前抄
	写《心经》,任汗水浸透发丝,一心躁动此刻却已渐平息,波澜不生。洗好几枚色泽鲜亮的桃子,切开
	一只雪白的蜜瓜,慢慢享用甜美的汁水,感谢植物奉献的果实。

		这些事,是适宜在夏天去做的。在苦夏的煎熬中,依旧漫不经心地去体味每一天,不紧不慢地度过
	日月的长短。

		如果,我们能够在热天里没有抱怨,没有烦躁,如果,我们能够保持着安详与淡定,那么,每一天
	都是一种长进。

		所谓舍弃肉体的安逸,而去荣获灵魂的完满。苦行僧似的修行自然不必,但肉体上适当的苦痛,大
	约对于我们的清醒确是有益的。

		夏天,在自然的环境下,给了我们这样的机会。

		在苦夏中,静定下身心,去度过焦灼难耐的日子,去忍受汗水的洗礼。这只是小小的练习。\par
		不再执着于身体的舒适,然后,我们能够牺牲一个太多贪恋的自我,去体味智慧的清明。

		小鹿说,在夏天,她总想读法国的小说。\par
		田说,我总是在夏天有写小说的冲动。\par
		也许,夏天,是适合于小说的,夏天,是容易引人走入一种幻境的。\par
		刺目的日光下一棵浓荫的树,路过的穿白裙的女孩。雨天里的十字路口,一盏亮起的绿灯。\par
		这些,不过是平凡的场景,在夏天,却令人有了许多遐想。

		想一个女孩,在烈日下如水的心事,想她淡淡描了的眉毛。一场青涩的青春,一个日后反复说起的
	夏天。

		也许,那便是曾经的自己,以另外的身影,在现实,在幻想中的显现。仿佛没有了记忆,在文字里
	,我的从前被抛掷一空,全然成为了别人的故事,如此陌生。你永远读不出,哪一句是真,哪一句又是
	假。

		田从不是善于讲故事的人。\par
		我总是遗忘。我总是擅自臆造出那些过往。过于美的,过于悲伤的,或者,过于失真的。

		夏天,写下小说,然后锁入抽屉,不再去读,亦不拿给谁去读。

		好像太多的欢乐,太多的苦难,不需要被展示,被了解,只要在心中默默生长,如那丛门外的小黄
	花,幽幽地开放。

		在平静中度过夏天。\par
		这是一年中最明亮的季节。\par
		于是,我们也该有明亮的心,去感谢一切。\par
		花都开好了。\par
		只等一个微笑。

	\endwriting


	\writing{华盖之下}{2007年07月01日 ~ 21:05} %<<<2

		看到一句话:但愿快乐,不是你忧伤的华盖。令我感觉触目惊心。

		多少时候,微笑的面孔下,掩藏的是分明的忧伤?

		多少时候,将痛苦轻描淡写的我们,独自将一杯杯苦酒饮下?

		一时间,想起太多的人。表面看来他们无一例外是如此坚强。

		面对疾病和苦难,紧咬住唇,依旧不说一句泄气的话。好像是病久了,人也便生出对于苦难的免疫。

		互相说着鼓励的话,宽慰的话,其实,谁都明白,这貌似坚不可摧的意志后边,是一颗分外脆弱的心。

		那些闪闪发光,充满了希望的劝慰,不是说给对方听,而是说给自己听的。

		这样的坚强,剥去了伪饰的坚硬外壳,显得如此颓唐狼狈。

		快乐,成为忧伤的华盖。它越是美丽,越是暴露出那忧伤的沉痛。

		如果我们能够真正地笑面这一切,那么,一定是因为对于生命更深的理解。

		看到子尤,那个身患癌症,依然昂起头来问一句“谁的青春有我狂”的天才少年,他的文字,他的苦难。

		他在疾病与死亡面前的勇敢,令我羞愧。我甚至自责,自己的悲观,自卑,和不堪一击的内心。

		子尤的世界里,是疾病蔓延的黑暗,他却用他年轻的光芒,把过于匆忙的生命照得雪亮。

		我的直觉告诉我,那不是一顶绘了图案的华盖。他的青春是真实的,他的坚强是真实的。

		因为,我懂得痛苦的掩盖是多么矫情而不堪的一种模样,然而,在子尤的眼神里,我没有发现一丝痕迹。

		他是真正懂得了生命的人。他没有怨恨命运的安排。他将自己的病,称为上帝赠与的一颗金色肿瘤。

		多少被痛苦折磨的日夜,多少次昏迷与清醒间的临界。我没有经历,但我的经历已足以令我能够想象到他曾承受的苦难。

		子尤爱生命,他真正爱生命,所以,一切的痛苦与不幸,都不能阻止他的快乐,他的青春,他的飞扬。

		他令我相信,没有什么,能够剥夺你去生活的权利,只要,是真的生活过,只要,你真的爱着。

	\endwriting


	\writing{不怕不怕}{2007年07月04日 ~ 14:51} %<<<2

		\longpoem{}{}{}
		紧握的蓓蕾不会开花别太傻 \\
		放开手让他拥抱风吹和雨打 \\
		顶着风不要害怕让你哭泣的那粒沙 \\
		流泪吧,微笑吧,就长大
		\endlongpoem

		偶然听到这样一首歌,《暴雨纪念日》。

		\longpoem{}{}{}
			顶着风,不要害怕,让你哭泣的那粒沙
		\endlongpoem

		令我想到,很多年前,当我还是那个穿着白色连裤袜的小女孩,听到的那一首《水手》。

		那一年,似乎所有大街小巷,都在飘动那样一个悲壮而坚硬的歌声。

		\longpoem{}{}{}
			风雨中,那些痛算什么,擦干泪,不要问为什么
		\endlongpoem

		06年的夏天,在协和住院,辗转难眠的夜里,听到楼下有人在唱这首歌。

		不要问为什么。那个歌声这样远,却又仿佛近在咫尺。我侧卧在寂静的病床,默默地听。

		当恐惧在我心中肆虐蔓延的时候,当绝望一丝丝吞噬掉对于未来的期盼时,这歌声简直是一场救赎。

		不知觉中,泪已浸湿了苍白的床单。却不是因为悲伤。

		小时候,我只是跟着哥哥,浅浅地唱着《水手》,看着电视上拄着双拐走下轮椅的男人。

		现在,我开始懂得这样一首歌。我也开始相信,疼痛终于会将我打磨成一块美丽的宝石。

		对自己说,不怕不怕。

	\endwriting


	\writing{依旧}{2007年07月04日 ~ 20:27:31} %<<<2

		夏花的绚烂,七月,被网住的情绪,无处停靠。

		\blankrev
		让生如夏花之绚烂。我喜欢泰戈尔的这一句诗。\par
		想起01年的夏天,一个人在病中读那本《漂鸟集》。\par
		我记得全书最后的一行文字:我信赖你的爱。

		远去的01年,淹没在太多人因申奥成功而激起的如潮欢乐中。\par
		而我15岁的夏天,却如一缕淡淡的烟,在时光中不断抽离,终于面目模糊。\par
		好像镜中的自己,转眼间肥胖浮肿的面孔,显得虚假且狼狈。

		如今,当我站在远处,望着那个自己,心中的忧郁早已不见痕迹。\par
		只仿佛是观看一场悲伤的电影,读一本行文冰冷的小说。\par
		一个女孩,在花一样的年纪上,患上无法治愈的疾病。一个女孩,注定了用后来的日子,与险恶的命运正面交锋。\par
		如此而已,或许,是早已滥觞的情节,早已令人厌倦的题材。\par
		但是,这些,只有在远处,只有在局外,你才能够如此轻描淡写地看待。\par
		当我成为了那个镜子中面目全非的女孩,吞下一粒粒药片的时候,才懂得,苦难永远无法真正地被了解。

		15岁的女孩,还没有做好足够的准备,去面临生命里的种种艰险。\par
		镜子里,她哭了,她反复问为什么,为什么是她。而所谓命运,便是从无解答。\par
		在炎炎的夏天,躲在房间读泰戈尔的诗歌。那里,有飞翔的鸟,有漂泊的云,有盛开的花,和闪烁的繁星。\par
		爱生命吧,并发现爱吧。一时间,心中溢满了这样的两句话。\par
		放下怨恨,放下恐惧,我闭上眼睛,默念着:让生如夏花之绚烂。

		01年,一处鲜明的坐标,在我的世界,刻出深深的疤痕,却在结出的血痂上长出一朵洁白的小花。\par
		我知道,生命在用一种特殊的方式,将我打磨。\par
		好像,一位信基督的朋友曾对我说:上帝这样安排,一定有他的良苦用心。\par
		我笑了,疾病没有令我被隔离抛弃,反而成为神的宠儿。

		这样远了,远去的15岁,后来的花季和雨季。

		有时,我翻看曾经的日记,却只相遇了一个个陌生的自己。\par
		夏天里独自去紫竹院看荷花的女孩,我竟然忘记,自己在本子上画下过那样一幅荷花的速写。\par
		冬天里一个人踏着雪走回家的女孩,我竟然忘记,自己用舌头去迎接第一场大雪的甜美,雪花是冰凉的。

		我似乎是自己感动着自己。

		一次次的蜕变,令我成为一个个截然不同的人,截然不同的自己。\par
		我喜欢日记里的女孩,我也想念她们,我为自己曾是她们而感到骄傲。\par
		虽然,她总是不够坚强,在病痛里不争气地掉眼泪,写下过绝望的话,悲哀的诗。\par
		但我原谅她,因这不该是一个与疾病相关的年纪。\par
		她该好好地享用青春的光华,毫无顾忌地去挥霍时光,不是么。然而,许多的欢乐,就这样轻易而决绝地错失。

		如果,如果,我反复假设,另外一种可能,千万种可能。我会更幸福吧,我会多么幸福啊。\par
		然而,没有谁给我们任何假设的机会。

		人生是一条寂寞的单行道。在命运中,我们只有独行。

		如今,我的生活依旧,夏天依旧。

		身体以一种费力的姿态,延续着她的工作,我时常深深感谢,我的心,我的肺,我的胃,和一切参与其中的器官。\par
		我知道,在默默中,他们比我承受了更多来自疾病的伤害。\par
		我爱他们。我信赖你的爱。\par
		和自己说话,这样的事情也许荒唐,却总令我倍感安慰。\par
		我说,我要呵护你们,不让你们再受折磨与痛苦。

		有人曾问我,如果生命满是欢乐,你爱它,如果生命只是平淡,你也爱它,但倘若生命是接踵的不幸呢?\par
		那天,我没有回答,我沉默了很久,说不出一句话。\par
		今天,我却想说,我依旧爱他。

		因为,那是属于我的。

	\endwriting


	\writing{观看}{2007年07月05日 ~ 11:01} %<<<2

		朱打越洋电话来,说起澳洲的寂寞,说起对北京的想念。\par
		你的窗外不是有海么,怎么还会有寂寞。我问。\par
		我时常想像,当月光洒满了海面,朱一个人倚窗站着,怀着淡淡的乡愁,一颗颗辨认南半球的星座。\par
		那是多美的画面,独自身在异乡的女孩子,度过着一段或许过于平静,但却诗意盎然的生活。\par
		朱笑了。哪里如你想像的一样。所谓的海,倒不如说是像个人工湖,没有月光,只有两盏路灯。\par
		北京多好啊,你半夜饿了,总有地方去吃饭,在这里,什么也没有。朱没好气地抱怨。\par
		还有,这里风刮得太可怕,潮气很重,东西总是不干。\par
		电话这一头,我一句句听她带着几分委屈的诉说。北京的窗外正是艳阳高照,火热泼辣的夏天。\par
		我想像中诗意的生活,原来只可以在这彼岸观看。真正处身其中的人,感受的,唯有生活的真实琐碎。\par
		这些真实的,关乎了柴米油盐,洗衣做饭的事情,往往是狼狈,甚至不堪的。\par
		一个人在异乡的生活,必然是有许多的不适和艰辛。\par
		我想像中,那涛声夜夜的海,也许丝毫没有消减这其中的寂寞,却反而是加深了寂寞。\par
		很多时候,我们总觉得幸福都在别人的生活中,也是因为站在彼岸观看的原因。\par
		远看,是桃源仙境,走近,却是肮脏泥泞的鸡舍原田。\par
		在别人的岸上望你,你的世界,或许也是火树银花,如梦似幻。\par
		其实,谁的幸福能比谁多多少呢。我总相信每个人快乐与悲伤的总量是守恒的。\par
		不都是平常的日子,琐碎繁复的种种。

	\endwriting


	\writing{温度}{2007年07月11日 ~ 11:54:53} %<<<2

		冷暖自知。心存感激。

		\blankrev \blankrev
		七月的天气,白花花的日光,把我的窗口映得雪白通明。总是在清晨五点便恍惚醒来。\par
		喝一杯凉开水,然后看看这完好的世界,又在黑夜的彼端悄无声息地复苏。\par
		打开关闭了一夜的手机。\par
		关心雨水和温度,定制的预报短信总是准时发来,我却总是留到第二天早上看。\par
		喜欢那千篇一律的开头:北京移动提示您注意天气变化……\par
		喜欢在预测了明日的风向和温度后,偶尔充满温情的一句:天气炎热,外出请注意防晒。\par
		大约不过都是本无深意的模式和客套,细细想来,却也有温馨。虽是无心,若在我们这里生出花朵,不是很好的事么。\par
		或许,我只是愿意听这样的关爱,这样嘘寒问暖的话。于是,纵使是刻板的天气提示,也能令我满心欣喜。

		\blankrev
		北方的四季分明,温度的转变在换季的日子总是急骤得令人难免慌忙。\par
		由冬到春,总是要经历几场“倒春寒”,才得彻底地温煦明媚起来。\par
		夏天,又是雷雨频仍,方才朗朗天空,忽而便乌云压城,风雨潇潇。\par
		还有秋日的风,冬天的寒流,都是毫无形迹地来到,不必对谁事先预告。\par
		北方的天气是任性的,我行我素地横冲直闯,乐于雨便有肆意滂沱,乐于雪又是满目飞霜。\par
		对于身体不够强健的人,在分明的四季中生活是要小心翼翼的。医生对我说,你每天一定要看天气预报。\par
		有时,我也怀着笑意想:是否孱弱的身体更容易做到天人合一?天地丝毫的风吹草动,我的身体都有敏锐的感应。

		\blankrev
		温度,大约是区分季节最鲜明的标尺。\par
		冬的寒冷,夏的炎热,竟能相差将近四十度。地球的公转自转,让这世间有了奇妙的变化。\par
		中学时,地理考试中有一道题,问如果地轴倾角增大或减小会对气候有何影响。\par
		答案似乎是热带的区域会增加或减小,已记不得了。却在那个时候,对宇宙充满了神秘的敬畏。\par
		是谁,在茫茫星空中安放了这一颗蓝色的星球?是谁用巨手调整好一切的角度和速度,让这世界如此丰富。\par
		坐地日行八万里,这样想着的时候,更觉自己是时空间的旅客。眼前的悲欢种种不过车窗外匆匆退去的风景。\par
		而我们,永远是风景中的人。逃不过风雨和日晒,躲不去一季季的冷暖更迭。\par
		自己的小日子,在宇宙的帷幕下,或许显得卑微可笑。还有什么是值得得意的呢,又有什么是值得痛哭的呢。\par
		一切都在我们手上,一切又是空空如也的安静。人何必争执,何必不舍。

		\blankrev
		七月的一个晚上,梦到自己在冬夜里独自站在偌大的阳台上,看到漫天闪烁的星斗,在深暗的背景里放出寒冷的光芒。\par
		那是带着时空距离的寒冷,是寒冷的,却又是醉人的美,如此洁净的光。\par
		梦里,我在那里站了许久,痴痴地仰头,心中充满了得到安慰一样的喜悦。\par
		醒来,我依然清晰记得这个梦,并喋喋不休地说给身旁的朋友们听。\par
		不过是个梦。也许所有人都会这样想,并没有注意到我在说起它的时候,那语气中的惊奇。\par
		我不曾见过那样的星空。虽然,我也曾无数次想像,在远离城市喧嚣的地方,我可以躺在草丛里,看一夜的星光。\par
		星星的光芒,却离我是这样远了。我与它们的距离,要用比光年更遥远的单位去衡量。\par
		星光在我眼中是寒冷,事实却是人类无以估计的剧烈燃烧的热。\par
		因为时间与空间的阻隔,令这温度的感触,有了截然相反的体会。或许一切事,莫不会被它扭曲变形。

		\blankrev
		无论寒暑,我总是通体冰凉。仿佛像一条蛇一样,是冷的血液。\par
		而我知道,这只是因为我的血流缓慢。如果每个人的身体是一条河,那么我便是在山涧里缓缓流淌。\par
		也许,我和那遥遥的星光一样,看似是寒冷,实则是燃烧的。

		\blankrev
		母亲总在睡前发短信来提醒我:明日降温,多穿衣服。……明天下雨,记得带伞。……\par
		这时候,我便躺在床上想像她吃力地按着拼音的模样。母亲本不会发短信,却硬是强迫着自己在一次次练习里学会。\par
		只为了最快捷地告诉我,明日的天气变化,只为了保护她的孩子,在骤变的天气中免受伤害。\par
		母亲的话,总是那样几句,于是,我调侃说:你干脆把那几条短信存起来,每天直接发好了。\par
		她笑笑:我每天写,还能练练拼音,动动脑子呢。许多时候,我发觉母亲越来越像一个孩子了。

		\blankrev
		我渐渐懂得。那个在睡前提醒你天凉加件衣的人,一定是真正爱你的人。\par
		因为,他把你的温度安放在了自己的心里。

	\endwriting


	\writing{田,田}{2007年7月17日 ~ 17:16} %<<<2

		我喜欢你们这样叫我,田。那语气轻轻的,像捧起一掌清水。\par
		有时,我竟慌乱。\par
		如同面对镜子时偶然的陌生,我不知道,田究竟是谁,是怎样一个女孩,度过着怎样的日月。\par
		田在何处。田的心是快乐的么,抑或悲伤?\par
		我仿佛成为站在远处的行人。远远地望她,她迟疑的脚步,苍白的面孔,洞张的眼睛。\par
		在田的身体之外,我看到另一个她。比我想象中可爱。\par
		因为田的苦痛与狼狈,被忽略了,因为田真实琐碎的生活,被遮蔽了。\par
		我只是远远地,望见她,走过夏日街道的女孩,行走在诗与爱的执迷间的女孩。\par
		我不懂得她的苦难,于是,可以想当然地以为,她的生活充满了轻易的幸福。\par
		田在这里,知道艰辛与跋涉,知道貌似平静的背后,是怎样的暴风骤雨。\par
		隐匿在美丽表象下的危险,一丝丝侵蚀着许多幸福的可能,逼迫你,去放弃,又报以微笑。\par
		田,田。我愿意听,你们这样叫我。\par
		那个时候,我便成为远远望着的行人,感觉田的生活,是如一首情歌般深情温存的。\par
		也许,那真的是我。\par
		田。

	\endwriting


	\writing{如鱼}{2007年07月24日 ~ 13:42:33} %<<<2

		鱼在水中,云在天。

		\blankrev \blankrev
		总觉得金鱼是属于夏天的生物。于是,几乎每一年暑假总要买回两只,放在书桌上,精心养起来。\par
		窗上是不绝的蝉声,窗下是已攀上栏杆的牵牛,这样的午后,百无聊赖的闲散,像水墨的留白,因空而丰富。

		\blankrev
		看我的鱼在圆柱体的鱼缸中来来回回。

		\blankrev
		鱼是快乐的么。子非鱼,安知鱼之乐。\par
		鱼是寂寞的么。我不是鱼,却知鱼的寂寞。\par
		两条鱼在狭小空间中头尾交错,又擦身而去,怎么看,怎么像世间的太多相遇。\par
		如此匆忙,如此拥挤,又是恒久的无言。

		\blankrev
		鱼大约是这世界最沉默的生物,除了极偶然跃出水面激起水花,它们几乎不发出一丝声响。\par
		当然,这声响局限于人耳所能接收到的频率波段。在我眼中,鱼的沈静,分明隐秘着生命原始的寂寞。\par
		在水的围困下,在水的拥抱下,它们不忧不惧地度过着自己的生涯。

		\blankrev
		据说,看鱼游水的姿态能够令人心神愉悦。\par
		鱼的姿态确是优雅的。特别是金鱼,如花绽放般的尾巴,纱裙一样的轻柔飘逸。\par
		看我的鱼,看它们的快乐或者寂寞。\par
		在城市的喧嚣之外,鱼有自己安然的生活。

		\blankrev
		他们说,鱼没有眼泪,他们说,鱼的记忆只有七秒钟。\par
		水中的鱼,即使哭泣,又怎么会有人知道,即使记得,又如何对什么人说起回忆。

		\blankrev
		鱼从不让谁看穿它的心事。\par
		于是,人以为鱼是忘情的,鱼是没有烦忧的。\par
		倘若记忆真的只有七秒,该有多少悲伤刹那里烟消云散,却也有多少欢乐瞬息间不知所踪。

		\blankrev
		回忆,总是一半疼痛,一半甜蜜。

		\blankrev
		鱼的心事,埋藏在水下,不去诉说,不去哀怨。鱼的沉默里,是隐忍的坚强。\par
		鱼也许是个哲学家,它的智慧无声息,来来去去,真正是子非鱼安知鱼。\par
		看我的鱼,越发觉得我无法参悟透它们的世界。\par
		或许,这无言的生物,是佛陀安排在世间的使者,来给人以启示。\par
		虽然,多数的时候,我们忽略了它们的存在,只是混沌无知地经过,而没有足够的觉醒。

		\blankrev
		父亲喜欢钓鱼。\par
		童年的记忆中,很多的夏天傍晚,他都是带着一身鱼腥,风尘仆仆地归来。\par
		然后,是厨房中的一阵忙乱,然后,是鱼肉的香味弥漫在小小的院落。\par
		那是一些有星星的晚上。一家人坐在小院中,分享一盘红烧鱼。

		\blankrev
		我不曾想过鱼钩穿透鱼嘴时鱼的疼痛。\par
		我只陶醉在鱼肉的美味。而现在想起来,却觉得人捕获鱼的方法未免残忍。\par
		人终于不是鱼,人终于无法将鱼的疼痛感同身受,人终于还是要吃鱼,享用它的鲜美。

		\blankrev
		顾城曾给他的法文翻译尚德兰女士写了两幅字,一副是“人可生如蚁,而美如神”,另一幅则是“鱼在盘子里想家”。\par
		诗人盘子里的鱼,是多情的远行者。它迷路在远方了,再回不去。\par
		读到这一句话,我仿佛见到那一条躺在白色磁盘中急促呼吸的鱼,它洞张的,不会流泪的眼睛,充满了令人惊心的悲伤。\par
		但即便如此,鱼依旧不发出一丝的声响,它以沉默面对生死之界。

		\blankrev
		曾是悠游于水的鱼,在无限眷恋中离开,诗人的心总是触及到那些我们视而不见的疼痛。\par
		鱼在盘子里想家。\par
		我渐渐已不忍心读这一句话。

		\blankrev
		庄子《大宗师》中有言:“泉涸,鱼相与处于陆,相呴以湿,相濡以沫,不如相忘于江湖。”\par
		庄子的话,本是论道,却被后人引做他用。

		\blankrev
		人们说,与其相濡以沫,不如相忘于江湖。

		\blankrev
		这话说得看似洒脱,实则万般艰难。分明是落着两行泪水,道出这样一句决绝的离别。\par
		看似决绝的人,往往是最狠不下心肠的人,所以才要用冷的面孔,冷的言语,粉饰和掩盖那一心的不舍。\par
		相忘于江湖,然后,或许彼此能够拥有各自的欢乐。\par
		但此种种,也不过一厢情愿的猜测。\par
		从此后,是海阔山遥,从此后,是汪洋中的各自沉浮。

		\blankrev
		人的错失,有时,大约真如鱼与鱼的擦身。\par
		只是,若鱼的记忆真的只有七秒钟,在江湖之上便真可相忘。\par
		而人,人太过发达的神经,如何去真正无所留恋地忘情。

		\blankrev
		因此,人无法如鱼。\par
		如鱼沉默,如鱼悠游,如鱼埋藏了心事,安然于自己的生活。

		\blankrev
		看仰韶文化陶器上的鱼形纹,让我知道在那么遥远的年代里,人在心中对于鱼就充满了美的想象。\par
		不只是器物上绘画的花纹,还有那太多美妙的传说和无邪的诗歌。

		《列仙传》上载赵人琴高行神仙道术,曾乘赤鲤来,留月余处复入水去。\par
		那月明的夜晚,水仙乘鲤而来,乘鲤而去,水面的清辉,清越的飞浪,该是怎般的飞逸动人。

		鲤鱼大约是最有仙骨的鱼,它们的跃起,总有传奇发生。

		读唐诗,翻到戴叔伦的一首《兰溪棹歌》:

		\shortpoem{}{}{}
		凉月如眉挂柳湾,越中山色镜中看。\\
		兰溪三日桃花雨,半夜鲤鱼来上滩。\\
		\endshortpoem

		一片片如粉的桃花,就扑入我梦中来,夹着轻轻雨丝,在凉月初升的夜半,浸湿一身衣衫。
		鲤鱼在这诗中,在涨起的春水中,激荡着层层水花灵动。

		鲁昭公赐孔子一尾鲤鱼,于是孔子的儿子因此而得名孔鲤,字伯鱼。
		鲤鱼大约也因此沾染了些圣人的灵气,而显得特别。古人的朴拙可爱在这名字中也可见一斑。

		感动于《乐府诗集》中那一首《饮马长城窟行》。

		\shortpoem{}{}{}
		…… \\
		客从远方来,遗我双鲤鱼。\\
		呼儿烹鲤鱼,中有尺素书。\\
		长跪读素书,书中竟何如?\\
		上言加餐饭,下言长相忆。\\
		\endshortpoem

		一双鲤鱼,藏匿着爱人远方的消息。一封家书,承载了多少千山万水的惆怅深情。\par
		她长跪在地,读这一封信,读着琐碎的嘱托:多吃些饭,莫因思念消瘦了身体。\par
		素白无华的诗句,古老真挚的爱情,在鱼的腹中成就着时光的永恒。\par
		千年之后,当再读起如此的诗,心中仍是一阵温暖的凄恻。

		\blankrev
		所谓爱情,不只是一句“执子之手,与子偕老”的承诺,而是长相忆的不变情怀。\par
		让我爱你,用鱼传尺素一般的心。

		\blankrev
		我不知道,这世间是否真的存在人鱼。但我希望那不只是人们的一种想象。

		\blankrev
		多少人为了小美人鱼化作泡沫的故事黯然下泪,多少人在梦境的海上听到人鱼忧伤的歌声。\par
		在一些传说中,人鱼是凶恶的海妖,但更多的故事里,她们是美丽善良的姑娘。

		\blankrev
		曾看过一部电影,是人鱼在现代的故事。\par
		美貌的人鱼爱上了人类的男子,于是幻化出双腿与他相爱。\par
		但每一天,她都要在浴缸中恢复鱼的身躯,才得以继续生存。她的双腿不可以沾水,否则便会显露鱼的形态。\par
		这个秘密终于被一个嫉妒她的女人发现,于是,在一次宴会上,她将一杯水泼向了人鱼的下身。\par
		后来的情节我已记不清晰,只记得众目睽睽之下,人鱼倒在地上那无助的眼神。\par
		人的丑恶在那一刻被剥离在空气之下,令人毛骨悚然。

		\blankrev
		这样的故事似曾相识,人好像总是要把异类打回原型才痛快安心。\par
		而所谓异类,那些被称作妖与怪的生灵,却分明在无言里对照出人的阴暗卑劣。\par
		多少梦中,我见到海上飘浮的五彩泡沫,多少梦中,我坐在礁石之上,听人鱼们月光里的歌声。

		\blankrev
		太多的美与爱,在童话里,在我们心中,无论真假,只要你相信了,它便是存在。\par
		人该如那些美丽的人鱼一样,执着无悔,充满勇气地去追寻真爱。

		\blankrev
		看我的鱼,安静地发呆,无知觉地度过又一个夏日的午后。\par
		从前,北京的四合院里会安放几缸石刻的金鱼缸,里边栽上睡莲,然后养上色彩斑斓的金鱼。\par
		那是多么诗意的设置。想象着在一个同样的夏日午后,立在漏下清澈日光的院中,看鱼在莲叶间时隐时现的穿梭。\par
		鸽哨飞过晴空,在云上洒下清脆悠远的回声。那时的北京,少了生活的仓皇,多了如鱼的从容。\par
		大概再没有那样的一处院落,因为,没有了那样一种情怀与心境。

		\blankrev
		小时候,家里的大鱼缸中曾养着一群热带鱼。\par
		我常常用小网捞起来,一条条细心地抚摸,如抚摸一只小猫或小狗那样。\par
		不多久,那一群鱼便相继死去。后来我才知道,鱼是经不起那样每日的抚摸的,特别是本身就娇嫩的热带鱼。

		\blankrev
		我的爱,竟然成为了致命的伤害。\par
		但是,那时的我,抚摸的初衷确实是出于单纯的喜爱。\par
		长大的我,才慢慢懂得,这世间太多的事,是由不得一厢情愿的。

		\blankrev
		我的鱼,两条沉默的金鱼,摆动着纱裙一样的尾巴,在我的书桌上度过这个寂静浮躁的七月。\par
		我读几页书,写几行字,想些无关痛痒的心事。\par
		有时,因为沉默,我竟觉得自己也仿佛是一条鱼了,一样是擦身与错失,被水围困,也被水拥抱着。\par
		只是,我如何能如鱼般,在水压之下,也从容优雅,我如何能如鱼般,不忧不惧地绽放生命,心无旁骛。

		\blankrev
		当这世界上还没有人,便有了鱼。\par
		关于鱼的一切,是天地留给我们的一道谜题。

	\endwriting


	\writing{七月二十六日}{2007年7月26日} %<<<2

		此时的小鹿,大约正一个人在宿舍中对着电脑屏幕,看一部又一部电影。

		而兔子小姐,与两千多人挤在交大的礼堂中听着长达9小时的考研政治。

		想象着她们各自的生活,在这个半明半暗的7月。我在这困顿的城市一角,9楼之上,时而昏睡,时
	而清醒。

		写些恣情恣意的文字,流水帐一样记录的心事,无关任何人,只关我自己的东西。

		躺在床上,头脑中却是不断地自言自语。母亲总说,我是心思太重的孩子。

		心中的那个声音,却始终不断地想起,很多年了,碎碎地说着话。那些话,一些被我写下来,一些
	永久地流失在记忆的河水。

		夏天是最虚妄的季节。日光,雨水,雾气的氤氲,一切都让我感觉生命是如梦幻一场。

		人如何能够感受到这所有,看到绿树婆娑,听到蝉声聒噪,触摸到风,感受到欢乐与悲伤。

		这些看似平常的事,细细思量,不是充满惊奇的么?我想,当我还是个婴儿,当我第一次知觉了这
	个世界,一定是满心欣喜的。

		好像此时,我欣喜地望着这个夏天,想起许多个夏天,属于我的夏天。

		它们像一部小说,书写着我的青春,这样匆忙,甜蜜又哀伤的青春。

		学校的宿舍楼下,栽种了一排向日葵。小鹿向我感叹,在南方没有见过。也许是因为阳光少吧,她
	推测说。

		没有在南方常住过,却在几次短短的旅途中领略到潮湿多雨的气候。一位师兄曾对我说,他是喜欢
	北京的晴天,因为在家乡,一年里没有几天能看到太阳的。

		向日葵,是向阳的植物,该是生长在北方这干燥得或许有些焦渴的土地。一方水土养育一方人,从
	植物身上,已体现得这样明显。

		南方北方,或许真的有太大的区别。南方话总是细语绵绵,带着水汽的柔软温润。

		北方的女孩不喜欢南方的男人,好像南方男人对北方女孩存在的偏见。

		读我的文字,一个网上的朋友问我说,你是哪里人?当我说我是北京人时,他大惑不解地惊叹,怎
	么可能,怎么可能,你别骗我了。

		我不知道,北京女孩写字该是什么样子。大约,不该有如我这般轻柔含蓄,而该直来直往,偶尔夹
	杂一两句粗口的?良说,他第一次来北京,在地铁中就遇到两个旁若无人讲粗口的北京女孩。他于是对
	北京女孩的印象很不好,大学几年都与班里的北京女孩没什么接触。

		这件事听得我心中难免愤愤。我认识的多数北京女孩,都是从不会讲脏话的,像兔子小姐,一个细
	心善良的女孩子,是多么可爱。

		人对于家乡必然是有特殊的感情的。你自己可以骂它百般的不是,却容不得别人说半句它的坏话。

		好像我整日抱怨着北京的拥挤局促,在想起那些绿树红墙琉璃瓦,却还是无比亲切温暖。

		走在南方的街上,听着听不懂的话,深切地感受自己是个外来人,就开始想念起北京。

		也许,小鹿在北方也是同样的感受,3年了,她却就这样无声息地承受下来。而今,她终于决定要
	回到南方去读研究生了。她说着这件事的时候,眼睛亮亮的。我知道,她是太想家的了。

		都说北方人恋家,其实,哪里的人不是一样的呢?

		这个7月,我看着朋友们的奔波,也看着自己的无所事事。

		每个人都在为自己的生存努力,只是,我的奢求少了许多。现在,我只想平安地度过多一些的时光
	,珍惜多一些的美好,陪在母亲身边,让她安心,不再为我操劳。

		我知道,这些是我生命中最重要的东西,我要紧紧握住。

		最近的日子很好,平静,清晰,虽然好多时候,也难免沮丧,在失眠的枕上落下一两滴泪来。

	\endwriting


	\writing{友情卡片}{2007年7月30日 ~ 20:55} %<<<2

		从车窗里,看这个热络的世界,北京,满地落花的街道。

		槐树,细小的洁白花朵,在这个夏天,落满这座城。

		播放器里,是朱也喜欢的那首《友情卡片》。“好怀念那夏天……”

		你有没有听到一首歌就会在心里轻轻疼痛的经验。你有没有想着想着会流下眼泪的朋友。

		朱,在七月,我无法不想念。

		朱,海的那一边有没有落花的街道,有没有隐隐的忧伤,随风飘逝。

		你还好吗。快乐吗。幸福吗。你的小爱人对你呵护备至吗。

		如果真的可以,我要永远和你住在那段回忆里。

		看你纯真的脸,看我纯真的脸,看我们的春夏秋冬,快乐的,悲伤的,一切一切。

	\endwriting


	\writing{田的碎珠链。一}{2007年08月02日 ~ 19:04:30} %<<<2


		听我的自言自语,听我的一心透明。


		\subpart{一。旋转}

		永不停歇的红舞鞋,飞驰欢乐的木马,一支被循环播放的歌,令人目眩的世界,在我眼前画着圆圈
	,旋转,旋转。

		小时候,我最怕坐转椅,每次坐总是剧烈头晕。四周景物渐渐模糊,仿佛将在速度中消失。\par
		当别人坐在转椅上旋转欢笑的时候,我是人群外模样伶仃的孩子,轻轻咬住粉红的下唇。

		\blankrev
		后来,我听到一个叫做红舞鞋的童话。\par
		后来,我爱上一种叫做旋转木马的游乐器。\par
		后来,我听到一首叫做《旋木》的歌,和歌声背后的故事。

		\blankrev
		那个残忍的童话,一双找了魔的红舞鞋,让我看到欲望与诱惑是多么可怕。\par
		我开始想要一颗清水做的心,澄澈,明净,没有躁动与贪心。\par
		当我坐在木马的背上,当我看到你回过头,微笑着为我拍照,我以为,我们依旧是孩子。\par
		是可以穿着洁白的小纱裙,任性地插起腰,撅起嘴的小公主。\par
		夏天的风,吹乱你的发,那一天,我们一次次坐旋转木马,不厌其烦,乐此不疲。\par
		木马在旋转的,分明是一场梦,童年里毫不迟疑的烂漫天真。\par
		17岁的冬天,我的cd反复播放那一首歌,《旋木》。Faye透明的歌声,刺破了耳膜与神经,直入我骨髓深处。\par
		在十二月的灰天空下,我想象着亮起彩灯的旋转木马,想一个穿纱裙的女孩,乌黑的眼睛,闪闪烁烁。\par
		一些是甜美,一些是忧伤。夜空下,星光将整个世界的安静收集,编制成一张温柔而明亮的网。\par
		21岁的六月,在KTV包厢中点唱这一首歌。画面更迭,灰白的色调,出现一张清癯苍白的脸孔。\par
		“知道么,他便是这首歌的作者,患了罕见的肿瘤,24岁,也就是04年就去世了。”\par
		静静听着同伴的话,眼前是消瘦的男子,温和恬淡的笑。\par
		原来,歌声背后藏着这样美好却匆忙的生命。\par
		天堂里,有没有乐音起伏。生命的圆圈,在我眼前,一条完美的弧,一次人间的行旅。\par
		旋转。旋转。我有几分晕眩。\par
		小学时,我们喜欢玩面对面拉着手旋转的游戏。\par
		看对面那张大笑不止,紧闭了双眼的脸,看模糊的背景,消失的树木和房屋。\par
		现在,谁会愿意陪你,玩这童年的游戏,在人世纷纷。\par
		谁拉紧你的手,同你一起,旋转出日子的一个个圆圈,一场场欢笑,或泪水。


		\subpart{二。天光}

		背景是亮的,树木是暗的,漏下来的,是淡蓝的天光,淡到仿若无物。\par
		红砖铺砌的人行步道上,落满槐树细小的白色落花。\par
		八月的城,在几场彻夜的雷雨之后显得淡漠而温柔。\par
		那几夜,我躺在黑暗里,看闪电划亮了窗口,又在瞬间里熄灭。\par
		耳畔是欧波冰凉的歌声。他唱:

		\longpoem{}{}{}
		深深亲吻吧 ~~ 紧紧拥抱吧 \\
		再一次对你所爱的人吧 \\
		深深亲吻吧 ~~ 紧紧拥抱吧 \\
		再看一眼你深爱的人吧 \\
		擦干眼泪吧  ~~ 采束百合花 \\
		如果你永不会忘记他 \\
		送给他鲜花 ~~ 为他歌唱吧 \\
		如果你会永远爱着他
		\endlongpoem

		于是,就这样迷恋一个男人的声音,无力自拔。任音符一寸寸浸在肌肤深处,变得像一场疯狂的爱
	情那样铭心刻骨。

		我想象着,有个人,同样声音冰凉的男人,在寂静的夜晚为我唱一首忧伤的情歌。\par
		让我们都朦胧了一双泪眼,为了相爱的疼痛。\par
		为什么一定要是忧伤的呢。司汤达说,真正的爱是不笑的。\par
		亲爱的人,我却要微笑着,与你相对凝视,用尽青春,用尽今生,哪怕,是一路的颠沛流离。\par
		枕上的梦里,谁在天光未息的花圃为我采下一束百合花。\par
		谁将我紧紧拥抱,用深深的吻,唤醒我在飘零无助的噩梦一场。\par
		我的世界一瞬间如此淡了,淡到仿若无物。\par
		雨声连绵,要用多少滂沱如注的夜晚,才能冲刷净一面心灵,才能淹没了欲望与贪心。\par
		我在枕上听,我在枕上昏睡,我在枕上清醒。

		\blankrev
		亲爱的人,这一夜,可有凉风扑入你的怀中。


		\subpart{三。海}

		夏天,我们该去海上。

		\blankrev
		看远天膨胀的云,看细细的桅杆,看海鸥的翅,浪花的舌。\par
		该站在你的身边,戴一顶宽沿的草帽,让长长的蝴蝶结丝带在海风中飘呀飘。\par
		想穿一条白色的吊带裙,想赤着双脚,想在炽热的沙滩上一路跑去,再重重跌倒在涨潮的浪中。

		\blankrev
		你会捡来贝壳,细心地一只只穿好,挂在我的脖子上。\par
		然后,像欣赏一件艺术品那样,充满赞许与骄傲地看着我,直到我已双颊通红。\par
		你会笑,一张被日光曝晒得健康非常的脸,一件皱巴巴的花衬衫,那笑容肆无忌惮。\par
		我说,我想看海上的月亮。\par
		于是,我们等待着夜晚,在路边买一瓶瓷瓶酸奶,坐在阳伞下慢慢喝着,直到日光淡去,月光亮起。\par
		夏天,该是色彩浓郁的油画,带着海的腥味,海的怒气和温柔,有时喧哗,有时却又是寂静。\par
		该拍下许多照片,快乐的,疯狂的照片。该亮出闭合太久的口腔,在镜头中尽情尽兴地龇牙咧嘴。\par
		夏天,该是恣意的,为所欲为。

		\blankrev
		什么时候,我能够拥有那样一个夏天。

		\blankrev
		你说,相爱是一件轻易的事么。你说,生命是一出荒诞的演出么。\par
		你没有答案,我没有答案。夏天的头脑,总是发烧一样,充满了幻觉和混乱。\par
		于是,好多时候,我以为自己是在海上。好像一只漂流瓶,身体中装上秘密的信件,漂洋而去。\par
		有一天,我会到达一处彼岸。\par
		那里,有没有传说中的花树繁茂,有没有你,向我挥手微笑。\par
		海在我的世界,是如此远,又如此亲近。

		\blankrev
		夏天,我在陆地上想念海。\par
		我在文字里想念你。

	\endwriting


	\writing{田的碎珠链。二}{2007年08月07日 ~ 17:42:51} %<<<2


		花影轻摇的下午,谁来欢喜我的幸福,谁来心疼我的悲伤。


		\subpart{四。石榴}

		从车窗里望见路旁一株一树绯红的石榴花,翠色的枝上已生了玲珑的果实。\par
		在这条车水马龙的街上,她站在那里,显得无助,却又是高傲。

		让我想起春天里,中关村东路上那一路樱花。飘零在四月的风中,和了脂粉的泪一样。却没有人
	去疼惜,身旁,总是绝尘而去的车流。

		绿灯亮起,所有的车子在瞬间里启动,石榴花从在我视野里渐渐远了,远了,终于不见。\par
		有多人人会在经过时,如我一般注意到她的存在?

		一树绯红的花,像一心热烈的期许,在夏日的街头绽放。在我眼中,她是历尽红尘的女子,一袭
	红裙,望这依旧形色匆忙的世界,轻轻一笑。

		\blankrev
		有一句话,叫做“拜倒在石榴裙下”。常常,这话之前还要加上“多少英雄豪杰”。\par
		据传,这石榴裙的来历,与杨贵妃颇有关系,这却并不是我所关心的。\par
		我想象着的,是那石榴裙的真容,是那穿石榴裙的女子的芳泽。\par
		被染做石榴色的裙,穿在唐代女子的身上,毫不掩饰的青春,是那个遥远年代的俏丽多情。

		是一场梦回长安般的行旅,又仿佛追忆着自己一段虚无缥缈的前生,我读着石榴裙这三个字,竟就
	望见镜里的黛眉花钿,发上的金钗步摇。

		华清宫中曾绽放如霞的石榴,今日是否依旧。

		不经意的一次转眼,却已是风云流散的千年时光。穿石榴裙的女子,流转的美目不再,如铃的巧笑
	不再。

		唯留一份可堪琢磨,可堪怅惘的美丽,映衬在那个熠熠发光的时代中,容你我凭吊追忆。\par
		谁不愿是穿石榴裙的女子。\par
		谁不愿英雄豪杰拜倒在自己的石榴裙下。\par
		这是只属于女人们的童话。

		\blankrev
		旧宅的西房前有一株石榴。儿时的我并不曾在意花开的盛丽,只垂涎那一只只饱满开裂的果实。\par
		母亲会摘下它们,在柜子上一只只并排着安放好。\par
		每晚去家附近的试验田中散步时便带上两只。我们坐在田垄边吃那一颗颗甜美多汁的种子。\par
		蜻蜓在身边飞舞,孩子们追逐着,一路嬉闹地跑过。稻田带着水汽,散发着草木的清香。\par
		那是一些多好的夜晚。\par
		有时,还有一场缤纷的火烧云在西天上演。\par
		现在,我常常想念童年的夏天。没有浮躁,没有不安,没有城的喧嚣和匆忙。\par
		石榴甜美的汁水浸满唇齿,一棵树,把生命的蜜无保留地奉献给我。\par
		长大后,再没有吃到过同样的石榴。\par
		搬家的时候,石榴被掘起,包扎,转送他人。听说不多久便死去了。\par
		母亲说,草木亦是有情的,换了水土和主人,往往长不好。\par
		那是一株深情的石榴。\par
		现在,我不再吃石榴。


		\subpart{五。阳台}

		每一家的阳台都用塑钢的门窗封起,底层的几家,还安装了铁笼似的护栏。\par
		只有四层的一户,阳台四面通透,没有加装任何。\par
		我仰起头,看这一栋旧去的六层砖楼。它全然一副戒备的紧张,只在四层轻轻舒了口深长的呼吸。

		那一户是不是没有人居住?窗台上依稀有花影摇动,玻璃窗也擦得晶亮,几只雪白的袜子在夏风里
	等待风干。

		那么,主人为什么不封起阳台,如所有的邻居一般?\par
		我不得而知,那一个四面通透的阳台却把我深深打动。

		阳台,本是居住在局促住宅中的人的一处喘息之地。它从水泥的囚笼里伸出,给你一个空间,把身
	体浸泡在外界的空气中。

		阳台,本该是我们的世外桃源,本该有一张藤椅,一盏清茶,一帘明月。\par
		让四面的风吹来,让冬日的雪花落满,这小小天地,该纵容着自己,也纵容着自然。\par
		在日影斑驳里,懒洋洋地读一卷闲书,朦胧着头脑和耳目,不求甚解。

		或者,探头出去,看看楼下的人来人往,看这个琐碎的世俗世界的嬉笑怒骂,然后,以旁观者的身
	份笑一场。

		也许,也只有旁观之时,你我才得看清人生荒诞。\par
		我不知那四层的主人对于阳台也有如我的看法。\par
		有星星的晚上,他会站在阳台上等待一颗流星的划过么。\par
		隆冬,他会在阳台上撒一把小米,等着麻雀来啄食,一个人悄悄躲在玻璃窗后看着,微笑么。\par
		那究竟是怎样一个人。\par
		或许,是一个乌发覆额的女孩,或许,是独居多年的妇人,或许,是漂泊半生的老者,或许……\par
		阳台背后藏着的那个人,是一个永不必解答的疑问。\par
		不知,那些把自己围困在自设的铁笼之中的领人,会不会发觉自己的可笑。\par
		大概,非但不会,反而会对如此的高明赞许不已。\par
		人,多数时候是被自己所囚禁而毫不自知,原来,这是真的。


		\subpart{六。寂静}

		那一年,我们一起痴迷聂鲁达的诗《我喜欢你是寂静的》。

		\longpoem{}{}{}
		我喜欢你是寂静的,彷佛你消失了一样,\\
		你从远处聆听我,我的声音却无法触及你。\\
		好像你的双眼已经飞离远去,\\
		如同一个吻,封缄了你的嘴。\\
		……
		\endlongpoem

		我们反复读着,这样美丽的字句,深深沉浸其中。我们想,原来爱是这样哀伤惆怅的缠绵。\par
		那一年,我对你说,未来,若我爱上什么人便只会远远地望他,而不靠近。\par
		不与他说一句话,不交汇一处眼神,不在他的记忆出现,留下任何痕迹。\par
		我将沉默着爱他,在他的全然不知中。\par
		你笑,你问我,你能做到么。\par
		我没自信地摇摇头。毕竟,我曾是那个大声宣布,要将他的回忆全部霸占的女孩。\par
		我爱,于是,我贪得无厌,于是,我容不得任何的疏忽和瑕疵。我总是爱得自私而贪婪。\par
		而今,我却说,要沉默地去爱一个人。

		我怎么会甘心,甘心站在他的对面,却是永远的陌生,甘心在他的世界里,连我的名字也不曾出现
	,哪怕一瞬。

		他们说,真正的爱,是“我爱你与你无关”。\par
		然而,除非你从未知晓,不然,你又如何忍心让这份爱恋与你无关。\par
		我终于不是那种能将爱情溶于寂静的女子。

		\blankrev
		那些寂静的爱,却令我神往。\par
		好像那一个听来的故事。

		时过中年的女人,给出版社写信转作者,表达对新出版的一本诗集的喜爱。那是一位成名不久的小
	说家的诗集。

		出版商趁着他小说的风潮,找来他早年的诗作,合编一册推入市场。

		女人在信中写:感谢你的诗,让我仿佛回到了青春的年代…… 你的情诗,使我悔恨自己不曾炽烈地
	爱过一次。

		她用蓝色的墨水书写,娟秀的小字,好像出自年轻的女孩之手一般。\par
		这个年代,还有人会手写一封信,来表达对一位作家的喜爱,出版社的工作人员有些意外。\par
		当小说家怀着同样的意外,拆开那一封信,他的嘴角浮起了笑意。\par
		他将那一封信夹入自己那个写满了诗行的旧日记本里。\par
		那一夜,小说家失眠了。好几次,他来到书桌前,提起了笔却又迟疑着缓缓放下。\par
		他想给女人回一封信,却终于没有。他醒着,直到天光亮起。\par
		最后,他拿出女人的信,在页尾的空白处写下:你永远不会知道,那些诗,都是为你写的。\par
		从他展开信纸的那一刻便认出了那熟悉的字迹,结尾的署名更令他一心怅惘。\par
		小说家将一切的爱埋藏了。\par
		用最寂静的方式爱着。\par
		女人读着诗的时候,会感动,会流泪,会记起自己如花的青春,却不会知道,那一份寂静的深情。

		事实上,她从不知道小说家,甚至,没有听说过他的名字,那一年,他只是隔壁班一个默默无闻的
	男孩。

		\blankrev
		有时,我羡慕那个女人。有时,我又为她遗憾。\par
		但也许,寂静,的确是爱情最美的样子。\par
		是否,在你的青春里,也有一个寂静的人,在人群之中将你的所有悉心珍藏。\par
		是否,也有一双你从未察觉的眼,跟随着你,让你就这么轻易,将他所有关于青春的回忆霸占。\par
		就让这寂静的爱成为一生的秘密,归于尘土。\par
		或者,直到某天,时光老去,有什么人对你说起:他曾经爱你。

	\endwriting


	\writing{咳嗽}{2007年8月8日 ~ 20:00} %<<<2

		如果不咳嗽该多么好。不会半夜一次次醒来,咳到心肺俱裂。\par
		如果不生病多么好。不会流着汗,怅惘地荒废掉夏天。\par
		可以去海上,可以一个人旅行,可以满心欢喜地打扮,看自己镜中美丽的模样。\par
		什么时候,我从自恋转向了自怜。\par
		我要好起来,我对自己说。\par
		肉体的痛苦,是在将我的灵魂度化么?

	\endwriting


	\writing{如果}{2007年8月9日 ~ 16:06} %<<<2

		零七。夏。苍白依旧。却努力。要自己有安宁的心。田。一定要相信。

		\longpoem{}{}{}
		如果。\\
		能够不因丑陋而自卑。\\
		不因病弱而哀伤。\\
		不因困顿而绝望。\\
		生命。本可以很明亮。

		如果。\\
		爱情有前生今世。\\
		有姻缘注定。\\
		有不悔执迷。\\
		幸福。是如何轻易。又如何艰难。
		\endlongpoem

	\endwriting


	\writing{菲。夏}{2007年8月10日 ~ 16:10} %<<<2

		在音乐盒里放上 Faye 的几支歌。\par
		反反复复唱着的,陪伴我许多欢乐与悲伤的歌。\par
		一些是安静,一些是喧嚷,一些是贴入心房的冰凉和温暖。\par
		听 Faye 的人精神上大约都是寂寞的。\par
		自己的小世界,不开一扇窗。只有天光透进,清风穿堂。\par
		让她的歌声陪伴着。然后,一个人去思想。

		\blankrev
		这个夏天好好的。\par
		生病的只是我自己。\par
		一场场沉睡,不得安稳的夜晚。\par
		有时,我很累了。\par
		有时,我想拒绝悲伤与哭泣了。

		\blankrev
		在我的右眼下有一颗痣。那是一颗会使人流泪的痣。\par
		如果可以,只让我的右眼去流泪吧。\par
		另一只眼睛,让她拥有明媚与微笑。

		\blankrev
		骤然风起的午后。\par
		我始终沉默的风铃,耐不住寂寞。

	\endwriting


	\writing{隐藏}{2007年8月12日 ~ 20:47} %<<<2

		想把自己隐藏起来。\par
		好多事力不从心。\par
		没情绪。\par
		毫无情绪。\par
		混沌 ……

	\endwriting
	%2>>>
	%1>>>

	\newsect{其它} %<<<1

	\writing{雪·雨}{现场写作,2007年1月15日} %<<<2

		无论雪或雨,都总是带着清寒的气味,从天而降。覆盖草木浓密的原野,覆盖连绵苍茫的远山,覆
	盖人们漂泊躁动的心,也覆盖了一个个飘起雪花或蒙上水雾的日子。有多少次,我是从自己那扇小小的
	窗台望出去,只一个人,安静在雪与雨的视野中。他们分明是精灵,水做的精灵,总引起那些无端而起
	的思绪。一张图画,一幅面孔。一种丢失许久的气味,都随着雪雨的到来翩然起舞。这些时候,我感谢
	他们,令我与记忆的距离如此贴近了,仿若触手可及。

		落雪的冬天,永远是一场纯白色的梦一样。雪却是那样一些细微的声音。是在清早的被里听到的扫
	雪声;是孩子们跑在雪地里,鞋底想起的咯吱声;是打起雪仗来,一个雪球飞过耳际的声音。我喜欢在
	冬天堆一个雪人,喜欢看他雪白肥胖的模样,站在院子中央。那些冬天里,祖母总是会坐在小小的炉火
	旁,烧一壶汩汩冒气的热水,热一两只被烤出焦黄的馒头。她是那样一个羸弱的身影,在冬天里显得单
	薄而伶仃。那双布了皱纹老得嶙峋的手却总是暖的。拉起我的小手,涂抹上肥皂,洗净,再用毛巾细心
	擦干。在冬天来临前,她为我做好棉衣棉裤,一年又一年。总是厚厚的棉花,绵密的针脚。祖母要我试
	穿上去,再左左右右地看好一阵。她总是怕我冷,好几次喃喃地责怪母亲:“怎么不给孩子穿双棉鞋,
	冻坏了脚可怎么办……”而我记忆里的冬天却从未寒冷过,我是被包裹得平平实实的,不知人间的阴晴冷
	暖。下雪了,我和哥哥便鱼贯而出。哥哥大我四岁,我于是总是像个小“跟屁虫”那样,跟在他的身后。
	雪是悠悠落下的,我用小小的手掌接住它们的下落,看那六棱的花朵霎那间化作一滴清水。雪地原是时
	间最脆弱的花朵,比昙花的一现,更加匆忙决绝。哥哥问我,你最喜欢什么花?我的回答便是雪花。他
	皱了皱眉头,“那叫什么花呀?哪有月季花好看,还香。”雪花不是花么?那一年,我七岁,看着手掌中
	那近乎完美的六棱形,一心疑惑。雪花不是花么?我不要问这个问题。甚至,满满地不去想了。后来,
	我竟几乎忘了,忘记那朵在手心中化作清水的花朵,也忘记了儿时想起扫雪声的院落。它们都远了,小
	院子在一次城市改造中被拆掉,建起了高楼大厦。搬家的那天,我站在已空荡的院子,心竟也是空的。
	我们都长大了。

		我们竟然都长大了。十九岁的夏天。已经参加工作的哥哥对我说,时间老人真坏。我笑了,我想问
	他,哥哥,你最喜欢的是什么花?你还会不会想起,祖母种的月季,会不会记得,那一种平凡却贴近着
	我们记忆的花。祖母在花坛前忙碌的身影,是一样的单薄和伶仃。花却是一派的明媚和鲜亮。好像我们
	一天天在度过的生活。她的日子,是不紧不慢地过着,用含了泪似的目光望着我们的长大。谁也不知道
	,祖母的心中藏了多少的岁月,谁也不知道,祖母亲见了多少场雪雨的绽放与凋败。她默不作声,她缝
	合着许多我们所不了解的时光,一针针,一线线,一种种悲伤和喜悦。她有一张旧照片,压在抽屉最下
	边,我和哥哥曾翻出来看,被她慌忙得夺去了。后来,我们再没有见到那张照片,记忆中,它已泛黄,
	也许是淋过水的缘故,显得迷糊而褶皱。依稀间,却仍可见两张面孔,一男一女,中年的模样,头向里
	微倾着,笑意淡淡。

		我渐渐明白,是什么凝结在那张照片里。

		祖母离开的那年夏天,下了很多雨,使她疼痛得整夜无法入睡的疾病终于带走了她。那间阳光并不
	充裕的小屋,显得更加阴暗了。放学归来的我,只见到那一床被整理得平整的被褥。祖母好像没有发出
	一丝声响,就如此轻易地从世界消失了。傍晚的时候,又下起雷阵雨,我从她那扇小窗望出去,又是雷
	电交加。湿润的空气,混杂了六月里的青草味,微微苦的,一阵阵涌入房间。我终于安静下来,坐在她
	的床上。祖母似乎什么也没有带走,却又似乎带走了一切。生死之界限,如此分明了,一处是回忆,一
	处是此刻的无言,后事操办得简单。忙于学业的哥哥从城南赶回,与我默默相对。我们都记起了什么,
	我们又遗忘了什么,谁也没出声。

		我被一种巨大的力量覆盖了。好像厚厚的雪被,又好像雷电的轰响。又下雪了么?我仿佛如梦方醒
	。窗外却又已是银装一片,玉花翻飞。清寒,我只在心中念出这两个字。我只想到祖母跳动着火光的侧
	脸,想到雪中那些细小的声音。记忆是一朵雪花么?总也经不起人间的温度,不容你深情地触摸,只是
	转瞬之间,已化清水。

		现在的哥哥在哪里呢,是不是依旧为了生计奔波,类瘦了身子。已经很久没有见到他。下雪的时候
	,他会同我一样,轻轻地想念吗。我们也在一针一线。密密缝合着各自的生活。我们都远了,真的远了
	。所有的记忆都像那张淋湿的照片,停在了原地,被用心的人藏起,埋入时光的底层。

		让雪落吧,落吧。让雨水蒙住我们回首的视线。原野与山川,观看着,也承受着。我的心,有时竟
	也如一片雪花了,如此轻,却又是最令人劳累的重量。

		就让我缓缓将你捧起吧,再轻轻放下。

		我从窗口望出去,世界是这样完好的。

	\endwriting
	%2>>>

	\iffalse %<<<2 一些其它
	\writing{雷}{2005年08月19日 ~ 09:50:52}
		% <重复,忘了哪里找到的,这个时间>
	\endwriting
	\fi %2>>>
	%1>>>

\end{document}


% vi: set ft=tex ts=4 sts=4 textwidth=94 iskeyword+=_ formatoptions+=2mM :
% vi: set foldmethod=marker foldmarker=<<<,>>> autoindent readonly :
